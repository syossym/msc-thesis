\selectlanguage{english}%

\chapter{Lindhard Screening Model}

\label{cha:Appendix_Screening}To implement the screened Hartree-Fock
approximation in chapter \ref{cha:Coulomb_Correlated_Optical_Transitions},
we need a screening model. One approach is to use a self-consistent
quantum theory of plasma screening involving arguments from classical
electrodynamics and quantum mechanics \citet{Chow1994}. Given an
electron at the origin of our coordinate system, we wish to know what
effect this electron has on its surroundings. To find out, we introduce
a test charge, i.e., a charge sufficiently small as to cause negligible
perturbation. In vacuum, the electrostatic potential due to the electron
is $\phi(r)=e/r$. However, in a semiconductor there is a background
dielectric constant $\epsilon_{b}$ which is due to everything in
the semiconductor in the absence of the carriers themselves. Furthermore,
there is the carrier distribution that is changed by the presence
of the test electron at the origin %
\begin{figure}
\begin{centering}
\includegraphics{\string"C:/Users/Yossi Michaeli/Documents/Thesis/Documents/Thesis/Figures/4_Lindhard_Screening\string".eps}
\par\end{centering}

\caption{\label{fig:Lindhards_Formula_Schematic}Change in the carrier distribution
due to an electron at the origin (after \citet{Chow1994}).}

\end{figure}
 (see Fig. \ref{fig:Lindhards_Formula_Schematic}). The new carrier
distribution, $\left\langle n_{s}(\mathbf{r})\right\rangle $, in
turn changes the electrostatic potential. We denote the carrier density
distribution as an expectation value since we plan to calculate it
quantum mechanically. To derive the induced carrier distribution,
we first simplify the problem by assuming that the screening effects
of an electron-hole plasma equal the sum of the effects resulting
from the separate electron and hole plasmas. As such we neglect excitonic
screening, which is not a bad approximation for the elevated carrier
densities present in conventional semiconductor heterostructures.
The density distribution operator can be defined as \begin{eqnarray}
\hat{n}(\mathbf{r}) & = & \psi^{\dagger}(\mathbf{r})\psi(\mathbf{r})\nonumber \\
 & = & \frac{1}{V}\sum_{\mathbf{k},\mathbf{k}'}e^{i\left(\mathbf{k}-\mathbf{k}'\right)\cdot\mathbf{r}}\hat{a}_{\mathbf{k}'}^{\dagger}\hat{a}_{\mathbf{k}}\nonumber \\
 & = & \sum_{\mathbf{q}}n_{\mathbf{q}}e^{i\mathbf{q}\cdot\mathbf{r}},\label{eq:Density_Distribution_Operator}\end{eqnarray}
where \begin{equation}
n_{\mathbf{q}}=\frac{1}{V}\sum_{\mathbf{k}}\hat{a}_{\mathbf{k}-\mathbf{q}}^{\dagger}\hat{a}_{\mathbf{k}}\end{equation}
is the Fourier amplitude of the density distribution operator. Starting
with the electron plasma, we can note from \ref{eq:Density_Distribution_Operator}
that the corresponding quantum-mechanical operator for the screened
electron charge distribution is $e\hat{n}_{s}(\mathbf{r})$ with\begin{eqnarray}
\hat{n}_{s}(\mathbf{r}) & = & \frac{1}{V}\sum_{\mathbf{k},\mathbf{k}'}e^{i\left(\mathbf{k}-\mathbf{k}'\right)\cdot\mathbf{r}}\hat{a}_{\mathbf{k}-\mathbf{q}}^{\dagger}\hat{a}_{\mathbf{k}}\\
 & = & \sum_{\mathbf{q}}n_{s\mathbf{q}}e^{i\mathbf{q}\cdot\mathbf{r}}.\end{eqnarray}
Here, the Fourier transform of the density operator is given by $n_{s\mathbf{q}}=\frac{1}{V}\sum_{\mathbf{k}}\hat{a}_{\mathbf{k}-\mathbf{q}}^{\dagger}\hat{a}_{\mathbf{k}}$,
and $V$ is the volume of the semiconductor medium. In a rigorous
treatment we would use the electronic part of the many-body Hamiltonian
to obtain an equation of motion for $n_{s\mathbf{q}}$. At the level
of a self-consistent Hartree-Fock approach, we can treat screening
effects on the basis of an effective single-particle Hamiltonian \begin{equation}
\hat{H}_{eff}=\sum_{\mathbf{k}}E_{e}(\mathbf{k})\hat{a}_{\mathbf{k}}^{\dagger}\hat{a}_{\mathbf{k}}\mathbf{-}V\sum_{\mathbf{q}}V_{s\mathbf{q}}n_{s,-\mathbf{q}},\label{eq:Effective_Single_Particle_Hamiltonian}\end{equation}
where \begin{equation}
V_{s\mathbf{q}}=\frac{1}{V}\int d^{3}rV_{s}(r)e^{-i\mathbf{q}\cdot\mathbf{r}},\end{equation}
with $V_{s}(r)=e\phi_{s}(r)$, and $\phi_{s}(r)$ is the screened
electrostatic potential.

With the effective Hamiltonian \ref{eq:Effective_Single_Particle_Hamiltonian},
we get the equation of motion \begin{eqnarray}
i\hbar\frac{d}{dt}\hat{a}_{\mathbf{k}-\mathbf{q}}^{\dagger}\hat{a}_{\mathbf{k}} & = & \left[\hat{a}_{\mathbf{k}-\mathbf{q}}^{\dagger}\hat{a}_{\mathbf{k}},\hat{H}_{eff}\right]\nonumber \\
 & = & \left(E_{e}(\mathbf{k})-E_{e}(\mathbf{k}-\mathbf{q})\right)\hat{a}_{\mathbf{k}-\mathbf{q}}^{\dagger}\hat{a}_{\mathbf{k}}\nonumber \\
 &  & +\sum_{\mathbf{p}}V_{s\mathbf{p}}\left(\hat{a}_{\mathbf{k}-\mathbf{q}}^{\dagger}\hat{a}_{\mathbf{k}+\mathbf{p}}-\hat{a}_{\mathbf{k}-\mathbf{q}-\mathbf{p}}^{\dagger}\hat{a}_{\mathbf{k}}\right).\end{eqnarray}
Taking the expectation value and keeping only slowly varying terms,
namely those with $\mathbf{p}=-\mathbf{q}$, we get \begin{eqnarray}
i\hbar\frac{d}{dt}\left\langle \hat{a}_{\mathbf{k}-\mathbf{q}}^{\dagger}\hat{a}_{\mathbf{k}}\right\rangle  & = & \left(E_{e}(\mathbf{k})-E_{e}(\mathbf{k}-\mathbf{q})\right)\left\langle \hat{a}_{\mathbf{k}-\mathbf{q}}^{\dagger}\hat{a}_{\mathbf{k}}\right\rangle \nonumber \\
 &  & +V_{s\mathbf{q}}\left(n_{\mathbf{k}-\mathbf{q}}-n_{\mathbf{q}}\right).\label{eq:Distribution_Function_Eq_Motion}\end{eqnarray}
We suppose that $\left\langle \hat{a}_{\mathbf{k}-\mathbf{q}}^{\dagger}\hat{a}_{\mathbf{k}}\right\rangle $
has solution of the form $e^{\left(\delta-i\omega\right)t}$, where
the infinitesimal $\delta$ indicates that the perturbation has been
switched on adiabatically, i.e., that we had a homogeneous plasma
at $t=-\infty$. We further suppose that the induced charge distribution
follows this response. This transforms \ref{eq:Distribution_Function_Eq_Motion}
to \begin{equation}
\left\langle \hat{a}_{\mathbf{k}-\mathbf{q}}^{\dagger}\hat{a}_{\mathbf{k}}\right\rangle =V_{s\mathbf{q}}\frac{n_{\mathbf{k}-\mathbf{q}}-n_{\mathbf{q}}}{\hbar\left(\omega+i\delta\right)-E_{e}(\mathbf{k})+E_{e}(\mathbf{k}-\mathbf{q})}\end{equation}
and \begin{equation}
\left\langle n_{s\mathbf{q}}\right\rangle =\frac{V_{s\mathbf{q}}}{V}\sum_{\mathbf{k}}\frac{n_{\mathbf{k}-\mathbf{q}}-n_{\mathbf{q}}}{\hbar\left(\omega+i\delta\right)-E_{e}(\mathbf{k})+E_{e}(\mathbf{k}-\mathbf{q})}.\label{eq:Distribution_Expectation_Value}\end{equation}
The induced charge distribution is a source in Poisson's equation
\begin{equation}
\nabla^{2}\phi_{s}(r)=-\frac{4\pi e}{\epsilon_{b}}\left[n_{e}(\mathbf{r})+\left\langle n_{s\mathbf{q}}\right\rangle \right].\end{equation}
The Fourier transform of this equation is \begin{equation}
\phi_{s\mathbf{q}}=\frac{4\pi e}{\epsilon_{b}q^{2}}\left(\frac{1}{V}+\left\langle n_{s\mathbf{q}}\right\rangle \right),\label{eq:Potential_Fourier_Transform}\end{equation}
where for a point charge at the origin \begin{equation}
n_{e\mathbf{q}}=\frac{1}{V}\int d^{3}r\delta^{3}\left(\mathbf{r}\right)e^{-i\mathbf{q}\cdot\mathbf{r}}=\frac{1}{V}.\end{equation}
Using $V_{s\mathbf{q}}\equiv e\phi_{s\mathbf{q}}$, substitute \ref{eq:Distribution_Expectation_Value}
into \ref{eq:Potential_Fourier_Transform} and sove for $V_{s\mathbf{q}}$
to find \begin{equation}
V_{s\mathbf{q}}=V_{\mathbf{q}}\left(1-V_{q}\sum_{\mathbf{k}}\frac{n_{\mathbf{k}-\mathbf{q}}-n_{\mathbf{q}}}{\hbar\left(\omega+i\delta\right)-E_{e}(\mathbf{k})+E_{e}(\mathbf{k}-\mathbf{q})}\right)^{-1},\end{equation}
where $V_{q}$ is the unscreened Coulomb potential. Repeating the
derivation for the hole plasma, and adding the electron and hole contributions,
we find the screened Coulomb potential energy between carriers \ \begin{equation}
V_{s\mathbf{q}}=\frac{V_{\mathbf{q}}}{\epsilon_{\mathbf{q}}(\omega)},\end{equation}
where the longitudinal dielectric function is given by \begin{equation}
\epsilon_{\mathbf{q}}(\omega)=1-V_{\mathbf{q}}\sum_{n\mathbf{k}}\frac{n_{n,\mathbf{k}-\mathbf{q}}-n_{n,\mathbf{q}}}{\hbar\left(\omega+i\delta\right)-E_{n}(\mathbf{k})+E_{n}(\mathbf{k}-\mathbf{q})}\end{equation}
This equation is the Lindhard formula. It describes a complex retarded
dielectric function, i.e., the poles are in the lower complex frequency
plane, and it includes spatial dispersion ($\mathbf{q}$ dependence)
and spectral dispersion ($\omega$ dependence). \selectlanguage{english}

