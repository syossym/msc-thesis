
\chapter{Symmetry Properties of Wavefunctions}

\label{cha:Appendix_Symmetry_Properties}

%
\begin{lyxgreyedout}
Put some intro here...
\end{lyxgreyedout}


Let $H(\mathbf{r})=T+U(\mathbf{r})$ be the Hamiltonian in \ref{eq:zero_center_Bloch_Equation}
with the kinetic operator $T$ and crystal potential $U(\mathbf{r})$,
and let $G$ be the point group%
\footnote{A \emph{crystallographic point group} is a set of symmetry operations,
like rotations or reflections, that leave a point fixed while moving
each atom of the crystal to the position of an atom of the same kind.
That is, an infinite crystal would look exactly the same before and
after any of the operations in its point group. In the classification
of crystals, each point group corresponds to a crystal class.%
} of the crystal such that the symmetry operator $g\in G$, the crystal
potential is invariant, i.e. \begin{equation}
\forall g\in G,\,\,\, U(g^{-1}\mathbf{r})=U(\mathbf{r}).\end{equation}
The kinetic operator $T$ is naturally invariant under the action
of the elements in the point group $G$. Assuming that $\psi(\mathbf{r})$
is an eigenfunction of $H$ with $H(\mathbf{r})\psi(\mathbf{r})=E\psi(\mathbf{r})$
and applying the symmetry operator $g$, leads to \begin{equation}
H(g^{-1}\mathbf{r})\psi(g^{-1}\mathbf{r})=H(\mathbf{r})\psi(g^{-1}\mathbf{r})=E\psi(g^{-1}\mathbf{r}).\end{equation}
Obviously $\psi'(\mathbf{r})=\psi(g^{-1}\mathbf{r})$ is also an eigenfunction
of the Hamiltonian operator $H(\mathbf{r})$ with eigenvalue $E$.
The wavefunction $\psi\lyxmathsym{\textasciiacute}(\mathbf{r})$ might
be the same wavefunction as $\psi(\mathbf{r})$ or might be linearly
independent. Applying all symmetry operations $g\in G$ therefore
results in a set of wavefunctions with the same eigenvalue $E$, spanning
a function space\begin{equation}
I_{\psi}=\left\{ \psi'(\mathbf{r})|\psi'(\mathbf{r})=\psi'(g^{-1}\mathbf{r})\,\,\,\forall g\in G\right\} .\end{equation}
Now, the point group $G$ is a closed group, meaning that for $g,\, f\in G:\, gf\in G$.
Therefore, the function space $I_{\psi}$ is invariant under the action
of $G$, \begin{equation}
\psi(\mathbf{r})\in I_{\psi}\Rightarrow\psi(g^{-1}\mathbf{r})\in I_{\psi}.\end{equation}
Suppose that $\left\{ \psi_{0},\psi_{1},\ldots,\psi_{N}\right\} $
is the basis of the subspace $I_{\psi}$, which is assumed to be of
the demension $N$. The basis is denoted as being the\emph{ irreducible
representation} of the group $G$. The desirable property deduced
from group theory is that for every symmetry group, there are only
a few distinct irreducible representations. Every irreducible representation
describes an unique way how wavefunctions are transformed under the
action of the elements of a symmetry group $G$. Therefore, the eigenfunctions
of the crystal Hamiltonian $H$ can be classified according to the
irreducible representation they form. It is clear that each eigenfunction
does only belong to one irreducible representation and it is clear
that eigenfunctions corresponding to different irreducible representations
are always orthogonal. In the following, the theory will be focused
on the zinc-blende crystal, for which the symmetry group of the Hamiltonian
is $T_{d}$, the tetrahedral group. The elements are given by the
24 symmetry operations mapping a tetrahedron to itself. For an appropriate
alignment and orientation, the symmetry operations are listed in Table
\ref{tab:Symmetry_operations_Td}. $T_{d}$ has in total five distinct
irreducible representations, which are commonly labeled as $\Gamma_{1}$,
$\Gamma_{2}$, $\Gamma_{12}$, $\Gamma_{15}$ and $\Gamma_{25}$.
$\Gamma_{1}$ and $\Gamma_{2}$ are one dimensional representations,
$\Gamma_{12}$ is a two dimensional and $\Gamma_{15}$ and $\Gamma_{25}$
are three dimensional. The basis functions for all irreducible representations
are given in Table \ref{tab:Basis_functions_Td}. The representation
$\Gamma_{1}$ is the identity representation, also denoted as trivial
representation, leaving the wavefunction invariant. An atomic wavefunction
with $s$-like symmetry transforms accordingly, i.e. is left unchanged
under the action of the elements of the group $T_{d}$. An example
of the $\Gamma_{1}$ representation is given by the conduction band
at the $\Gamma$ point of %
\begin{table}
\begin{centering}
\begin{tabular}{cc}
\toprule 
\textbf{Type} & \textbf{Operation} $\left(xyz\right)\rightarrow\left(\ldots\right)$\tabularnewline
\midrule
\midrule 
$E$ & $\left(xyz\right)$\tabularnewline
\midrule 
$3C_{2}$ & $\left(\bar{x}\bar{y}z\right),\left(xy\bar{z}\right),\left(\bar{x}y\bar{z}\right)$\tabularnewline
\midrule 
$8C_{3}$ & $\left(zxy\right),\left(yzx\right),\left(\bar{y}z\bar{x}\right),\left(\bar{z}\bar{x}y\right),\left(\bar{y}\bar{z}x\right),\left(z\bar{x}\bar{y}\right),\left(y\bar{z}\bar{x}\right),\left(\bar{z}x\bar{y}\right)$\tabularnewline
\midrule 
$6C_{4}$ & $\left(\bar{x}z\bar{y}\right),\left(\bar{x}\bar{z}y\right),\left(z\bar{y}\bar{x}\right),\left(\bar{z}\bar{y}x\right),\left(\bar{y}x\bar{z}\right),\left(y\bar{x}\bar{z}\right)$\tabularnewline
\midrule 
$6\sigma$ & $\left(\bar{y}x\bar{z}\right),\left(yxz\right),\left(\bar{z}y\bar{x}\right),\left(zyx\right),\left(xzy\right),\left(x\bar{z}\bar{y}\right)$\tabularnewline
\bottomrule
\end{tabular}
\par\end{centering}

\caption{\label{tab:Symmetry_operations_Td}Symmetry operations of the group
$T_{d}$ using the Sch\"{o}nflies notation (notations after \citet{Yu2005}).}

\end{table}
 zinc-blende, direct bandgap III-V semiconductors. The conduction
band at the $\Gamma$ point is non-degenerate with a wavefunction
obeying $s$-type symmetry. Another important irreducible representation
is given by the top of the valence band at the $\Gamma$ point. Neglecting
the later considered spin-orbit splitting, the valence band is threefold
degenerate, with $p$-type basis functions $x,y$ and $z$, transforming
under the action of $T_{d}$ according to the elements of a vector.
The $p$-type basis functions correspond to the representation $\Gamma_{15}$.
The goal is now to use the introduced group theory to analyze the
properties of the momentum matrix elements \ref{eq:momentum_matrix_element}.
The momentum operator $\mathbf{p}$ is given by a vector of three
operators\[
\mathbf{p}=\left(\begin{array}{c}
p_{x}\\
p_{y}\\
p_{z}\end{array}\right),\]
obviously transforming like the elements of a vector. Therefore, the
momentum operator forms an irreducible representation of $\Gamma_{15}$.
The action of the momentum operator on wavefunctions of the irreducible
representation $\Gamma_{j}$ leads to a new expression. The corresponding
representation is given by the \emph{direct product} $\Gamma_{15}\otimes\Gamma_{j}$
(see \citet{Yu2005}, p. 46), as one has $p_{u}$ acting on the basis
function $\psi_{v}$ for $u=x,y,z$ and $v=1,...,N$. The direct product
is not \emph{irreducible}, but can be decomposed into a \emph{direct
sum} of irreducible representations %
\begin{table}
\begin{centering}
\begin{tabular}{ccc}
\toprule 
\textbf{$\Gamma_{i}$} & \textbf{Dimension} & \textbf{Basis functions}\tabularnewline
\midrule
\midrule 
$\Gamma_{1}$ & 1 & $xyz$\tabularnewline
\midrule 
$\Gamma_{2}$ & 1 & $x^{4}\left(y^{2}-z^{2}\right)+y^{4}\left(z^{2}-x^{2}\right)+z^{2}\left(x^{2}-y^{2}\right)$\tabularnewline
\midrule 
$\Gamma_{12}$ & 2 & $\left(x^{2}-y^{2}\right),\, z^{2}-\frac{1}{2}\left(x^{2}+y^{2}\right)$\tabularnewline
\midrule 
$\Gamma_{15}$ & 3 & $x,\, y,\, z$\tabularnewline
\midrule 
$\Gamma_{25}$ & 3 & $x\left(y^{2}-z^{2}\right),\, y\left(z^{2}-x^{2}\right),\, z\left(x^{2}-y^{2}\right)$\tabularnewline
\bottomrule
\end{tabular}
\par\end{centering}

\caption{\label{tab:Basis_functions_Td}Basis functions of the tetrahedral
symmetry group $T_{d}$.}

\end{table}
\begin{equation}
\Gamma_{15}\otimes\Gamma_{j}=\sum_{u}\oplus\Gamma_{i}.\label{eq:direct_sum}\end{equation}
Recall that functions not belonging to the same irreducible representation
are orthogonal. Therefore, the matrix element $\left\langle \psi^{\Gamma_{i}}\mid\mathbf{p}\mid\psi^{\Gamma_{j}}\right\rangle $
between two wavefunctions belonging to the irreducible representation
$\Gamma_{i}$ and $\Gamma_{j}$ is nonzero only if the direct sum
\ref{eq:direct_sum} of the direct product $\Gamma_{15}\otimes\Gamma_{j}$
contains $\Gamma_{i}$. For the tetrahedral symmetry, the decomposition
of the direct product into the direct sum is given in Table \ref{tab:Direct_products_Gamma15},
from which the vanishing momentum matrix elements \ref{eq:momentum_matrix_element}
can be calculated. They are given by the matrix in \ref{tab:The_non_vanishing_momentum_elements_Td}.
Within the matrix, X denotes a non-vanishing and the 0 denotes a vanishing
matrix element. In order to further reduce the number of unknowns,
equivalent matrix elements can be determined using the basis functions
defined in Table \ref{tab:Basis_functions_Td} and symmetry operations
of $G$. As an example, %
\begin{table}
\begin{centering}
\begin{tabular}{cc}
\toprule 
\textbf{Direct product} & \textbf{Direct sum}\tabularnewline
\midrule
\midrule 
$\Gamma_{15}\otimes\Gamma_{1}$ & $\Gamma_{15}$\tabularnewline
\midrule 
$\Gamma_{15}\otimes\Gamma_{2}$ & $\Gamma_{25}$\tabularnewline
\midrule 
$\Gamma_{15}\otimes\Gamma_{12}$ & $\Gamma_{15}\oplus\Gamma_{25}$\tabularnewline
\midrule 
$\Gamma_{15}\otimes\Gamma_{15}$ & $\Gamma_{15}\oplus\Gamma_{25}\oplus\Gamma_{12}\oplus\Gamma_{1}$\tabularnewline
\midrule 
$\Gamma_{15}\otimes\Gamma_{25}$ & $\Gamma_{15}\oplus\Gamma_{25}\oplus\Gamma_{12}\oplus\Gamma_{2}$\tabularnewline
\bottomrule
\end{tabular}
\par\end{centering}

\caption{\label{tab:Direct_products_Gamma15}Direct products of the $\Gamma_{15}$
representation with all representations of $T_{d}$ (after \citet{Yu2005}).}

\end{table}
 for the $\Gamma_{1}$ type conduction band, the only non-vanishing
momentum matrix elements involve - according to Table \ref{tab:The_non_vanishing_momentum_elements_Td}
- only bands belonging to $\Gamma_{15}$. $\Gamma_{1}$ is represented
by the function $xyz$ and $\Gamma_{15}$ by $x,y$ and $z$. The
only non-vanishing matrix element is of the type $\left\langle xyz\mid p_{x}\mid x\right\rangle $.
For the other, e.g. $\left\langle xyz\mid p_{x}\mid z\right\rangle $,
a rotation of the crystal by $180\lyxmathsym{\textdegree}$ around
the rotation axis $[001]$ results in\[
\left\langle xyz\mid p_{x}\mid x\right\rangle =-\left\langle xyz\mid p_{x}\mid z\right\rangle ,\]
which can only be met if the matrix element is zero.%
\begin{table}
\begin{centering}
\begin{tabular}{cccccc}
\toprule 
$\mathbf{p}$ & $\left|\psi^{\Gamma_{1}}\right\rangle $ & $\left|\psi^{\Gamma_{2}}\right\rangle $ & $\left|\psi^{\Gamma_{12}}\right\rangle $ & $\left|\psi^{\Gamma_{15}}\right\rangle $ & $\left|\psi^{\Gamma_{25}}\right\rangle $\tabularnewline
\midrule
\midrule 
$\left\langle \psi^{\Gamma_{1}}\right|$ & 0 & 0 & 0 & X & 0\tabularnewline
\midrule 
$\left\langle \psi^{\Gamma_{2}}\right|$ & 0 & 0 & 0 & 0 & X\tabularnewline
\midrule 
$\left\langle \psi^{\Gamma_{12}}\right|$ & 0 & 0 & 0 & X & X\tabularnewline
\midrule 
$\left\langle \psi^{\Gamma_{15}}\right|$ & X & 0 & X & X & X\tabularnewline
\midrule 
$\left\langle \psi^{\Gamma_{25}}\right|$ & 0 & X & X & X & X\tabularnewline
\bottomrule
\end{tabular}
\par\end{centering}

\caption{\label{tab:The_non_vanishing_momentum_elements_Td}The non-vanishing
momentum matrix elements between the states corresponding to different
irreducible representations of the tetrahedral symmetry group. A zero
denotes a vanishing and X a non-vanishing element.}

\end{table}

