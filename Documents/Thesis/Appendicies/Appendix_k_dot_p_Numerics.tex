
\chapter{Two Band Model Numerical Implementation}

\label{cha:Appendix_Two_Band_Numerics}The method used to solve the
Hamiltonian equations in the two-band model is based on the shooting
method formulated in \citealp{harrison_quantum_2000} for the simple
case of the conduction band.

As a starting point, we consider the general one-dimensional form
of the Hamiltonian equation \ref{eq:Effective_Mass_Equation}. In
order to allow for a location dependent effective mass, we rewrite
this equation as\begin{equation}
\left(\begin{array}{cc}
\hat{H}_{hh}+V(z) & \hat{W}\\
\hat{W}^{\dagger} & \hat{H}_{lh}+V(z)\end{array}\right)\left(\begin{array}{c}
F_{hh}\\
F_{lh}\end{array}\right)=E(\mathbf{k})\left(\begin{array}{c}
F_{hh}\\
F_{lh}\end{array}\right),\end{equation}
with \begin{eqnarray}
\hat{H}_{lh} & = & -\frac{\partial}{\partial z}\left(\gamma_{1}(z)+2\gamma_{2}(z)\right)\frac{\partial}{\partial z}+\left(\gamma_{1}(z)-\gamma_{2}(z)\right)k_{t}^{2},\\
\hat{H}_{hh} & = & -\frac{\partial}{\partial z}\left(\gamma_{1}(z)-2\gamma_{2}(z)\right)\frac{\partial}{\partial z}+\left(\gamma_{1}(z)+\gamma_{2}(z)\right)k_{t}^{2},\\
\hat{W} & = & \left\{ \begin{array}{c}
\begin{array}{c}
\sqrt{3}k_{t}\left(\gamma_{2}(z)k_{t}-2\gamma_{3}(z)\frac{\partial}{\partial z}\right)\,\,\,\textrm{for}\,[100]\end{array}\\
\begin{array}{c}
\sqrt{3}k_{t}\left(\gamma_{3}(z)k_{t}-2\gamma_{3}(z)\frac{\partial}{\partial z}\right)\,\,\,\textrm{for}\,[110]\end{array}\end{array}\right.\end{eqnarray}
The potential $V(z)$ describes the valence band edge of the quantum
well structure (in terms of hole energy), and $F_{hh}$ and $F_{lh}$
represent the hole wavefunction while under the effective mass and
envelope function approximations.

We can rewrite the effective mass equations as \begin{equation}
\left(\begin{array}{cc}
\hat{H}_{hh}+V(z)-E & \hat{W}\\
\hat{W}^{\dagger} & \hat{H}_{lh}+V(z)-E\end{array}\right)\left(\begin{array}{c}
F_{hh}\\
F_{lh}\end{array}\right)=0.\end{equation}
The problem now is to find a numerical method for the the solution
of both the energy eigenvalues $E$ and the eigenfunctions $F$ for
any $V(z)$. For this purpose, we can expand the first and second
derivatives in terms of finite differences. The first derivative of
a function $f(z)$ can be approximated to \begin{equation}
\frac{df}{dz}\approx\frac{\Delta f}{\Delta z}=\frac{f(z+\delta z)-f(z-\delta z)}{2\delta z}.\label{eq:Finite_Dif_1_Der}\end{equation}
The second derivative follows as \begin{eqnarray}
\frac{d^{2}f}{dz^{2}} & \approx & \frac{\left.\frac{df}{dz}\right|_{z+\delta z}-\left.\frac{df}{dz}\right|_{z-\delta z}}{2\delta z}\nonumber \\
 & = & \frac{f(z+2\delta z)-2f(z)+f(z-2\delta z)}{(2\delta z)^{2}}.\label{eq:Finite_Diff_2_Der}\end{eqnarray}
As $\delta z$ is an, as yet, undefined small step along the $z$-axis,
and as it only appears in equation \ref{eq:Finite_Diff_2_Der}with
the factor 2, then we can simplify this expression by substituting
$\delta z$ for $2\delta z$\begin{equation}
\frac{d^{2}f}{dz^{2}}\approx\frac{f(z+\delta z)-2f(z)+f(z-\delta z)}{(\delta z)^{2}}.\end{equation}


Let us focus on the term $\hat{H}_{lh}^{0}=-\frac{\partial}{\partial z}\left(\gamma_{1}+2\gamma_{2}\right)\frac{\partial}{\partial z}$
in the light hole Hamiltonian, and express it in terms of finite differences.
We can rewrite this term as\begin{equation}
\hat{H}_{lh}^{0}=-\frac{\partial}{\partial z}\left(\gamma_{1}(z)+\gamma_{2}(z)\right)\frac{\partial F_{lh}}{\partial z}+\left(\gamma_{1}(z)+\gamma_{2}(z)\right)\frac{\partial^{2}F_{lh}}{\partial z^{2}}.\end{equation}
However, the shooting equations derived from this point by expanding
the derivatives in terms of finite differences have led to significat
computational inaccuracies in systems with a large discountinuous
change in the effective mass (the Luttinger parameters), as occurs
in the $GaAs/AlGaAs$ system. The source of the inaccuracies is thought
to arise from the $\delta$-function nature of the effective mass
derivative.

A more robust shceme can be derived by expanding $\hat{H}_{lh}^{0}$
starting from the left-hand derivative \begin{equation}
\hat{H}_{lh}^{0}\approx\frac{\left.\left(\gamma_{1}+\gamma_{2}\right)\frac{\partial F_{lh}}{\partial z}\right|_{z+\delta z}-\left.\left(\gamma_{1}+\gamma_{2}\right)\frac{\partial F_{lh}}{\partial z}\right|_{z-\delta z}}{2\delta z}.\end{equation}
Recalling the centered finite difference expansion for the first derivative
\ref{eq:Finite_Dif_1_Der}, we can write the numenator of the above
expression as \begin{eqnarray}
2\delta z\hat{H}_{lh}^{0} & = & \left.\left(\gamma_{1}+2\gamma_{2}\right)\right|_{z+\delta z}\frac{F_{lh}(z+2\delta z)-F_{lh}(z)}{2\delta z}\nonumber \\
 &  & \left.\left(\gamma_{1}+2\gamma_{2}\right)\right|_{z-\delta z}\frac{F_{lh}(z)-F_{lh}(z-2\delta z)}{2\delta z},\end{eqnarray}
or \begin{eqnarray}
(2\delta z)^{2}\hat{H}_{lh}^{0} & = & \left.\left(\gamma_{1}+2\gamma_{2}\right)\right|_{z+\delta z}\left[F_{lh}(z+2\delta z)-F_{lh}(z)\right]\nonumber \\
 &  & \left.\left(\gamma_{1}+2\gamma_{2}\right)\right|_{z-\delta z}\left[F_{lh}(z)-F_{lh}(z-2\delta z)\right].\end{eqnarray}
Making the transformation $2\delta z\rightarrow\delta z$ then yields
\begin{eqnarray}
\hat{H}_{lh}^{0} & = & \frac{1}{(\delta z)^{2}}\left[\left(\gamma_{1}-2\gamma_{2}\right)^{+}F_{lh}(z+\delta z)\right.\nonumber \\
 &  & -\left(\left(\gamma_{1}-2\gamma_{2}\right)^{+}+\left(\gamma_{1}-2\gamma_{2}\right)^{-}\right)F_{lh}(z)\nonumber \\
 &  & \left.+\left(\gamma_{1}-2\gamma_{2}\right)^{-}F_{lh}(z-\delta z)\right],\end{eqnarray}
with \begin{eqnarray}
\left(\gamma_{1}+2\gamma_{2}\right)^{+} & = & \left.\left(\gamma_{1}+2\gamma_{2}\right)\right|_{z+\delta z/2},\\
\left(\gamma_{1}+2\gamma_{2}\right)^{-} & = & \left.\left(\gamma_{1}+2\gamma_{2}\right)\right|_{z-\delta z/2},\\
\left(\gamma_{1}-2\gamma_{2}\right)^{+} & = & \left.\left(\gamma_{1}-2\gamma_{2}\right)\right|_{z+\delta z/2},\\
\left(\gamma_{1}-2\gamma_{2}\right)^{-} & = & \left.\left(\gamma_{1}-2\gamma_{2}\right)\right|_{z-\delta z/2}.\end{eqnarray}
We now substitute the finite difference expressions for $\partial/\partial z$,
$\hat{H}_{lh}^{0}$ and a similar expression for the heavy-hole counterpart
into the effective mass equations, and obtain\begin{eqnarray}
0 & = & -\frac{\left(\gamma_{1}-2\gamma_{2}\right)^{+}}{(\delta z)^{2}}F_{hh}(z+\delta z)+\frac{\left(\gamma_{1}-2\gamma_{2}\right)^{+}+\left(\gamma_{1}-2\gamma_{2}\right)^{-}}{(\delta z)^{2}}F_{hh}(z)\nonumber \\
 &  & -\frac{\left(\gamma_{1}-2\gamma_{2}\right)^{-}}{(\delta z)^{2}}F_{hh}(z-\delta z)+(\gamma_{1}+\gamma_{2})k_{t}^{2}F_{hh}(z)\nonumber \\
 &  & +\left(V(z)-E\right)F_{hh}(z)+\sqrt{3}\gamma_{2}k_{t}^{2}F_{lh}(z)-2\sqrt{3}\gamma_{3}k_{t}\frac{F_{lh}(z+\delta z)-F_{lh}(z-\delta z)}{2\delta z},\\
0 & = & \sqrt{3}\gamma_{2}k_{t}^{2}F_{hh}(z)+2\sqrt{3}\gamma_{3}k_{t}k_{t}\frac{F_{hh}(z+\delta z)-F_{hh}(z-\delta z)}{2\delta z}\nonumber \\
 &  & -\frac{\left(\gamma_{1}+2\gamma_{2}\right)^{+}}{(\delta z)^{2}}F_{lh}(z+\delta z)+\frac{\left(\gamma_{1}+2\gamma_{2}\right)^{+}+\left(\gamma_{1}+2\gamma_{2}\right)^{-}}{(\delta z)^{2}}F_{lh}(z)\nonumber \\
 &  & -\frac{\left(\gamma_{1}+2\gamma_{2}\right)^{-}}{(\delta z)^{2}}F_{lh}(z-\delta z)+(\gamma_{1}-\gamma_{2})k_{t}^{2}F_{lh}(z)+\left(V(z)-E\right)F_{lh}(z).\end{eqnarray}
The Luttinger parameters $\gamma_{i}$ can be found at the intermediary
points $z\pm\delta z/2$ by taking the mean of the two neighboring
points $z$ and $z\pm\delta z$.

It can be seen that we draw up a set of finite difference equations
if we map the potential $V(z)$ and the Luttinger parameters $\gamma_{i}$
to a grid along the $z$-axis. To solve these coupled equations and
find the energies $E$ and functions $F$ we assume a equidistant
grid $z_{i}$, with a grid step $\delta z$, we can substitute $z\rightarrow z_{i}$,
$z-\delta z\rightarrow z_{i-1}$ and $z+\delta z\rightarrow z_{i+1}$.
If we assume a given energy $E$, we are still left with 6 unknown
parameters in the finite difference equations. However, we can rewrite
these equations so that we are able to find $F_{lh}(z_{i+1})$ and
$F_{hh}(z_{i+1})$ from their values at the two previous nodes, $z_{i-1}$
and $z_{i}$\begin{multline}
F_{hh}(z_{i+1})\left[1+3\frac{\gamma_{3}^{2}}{\left(\gamma_{1}+2\gamma_{2}\right)^{+}\left(\gamma_{1}-2\gamma_{2}\right)^{-}}k_{t}^{2}(\delta z)^{2}\right]=\\
F_{hh}(z_{i-1})\left[-\frac{\left(\gamma_{1}-2\gamma_{2}\right)^{-}}{\left(\gamma_{1}-2\gamma_{2}\right)^{+}}+3\frac{\gamma_{3}^{2}}{\left(\gamma_{1}+2\gamma_{2}\right)^{+}\left(\gamma_{1}-2\gamma_{2}\right)^{+}}k_{t}^{2}(\delta z)^{2}\right]\\
+F_{lh}(z_{i-1})\left[\sqrt{3}\frac{\gamma_{3}}{\left(\gamma_{1}-2\gamma_{2}\right)^{+}}k_{t}\delta z\left(1+\frac{\left(\gamma_{1}+2\gamma_{2}\right)^{-}}{\left(\gamma_{1}+2\gamma_{2}\right)^{+}}\right)\right]\\
F_{hh}(z_{i})\left[\frac{\left(\gamma_{1}-2\gamma_{2}\right)^{+}+\left(\gamma_{1}-2\gamma_{2}\right)^{-}}{\left(\gamma_{1}-2\gamma_{2}\right)^{+}}+\frac{\gamma_{1}+\gamma_{2}}{\left(\gamma_{1}-2\gamma_{2}\right)^{+}}k_{t}^{2}(\delta z)^{2}\right.\\
+\left.\frac{V(z_{i})-E}{\left(\gamma_{1}-2\gamma_{2}\right)^{+}}(\delta z)^{2}-3\frac{\gamma_{3}\gamma_{2}}{\left(\gamma_{1}+2\gamma_{2}\right)^{+}\left(\gamma_{1}-2\gamma_{2}\right)^{+}}k_{t}^{3}(\delta z)^{3}\right]\\
+F_{lh}(z_{i})\left[\sqrt{3}\frac{\gamma_{2}}{\left(\gamma_{1}-2\gamma_{2}\right)^{+}}k_{t}^{2}(\delta z)^{2}\right.\\
-\sqrt{3}\frac{\gamma_{3}}{\left(\gamma_{1}-2\gamma_{2}\right)^{+}}\frac{\left(\gamma_{1}+2\gamma_{2}\right)^{+}+\left(\gamma_{1}+2\gamma_{2}\right)^{-}}{\left(\gamma_{1}+2\gamma_{2}\right)^{+}}k_{t}\delta z\\
-\sqrt{3}\frac{\gamma_{3}\left(\gamma_{1}-\gamma_{2}\right)}{\left(\gamma_{1}+2\gamma_{2}\right)^{+}\left(\gamma_{1}-2\gamma_{2}\right)^{+}}k_{t}^{3}(\delta z)^{3}\\
\left.-\sqrt{3}\frac{\gamma_{3}}{\left(\gamma_{1}+2\gamma_{2}\right)^{+}\left(\gamma_{1}-2\gamma_{2}\right)^{+}}\left(V(z_{i})-E\right)k_{t}(\delta z)^{3}\right],\end{multline}
\begin{multline}
F_{lh}(z_{i+1})\left[1+3\frac{\gamma_{3}^{2}}{\left(\gamma_{1}+2\gamma_{2}\right)^{+}\left(\gamma_{1}-2\gamma_{2}\right)^{+}}k_{t}^{2}(\delta z)^{2}\right]=\\
F_{lh}(z_{i-1})\left[-\frac{\left(\gamma_{1}+2\gamma_{2}\right)^{-}}{\left(\gamma_{1}+2\gamma_{2}\right)^{+}}+3\frac{\gamma_{3}^{2}}{\left(\gamma_{1}+2\gamma_{2}\right)^{+}\left(\gamma_{1}-2\gamma_{2}\right)^{+}}k_{t}^{2}(\delta z)^{2}\right]\\
+F_{hh}(z_{i-1})\left[\sqrt{3}\frac{\gamma_{3}}{\left(\gamma_{1}+2\gamma_{2}\right)^{+}}k_{t}\delta z\left(1+\frac{\left(\gamma_{1}-2\gamma_{2}\right)^{-}}{\left(\gamma_{1}-2\gamma_{2}\right)^{+}}\right)\right]\\
F_{lh}(z_{i})\left[\frac{\left(\gamma_{1}+2\gamma_{2}\right)^{+}+\left(\gamma_{1}+2\gamma_{2}\right)^{-}}{\left(\gamma_{1}+2\gamma_{2}\right)^{+}}+\frac{\gamma_{1}-\gamma_{2}}{\left(\gamma_{1}+2\gamma_{2}\right)^{+}}k_{t}^{2}(\delta z)^{2}\right.\\
+\left.\frac{V(z_{i})-E}{\left(\gamma_{1}+2\gamma_{2}\right)^{+}}(\delta z)^{2}+3\frac{\gamma_{3}\gamma_{2}}{\left(\gamma_{1}+2\gamma_{2}\right)^{+}\left(\gamma_{1}-2\gamma_{2}\right)^{+}}k_{t}^{3}(\delta z)^{3}\right]\\
+F_{hh}(z_{i})\left[\sqrt{3}\frac{\gamma_{2}}{\left(\gamma_{1}+2\gamma_{2}\right)^{+}}k_{t}^{2}(\delta z)^{2}\right.\\
+\sqrt{3}\frac{\gamma_{3}}{\left(\gamma_{1}+2\gamma_{2}\right)^{+}}\frac{\left(\gamma_{1}-2\gamma_{2}\right)^{+}+\left(\gamma_{1}-2\gamma_{2}\right)^{-}}{\left(\gamma_{1}-2\gamma_{2}\right)^{+}}k_{t}\delta z\\
+\sqrt{3}\frac{\gamma_{3}\left(\gamma_{1}+\gamma_{2}\right)}{\left(\gamma_{1}+2\gamma_{2}\right)^{+}\left(\gamma_{1}-2\gamma_{2}\right)^{+}}k_{t}^{3}(\delta z)^{3}\\
\left.+\sqrt{3}\frac{\gamma_{3}}{\left(\gamma_{1}+2\gamma_{2}\right)^{+}\left(\gamma_{1}-2\gamma_{2}\right)^{+}}\left(V(z_{i})-E\right)k_{t}(\delta z)^{3}\right].\end{multline}
These equations imply that, if the wavefunctions are known at the
two points $z-\delta z$ and $z$, then the value at $z+\delta z$
can be determined for any energy $E$. This iterative equation forms
the bais of a standard method of solving equations numerically, and
is known as the \emph{shooting method}.

The equations can be rewritten in a matrix notation, which allows
easy programical implementation. Using a coefficient notation for
these two equations of the form\begin{eqnarray}
F_{hh}(z_{i+1}) & = & a_{1}F_{hh}(z_{i-1})+a_{2}F_{lh}(z_{i-1})+a_{3}F_{lh}(z_{i})+a_{4}F_{hh}(z_{i}),\\
F_{lh}(z_{i+1}) & = & b_{1}F_{hh}(z_{i-1})+b_{2}F_{lh}(z_{i-1})+b_{3}F_{hh}(z_{i})+b_{4}F_{lh}(z_{i}),\end{eqnarray}
the effective mass equations can be written in a recursive transfer
matrix expression\begin{equation}
\left(\begin{array}{cccc}
0 & 0 & 1 & 0\\
0 & 0 & 0 & 1\\
a_{1} & a_{2} & a_{3} & a_{4}\\
b_{1} & b_{2} & b_{3} & b_{4}\end{array}\right)\left(\begin{array}{c}
F_{hh}(z_{i-1})\\
F_{lh}(z_{i-1})\\
F_{hh}(z_{i})\\
F_{lh}(z_{i})\end{array}\right)=\left(\begin{array}{c}
F_{hh}(z_{i})\\
F_{lh}(z_{i})\\
F_{hh}(z_{i+1})\\
F_{lh}(z_{i+1})\end{array}\right).\end{equation}


Provided that we have intial values for the wave functions at the
first and second nodes, we can determine the wavefunction values at
any node by an iterative procedure. By multiplying matrices, it is
possible to obtain an expression for the wavefunction values at any
node (as a function of the intial values)\begin{equation}
\left(\begin{array}{c}
F_{hh}(z_{n})\\
F_{lh}(z_{n})\\
F_{hh}(z_{n+1})\\
F_{lh}(z_{n+1})\end{array}\right)=\mathbf{M}_{n+1}\mathbf{M}_{n}\mathbf{M}_{n-1}\cdots\mathbf{M}_{3}\mathbf{M}_{2}\left(\begin{array}{c}
F_{hh}(z_{0})\\
F_{lh}(z_{0})\\
F_{hh}(z_{1})\\
F_{lh}(z_{1})\end{array}\right).\label{eq:Transfer_Matrix_Formalism}\end{equation}
The questions that remain are what is the suitable choise for these
initial values, and how to determine whether an energy is an eigenenergy
of the system.

Using four known values of the wavefunction components at $z$ and
$z+\delta z$, a fifth and sixth values can be predicted. Using the
new point togather with the known wavefunction components at $z$,
we can subsequently find the wavefunctions at $z+2\delta z$, and
so on. Hence the complete wave function solution can be found for
any particular energy. The solutions for steady states have wavefunctions
that satisfy the standard boundary conditions \begin{equation}
F\rightarrow0\,\,\,\textrm{and}\,\,\,\frac{\partial}{\partial z}F\rightarrow0,\,\,\,\textrm{as}\, z\rightarrow\pm\infty.\label{eq:Shooting_Method_Boundary_Conditions}\end{equation}
As argued in \citet{harrison_quantum_2000}, in the one-band case
of the conduction band only two initial values are required, and the
suitable choise is 0 for the first node, and 1 for the second. The
1 can be any arbitrary number, as changing it will only scale the
wavefunction (the finite difference equations are linear) and this
does not affect the eigenenergy. The valence band case is a bit more
complicated, as now there are two coupled wavefunction components,
and one cannot be scaled independently from the other. Therefore,
we choose the initial values to be 0 and 1 for one subband, and 0
and $c$ for the other. Here $c$ is a parameter, which is to be determined
when the equations are solved. 

The energy is varied systematically until both wavefunction components
switch from diverging to $\pm\infty$ to $\mp\infty$, satisfying
the boundary conditions. However, an additional problem araises from
the parameter $c$ defined above. On top of that, in many cases one
of the wavefunction components exhibit a very sharp sign switching,
often twice within a single energy search step. In order to work around
these problems, we minimize the amplitude of the wavefunction at the
end of the grid. The function to be minimized can be found by generating
the transfer matrix which propagates the wavefunction from the first
two nodes to the last two nodes\begin{equation}
\left(\begin{array}{c}
F_{hh}(z_{N-1})\\
F_{lh}(z_{N-1})\\
F_{hh}(z_{N})\\
F_{lh}(z_{N})\end{array}\right)=\left(\begin{array}{cccc}
m_{11} & m_{12} & m_{13} & m_{14}\\
m_{21} & m_{22} & m_{23} & m_{24}\\
m_{31} & m_{32} & m_{33} & m_{34}\\
m_{41} & m_{42} & m_{43} & m_{44}\end{array}\right)\left(\begin{array}{c}
0\\
0\\
1\\
c\end{array}\right).\end{equation}
Minimizing the wavefunction amplitude at the final node leads to\begin{equation}
\left(m_{33}+m_{34}c\right)^{2}+\left(m_{43}+m_{44}c\right)^{2}\rightarrow c_{min},\end{equation}
and then the minimum of $c_{min}(E)$ is searched to obtain the energy.
A solution of the Hamiltonian equations is found when this minimum
wavefunction amlitude is smaller than a certain threshold value. This
guarantees a converging wavefunction, which can be found by substituting
$c_{min}$ into \ref{eq:Transfer_Matrix_Formalism}. 

Note that the wavefunctions obtained in this procedure are not properly
normalized and should be transformed into \begin{equation}
\left(\begin{array}{c}
F_{hh}(z)\\
F_{lh}(z)\end{array}\right)\rightarrow\left(\begin{array}{c}
F_{hh}(z)\\
F_{lh}(z)\end{array}\right)\frac{1}{\sqrt{\int\left(F_{hh}^{2}(z)+F_{lh}^{2}(z)\right)dz}}\end{equation}

