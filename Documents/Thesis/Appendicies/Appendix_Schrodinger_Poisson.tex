
\chapter{Self-Consistent Solution of Schr\"{o}dinger -Poisson Model}

\label{cha:Appendix_Schrodinger_Poisson}For a quantitative discussion
of carriers that are strongly confined to a small area, it is necessary
to consider not only the band-edge and external potential, but also
the carrier-carrier electrostatic potential. We start by considering
the Schr\"{o}dinger  equation in the slowly varying envelope approximation
in one dimension\begin{equation}
\frac{\hbar^{2}}{2}\frac{d}{dz}\left(\frac{1}{m^{*}(z)}\frac{d}{dz}\right)\varphi(z)+V(z)\varphi(z)=E\varphi(z)\label{eq:Slowly_Varying_Envelope_Approximation_Sch_Eq}\end{equation}
where $V(z)$ is the overall potential and $\varphi(z)$ is the slowly
varying envelope. The Poisson equation \ref{eq:Poissons_Eq} can be
formulated as\begin{equation}
\frac{d}{dz}\left(\epsilon(z)\frac{d}{dz}\right)V_{\rho}(z)=-e[N_{D}^{+}(z)-n(z)-N_{A}^{-}(z)+p(z)],\label{eq:Poisson_Equation_CB}\end{equation}
where $\epsilon(z)$ is the position dependent dielectric constant,
$V_{\rho}(z)$ is the electrostatic potential and $e$ denotes the
elementary charge. N. $N_{D}^{+}(z)$ represents the ionized donor
distribution and $n(z)$ is the electron distribution. For now, we
only consider the conduction-band, so we set $N_{A}^{-}(z)=p(z)=0$.

The first coupling term between these two equations is the overall
potential\begin{equation}
V(z)=V_{CB}(z)-eV_{\rho}(z)\end{equation}
where $V_{CB}(z)$ represents the conduction-band profile given by
the material composition. The second coupling term is the electron
concentration $n(z)$, which is calculated from the envelope fucntion
$\varphi$and the Fermi level $E_{F}$\begin{equation}
n(z)=\frac{m^{*}(z)k_{B}T}{\pi\mathcal{L}\hbar^{2}}\sum_{\mathbf{k}}\left|\varphi_{\mathbf{k}}(z)\right|^{2}\ln\left(1+e^{\frac{E_{F}-E_{\mathbf{k}}}{k_{B}T}}\right),\label{eq:Fermi_Level_Eq}\end{equation}
where $k_{B}$ is the Boltzmann constant, $T$ is the temperature,
$\mathcal{L}$ is the length of the heterostructure and $E$ is the
eigenenergy. The summation over $\mathbf{k}$ represents the summation
over all eigenstates (including spin). As a consequence, we have to
solve for a given heterostructure with given donor concentration \ref{eq:Slowly_Varying_Envelope_Approximation_Sch_Eq}
and \ref{eq:Poisson_Equation_CB} self-consistently. We restrict ourselves
to model only the mesoscopic part and not the whole device, so we
have to define appropriate boundary conditions for the envelope function
and the electrostatic potential. The boundary conditions for the envelope
function are given by \ref{eq:Shooting_Method_Boundary_Conditions},
whereas we set the electrostatic potential $V_{\rho}$ to zero at
the boundaries \citet{Chuang1995}, which is equivalent to assume
the device in equilibrium.

To solve \ref{eq:Slowly_Varying_Envelope_Approximation_Sch_Eq} and
\ref{eq:Poisson_Equation_CB} numerically, we apply the shooting method
from Appendix \ref{cha:Appendix_Two_Band_Numerics} to the Schr\"{o}dinger 
equation and to the Poisson equation a finite-difference scheme, which
reads\begin{eqnarray}
0 & = & \frac{1}{2\left(\delta z\right)^{2}}\left((\epsilon_{i}+\epsilon_{i-1})V_{\rho,i-1}-(\epsilon_{i-1}+2\epsilon_{i}+\epsilon_{i+1})V_{\rho,i}+(\epsilon_{i}+\epsilon_{i+1})V_{\rho,i+1}\right)\nonumber \\
 &  & +e(N_{D,i}^{+}-n_{i}),\end{eqnarray}
where $\delta z$ is the spatial discretization. Equilibrium conditions
require us to choose the Fermi level appropriate to allow that\begin{equation}
\int_{-\infty}^{\infty}N_{D}^{+}(z)dz=\int_{-\infty}^{\infty}n(z)dz.\label{eq:Equalibrium_Condition}\end{equation}
To ensure this equilibrium condition, we solve \ref{eq:Fermi_Level_Eq}
with the constraint \ref{eq:Equalibrium_Condition} to obtain the
Fermi level $E_{F}$ .

%
\begin{figure}
\begin{centering}
\includegraphics[scale=0.6]{\string"C:/Users/Yossi Michaeli/Documents/Thesis/Documents/Thesis/Figures/Appendix_Schr_Poisson_Schematic\string".eps}
\par\end{centering}

\caption{\label{fig:Program_Flow_Schr_Poisson}Program flow for self-consistent
solution of Schr\"{o}dinger -Poisson under equilibrium condition
with given donator concentration.}



\end{figure}
Figure \ref{fig:Program_Flow_Schr_Poisson} shows the program flow
to obtain the self-consistent solution of the Schr\"{o}dinger -Poisson
system under equilibrium condition. First, Schr\"{o}dinger \textquoteright{}s
equation is solved assuming an initial potential $V(z)=V_{CB}(z),\, V_{\rho}(z)=0$.
Next, the Fermi level $E_{F}$ and the carrier distribution $n(z)$
are obtained by solving \ref{eq:Fermi_Level_Eq} with respect to \ref{eq:Equalibrium_Condition}.
Then, the Poisson equation is solved resulting in a new electrostatic
potential $V_{\rho,\textrm{new}}(z)$. If $|V_{\rho,\textrm{new}}(z)-V_{\rho}(z)|$
is smaller than a pre-defined value $\lyxmathsym{\textgreek{d}}$,
we obtain the converged solution. Otherwise, the electrostatic potential
is updated $V_{\rho}(z)(z)=V_{\rho}(z)+\lyxmathsym{\textgreek{G}}(V_{\rho,\textrm{new}}(z)-V_{\rho}(z))$,
where 0 < $\lyxmathsym{\textgreek{G}}$ < 1 is a damping parameter
used to improve convergence.
