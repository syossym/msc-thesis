\chapter{Uncertainty singularities in parallel mechanisms}
\label{chap4}

\textsf{Submitted to \textsl{IEEE Transactions on Robotics}, 2006.} \\
\textsf{Co-authors: David Blanc and Moshe Shoham.}\\

We study singularities for a parallel mechanism with spherical
joints and moving planar platform, giving a necessary condition
for an uncertainty singularity (defined topologically, in terms of
the \cspace) to occur, and describe the corresponding
instantaneous kinematic singularity.  An explicit example is
provided.



\sect{Introduction} \label{cint}
%
Kinematic singularities of parallel mechanisms have been studied
extensively in the literature (cf.\ \cite{Hu2}, and \cite{Hu1,WW}
for the example of open kinematic chains). On the other hand, many
authors have investigated \emph{topological} singularities of the
\cspace s of such mechanisms (see, e.g., \cite{KM,KT,MT1,SSB05}).

In this note, we study parallel mechanisms with rotational joints and
planar moving platform, and give a necessary
geometric condition  for uncertainty singularities in such mechanisms
(Theorem \ref{thmain}). We then explain how such topological
singularities give rise to instantaneous kinematic singularities
(Proposition \ref{pkinsing}).
Finally, the occurence of an uncertainty singularity is illustrated
visually in an explicit example (Section \ref{cve}).

\subsection{Topological singularities}
\label{sts}
%
By choosing appropriate local coordinates for a given mechanism $\Gamma$, we
can think of the set of all its configurations as a topological space
$\C$, called its \emph{\cspace}, and try to endow it with the
structure of a differentiable manifold. The points of $\C$  where this
cannot be done constitute the \emph{topological singularities} of
$\Gamma$. See Section \ref{cts} below for further details.

These are refered to in the robotics literature as \emph{uncertainty
singularities}: they occur when the \emph{instantaneous} mobility
is greater than the \emph{full cycle} mobility  (cf.\ \cite{Hu2}).
In such a position the mechanism as a whole gains an additional
degree of freedom. At times, such singularities are the only point
where two disjoint regions of $\C$ meet, allowing transition between
different operating modes. See \cite{ZBC} for an examination of such
transitions in the case of a constraint singularity for a $3$-URU
$3$-DOF parallel mechanism.

\subsection{Kinematic singularities}
\label{sks}
%
In general, a configuration $\Vc$ is naturally described by the vector \
$\bx=(x_{1},\dotsc,x_{N})$ \ whose coordinates \ $x_{i}$ \ correspond
to the positions of the various joints and links of the mechanism.
In practice, we choose a subset \ $\bx_{\iin}$ \
of \emph{input} coordinates (corresponding to the actuated joints),
which can serve as local coordinates for $\C$ around $\Vc$. In
addition, we often focus on a subset \ $\bx_{\out}$ \ of \emph{output}
coordinates of interest (which may describe the position of the end
effector). The structure of the mechanism imposes relations which must
hold among the coordinate \ $x_{i}$; \ in particular, we may assume that:
%
\begin{equation}\label{eq:constr}
%
F(\bx_{\iin},\bx_{\out})=0
%
\end{equation}
%
identically in a neighborhood of $\Vc$ in $\C$. The Jacobian \
$J:=\frac{\partial F}{\partial\bx}$ \ consists of two blocs \
$\frac{\partial F}{\partial\bx_{\iin}}$ \ and \
$J_{\out}:=\frac{\partial F}{\partial\bx_{\out}}$, \ where \
$J_{\out}$ \ must be nonsingular if the \ $\bx_{\iin}$ \ are to serve
as local coordinates near $\Vc$.

The kinematics of the mechanism are described by a path \ $\bx(t)$
\ ($t\in\RRR$) \ in the \cspace. Differentiating \
\eqref{eq:constr} \ with respect to $t$ we get \ $\frac{\partial
F}{\partial\bx}\,\dot{\bx}=0$, \ which can be written in the form:
%
\begin{equation} \label{eq:gosselin_sing}
%
J_{\iin}\ \dot{\bx}_{\iin}~=~J_{\out}\ \dot{\bx}_{\out}
%
\end{equation}
%
where \ $J_{\iin}:=-\frac{\partial F}{\partial\bx_{\iin}}$. \ If \
$J_{\out}$ \ is of maximal rank and \ $J_{\iin}$ \ is not, \
$\Vc$ is called a kinematic singularity \emph{of type I}: this
means that we cannot obtain every (infinitesimal) change in output we
may want by an appropriate change in the actuated joints.
On the other hand, if \ $J_{\iin}$ \ is of maximal rank and  \
$J_{\out}$ \ is not, $\Vc$ is called singular of \emph{of type II}: in
this case the actuated joints do not determine uniquely the behavior
of the outputs.
Finally, if neither \ $J_{\out}$ \ nor \ $J_{\iin}$ \ is of maximal
rank, $\Vc$ is called singular \emph{of type III} (cf.\ \cite{GA}).
Gosselin and Angeles give examples of all three types of
singularities for a parallel \ $3$-RRR \ planar mechanism.

However, as Zlatanov, Fenton and Benhabib observe, the choice of \
$\bx_{\iin}$ \ and \ $\bx_{\out}$ \ is arbitrary; moreover, in
general it need not fully determine the mechanism, which may have
additional ''passive coordinates'' of interest. They therefore provide
a different approach to the singularity analysis of an arbitrary
kinematic chain (see \cite{ZFB1}).

%
%s2       Topological singularities
%
\sect{Topological singularities} \label{cts}

We consider \emph{polygonal} mechanism $\Gamma$ in \ $\RRR^{d}$ \
($d=2,3$), \ consisting of a moving platform $\Pc$ (planar
$k$-polygonal), with $k$ branches attached to its vertices. The
$i$-th branch is a sequence of \ $\nj{i}$ \ concatenated links of
lengths \ $\lj{i}_{j}$ \ ($j=1,\dotsc\nj{i}$), \ connected by
spherical joints. One end of the branch is attached to a vertex of
the moving platform, and the other is fixed  at \
$\xj{i}\in\RRR^{d}$.

\begin{defn}
%
A \emph{configuration} for an $n$-link branch consists of $n$
vectors \ $(\bv_{1},\dotsc,\bv_{n})$ \ in \ $\RRR^{d}$ \ of
specified lengths \ $\|\bv_{j}\|=\ell_{j}$ \ ($j=1,\dotsc n$). \

A branch configuration is said to be \emph{aligned} if all the
vectors \ $\bv_{i}$ \ are scalar multiples of $\bv$, which is
called the \emph{direction vector} of $V$. The \emph{direction
line} for $V$ is \ $\Line:=\{\bx+t\bv~|\ t\in\RRR^{}\}$.

A configuration for $\Gamma$ consists of a set \
$\Vc=(\Vj{1},\dotsc,\Vj{k})$ \ of configurations for each of the $k$
branches, such that the $k$ endpoints
%
$$
\ppj{i}:=\xj{i}+\sum_{i=1}^{n}\,\bv_{i} \ \ \ (i=1,\dotsc,k)
$$
of the corresponding branch configurations form a polygon congruent to the
given moving platform $\Pc$. The set \ $\C=\C_{\Gamma}$ \ of all such
configurations, topologized in the obvious way, is the \emph{\cspace}
of $\Gamma$.
%
\end{defn}

\textit{Convention:} \
%
Parenthesized superscripts indicate the branch
number, and subscripts indicate the link number. For example, \
$\ell_{j}^{(i)}$ \ denotes the length of the $j$-th link of the $i$-th
branch.

\begin{defn}
%
A configuration $\Vc$ for $\Gamma$ is called \emph{singular of type
  (a)} if two of its branch configurations \
$\Vj{i_{1}}$ \ and \ $\Vj{i_{2}}$ \ are aligned, with coinciding
direction lines: \ $\Line\up{i_{1}}=\Line\up{i_{2}}$. \ See Fig.
\ref{fig:sing1}.

\begin{figure}[ht]
\centering \epsfysize=5.5cm  \epsffile{fig_4/singular_point1.eps}
\caption{A singular configuration of type (a)} \label{fig:sing1}
\end{figure}

It is \emph{singular of type (b)} if three of its branch
configurations are aligned, with direction lines in the same
plane meeting in a single point (this is referred to in literature as
a \emph{flat pencil}). See Fig. \ref{fig:sing2}.

\begin{figure}[ht]
\centering \epsfysize=6cm  \epsffile{fig_4/singular_point2.eps}
\caption{A singular configuration of type (b)} \label{fig:sing2}
\end{figure}

It is \emph{singular of type (c)} if four of its branch configurations
are aligned, with direction lines in the same plane.
%
\end{defn}

We can now formulate our main result:

\begin{thm}\label{thmain}
%
A necessary condition for a configuration \ $\Vc=(\Vj{1},\dotsc,\Vj{k})$ \
of a polygonal mechanism $\Gamma$ to be an uncertainty singularity is that it
be singular of type (a), (b) or (c).
%
\end{thm}

\begin{examples}\label{eks}
%
Before sketching the proof, consider the following two examples\vsm:

\noindent (1) \ If $\Vc$ is singular of type (a), as in Fig.
\ref{fig:sing1}, we can construct an open neighborhood \ $U\subset
\C$ \ of $\Vc$ as a product \ $U_{\alignd} \times U_{\rest}$. \
Here \ $U_{\alignd}$ \ may be identified with an neighborhood of
$\Vc$ in the \cspace\ of a closed chain mechanism \ $\Gamma'$, \
consisting of the aligned branches of $\Gamma$. In the case where
the aligned branches have one and two links respectively,  \
$U_{\alignd}$ \ is a one-point union of two $2$-discs (see
\cite[Proposition 4.1]{SSB2}), so it is singular.

On the other hand, \ $U_{\rest}$ \ is a neighborhood of $\Vc$ in the
\cspace\ of the mechanism obtained from $\Gamma$ by omitting the
aligned branches, which is non-singular (generically)\vsm.

\noindent (2) \ Consider a mechanism with a triangular platform, and
two branches with one link each. In this case, the workspace
for the third vertex of the platform is the coupler curve $\gamma$ for the
corresponding $4$-chain mechanism (see \cite[Ch.\ 4]{Ha}), while the
workspace for the third branch is an annulus $A$.

For suitable parameters, the boundary \ $\partial A$ \ (where the
third branch is aligned) will be tangent to $\gamma$. The
corresponding configuration $\Vc$ for $\Gamma$ will be singular of
type (b), as in Fig. \ref{fig:second case intuitive}.

\begin{figure}[htb]
\centering \epsfysize=6cm \leavevmode
\epsffile{fig_4/intantanious_rotation.eps} \caption{Coupler curve
tangent to branch workspace boundary} \label{fig:second case
intuitive}
\end{figure}

Note that near $\Vc$ each point in $\gamma$ has two
corresponding configurations, associated to ``elbow up/down''
positions of the third branch, which coalesce at $\Vc$ itself; thus
$\Vc$ has a singular neighborhood consisting of two transverse
intervals\vsm.
%
\end{examples}


\textit{Sketch of proof of Theorem \ref{thmain}}\vsm:

A configuration $\Vc$ for the polygonal mechanism $\Gamma$ in \
$\RRR^{d}$ \ ($d=2,3$) \ is determined by the \ $Nd$ \ Cartesian
coordinates of the endpoints of the \ $N=\sum_{i=1}^{k}\,\nj{i}$ \
links of $\Gamma$. However, there are \ $M:=N+dk-3(d-1)$ \
relations among these imposed by the lengths of the links and a
certain minimal collection $\Ic$ of edges and diagonals of the
polygonal platform $\Pc$. Thus the \cspace\ $\C$ is the preimage
of a certain vector \ $Z\in\RRR^{M}$ \ under the \emph{constraint
map}  \ $F:\RRR^{dN}\to\RRR^{M}$, \ defined:
%
\begin{equation}\label{econstr}
\begin{split}
%
F(\Vc)~:=&(f_{\nj{1}}(\Vj{1}),\dotsc,f_{\nj{k}}(\Vj{k}),\\
&\|\aj{1,2}\|^{2},\dotsc,\|\aj{k-1,k}\|^{2})~,
%
\end{split}
\end{equation}
%
where \
$f_{n}(\bv_{1},\dotsc,\bv_{n}):=(|\bv_{1}|^{2},\dotsc,|\bv_{n}|^{2})$ \
and \ $\aj{i,j}$ \ is the \ $(i,j)$-diagonal (or edge) of $\Pc$
spanned by the endpoints of branches $i$ and $j$\vsm.

\noindent\textbf{Step I.}~~~By the Regular Value Theorem, $\C$ will be
a smooth manifold if \ $\dF_{\Vc}$ \ is of rank $M$, where \
$\dF_{\Vc}/2$ \ is:
%
%\setlength{\arraycolsep}{0pt}
\begin{equation}\label{edf}
%
\begin{pmatrix}
%
A\up{1}    & 0         &  0           & \dotsc  & 0        \\
 0         & A\up{2}   &  0           & \dotsc  & 0        \\
 \vdots    & \vdots    &  \vdots      & \ddots  & \vdots   \\
 0         & 0         &  0           & \dotsc  & A\up{k}  \\
\bj{1}{2}  & \bj{2}{1} & 0            & \ldots  & 0        \\
\bj{1}{3}  & 0         & \bj{3}{1}    & \ldots  & 0        \\
 0         & \bj{2}{3} & \bj{3}{2}    & \ldots  & 0        \\
\vdots     & \vdots    &  \vdots      & \ddots  & \vdots \\
 0         & \dotsc    & 0            & \ldots  & \bj{k}{k-1}
\end{pmatrix}\vspace{4mm}\quad
%
\end{equation}
%
and \ $A\up{i}$ \ is the \ $\nj{i}\times d\nj{i}$ \ matrix:
%
\begin{equation}\label{eai}
\begin{pmatrix}
%
\vj{i}_{1} & \dotsc    & 0              \\
\vdots     & \ddots    & \vdots         \\
0          & \dotsc    & \vj{i}_{\nj{i}}
\end{pmatrix}~.
%
\end{equation}

Each edge \ $\aj{i,j}\in\RRR^{d}$ \ appears \ $\nj{i}$ \ times in
the vector
%
\begin{equation}\label{ebj}
%
\bj{i}{j}~:=~\underbrace{(\aj{i,j},\aj{i,j},\dotsc,\aj{i,j})}_{\nj{i}}\vsm.
%
\end{equation}

\noindent\textbf{Step II.}~~~
%
Let \ $\Vc\in\C=F^{-1}(Z)$, \ and consider a vanishing linear
combination of the rows of \ $\dF_{\Vc}$. \ If some row of \ $A\up{i}$ \
has a non-zero coefficient, so does at least one of the \ $M-N$ \
bottom rows; but since these are all repeated, as in \ \eqref{ebj}, \
we see that all the direction vectors \ $\vj{i}_{j}$ \
for the $i$-th branch must be equal \ -- \ that is, this
branch is aligned, with the direction vector \ $\vj{i}$ \ a linear
combination of appropriate diagonals \ $\aj{k,\ell}$ \ of $\Pc$, and
thus lying in the plane $\Ec$ determined by $\Pc$.

Moreover, from \ \eqref{edf} \ we see that \ $\sum_{i=1}^{k}\,\vj{i}=\vz$. \
Therefore, if \ $\vj{i}=\vz$ \ for \ $i\neq i_{0},i_{1}$, \ then \
$\vj{i_{0}}+\vj{i_{1}}=\vz$, \ and thus branches \ $i_{0}$ \ and \
$i_{1}$ \ are co-aligned with direction vector \
$\aj{i_{0},i_{1}}$, \ yielding a singularity of type (a).

On the other hand, if at least four branches are aligned (with
direction vectors in $E$), we obtain a singularity of type (c)\vsm.

\noindent\textbf{Step III.}~~~
%
Now assume we have exactly three aligned branches, say \ $1,2$ \
and $k$. Let $\hC$ denote the \cspace\ of the mechanism obtained
by omitting the last ($k$-th) branch \ -- \ again given as the
fiber of an appropriate constraint map \
$\hF:\RRR^{(N-\nj{k})d}\times\Sc\to\RRR^{K}$ \ (where \
$\Sc=S^{d-2}$ \ if \ $k=3$, \ and a point otherwise), with \
$\hi:\hC\to\RRR^{(N-\nj{k})d}$ \ the inclusion. Similarly, let
$\Ck$ be the \cspace\ for the $k$-th branch (an \ $\nj{k}$-torus,
with the obvious embedding \ $i:\Ck\to\RRR^{\nj{k}d}$).

The \emph{workspace} of both mechanisms \ -- \ that is, the
possible locations for the $k$-th vertex of the moving platform \
-- \ are in  \ $\RRR^{d}$, \ and we denote the \emph{work maps} by
\ $\psi:\hC\to\RRR^{d}$ \ and \ $\phi:\Ck\to\RRR^{d}$.

Let \ $X:=\Ck\times\RRR^{(N-\nj{k})d}\times\Sc$, \
$Y:=\RRR^{d}\times\RRR^{(N-\nj{k})d}\times\Sc$, \ and define \
$h:X\to Y$ \ to be the product map \ $\phi\times\Id$, \ and \
$g:\hC\to Y$ \ to be \ $(\psi,\hi)$. \ Thus $\C$ (the \cspace\ for
$\Gamma$) is the preimage of the submanifold \ $\hC\subseteq Y$ \
under $h$.

Let \ $\hV\in\hC$ \ be a configuration where branches $1$ and $2$
are aligned, with distinct direction vectors \ $\oj{1}$, $\oj{2}$ \ in $\Ec$
(the plane of $\Pc$), and let \ $\Vj{k}\in\Ck$ \ be an aligned
configuration with direction vector \ $\oj{k}\in\Ec$. \ Assume that \
$\psi(\hV)=\phi(\Vj{k})$, \ and let \ $\bx\in X$ \ be the
configuration \ $(\Vj{k},\hi(\hV))$.

To prove Theorem \ref{thmain}, we must show that if the point \
$\Vc\in\C$ \ defined by \ $(\hV,\Vj{k})$ \ is not singular of type
(b), then it is smooth. This would follow if $h$ is
locally transverse to $\hC$ at the points \ $\bx\in X$ \ and \
$\hV\in\hC$, \ in other words:
%
\begin{equation}\label{etransv}
%
\Image\dhh_{\bx}~+~T_{\hV}(\hC)~=~T_{\hV}(Y)~.
%
\end{equation}

But \ $T_{\Vj{k}}(\Ck)$ \ is just the null space of \
$A\up{k}$ \ of \eqref{eai}, \ namely: \ $(\oj{k}^{\perp})^{n}$, \
so it suffices to prove that \ $\oj{k}\in T_{\hV}(\hC)$. \
Choose unit vectors \ $\cj{i}$ \ spanning \ $\oj{i}^{\perp}$ \
($i=1,2$) \ in $\Ec$; then \ $\oj{k}\not\in T_{\hV}(\hC)$ \
only if \ $\oj{k}$ \ is proportional to: \
$(\aj{1,2}\cdot\cj{2})\,\oj{1}~-~(\oj{1}\cdot\cj{2})\,\aj{1,3}$, \
which is the vector connecting the meeting point of \
$\Line\up{1}$ \ and \ $\Line\up{2}$ \ with the end point
of \ $\Vj{k}$. \ See \cite{SSB2} for the details.\hfill$\Box$

%
%s3      Kinematic singularities
%
\sect{Kinematic singularities} \label{crks}

Theorem \ref{thmain} gives a \emph{necessary} condition for a
configuration to be singular (topologically): namely, that some
subset $\{i_{1},\dotsc,i_{m}\}$ \ of its branches be aligned, with
direction lines \ $(\Line\up{i_{j}})_{j=1}^{m}$. \ Moreover, the
Pl\"{u}cker line coordinates of these lines must span certain
types of \emph{varieties}, of positive codimension in \ $\RRR^{6}$
\ (see \cite[\S 5]{M}).

We now show how uncertainty singularities give rise to kinematic
singularities, for the polygonal mechanism discussed in Section \ref{cts}.

\subsection{Mechanism architectures}
\label{sarch}

Recall from \S \ref{sks} that the latter require an
\emph{actuation input choice} \ -- \ that is, a choice of input
coordinates \ $\bx_{\iin}$ \ (corresponding to the actuated
joints), and output coordinates  \ $\bx_{\out}$ \ (corresponding
to the end effector of the mechanism: in our case, the position
and orientation of the moving platform).

For the spatial case (i.e., $\Gamma$ embedded in \ $\RRR^{3}$, and
having spherical joints), \ there are three architectures to
consider as a result of platform mobility considerations:
%
\begin{enumerate}
\renewcommand{\labelenumi}{(\alph{enumi})}
%
\item Six branches \
  $\{i_{1},\dotsc,i_{6}\}\subseteq\{1,\dotsc,k\}$, \ where the $j$-th
  branch has \ $\nj{i_{j}}-1$ \ spherical actuators (say, at all
  joints but the last two).
%
\item One branch \ $i_{1}$ \ having a total of \ $\nj{i_{1}}$ \
  spherical actuators (at each joint but the last); and three more
  branches \ $i_{2}$, \ $i_{3}$, \ and \ $i_{4}$, \ with \
  $\nj{i_{j}}-1$ \ spherical actuators in the $j$-th branch \
  ($j=2,3,4$).
%
\item Two branches \ $i_{1}$, \ $i_{2}$ \ with actuators at each joint
    but the last; and one more branch \ $i_{3}$ \ with \
  $\nj{i_{3}}-1$ \ spherical actuators.
%
\end{enumerate}

\subsection{Kinematic analysis through screw theory}
\label{ssth}

We use screw theory (see \cite{DH}) to describe the forces operating
at each joint of our mechanism:

A \emph{screw} $\$ $ is a Pl\"{u}cker vector in \ $\RRR P^{5}$, \
describing a line \ -- \ or equivalently, the position and
direction of vector \ -- \ in \  $\RRR^{3}$ \ (cf.\ \cite[\S
5]{M}). Thus  Fig. \ref{fig:screws} depicts the equivalent
kinematic chain of an arbitrary branch $i$ of a mechanism, where
the movement of each spherical joint $j$\ is described by three
unit screws \ $\sj{i}_{j,1}$, \ $\sj{i}_{j,2}$, \ and \
$\sj{i}_{j,3}$ \ attached to its center. Note that just before a
passive joint, we can make do with two screws.

Considering the $i$-th branch as an open chain, we can express the
instantaneous twist of the end-effector as:
%
\begin{equation}
\label{eq:vector_loop_equation}
%
 \$_{p}~=~\sum_{j=0}^{\nj{i}}\,\sum_{k=1}^{3}\,
\sj{i}_{j,k} \cdot \dot{\theta}\up{i}_{j,k}
%
\end{equation}
%
where \ $\theta\up{i}_{j,k}$ \ is the input coordinate for the \
$\sj{i}_{j,k}$ \ screw (cf.\ \cite[\S 5.6]{T}). \newnot{4-1}

\begin{figure}[htb]
\centering \epsfysize=7cm \leavevmode \epsffile{fig_4/screws.eps}
\caption{Equivalent kinematic structure of a branch}
\label{fig:screws}
\end{figure}

In order to rid \ \eqref{eq:vector_loop_equation} \ of the passive
joints, for the $i$-th branch, we must multiply both sides by
appropriate reciprocal screws. More precisely, choose a basis \
$\{\sjp{i}_{t}\}_{t=1}^{\laj{i}}$ \ for the space \ $\Vj{i}$ \ of common
reciprocals of the screws of all passive joints for this branch:

\begin{enumerate}
\renewcommand{\labelenumi}{(\alph{enumi})}
%
\item If the branch has \ $\nj{i}-1$ \ spherical actuators, at all
  joints but the last two, the reciprocal screw for these two joints
  corresponds to the line passing through them (i.e., \ $\Vj{i}$ \ is
  one-dimensional).
%
\item If the branch has \ $\nj{i}$ \ spherical actuators, at all but
  the last joint, \ $\Vj{i}$ \ is   $3$-dimensional, with the
  corresponding lines all passing through the last joint (see \cite[Ch. 5]{T}).
%
\end{enumerate}

For the $i$-th branch we obtain a system of \ $\laj{i}$ \ linear
equations for \ $\$_{p}$:
%
\begin{equation}
\label{eq:one_branch_screw}
%
\sjp{i}_{t} \cdot \$_p~=~\sum_{j=0}^{\nj{i}}\,\sum_{k=1}^{3}\,
\sjp{i}_{t}\cdot \sj{i}_{j,k}\cdot \dot{\theta}\up{i}_{j,k}
%
\end{equation}
%
($t=1,\dotsc,\lj{i}$), \ in which of course the unactuated inputs \
$\dot{\theta}\up{i}_{j,k}$ \ have zero coefficient.

Combining the \ $\lambda=\sum_{i=1}^{k}\,\laj{i}$ \ equations \
\eqref{eq:one_branch_screw} \ for all $k$ branches, we
obtain equation \ \eqref{eq:gosselin_sing} \ for the chosen
architecture \ ($\lambda=6$ \ for the first two, and \ $\lambda=7$ \
for the third).

The \ $\lambda\times 6$ \ matrix \ $J_{\out}$ \ will take the form:
%
\begin{equation}
\label{eq:six_branches_Jout}
%
J_{\out}=
\begin{pmatrix}
\sjp{1}_{1}\\
\vdots\\
\sjp{1}_{\laj{1}}\\
\vdots\\
\sjp{k}_{\laj{k}}
\end{pmatrix}
%
\end{equation}
%
whose rows are the reciprocal screws of all actuated branches.

The matrix \ $J_{\iin}$ \ is bloc-diagonal:
$$
J_{\iin}~=~
\begin{pmatrix}
A\up{i_{1}} & \dotsc & 0     \\
\vdots      & \ddots & \vdots \\
0           & \dotsc & A\up{i_{k}}
\end{pmatrix}~,
$$
%
with the $i$-th bloc (for the $i$-th actuated branch, with $d$ passive
joints) a \ $\laj{i}\times(\nj{i}-d)$ \ matrix of the form:
$$
A\up{i}~:=~\begin{pmatrix}
\sjp{i}_{1}\cdot\sj{i}_{0,1} & \dotsc & \sjp{i}_{1}\cdot\sj{i}_{\nj{i}-d,3}\\
\vdots & & \vdots \\
\sjp{i}_{\laj{i}}\cdot\sj{i}_{0,1} & \dotsc &
\sjp{i}_{\laj{i}}\cdot\sj{i}_{\nj{i}-d,3}
\end{pmatrix}~.
$$

\begin{prop}\label{pkinsing}
%
For a polygonal mechanism $\Gamma$ (with no unactuated branches), there
is an instantaneous kinematic singularity of type I or III at any
uncertainty singularity.
%
\end{prop}

\begin{proof}
%
At an uncertainty singularity at least two branches are aligned,
so the reciprocals to the passive joint(s) are reciprocal to all
screws of these branches, and thus \ $A\up{i}=0$ \ for these branches. Since \
$\lambda\leq 7$, \ $J_{\iin}$ \ is singular.

Now by Theorem \ref{thmain}, an uncertainty singularity can have:
%
\begin{enumerate}
%
\item Two coaligned branches, each with a pair of unactuated joints:
  they have a common reciprocal (and \ $\lambda=6$), \ so \
  $J_{\out}$ \ has rank \ $\leq 5$. \ The same holds in the second
  architecture whenever two branches are coaligned.
%
\item Three aligned branches whose lines lie in a planar pencil, each with
  a pair of unactuated joints: in this case the corresponding screws
  are linearly dependent, so again \ $J_{\out}$ \ has rank \ $\leq 5$.
%
\item Four aligned branches whose lines are in one plane: the lines of
  those with a pair of unactuated joints each lie in a
  planar pencil (rank $3$). In the first architecture, the last two
  lines each add at most $1$ to the rank; in the second, adding the last
  line form a degenerate congruence (total rank $4$). Thus in any case \
  $J_{\out}$ \ has rank \ $\leq 5$.
%
\end{enumerate}
%
\end{proof}

%
%s4      An illustrative example
%
\sect{An illustrative example} \label{cve}

We now give an visual presentation of an uncertainty singularity in an
explicit $3$-URU $3$-DOF mechanism $\Gamma$, introduced
in \cite{ZFB}.

\begin{figure}[htb]
\centering \epsfysize=5.6cm \leavevmode \epsffile{fig_4/robot.eps}
\caption{Singular config. for a 3-DOF 3-URU mechanism}
\label{fig:triangle2} \end{figure}

The centers of the three U-joints at the base (platform) form an
equilateral triangle. The three R-joint axes fixed in the base
(platform) meet (not at the center of the base(platform) triangle) in
a "Y" shape. The three intermediate R joints in each branch are all
parallel.

\begin{remark}\label{rem:planarity}
%
To enable visualization near the singular configuration \ $\Vc\in\C$, \
we need a mechanism with instantaneous mobility near $\Vc$ equal to
$3$. We therefore make sure that the intersection point of the three
R-joint axes avoids the center of the base platform (otherwise
mobility would be increased by the additional spin dexterity of the
extended branch).
%
\end{remark}

In the region in $\C$ where $\Gamma$ acts as a planar mechanism
(see \cite{ZFB2}), the pose of the moving platform $\Pc$ (an equilateral
triangle) is determined by two coordinates (say, $x$ and $y$) of its
barycenter $\bp$, and its planar rotation $\phi$. Denote by $r$
the common distance \ $d(\bp,\ppj{i})$ \ ($i=1,2,3$).

The \wspace\ for each vertex \ $\ppj{i}$ \ of $\Pc$ is an annulus \
$\Aj{i}$ \ centered at the fixed endpoint \ $\xj{i}$ \ of the $i$-th
branch, with radii \ $\lj{i}_{1}\pm\lj{i}_{2}$ \ respectively.
Thus if we fix the orientation $\phi$ of $\Pc$, the resulting
constrained \wspace\ \ $\Wc_{\phi}$ \ for $\bp$ (the shaded area in
Fig. \ref{annuli}) is the intersection of three annuli (with centers
at \ $\Tj{i}$): \ namely, the displacements \ $\hAj{i}$ \ ($i=1,2,3$) \
of \ $\Aj{i}$ \ by a vector \ $\ppj{i}\bp=\xj{i}\Tj{i}$ \ of length $r$.

\begin{figure}[ht]
\centering
\epsfysize=7cm %\epsfxsize=6.4cm\leavevmode
\epsffile{fig_4/annuli.eps} \caption{constrained \wspace\ \
$\Wc_{\phi}$\ for $\bp$ with fixed $\phi$} \label{annuli}
\end{figure}

Thus $\C$ is described in the vicinity of $\Vc$ by the location of
$\bp$ in \ $\Wc_{\phi}$ and the orientation \ $\phi\in I$. \
Generically, the \wspace s $\Wc_{\phi}$ \ will all be homeomorphic to
a fixed space \ $\Wc'$ \ for nearby values of $\phi$, so that a
neighborhood of $\Vc$ will be a topological product \ $I\times\Wc'$. \
However, for some values of $\phi$, \ $\Wc_{\phi}$ \ may vanish, so
a larger neighborhood $U$ of $\Vc$ need not be connected (see Fig.
\ref{fig:2 boxes}).

\begin{figure}[ht]
\centering
\epsfysize=5.8cm %\epsfxsize=6.4cm
\leavevmode \epsffile{fig_4/before_boxes_kiss.eps}
\caption{\cspace\ for $3$-RRR mechanism} \label{fig:2 boxes}
\end{figure}

In addition, we need discrete data on the elbow up/down position of
each branch at $\Vc$, so the relevant region of $\C$ will actually
contain eight identical copies of $U$, each consisting of one or more
product ``boxes'' as above. Since the boundaries of \ $\Wc_{\phi}$ \
are at the boundaries of the annuli \ $\hAj{i}$, \ where the links of
the $i$-th branch are aligned, the faces of each box are identified
with a corresponding face in another box.
The aligned poses can be calculated analytically using the algorithm
in Gosselin and Merlet (cf.\ \cite{GM}), since each of the extreme
situations can be treated as an equivalent $3$-RPR robot, whose
link lengths are fixed and known.

For example, gluing faces for the situation depicted in Fig.
\ref{fig:2 boxes} yields two $3$-tori. (Of course, this is only
true in the region where the mechanism platform motion is planar \
-- \ see Remark \ref{rem:planarity}; so all we can conclude about
$\C$ near $\Vc$ is that it has two connected components, locally
isomorphic to \ $\RRR^{3}$.)

As the parameters vary, we find that the two connected components
of $U$ approach each other, and for an appropriate value they
touch at one point (see Fig. \ref{fig:box_kiss}).  As a result,
$\C$ is now locally homeomorphic to \ $\RRR^{3}
\vee_{\Vc}\RRR^{3}$ that is a singular point where two regions
meet.

\begin{figure}[ht]
\centering
\epsfysize=5.8cm %\epsfxsize=6.4cm
\leavevmode \epsffile{fig_4/box_kiss.eps} \caption{\cspace\ is
locally \ $\textbf{T}^3\vee_c \textbf{T}^3$} \label{fig:box_kiss}
\end{figure}


% conclusions

\section{Conclusions}

In this paper we derive necessary conditions for the existence of
topological (uncertainty) singularities in a class of spatial and
planar robots having arbitrary number of legs and arbitrary number
of spherical joints in each leg, all attached to a moving planar
platform. We conclude that a topological singularity emerges when
one of the following holds: two legs are aligned and have common
direction, three legs are aligned  and form a flat pencil or when
four legs are aligned and lie in the same plane.

We then relate these topological singular points with the well
known kinematics ones.

Lastly, these results are used in an illustrative example for a
topological singular point, where intrinsically the robot
actuation is not sufficient \--\ regardless of the actuators
placements.
