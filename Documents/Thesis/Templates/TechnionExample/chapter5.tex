
%% bare_conf.tex
%% V1.2
%% 2002/11/18
%% by Michael Shell
%% mshell@ece.gatech.edu
%%
%% NOTE: This text file uses UNIX line feed conventions. When (human)
%% reading this file on other platforms, you may have to use a text
%% editor that can handle lines terminated by the UNIX line feed
%% character (0x0A).
%%
%% This is a skeleton file demonstrating the use of IEEEtran.cls
%% (requires IEEEtran.cls version 1.6b or later) with an IEEE conference paper.
%%
%% Support sites:
%% http://www.ieee.org
%% and/or
%% http://www.ctan.org/tex-archive/macros/latex/contrib/supported/IEEEtran/
%%
%% This code is offered as-is - no warranty - user assumes all risk.
%% Free to use, distribute and modify.

% *** Authors should verify (and, if needed, correct) their LaTeX system  ***
% *** with the testflow diagnostic prior to trusting their LaTeX platform ***
% *** with production work. IEEE's font choices can trigger bugs that do  ***
% *** not appear when using other class files.                            ***
% Testflow can be obtained at:
% http://www.ctan.org/tex-archive/macros/latex/contrib/supported/IEEEtran/testflow


% Note that the a4paper option is mainly intended so that authors in
% countries using A4 can easily print to A4 and see how their papers will
% look in print. Authors are encouraged to use U.S. letter paper when
% submitting to IEEE. Use the testflow package mentioned above to verify
% correct handling of both paper sizes by the author's LaTeX system.
%
% Also note that the "draftcls" or "draftclsnofoot", not "draft", option
% should be used if it is desired that the figures are to be displayed in
% draft mode.
%
% This paper can be formatted using the peerreviewca
% (instead of conference) mode.


\documentclass[twocolumn]{IEEEtran}

\usepackage{amsmath,amstext,amssymb,epsf,verbatim}%subfigure}
%\usepackage{algorithm,algorithmic}
\newtheorem{assumption}{{\bf Assumption.}}
\newtheorem{Proposition}{{\bf Proposition.}}
\newtheorem{Lemma}{{\bf Lemma}}
\newtheorem{Theorem}{{\bf Theorem}}
\newtheorem{Corollary}{{\bf Corollary}}
\newtheorem{Remark}{{\bf Remark}}
\newtheorem{Algorithm}{{\bf Algorithm}}
\usepackage{graphicx}        % standard LaTeX graphics tool
                             % when including figure files
\usepackage{multicol}        % used for the two-column index
\usepackage[bottom]{footmisc}% places footnotes at page bottom
\newcommand{\reals}{{\Bbb R}}
\newcommand{\bv}{\mathbf{v}}
\def\V{\mathcal{V}}
\newcommand{\bw}{\mathbf{w}}
\newcommand{\Vj}[1]{V\up{#1}}
\newcommand{\vj}[1]{\bv\up{#1}}
\def\Vc{\mathcal{V}}
\def\C{\mathcal{C}}


% If the IEEEtran.cls has not been installed into the LaTeX system files,
% manually specify the path to it:
% \documentclass[conference]{../sty/IEEEtran}


% some very useful LaTeX packages include:

%\usepackage{cite}      % Written by Donald Arseneau
                        % V1.6 and later of IEEEtran pre-defines the format
                        % of the cite.sty package \cite{} output to follow
                        % that of IEEE. Loading the cite package will
                        % result in citation numbers being automatically
                        % sorted and properly "ranged". i.e.,
                        % [1], [9], [2], [7], [5], [6]
                        % (without using cite.sty)
                        % will become:
                        % [1], [2], [5]--[7], [9] (using cite.sty)
                        % cite.sty's \cite will automatically add leading
                        % space, if needed. Use cite.sty's noadjust option
                        % (cite.sty V3.8 and later) if you want to turn this
                        % off. cite.sty is already installed on most LaTeX
                        % systems. The latest version can be obtained at:
                        % http://www.ctan.org/tex-archive/macros/latex/contrib/supported/cite/

%\usepackage{graphicx}  % Written by David Carlisle and Sebastian Rahtz
                        % Required if you want graphics, photos, etc.
                        % graphicx.sty is already installed on most LaTeX
                        % systems. The latest version and documentation can
                        % be obtained at:
                        % http://www.ctan.org/tex-archive/macros/latex/required/graphics/
                        % Another good source of documentation is "Using
                        % Imported Graphics in LaTeX2e" by Keith Reckdahl
                        % which can be found as esplatex.ps and epslatex.pdf
                        % at: http://www.ctan.org/tex-archive/info/
% NOTE: for dual use with latex and pdflatex, instead load graphicx like:
%\ifx\pdfoutput\undefined
%\usepackage{graphicx}
%\else
%\usepackage[pdftex]{graphicx}
%\fi

% However, be warned that pdflatex will require graphics to be in PDF
% (not EPS) format and will preclude the use of PostScript based LaTeX
% packages such as psfrag.sty and pstricks.sty. IEEE conferences typically
% allow PDF graphics (and hence pdfLaTeX). However, IEEE journals do not
% (yet) allow image formats other than EPS or TIFF. Therefore, authors of
% journal papers should use traditional LaTeX with EPS graphics.
%
% The path(s) to the graphics files can also be declared: e.g.,
% \graphicspath{{../eps/}{../ps/}}
% if the graphics files are not located in the same directory as the
% .tex file. This can be done in each branch of the conditional above
% (after graphicx is loaded) to handle the EPS and PDF cases separately.
% In this way, full path information will not have to be specified in
% each \includegraphics command.
%
% Note that, when switching from latex to pdflatex and vice-versa, the new
% compiler will have to be run twice to clear some warnings.


%\usepackage{psfrag}    % Written by Craig Barratt, Michael C. Grant,
                        % and David Carlisle
                        % This package allows you to substitute LaTeX
                        % commands for text in imported EPS graphic files.
                        % In this way, LaTeX symbols can be placed into
                        % graphics that have been generated by other
                        % applications. You must use latex->dvips->ps2pdf
                        % workflow (not direct pdf output from pdflatex) if
                        % you wish to use this capability because it works
                        % via some PostScript tricks. Alternatively, the
                        % graphics could be processed as separate files via
                        % psfrag and dvips, then converted to PDF for
                        % inclusion in the main file which uses pdflatex.
                        % Docs are in "The PSfrag System" by Michael C. Grant
                        % and David Carlisle. There is also some information
                        % about using psfrag in "Using Imported Graphics in
                        % LaTeX2e" by Keith Reckdahl which documents the
                        % graphicx package (see above). The psfrag package
                        % and documentation can be obtained at:
                        % http://www.ctan.org/tex-archive/macros/latex/contrib/supported/psfrag/

%\usepackage{subfigure} % Written by Steven Douglas Cochran
                        % This package makes it easy to put subfigures
                        % in your figures. i.e., "figure 1a and 1b"
                        % Docs are in "Using Imported Graphics in LaTeX2e"
                        % by Keith Reckdahl which also documents the graphicx
                        % package (see above). subfigure.sty is already
                        % installed on most LaTeX systems. The latest version
                        % and documentation can be obtained at:
                        % http://www.ctan.org/tex-archive/macros/latex/contrib/supported/subfigure/

%\usepackage{url}       % Written by Donald Arseneau
                        % Provides better support for handling and breaking
                        % URLs. url.sty is already installed on most LaTeX
                        % systems. The latest version can be obtained at:
                        % http://www.ctan.org/tex-archive/macros/latex/contrib/other/misc/
                        % Read the url.sty source comments for usage information.

%\usepackage{stfloats}  % Written by Sigitas Tolusis
                        % Gives LaTeX2e the ability to do double column
                        % floats at the bottom of the page as well as the top.
                        % (e.g., "\begin{figure*}[!b]" is not normally
                        % possible in LaTeX2e). This is an invasive package
                        % which rewrites many portions of the LaTeX2e output
                        % routines. It may not work with other packages that
                        % modify the LaTeX2e output routine and/or with other
                        % versions of LaTeX. The latest version and
                        % documentation can be obtained at:
                        % http://www.ctan.org/tex-archive/macros/latex/contrib/supported/sttools/
                        % Documentation is contained in the stfloats.sty
                        % comments as well as in the presfull.pdf file.
                        % Do not use the stfloats baselinefloat ability as
                        % IEEE does not allow \baselineskip to stretch.
                        % Authors submitting work to the IEEE should note
                        % that IEEE rarely uses double column equations and
                        % that authors should try to avoid such use.
                        % Do not be tempted to use the cuted.sty or
                        % midfloat.sty package (by the same author) as IEEE
                        % does not format its papers in such ways.

%\usepackage{amsmath}   % From the American Mathematical Society
                        % A popular package that provides many helpful commands
                        % for dealing with mathematics. Note that the AMSmath
                        % package sets \interdisplaylinepenalty to 10000 thus
                        % preventing page breaks from occurring within multiline
                        % equations. Use:
%\interdisplaylinepenalty=2500
                        % after loading amsmath to restore such page breaks
                        % as IEEEtran.cls normally does. amsmath.sty is already
                        % installed on most LaTeX systems. The latest version
                        % and documentation can be obtained at:
                        % http://www.ctan.org/tex-archive/macros/latex/required/amslatex/math/



% Other popular packages for formatting tables and equations include:

%\usepackage{array}
% Frank Mittelbach's and David Carlisle's array.sty which improves the
% LaTeX2e array and tabular environments to provide better appearances and
% additional user controls. array.sty is already installed on most systems.
% The latest version and documentation can be obtained at:
% http://www.ctan.org/tex-archive/macros/latex/required/tools/

% Mark Wooding's extremely powerful MDW tools, especially mdwmath.sty and
% mdwtab.sty which are used to format equations and tables, respectively.
% The MDWtools set is already installed on most LaTeX systems. The lastest
% version and documentation is available at:
% http://www.ctan.org/tex-archive/macros/latex/contrib/supported/mdwtools/


% V1.6 of IEEEtran contains the IEEEeqnarray family of commands that can
% be used to generate multiline equations as well as matrices, tables, etc.


% Also of notable interest:

% Scott Pakin's eqparbox package for creating (automatically sized) equal
% width boxes. Available:
% http://www.ctan.org/tex-archive/macros/latex/contrib/supported/eqparbox/



% Notes on hyperref:
% IEEEtran.cls attempts to be compliant with the hyperref package, written
% by Heiko Oberdiek and Sebastian Rahtz, which provides hyperlinks within
% a document as well as an index for PDF files (produced via pdflatex).
% However, it is a tad difficult to properly interface LaTeX classes and
% packages with this (necessarily) complex and invasive package. It is
% recommended that hyperref not be used for work that is to be submitted
% to the IEEE. Users who wish to use hyperref *must* ensure that their
% hyperref version is 6.72u or later *and* IEEEtran.cls is version 1.6b
% or later. The latest version of hyperref can be obtained at:
%
% http://www.ctan.org/tex-archive/macros/latex/contrib/supported/hyperref/
%
% Also, be aware that cite.sty (as of version 3.9, 11/2001) and hyperref.sty
% (as of version 6.72t, 2002/07/25) do not work optimally together.
% To mediate the differences between these two packages, IEEEtran.cls, as
% of v1.6b, predefines a command that fools hyperref into thinking that
% the natbib package is being used - causing it not to modify the existing
% citation commands, and allowing cite.sty to operate as normal. However,
% as a result, citation numbers will not be hyperlinked. Another side effect
% of this approach is that the natbib.sty package will not properly load
% under IEEEtran.cls. However, current versions of natbib are not capable
% of compressing and sorting citation numbers in IEEE's style - so this
% should not be an issue. If, for some strange reason, the user wants to
% load natbib.sty under IEEEtran.cls, the following code must be placed
% before natbib.sty can be loaded:
%
% \makeatletter
% \let\NAT@parse\undefined
% \makeatother
%
% Hyperref should be loaded differently depending on whether pdflatex
% or traditional latex is being used:
%
%\ifx\pdfoutput\undefined
%\usepackage[hypertex]{hyperref}
%\else
%\usepackage[pdftex,hypertexnames=false]{hyperref}
%\fi
%
% Pdflatex produces superior hyperref results and is the recommended
% compiler for such use.



% *** Do not adjust lengths that control margins, column widths, etc. ***
% *** Do not use packages that alter fonts (such as pslatex).         ***
% There should be no need to do such things with IEEEtran.cls V1.6 and later.


% correct bad hyphenation here
\hyphenation{op-tical net-works semi-conduc-tor IEEEtran}


\begin{document}

% paper title
\title{Motion Planning for a Class of Planar Closed-Chain
Manipulators
%\thanks{The work of Trinkle and Liu was supported in
%part by the National Science Foundation under grants 0139701
%(DMS-FRG), 0413227 (IIS-RCV), and 0420703 (MRI) and by Rensselaer
%Polytechnic Institute.}
}

\author{Guanfeng Liu, J.C. Trinkle$^\dagger$, and N. Shvalb$^\ddagger$\\
Department of Computer Science, Stanford University, liugf@cs.stanford.edu\\
 $\dagger$  Department of Computer Science, Rensselaer Polytechnic Institute, trink@cs.rpi.edu\\
$\ddagger$ Department of Mechanical Engineering, Technion-Israel Institute of Technology, Israel,
shvalbn@techunix.technion.ac.il}
% author names and affiliations
% use a multiple column layout for up to three different
% affiliations
%\author{\authorblockN{Guanfeng Liu}
%\authorblockA{Department of Computer Science\\
%Stanford University\\
%Stanford, CA\\
%Email:liugf@cs.stanford.edu}
%  \and
%\authorblockN{J.C. Trinkle}
%\authorblockA{Department of Computer Science\\
%Rensselaer Polytechnic Institute\\
%110 8th St., Troy, NY 12180-3590\\
% Email: trink@cs.rpi.edu}
%\and
%\authorblockN{N. Shvalb}
%\authorblockA{Department of Mechanical Engineering\\
%Technion-Israel
%Institute of Technology, Israel\\
%Email: shvalbn@techunix.technion.ac.il
%}
%}

% avoiding spaces at the end of the author lines is not a problem with
% conference papers because we don't use \thanks or \IEEEmembership


% for over three affiliations, or if they all won't fit within the width
% of the page, use this alternative format:
%
%\author{\authorblockN{Michael Shell\authorrefmark{1},
%Homer Simpson\authorrefmark{2},
%James Kirk\authorrefmark{3},
%Montgomery Scott\authorrefmark{3} and
%Eldon Tyrell\authorrefmark{4}}
%\authorblockA{\authorrefmark{1}School of Electrical and Computer Engineering\\
%Georgia Institute of Technology,
%Atlanta, Georgia 30332--0250\\ Email: mshell@ece.gatech.edu}
%\authorblockA{\authorrefmark{2}Twentieth Century Fox, Springfield, USA\\
%Email: homer@thesimpsons.com}
%\authorblockA{\authorrefmark{3}Starfleet Academy, San Francisco, California 96678-2391\\
%Telephone: (800) 555--1212, Fax: (888) 555--1212}
%\authorblockA{\authorrefmark{4}Tyrell Inc., 123 Replicant Street, Los Angeles, California 90210--4321}}



% use only for invited papers
%\specialpapernotice{(Invited Paper)}

% make the title area
\maketitle

%\IEEEpeerreviewmaketitle
%\pagestyle{empty}
%\thispagestyle{empty}

\begin{abstract}
We study the motion problem for planar {\sl star-shaped}
manipulators. These manipulators are formed by joining $k$ ``legs"
to a common point (like the thorax of an insect) and then fixing the
``feet" to the ground.  The result is a planar parallel manipulator
with $k-1$ independent closed loops.  A topological analysis is used
to understand the global structure of the configuration space so that
planning problem can be solved exactly. The worst-case complexity of
our algorithm is $O(k^3N^3)$, where $N$ is the maximum number of
links in a leg.  A simple example illustrating our method is given.
\end{abstract}

% no keywords

% For peer review papers, you can put extra information on the cover
% page as needed:
% \begin{center} \bfseries EDICS Category: 3-BBND \end{center}
%
% for peerreview papers, inserts a page break and creates the second title.
% Will be ignored for other modes.
%
%\section*{To Do}
%\begin{enumerate}
%
%\item
%
%\end{enumerate}

\section{Introduction}
Due to the computational complexity and difficulty of implementing
general exact motion planning algorithms, such as Canny's
\cite{Can88}, today sample-base algorithms, such as Kavraki's
\cite{KSLO96} dominate motion planning research. However, there are
important classes of problems for which these algorithms do not
perform well. These arise in systems whose configuration space
(C-space) cannot effectively be represented as a set of parameters
with simple bounds, but rather is most naturally represented as a
variety of co-dimension one or greater embedded in a
higher-dimensional ambient space \cite{YLK01}. Examples of the
systems include manipulators with one or multiple closed loops,
whose configuration space is defined by loop closure constraints.
The RLG method \cite{Cortes02,CS03} improves the sampling techniques
through estimating the regions of sampling parameters. However, its
efficiency relies on the accuracy of the estimation, which often
varies case by case. Moreover, it ignores the global structure of
C-space, and may fail to sample globally important regions.
%and for which specialized exact methods have not been developed.
%This provides motivation to try to develop effective exact
%algorithms.


%Cortes et al.
%\cite{Cortes02} propose a RLG (Random Loop Generator) method for
%generating random configurations for a single loop, which was
%further extended for manipulators with multiple kinematic loops
%\cite{CS03}. This method relies on an estimation of the possible
%region of configuration variables.
Recent advances in the understanding of the global structure of
C-spaces of single-loop closed chains \cite{TM02,LTrss05} allows us
to develop an effective exact algorithm for a class of manipulators
with multiple loops, namely, the planar {\sl star-shaped}
manipulators. These manipulators are formed by joining $k$ planar
``legs" to a common point (like the thorax of an insect) and then
fixing the ``feet" to the ground.  The result is a planar parallel
manipulator with $k-1$ independent closed loops. Each independent
loop imposes an algebraic kinematic constraint equation on the
system, and so the C-space of star-shaped manipulators is an
algebraic variety embedded in the joint space. Walking robots whose
legs are SCARA robots with axes perpendicular to the ground can be
modeled as a planar star-shaped manipulators.
%The most relevant previous work is that of Milgram, Trinkle, and Liu
%\cite{TM02,LTMwafr04,LTrss05}.  In those papers, the C-space of
%manipulators forming a single closed loop are analysed from a global
%topological perspective, and the topological properties are used to
%guide the design of complete algorithms.  The range of problem
%covered includes three-dimensional single-loop closed chains with
%spherical joints without obstacles and two-dimensional single-loop
%closed chains with point obstacles.

%In the former case, the
%complexity of the planning algorithm given was $O(n^3)$, where $n$
%is the number of links in the loop.  The worst-case complexity of
%the algorithm obtained for the latter was $\Omega(m^{n-3})$ and $O(m^{2n-7})$
%with $m$ being the number of point obstacles in the workspace,
%so unfortunately its practicality is currently limited to
%closed chains with about 6 links and 10 obstacles.

 %previous topological methods for C-space connectivity analysis are
%extended to the case of planar star-shaped manipulators without
%obstacle.
Here, we extend the previous topological methods in
\cite{TM02,LTrss05} for C-space connectivity analysis to the case of
planar star-shaped manipulators without obstacle. We derive
important global properties of C-space and use these results to
derive a necessary and sufficient condition for path existence
problem.  A complete algorithm based on the properties is
implemented and examples are presented.

\section{Notation}
\begin{center}
\begin{tabular}{rl}
\hline \hline \multicolumn{2}{c}{Manipulator Notation} \\ \hline
${M}$ & - Manipulator \\
$A$ & - Root junction or thorax of ${M}$ \\
$o_i$ & - Grounding point of foot $i$ of ${M}$ \\
$n_j$ & - Number of links in ${M}_j$ \\
$l_{j,i}$ & - Length of link $i$ of ${M}_j$; $i=1,...,n_j$ \\
$\theta_{j,i}$ & - Angle of link $i$ relative to link $i-1$ \\
${M}_j$ & - Leg $j$ of ${M}$ with foot fixed at $o_j$ \\
               & and other end free, $j=1,...,k$ \\
$\tilde {M}_j(p)$ & - Leg $j$ of ${M}$ with foot fixed at $o_j$\\
               & and other end fixed at $p$ \\
$\tilde {M}(p)$ & - Manipulator with $A$ fixed at $p$\\
$L_j$ & - Sum of lengths of links of ${M}_j$ \\
${\cal L}_j(p)$ & - A set of long links of $\tilde {M}_j(p)$ \\
$|{\cal L}_j^*(p)|$ & - Number of long links of $\tilde {M}_j(p)$ \\
               \hline \hline
  \multicolumn{2}{c}{Workspace Notation} \\ \hline
$W_A$ & - Workspace of $A$ \\
$^dU_i$ & - Cell of dimension $d$ of $W_A$ \\
$p$ & - Point in the plane of ${M}$ \\
$\gamma = p(t)$ & - Curve in the plane of ${M}$ \\
$f$ & - Kinematic map of $A$ \\
$f_j$ & - Kinematic map of endpoint of $M_j$ \\
$\Sigma_A$ & - Singular set of $f$ in $W_A$  \\
$\Sigma_j$ & - Singular set of $f_j$ \\
                 \hline \hline
  \multicolumn{2}{c}{Configuration Space (C-space) Notation} \\
\hline
${\cal C}$ & - C-space of ${M}$ \\
$\tilde {\cal C}(p)$ & - C-space of $\tilde {M}(p)$ \\
${\cal C}_j$ & - C-space of ${M}_j$ \\
$\tilde {\cal C}_j(p)$ & - C-space of $\tilde {M}_j(p)$ \\
$c$ & - Point in C-space \\ \hline \hline
\end{tabular}
\end{center}

\section{Preliminaries}
\label{section1}

The class of planar manipulators studied here are refered to as
planar {\sl star-shaped} manipulators (see Fig.~\ref{X-shape}). A
star-shaped manipulator is composed of $k$ serial chains with all
revolute joints.  Leg $M_j$ is composed of $n_j$ links of lengths
$l_{j,i}, i=1,...,n_j$  with angles $\theta_{j,i}, i=1,...,n_j$. At
one end (the foot), $M_j$ is connected to ground by a revolute joint
fixed at the point $o_j$. At the other end, it is connected by
another revolute joint to a junction point denoted by $A$. Note that
when $k$ is one, a star-shaped manipulator is an open serial chain.
When $k$ is two, it is a single-loop closed chain.
\begin{figure}
  \centering
  \includegraphics[width=3in]{fig/x-shape.eps}
  \caption{Star-shaped manipulator with $k=4$.}
  \label{X-shape}
\end{figure}

Assuming that the foot of $M_j$ is fixed at $o_j$, let
$f_j(\Theta_j)=p$ denote the kinematic map of $M_j$, where $\Theta_j
= (\theta_{j,1}, \cdots, \theta_{j,n_j})$ is the tuple of joint
angles, and $p$ is the location of the endpoint of the leg (the
thorax end). When $M_j$ is detached from the junction $A$, the image
of its joint space is the reachable set of positions of the free end
of the leg, called the workspace $W_j$. In the absence of joint limits,
the workspace $W_j$ is an
annulus if and only if there exists one link with length strictly
greater than the sum of all the other link lengths. Otherwise it is
a disk. Clearly, the workspace $W_A$ of $A$ when all the legs are
connected to $A$ is given by:
\begin{equation}
\label{eq:defW}
   W_A = \bigcap_{j=1}^k W_j.
\end{equation}

In our study of ${\cal C}$, it will be convenient to refer to
several other C-spaces. The C-space of leg $M_j$ when detached from
the rest of the manipulator will be denoted by ${\cal C}_j$. When
the endpoint is fixed at the point $p$, leg $j$ will be denoted by
$\tilde M_j(p)$, where the tilde is used to emphasize the fact that
the endpoint has been fixed. Note that $\tilde M_j(p)$ is a
single-loop planar closed chain, about which much is known (see
\cite{TM02}), including global structural properties of its C-space,
denoted by $\tilde {\cal C}_j(p) = f_j^{-1}(p)$.

When the junction $A$ of a star-shaped manipulator is fixed at point
$p$, its C-space will be denoted by $\tilde {\cal C}(p)$. Since
collisions are ignored, the motions of the legs are independent, and
therefore the C-space of the manipulator (with fixed junction) is
the product of the C-spaces of the legs with all endpoints fixed at
$p$:
\begin{equation}
\label{eq:def_tildeCp}
   \left. \begin{array}{rcl}
   \tilde {\cal C}(p) & = & \tilde {\cal C}_1(p) \ \times
           \cdots \times \ \tilde {\cal C}_k(p) \\ [5pt]
          & = & f_1^{-1}(p) \ \times \cdots \times \ f_k^{-1}(p) \\ [5pt]
          & = & f^{-1}(p)
   \end{array} \right\}
\end{equation}
where by analogy, $f$ is a total kinematic map of the star-shaped
manipulator. Loosely speaking, the union of the C-spaces $\tilde
{\cal C}(p)$ at each point $p$ in $W_A$ gives the C-space of a
star-shaped manipulator:
\begin{equation}
\label{eq:def_C}
   {\cal C} = \bigcup_{p \in W_A} \tilde {\cal C}(p).
\end{equation}
%Denote by $c$, $c_j$, $\tilde c(p)$, or $\tilde c_j(p)$, a point in
%${\cal C}$, ${\cal C}_j$, $\tilde {\cal C}(p)$, or $\tilde {\cal
%C}_j(p)$, respectively.

Several properties of the C-spaces ${\cal C}_j$ and $\tilde {\cal
C}_j(p)$ are highly relevant and so are reviewed here before
analyzing the C-space of star-shaped manipulators.  It is well known
that the C-space of $M_j$ is a product of circles ({\em i.e.,}
${\cal C}_j = (S^1)^{n_j}$).  The workspace $W_j$ contains a
singular set $\Sigma_j$ which is composed of all points $p$ in $W_j$
for which the Jacobian of the kinematic map $Df_j(\Theta_j)$ drops
rank for some $\Theta_j \in f_j^{-1}(p)$. These points form
concentric circles of radii $|l_{j,1}\pm l_{j,2} \pm \cdots \pm
l_{j,n_j}|$, as shown in Fig~\ref{fig:single-leg}. When $A$ coincides
with a point in $\Sigma_j$, the links can be arranged such that they
are all colinear, in which case the number of instantaneous degrees
of freedom of the endpoint of the leg is reduced from two to one.
\begin{figure}
  \centering
  \includegraphics[width=3in]{fig/nir-paper-1b.eps}
  \caption{{\bf Left:} The workspace $W_j$ of a three-link open chain $M_j$
    based at $o_j$. The singular set $\Sigma_j$ of the kinematic map $f_j$ is four
    concentric circles.  The small circles, figure eights, and points at 12 o'clock
    show the topology of the C-space $\tilde {\cal C}_j(p)$ of the
    leg when its endpoint is fixed at a point in one of the seven regions
    delineated by the singular circles (one of the four circles or one of
    the three open annular regions between them).  {\bf Right:} The inverse
    image of the curve $\gamma$ - a ``pair of pants."}
 \label{fig:single-leg}
\end{figure}

Now consider the case where the endpoint of leg $j$ is fixed to the
point $p$.  In other words, we are interested in the C-space $\tilde
{\cal C}_j(p)$ of $\tilde M_j(p)$. In the 12 o'clock position in
Fig.~\ref{fig:single-leg}, points, circles, and figure eights are
drawn to represent the global structures of $\tilde {\cal C}_j(p)$
in the seven regions of $W_j$. Specifically, when $A$ is fixed to a
point $p$ on the outer-most singular circle, $\tilde {\cal C}_j(p)$
is a single point. For $p$ fixed to any point in the largest open
annular region, C-space is a single circle. Continuing inward, the
possible C-space types are a figure eight (on the second largest
singular circle), two disconnected circles, a figure eight again, a
single circle, and a single point (on the inner-most singular
circle).

A detailed analysis of $\tilde {\cal C}_j(p)$ with an arbitrary
number of links in $\tilde M_j(p)$ can be found in \cite{TM02}. The
results that will be particularly useful in the analysis of
star-shaped manipulators follow.  First, the connectivity of $\tilde
{\cal C}_j(p)$ is uniquely determined by the number of ``long
links."  Consider the augmented link set composed of the links of
$M_j$ and $\overline {o_jp}$, which will be called the fixed base
link with length denoted by $l_{j,0}$. Let $L_j$ be the sum of all
the link lengths including the fixed base link ({\em i.e.,} $L_j =
\sum_{i=0}^{n_j} l_{j,i}$). Further, let ${\cal L}_j(p)$ be a subset
of $\{0,1,...,n_j\}$ such that $l_{j,\alpha}+l_{j,\beta} > L_j / 2;
\ \alpha,\beta \in {\cal L}_j(p), \ \alpha \neq \beta$. Over all
such sets, let ${\cal L}_j^*(p)$ be a set of maximal cardinality.
Then the number of long links of $\tilde M_j(p)$ is defined as
$|{\cal L}_j^*(p)|$, where $| \cdot |$ denotes set cardinality.

\medskip
\begin{Lemma} {\bf Kapovich and Milson \cite{KM95}, Trinkle and
Milgram \cite{TM02}}\\
\label{lem-02} \rm The C-space $\tilde {\cal C}_j(p) = f_j^{-1}(p)$
has two components if and only if $|{\cal L}_j^*(p)|=3$, and is
connected if and only if $|{\cal L}_j^*(p)|=2$ or $0$. No other
cardinality is possible.
\end{Lemma}

\medskip

Let us return to the discussion of Fig.~\ref{fig:single-leg}.
Viewing $W_j$ as a base manifold and the C-space corresponding to
each end point location as a fibre, it is apparent that the singular
set $\Sigma_j$ partitions $W_j$ into regions over which the C-spaces
$\tilde {\cal C}_j(p)$ form a trivial fibration. The implications of
this observation are useful in determining the C-space of more
complicated mechanisms.  Consider a modification to $\tilde M_j(p)$
that allows the endpoint to move along a one-dimensional curve
segment $\gamma$ within $W_j$.  Then as long as $\gamma$ is entirely
contained in one of the regions defined by the singular circles,
$\tilde {\cal C}_j(\gamma) = \tilde {\cal C}_j(p) \times I$, where
$I$ is the interval.  If $\gamma$ crosses a singular circle
transversally, then $\tilde {\cal C}_j(\gamma) = (\tilde {\cal
C}_j(p_1) \times I) \bigcup \tilde {\cal C}_j(p_3) \bigcup (\tilde
{\cal C}_j(p_2) \times I)$, where $p_1$ is a point in one of the two
open angular regions containing $\gamma$, $p_2$ is a point in the
other, and $p_3$ is a point on the singular circle crossed by
$\gamma$, and $\bigcup$ denotes the standard ``gluing" operation. In
Fig.~\ref{fig:single-leg}, an example $\gamma$ and the corresponding
C-space $\tilde {\cal C}_j(\gamma)$ are shown.

\section{Analysis of Star-Shaped Manipulators}
\label{section-2} For star-shaped manipulators with one or two legs,
the global topological properties of the C-space ${\cal C}$ are
fully understood (for one, see \cite{Lat92}; for two, see
\cite{TM02,MT04}). The goals of this section are to study the global
properties of ${\cal C}$ when $M$ has more than two legs and to
derive necessary and sufficient conditions for solution existence to
the motion planning problem.

\subsubsection{Local Analysis}
As a direct generalization of the singular set of a single leg, we
define the singular set of a star-shaped manipulator as a subset
$\Sigma$ of $W_A$ such that for every $p \in \Sigma$, there exists a
configuration $c$ such that at least one of the Jacobians
$\{Df_1(c),\cdots,Df_k(c)\}$ drops rank.  By definition we have:
\begin{eqnarray}
\label{eqn-01}
 \Sigma=\left(\bigcup_{i=1}^k \Sigma_i \right)\bigcap W_A.
\end{eqnarray}
An advantage of this definition is that $\Sigma$ can be used to
stratify $W_A$ such that each stratum is trivially fibred.
Figure~\ref{fig:double-leg} shows a star-shaped manipulator with two
legs.  The singular set $\Sigma$ is the boundary of the lune formed by the
intersection of the outer singular circles of their individual
workspaces.  For every point interior to the lune, the fibre is two
circles (the direct product of two points with one circle). The
fibres associated to the vertices of the lune are single points,
which correspond to simultaneous full extension of the two legs.
%Note the singular set $\Sigma$ is the boundary of the lune.
\begin{figure}
  \centering
  \includegraphics[width=3in]{fig/nir-paper-2a.eps}
 \caption{The workspace $W_A$ of $A$ for a star-shaped manipulator with $k=2$
    is the intersection of the workspaces of $A$ for each leg considered
    separately.  The singular set $\Sigma$ is composed of the black
    circular arcs where they bound or intersect the gray area.
    }
 \label{fig:double-leg}
\end{figure}

Fig.~\ref{fig:chambers} shows a possible workspace for a star-shaped
manipulator with three legs. The singular set defines 65 distinct
sets $^dU_i$ of varying dimension $d$, where $i$ is an arbitrarily
assigned index that simply counts components. We will refer to these
sets as {\sl chambers}. There are 12 two-dimensional, 32
one-dimensional, and 21 zero-dimensional chambers, each of which is
trivially fibred. More generally, the intersections among the arcs
composing $\Sigma$ are zero-dimensional chambers, denoted $^0\!U_i$,
$i=1,\cdots,^0\!m$. Removing the $^0\!U_i$ from $\Sigma$ partitions
it into open one-dimensional chambers $^1\!U_i$, $i=1,\cdots,^1\!m$.
Removing $^0\!U_i$ and $^1\!U_i$ from $W_A$ yields open
two-dimensional sets $^2\!U_i, \ i=1,\cdots,^2\!m$, for which the
following relationships hold:
\begin{eqnarray}
\Sigma & = & \left(\bigcup_{i=1}^{^0\!m} {}^0\!U_i \right) \bigcup
\left(\bigcup_{i=1}^{^1\!m} {}^1\!U_i \right) \\[5pt]
W_A - \Sigma & = & \bigcup_{i=1}^{^2\!m} {}^2\!U_i.
\end{eqnarray}
\begin{figure}
 \vbox{
      \centering {\epsfysize=2.3in \epsffile{fig/nir-paper-3a.eps}}
      }
 \caption{Workspace (shaded gray) of a star-shaped manipulator with
 three legs. The singular set partitions $W_A$ into $12$ two-dimensional,
 32 one-dimensional, and 21 zero-dimensional chambers.}
 \label{fig:chambers}
\end{figure}

 \smallskip

\begin{Proposition}
\label{prop-01} \rm For all $d=0,1,2$ and $i$, $f^{-1}(^d\!U_i) \ =
\ ^d\!U_i \times f^{-1}(p)$, where $p$ is any point in $^d\!U_i$ and
the operator $\times$ denotes the direct product. Gluing the
$f^{-1}(^d\!U_i)$ for all $i$ and $d$ gives the total C-space ${\cal
C}$.
\end{Proposition}

\medskip

%{\bf Proof:} When $d=0$, ${}^0\!U_i$ contains a single point, the
%result follows. When $d=1$, ${}^1\!U_i$ belongs to one singular
%circle of one leg, say $\tilde M_j$. Any two points $p_1,p_2 \in
%{}^1\!U_i$ are related by a Euclidean rotation $p_2-o_j =
%R(p_1-o_j)$, indicating that $\tilde {\cal C}_j(p_1)$ and $\tilde
%{\cal C}_j(p_2)$ are homeomorphic. Thus $\tilde {\cal C}_j(p)$ for
%all $p \in {}^1\!U_i$ have equivalent topological structure. For the
%other legs $\tilde M_l$, $l \neq j$, according to \cite{MT04} (Lemma
%$6.1$ and Corollary~$6.5$) $\tilde {\cal C}_l(p)$ for all $p \in
%{}^1\!U_i$ have equivalent topological structures as ${}^1\!U_i$ is
%free of singular points of $\tilde M_l(p)$. Thus $f^{-1}(p) = \tilde
%{\cal C}_1(p)\times \cdots \times \tilde {\cal C}_k(p)$ for all $p
%\in {}^1\!U_i$ have equivalent topological structures. The case when
%$d=2$ can be proved by applying Lemma $6.1$ and Corollary $6.5$ of
%\cite{MT04} to all legs. \hfill$\blacksquare$

\medskip

Proposition \ref{prop-01} and the fact that ${}^d\!U_i$ is a simply
connected set, reveal that
%the connectivity of $f^{-1}({}^d\!U_i)$
%is same as that of a single fibre
%$f^{-1}(p)=f_1^{-1}(p) \times \cdots \times f_k^{-1}(p)$
%${\tilde {\cal C}}(p)$, $p\in {}^d\!U_i$.
%Moreover,
each component
of $f^{-1}({}^d\!U_i)$
%$\cap_{p \in {}^d\!U_i} {\tilde {\cal C}}(p)$
is a direct product of one component of ${\tilde {\cal C}}_j(p)$,
$j=1,\cdots,k$, with a $d$-dimensional disk. Using $|{\cal
L}_j^*(p)|$, $j=1,\cdots,k$ and Lemma \ref{lem-02}, one can show
that the number of components of $f^{-1}({}^d\!U_i)$ is $2^{k_0}$,
where $k_0 \leq k$ is the number of legs for which $|{\cal
L}_j^*(p)| = 3$.

\smallskip

\subsubsection{Local Path Existence}
Before considering the global path existence problem, consider
motion planning between two valid configurations $c_{\rm init}$ and
$c_{\rm goal}$ for which the junction $A$ lies in the same chamber.
Since the fibre over every point in ${}^d\!U_i$ is equivalent, path
existence amounts to checking the component memberships of the
configurations $c_{\rm init}$ and $c_{\rm goal}$.

For a single leg $\tilde{M}_j(p)$, if the number of long links
$|{\cal L}_j^*(p)|$ is not three, then any two configurations of
$\tilde{M}_j(p)$ are in the same component. When $|{\cal
L}_j^*(p)|=3$, choose any two long links and test the sign of the
angle between them (with full extension taken as zero).  There are
two possible signs, one corresponding to {\sl elbow-up} and the
other to {\sl elbow-down}. If for two distinct configurations of
$\tilde{M}_j$, $A$ lies in the same chamber, there is a continuous
motion between them while keeping $A$ in this chamber, if and only
if the elbow sign is the same at both configurations (naturally, one
must perform the sign test with the same two links and in the same
order for both configurations). Considering all the legs together, a
continuous motion of $A$ in ${}^d\!U_i$ exists if and only if a
motion exists for each leg individually. The previous discussion
serves to prove the following result.

\medskip

\begin{Proposition}
\label{prop-2} \rm Restricted to $f^{-1}({}^d\!U_i)$, two
configurations $c_1,c_2 \in f^{-1}({}^d\!U_i)$ are path connected if
and only if for each leg $\tilde M_j$ with $|{\cal L}_j^*|=3$ in
${}^d\!U_i$, the elbow angle of $\tilde M_j$ has the same sign at
$c_1$ and $c_2$.
\end{Proposition}

\medskip

Proposition \ref{prop-2} completely solves the path existence
problem if $W_A$ consists of a single chamber. However, things
become complex when $W_A$ has more than one chamber.

\subsubsection{Singular Set and Global C-space Analysis}

Recall that the C-space ${\cal C}$ is a union of
$f^{-1}({}^d\!U_i)$, $d \in \{0,1,2\}$, $i=1,\cdots,{}^d\!m$ and
that $f^{-1}(p)$, $p\in {}^d\!U_i$ for $d\neq 2$ and all $i$ is a
set containing at least a singularity of $f$.
Combining the local C-space and singular set analysis
yields the global structure of C-space.

\medskip

\begin{Proposition}
\label{prop-sing} \rm For all $p \in \Sigma_j$, $f_j^{-1}(p)$ is a
singular set containing isolated singularities. If a singularity
separates its neighborhood $V$ in $f_j^{-1}(p)$, then it is these
singularities which glue the two separated components in
$f_j^{-1}(q)$ where $q \in W_A-\Sigma_j$ is a point sufficiently
close to $p$.
\end{Proposition}

\medskip

%{\bf Proof:} First it is obvious that $f_j^{-1}(p)$ contains
%isolated singularities for there are finite ways to colinearize all
%the links of a close chain. Second, let
%\[
%   \gamma: (-\varepsilon,\varepsilon) \rightarrow W_A,\, \gamma(0)=p
%\]
%be a curve that is transverse to $\Sigma_j$. According to Corollary
%6.6 of \cite{MT04}, the distance function $s(\gamma(t))=\int_{0}^t
%|{\dot \gamma}|dt$ defines a Morse function on $f_j^{-1}(\gamma)$
%\[
%   s \circ f_j: f_j^{-1}(\gamma) \rightarrow \reals.
%\]
%Note that $0$ is a singular value of $s \circ f_j$ and the isolated
%singularities of $f_j^{-1}(p)$ are also singularities of $s \circ
%f_j$. The result of Morse theory applying to $s \circ f_j$ yields
%that $(s \circ f_j)^{-1}(0)=f_j^{-1}(p)$ is given by attaching a
%handle to $(s \circ f_j)^{-1}(\varepsilon_0)=f_j^{-1}(q)$ for a
%sufficiently small $\varepsilon_0$ and $q$ a point sufficiently
%close to $p$. The Proposition follows.
%  \hfill$\blacksquare$\medskip

Next, we establish necessary and sufficient conditions for the
connectivity of ${\cal C}$.  Let ${\cal J}$ be the index set such
that for all $j \in {\cal J}$, $|{\cal L}_j^*|=3$ for at least one
chamber ${}^d\!U_i$. We prove the following theorem.

\medskip

\begin{Theorem}
\label{them-1} \rm Suppose $W_A = \bigcup_{d=0}^2 \left(
\bigcup_{i=1}^{{}^d\!m} {}^d\!U_i \right)$.  Then ${\cal C} =
f^{-1}(W_A)$ is connected if and only if:
\begin{enumerate}
\item
    $W_A$ is connected;
\item
    $\Sigma_j \bigcap W_A \ne \emptyset$ for all $j \in {\cal J}$.
\end{enumerate}
\end{Theorem}

\medskip

{\bf Proof:(sketch)} Notice that ${\cal C}$ is connected if and only
if any two possible configurations of leg $j$ are connected for all
$j$. These are exactly what Item 1 and 2 imply.
%(i) ``Necessity": Since ${\cal C}$ is a fibration of
%the base manifold $W_A$, it can have one component only when $W_A$
%has one component.  Thus item $1$ of Theorem~\ref{them-1} is
%required.
%%
%Second, in order that ${\cal C}$ be connected, for each leg $M_j$
%restricted to $W_A$, the C-space ${\tilde {\cal C}}_j(W_A) =
%f_j^{-1}(W_A)$ must be connected. By definition, for all $j \in
%{\cal J}$, there exists a chamber ${}^d\!U_i$ such that $|{\cal
%L}_j^*|=3$. The result of Proposition \ref{prop-sing} means that
%${\tilde {\cal C}}_j(W_A)$ is connected only if $W_A \bigcap
%\Sigma_j \ne \emptyset$.
%
%(ii) ``Sufficiency": Item 1 and 2 imply that ${\tilde {\cal
%C}}_j(W_A)$ are path connected for all $j$. Moreover, ${\cal C}$ is
%a fibration over $W_A$. The result follows.
\hfill$\blacksquare$\medskip

Fig.~\ref{fibre} illustrates the global connectivity for an example
$W_A$ corresponding to a star-shaped manipulator with two legs and a
workspace for which there are two chambers $^2U_1$ and $^2U_3$ where
leg~1 has three long links and another chamber $^2U_4$ where both
legs have three long links.  Among these chambers, ${}^1\!U_1$ and
${}^1\!U_2$ belong to $\Sigma_1$, and ${}^1\!U_3$ belongs to
$\Sigma_2$. According to Theorem~\ref{them-1}, the C-space is path
connected.  In this example, the ${\cal C}$ is the product of the
two structures shown.
\begin{figure}
 \vbox{
      \centering {\epsfysize=2.8in \epsffile{fig/fibre.eps}}
      }
 \caption{C-space of a star-shape manipulator with two legs.
 For purposes of simplicity, only the portion of $f^{-1}(\gamma)$
 is shown, where $\gamma$ is a continuous curve in $W_A$
 that visits all chambers. }
 \label{fibre}
\end{figure}

\smallskip

\begin{Corollary}
\label{cor-1} \rm Two configurations $c_1$ and $c_2$ of a
star-shaped manipulator are in the same component if and only if
\begin{enumerate}
\item $f(c_1)$ and $f(c_2)$ are in the same component of $W_A$;
\item For each leg $j$ with $|{\cal L}_j^*|=3$ for all chambers
${}^d\!U_i$ in the component of $W_A$ which contains $f(c_1)$ and
$f(c_2)$, the elbow sign is same at both $c_1$ and $c_2$.
\end{enumerate}
\end{Corollary}

%\medskip

%\begin{Remark}
%\rm As a matter of fact, $\Sigma$ completely determines the
%connectivity of C-space. To plan a path between two given
%configurations, often motions of the junction to points on $\Sigma$
%are produced to adjust the signs of the leg angles. However,
%possible deviation of the junction from $\Sigma$ caused by numerical
%errors, somtimes makes it impossible to adjust the sign of the leg
%while fixing its end point. For these reasons, points in 2D chambers
%are preferred for sign adjustment.
%\end{Remark}

%
%\subsection{Subsection Heading Here}
%Subsection text here.
%
%\subsubsection{Subsubsection Heading Here}
%Subsubsection text here.
%
% Reminder: the "draftcls" or "draftclsnofoot", not "draft", class option
% should be used if it is desired that the figures are to be displayed while
% in draft mode.

% An example of a floating figure using the graphicx package.
% Note that \label must occur AFTER (or within) \caption.
% For figures, \caption should occur after the \includegraphics.
%
%\begin{figure}
%\centering
%\includegraphics[width=2.5in]{myfigure}
% where an .eps filename suffix will be assumed under latex,
% and a .pdf suffix will be assumed for pdflatex
%\caption{Simulation Results}
%\label{fig_sim}
%\end{figure}


% An example of a double column floating figure using two subfigures.
%(The subfigure.sty package must be loaded for this to work.)
% The subfigure \label commands are set within each subfigure command, the
% \label for the overall fgure must come after \caption.
% \hfil must be used as a separator to get equal spacing
%
%\begin{figure*}
%\centerline{\subfigure[Case I]{\includegraphics[width=2.5in]{subfigcase1}
% where an .eps filename suffix will be assumed under latex,
% and a .pdf suffix will be assumed for pdflatex
%\label{fig_first_case}}
%\hfil
%\subfigure[Case II]{\includegraphics[width=2.5in]{subfigcase2}
% where an .eps filename suffix will be assumed under latex,
% and a .pdf suffix will be assumed for pdflatex
%\label{fig_second_case}}}
%\caption{Simulation results}
%\label{fig_sim}
%\end{figure*}



% An example of a floating table. Note that, for IEEE style tables, the
% \caption command should come BEFORE the table. Table text will default to
% \footnotesize as IEEE normally uses this smaller font for tables.
% The \label must come after \caption as always.
%
%\begin{table}
%% increase table row spacing, adjust to taste
%\renewcommand{\arraystretch}{1.3}
%\caption{An Example of a Table}
%\label{table_example}
%\begin{center}
%% Some packages, such as MDW tools, offer better commands for making tables
%% than the plain LaTeX2e tabular which is used here.
%\begin{tabular}{|c||c|}
%\hline
%One & Two\\
%\hline
%Three & Four\\
%\hline
%\end{tabular}
%\end{center}
%\end{table}

\section{A Polynomial-Time, Exact, Complete Algorithm}
Our algorithm consists of two main routines, {\tt PathExists} and
{\tt ConstructPath}.  The logical flow of {\tt PathExists} is
illustrated in Figure~\ref{PathExists}. Its input is the topology
and link lengths of a star-shaped manipulator and two valid
configurations, $c_{\rm init}$ and $c_{\rm goal}$.  The output is
the answer to the path existence question. Below we will show that
the complexity of {\tt PathExists} is $O(k^3+kN)$, where $N$ is the
maximum number of links in a leg and $k$ is the number of legs.
\begin{figure}
  \centering
  \includegraphics[width=2.5in]{fig/star-shaped-flow-chart3.eps}
  \caption{Logical flow and complexity of the major steps
  of {\tt PathExists}.}
  \label{PathExists}
\end{figure}

The approach taken is to compute $W_A$ and then, for each leg with
its end point constrained to lie in $W_A$, to determine if its
initial and goal configurations are path connected.  Since the
C-space of a leg is guaranteed to be connected if one of its
singular circles $\Sigma_j$ intersects $W_A$, the most straight
forward way to test connectivity is to explicitly perform the
intersections.  However, since there are as many as $2^{n_j-1}$
singular circles, any algorithm based on this approach will have
worst-case complexity that is at least exponential in $N$. The key
idea of {\tt PathExists} is a polynomial-time algorithm for checking
the existence of an intersection between $W_A$ and a singular
circle.

\medskip

\noindent \framebox{1. Construct $W_A$} Recall that $W_A$ is the
intersection of the workspaces of the legs when they are
disconnected from $A$.  Each workspace is a disk or annulus, which
can be determined by finding the length of the longest link and
comparing it with the sum of all the other link lengths of that leg.
 The boundary circles of these annuli decompose the plane into 2-D
open cells, among which, those can be reached by all legs constitute
$W_A$. We adopt a cell-decomposition algorithm (e.g., the line
sweeping algorithm) to compute a graph representation for $W_A$.
%For each leg, computation of the workspace is $O(n_j)$, where recall
%that $n_j$ is the number of links in leg $j$. Since there are $k$
%legs, computing the workspaces of the legs is $O(kN)$. The last step
%in the construction of $W_A$ is the pairwise intersection of the
%$O(k)$ boundary circles of the legs. Thus the overall complexity of
%constructing $W_A$ is $O(k^2 + kN)$.
The complexity of constructing the graph is $O(k^2+kN)$.
\medskip \\
\framebox{2. Are $p_{\rm init}$ and $p_{\rm goal}$ in same component
of $W_A$?}  As an immediate consequence of the cell decomposition,
this can be answered directly by searching the cell graph. This
requires $O(k^2)$ since in worst case the number of nodes in the
graph is $O(k^2)$.
%One can compute $p_{\rm init}$ and $p_{\rm goal}$ from
%$c_{\rm init}$ and $c_{\rm goal}$ from the kinematic map of any leg,
%which is $O(N)$. Extend rays from $p_{\rm init}$ and $p_{\rm goal}$
%and count the number of intersections of each ray with the boundary
%of each component of $W_A$.  The component with an odd number of
%intersections (counting tangencies as two intersections) contains
%the point. Since there are at most $O(k^2)$ arcs defining $W_A$,
%this step is $O(k^2)$.
\medskip  \\
\framebox{3. Compute $J$} This step is used to filter out easy
solution existence checks, based on the cardinality and members of
the sets ${\cal L}_j^*(p_{\rm init})$ and ${\cal L}_j^*(p_{\rm
goal})$. For each leg $\tilde M_j(p_{\rm init})$, compute $L_j$ (see
Section \ref{section1}) and find the three longest links of the set
$\{l_{j,0},...,l_{j_{n_j}} \}$.  Denote these links by $(p_{\rm
init}; \lambda_{j,1},\lambda_{j,2},\lambda_{j,3})$.  Do the same for
$(p_{\rm goal})$ and define $(p_{\rm goal};
\lambda_{j,1},\lambda_{j,2},\lambda_{j,3})$. This requires $O(N)$
work. Finally, $|{\cal L}_j^*(p_{(\cdot)})| = 3$ if and only if
$\lambda_{j,2} + \lambda_{j,3} > L_j/2$.  If ${\cal L}_j^*(p_{\rm
init}) = {\cal L}_j^*(p_{\rm goal})$ and $|{\cal L}_j^*(p_{\rm
init})| = 3$, and if the signs of the long links are different at
$c_{\rm init}$ and $c_{\rm goal}$, then add $j$ into $J$.
Computing $J$ is $O(kN)$. \medskip  \\
%\framebox{4. Compute extreme values of $||p-o_j||$; $p \in W_A$,
%$j\in J$} If $o_j$ is interior to $W_A$, then the minimum length is
%zero. This will be determined as a byproduct of the ray-shooting
%test above. Otherwise the minimum and maximum can be computed by
%finding the minimum and maximum distances from $o_j$ to each arc on
%the boundary of $W_A$.  In worst case this requires $O(k^2)$
%intersections between a line segment and a circular arc, for each
%leg, so this step is $O(k^3)$. \medskip  \\
\framebox{4. Does the set of long links vary for all $j \in J$?} If
and only if a way point $p_j \in W_A$ exists such that ${\cal L}_j^*(p_j) \neq {\cal
L}_j^*(p_{\rm init})$, then it is possible to make the long links
colinear and thus change the signs of their relative angles. This
can be done by computing a point $p_j \in W_A$ on the boundary of the
cell which contains $p_{\rm goal}$ and keeps the same constant
${\cal L}_j^*(p)$ for all $p$ in this cell. This boundary is
characterized by $\lambda_{j,2}+\lambda_{j,3}=L_j/2$. Using the fact
that the base link is the only link with variable link length, $\{p_j \mid j \in J\}$
can be computed in $O(k^3+kN)$.
%If and only if $p \in W_A$ exists such that ${\cal L}_j^*(p) \neq
%{\cal L}_j^*(p_{\rm init})$, then it is possible to make the long
%links colinear and thus change the signs of their relative angles.
%This can be done by computing a base length $l_{j,0}$ satisfying
%$\lambda_{j,2}(l_{j,0})+\lambda_{j,3}(l_{j,0})=L_j(l_{j,0})/2$ (Note
%that $L_j$ varies linearly with $l_{j,0}$, and $\lambda_{j,2}$ and
%$\lambda_{j,3})$ vary discontinuously with $l_{j,0}$.)  Finding such
%a base length or determinining nonexistence of such can be computed
%in constant time for each leg, so this step is $O(k)$.

\medskip

The basic idea of {\tt ConstructPath} 
%is that when moving from
%$c_{\rm init}$ to $c_{\rm goal}$, those legs $j \in J$ require a
%change in the signs of relative angles between a pair of long links,
%some of the legs with three long
%links when $A$ is fixed at $p_{\rm init}$ and $p_{\rm goal}$ will
%require a change in the signs of relative angles between long links,
%which is always possible at the way point $p_j$ or other singular
%points of the corresponding leg. A natural approach then is 
is to use
two kinds of motion generation algorithms: {\sl accordion move} and
{\sl sign-adjust move}.  The former moves the thorax endpoint (at
$A$) along a specified path segment with all legs moving compliantly
so that all loop closures are maintained.
%without changing the sign of the relative
%angle between a pair of long links in legs.
The latter keeps the endpoint fixed at a way point $q_j \in
\Sigma_j$ while moving
leg $j$ into a colinear configuration and then to a nearby
configuration with the sign of the relative angle between a pair of
long links in this leg chosen to match those of $c_{\rm goal}$.

The input of {\tt ConstructPath} is $W_A$ and its cell graph,
$c_{\rm init}$, $c_{\rm goal}$, and the set of way points $p_{j}\,
\in W_A, \, j\in J$ computed during the execution of {\tt
PathExist}.

\noindent \framebox{1. Construct an initial path} {\tt
ConstructPath} explores the cell graph of $W_A$, and constructs a
path in $W_A$ connecting $p_{\rm init}$ to $p_{\rm goal}$ and
visiting all of the way points. Since there are at most $k$ way
points, this can be done in $O(k^3)$ time (the path has $k$ segments
each with $O(k^2)$ arcs).

\noindent \framebox{2. Construct {\em guards} and insert the {\rm
guards} into the path} Notice that accordion moves keep the 
sign of the relative angle between a pair of long links of leg $j$ 
only when the thorax endpoint moves in a cell where ${\tilde C}_j$ has two components. 
%Notice that when one accordion moves a leg in
%a cell in which the number of long links is not $3$ (called
%one-component cell), neither the signs of concatenating angles, nor
%the sign between any pair of links in legs will be kept invariant.
%Thus even the sign between a pair of long links is adjusted to the
%desired one at a way point (e.g., $p_j$), it still could change if
%the leg keeps moving accordionly in this one-component cell. 
For this reason, we set {\em guards} $\{q_j\}$ for legs which have three long
links at $p_{\rm goal}$. The set of such legs is denoted $I$. 
$\{q_j\}$ is computed as the last intersection point between the above constructed path
in $W_A$ and the boundary of the two-component cell of leg $j$
containing $p_{\rm goal}$, for all leg $j \in I$. 
%Thus the number of guards ($q_j$'s) may be more than
%the number of way points. 
Inserting these {\em guards} into the
path. 
%Later when we construct the path in ${\cal C}$, sign-adjust
%moves are only performed at {\em guards} $q_j$ (but not $p_j$) for
%after that the thorax endpoint gets into the two-component cell and
%the sign between a pair of long links will not change during
%accordion moves, i.e., the leg only stays in one component its
%C-space. 
Assuming each arc in the path is approximated by fixed
number of line segments, finding guards is $O(k^3)$.

\noindent \framebox{3. Accordion moves and sign-adjust moves} The
path in ${\cal C}$ then is produced by using accordion moves along
the path and sign-adjust moves at the {\em guards}. At each {\em
guard} $q_j$, $j\in I$, one checks the sign between a pair of long links of leg $j$. 
If it does not match the goal one, do sign-adjust
move, otherwise, accordion moves keep going on. Once $A$ is
coincident with $p_{\rm goal}$, one is assured by the previous
steps, that with $A$ fixed at $p_{\rm goal}$, the configuration of
each leg is in the same component of its current C-space $\tilde
C_j(p_{\rm goal})$ as $c_{\rm goal}$. The final move can be
accomplished using a special accordion move algorithm found in
\cite{TM02}. 
%At this stage, we remark that finding the set of way
%points $p_j$ and planning an initial path visiting all $p_j$ is
%necessary for otherwise, an arbitrary path between $p_{\rm init}$
%and $p_{\rm goal}$ may not intersect the boundary of the
%two-component cell of a leg that contains $p_{\rm goal}$.

The complexity of the accordion move algorithms reported in
\cite{TM02} are $O(N^3)$. Since the path has $O(k^3)$ line segments
%Since there are $O(k^2)$ boundary segments, the number of path
%segments is also $O(k^2)$.
%If we
%assume that each boundary segment in the path is approximated by a
%constant number of line segments, then
the complexity of {\tt ConstructPath} is $O(k^3N^3)$.  Note that
accordion move algorithms with the required behavior can be designed
to be $O(N^2)$, so the complexity of {\tt ConstructPath} could be
reduced.

Overall, our path planning algorithm is  $O(k^3N^3)$.


%The basic idea of {\tt ConstructPath} is that when moving from
%$c_{\rm init}$ to $c_{\rm goal}$, some of the legs with three long
%links when $A$ is fixed at $p_{\rm init}$ and $p_{\rm goal}$ will
%require a change in the signs of relative angles between long links, which is
%always possible at a singular point of the corresponding leg. A
%natural approach then is to use two kinds of motion generation
%algorithms: {\sl accordion move} and {\sl sign-adjust move}.  The
%former moves the thorax endpoint (at $A$) along a specified path
%segment without changing the sign of the ralative angle
%between a pair of long links in legs.
%The latter keeps the endpoint fixed at a way point
%$p \in \Sigma_j$ while moving leg $j$ into a colinear configuration
%and then to a nearby configuration with the sign of the relative angle
%between a pair of long links in this leg
%chosen to match those of $c_{\rm goal}$.

%The input of {\tt ConstructPath} is $W_A$, $c_{\rm init}$, $c_{\rm
%goal}$, and the set of way points $p_{j}\, \in W_A, \, j\in J$
%computed during the execution of {\tt PathExist}.  {\tt
%ConstructPath} constructs a path in $W_A$ connecting $p_{\rm init}$
%to $p_{\rm goal}$ and visiting all of the way points.  Since there
%are at most $k$ way points, this can be done in $O(k^3)$ time by
%connecting all these points with straight lines to the boundary of
%$W_A$.  Though crude, a path can be constructed by moving from
%$c_{\rm init}$ to the boundary, around the boundary to the line
%segment connected to one of the way points, along the line to the
%way point, back to the boundary, and so on, until all way points
%have been visited and $p_{\rm goal}$ has been reached.  The path in
%${\cal C}$ then is produced by using accordion moves along the path
%and sign-adjust moves at the way points.  Once $A$ is coincident
%with $p_{\rm goal}$, one is assured by the previous steps, that with
%$A$ fixed at $p_{\rm goal}$, the configuration of each leg is in
%the same component of its current C-space $\tilde C_j(p_{\rm
%goal})$ as $c_{\rm goal}$.  The final move can be accomplished using
%a special accordion move algorithm found in \cite{TM02}.

%The complexity of the accordion move algorithms reported in
%\cite{TM02} are $O(N^3)$.  Since there are $O(k^2)$ boundary
%segments, the number of path segments is also $O(k^2)$.  If we
%assume that each boundary segment in the path is approximated by a
%constant number of line segments, then the complexity of {\tt
%ConstructPath} is $O(k^2N^3)$.  Note that accordion move algorithms
%with the required behavior can be designed to be $O(N^2)$, so the
%complexity of {\tt ConstructPath} could be reduced.
%Overall, our path planning algorithm is  $O(k^3 + k^2N^3)$.
\section{Example}
In this example, the manipulator has three three-link legs, one of
which has three long links when $A$ is fixed at 
$p_{\rm goal}$. Figure~\ref{start-goal} shows the manipulator in
it's starting and goal configurations.  Since when $A$ is fixed at
$p_{\rm goal}$, the C-space of two of the legs have one component,
our algorithm requires that $A$ move from its initial location to
one guard, and then to the goal.  At the guard, the two lower legs can remain fixed while the other leg is
moved to adjust the signs of its relative angles.  
%In this
%particular example, we chose to move $A$ to a point on the singular
%set of the leg in question.  
This leg was then moved to make all its
links colinear. Before leaving the guard via the next
accordion move, leg's angles were adjusted to match the relative
signs in the goal configuration.
Figures~\ref{accordion},~\ref{adjust_sign},~\ref{accordion_goal},
and~\ref{TMalg} show the progress of the manipulation plan as the
steps of the complete planning algorithm are carried out.
 Animation of the motion in
this example can be found in
\verb$http://www.cs.rpi.edu/~liugf/multiloop$.
\begin{figure}
  \centering
  \includegraphics[width=2.2in]{fig/two_config.eps}
  \caption{Manipulator's initial configuration (junction on the right, drawn red)
  and goal configuration (junction just below the top foot, drawn blue.}
  \label{start-goal}
\end{figure}

\begin{figure}
  \centering
  \includegraphics[width=2.2in]{fig/accordion.eps}
  \caption{All legs use an accordion move to move the junction $A$ to the guard on
  a singular circle of leg~2.}
  \label{accordion}
\end{figure}

\begin{figure}
  \centering
  \includegraphics[width=2.2in]{fig/adjust_sign.eps}
  \caption{With the junction $A$ at the guard of
  leg~2, its join angles can be adjusted to achieve the signs required
  at the goal configuration.}
  \label{adjust_sign}
\end{figure}

\begin{figure}
  \centering
  \includegraphics[width=2.2in]{fig/adjust_goal.eps}
  \caption{All legs use an accordion move to move the junction $A$ to its
  goal location.  The sign of leg 2 are preserved guaranteeing
  that leg~2 will be in the correct C-space component once $A$ is fixed at
  the goal position.}
  \label{accordion_goal}
\end{figure}

\begin{figure}
  \centering
  \includegraphics[width=2.2in]{fig/TMalg.eps}
  \caption{All legs use the Trinkle-Milgram algorithm to achieve their
  goal configurations with the junction $A$ fixed.}
  \label{TMalg}
\end{figure}

\section{Conclusion}
In this paper, we study the global structural properties of planar star-shaped
manipulators. Via the analysis of the singular set $\Sigma$,
we derive the global connectivity of the C-space, and necessary and sufficient conditions
for path existence. Based on these results, we devise a complete algorithm for motion planning.
Simulation examples are used to illustrate the key ideas behind the motion planning problem
of planar star-shaped manipulators.


% conference papers do not normally have an appendix

% use section* for acknowledgement
\section*{Acknowledgment}
% optional entry into table of contents (if used)
%\addcontentsline{toc}{section}{Acknowledgment}
%The authors would like to thank Nir Shvalb and Moshe Shoham for
%their insightful comments on aspects of this work, 
The authors would like to thank Ryan Trinkle for
help on the analysis of the solution existence algorithm, and the
National Science Foundation for its support through grants 0139701
(DMS-FRG), 0413227 (IIS-RCV), and 0420703 (MRI).

% trigger a \newpage just before the given reference
% number - used to balance the columns on the last page
% adjust value as needed - may need to be readjusted if
% the document is modified later
%\IEEEtriggeratref{8}
% The "triggered" command can be changed if desired:
%\IEEEtriggercmd{\enlargethispage{-5in}}

% references section
% NOTE: BibTeX documentation can be easily obtained at:
% http://www.ctan.org/tex-archive/biblio/bibtex/contrib/doc/

% can use a bibliography generated by BibTeX as a .bbl file
% standard IEEE bibliography style from:
% http://www.ctan.org/tex-archive/macros/latex/contrib/supported/IEEEtran/bibtex
%\bibliographystyle{IEEEtran.bst}
% argument is your BibTeX string definitions and bibliography database(s)
%\bibliography{IEEEabrv,../bib/paper}
%
% <OR> manually copy in the resultant .bbl file
% set second argument of \begin to the number of references
% (used to reserve space for the reference number labels box)

\begin{thebibliography}{10}

\bibitem{Can88}
J.F. Canny, \emph{The Complexity of Robot Motion Planning}. \hskip
1em plus 0.5em minus 0.4em\relax Cambridge, MA: MIT Press, 1988.

\bibitem{Cortes02}
J. Cort\'es, T. Sim\'eon, and J.P. Laumond, \emph{A Random Loop
Generator for Planning the Motions of Closed Kinematic Chains using
PRM Methods}. \hskip 1em plus 0.5em minus 0.4em\relax In Proceedings
of the 2002 IEEE International Conference on Robotics and
Automation, pages 2141-2146, 2002.

\bibitem{CS03}
J. Cortes and T. Sim\'eon, \emph{Probabilistic Motion Planning for
Parallel Mechanisms}. \hskip 1em plus 0.5em minus 0.4em\relax In
Proceedings of the 2003 IEEE International Conference on Robotics
and Automation, pages 4354-4359, 2003.

\bibitem{KM95}
M. Kapovich and J. Millson, \emph{On the moduli spaces of polygons
in the Euclidean plane}. \hskip 1em plus 0.5em minus 0.4em\relax
Journal of Differential Geometry, Vol. 42, PP. 133-164, 1995.

\bibitem{KSLO96} L.E. Kavraki, P. $\check{S}$vestka, J.C. Latombe,
and M.H. Overmars, \emph{Probablistic Roadmaps for path planning in
high-dimensional configuration space}. \hskip 1em plus 0.5em minus
0.4em\relax IEEE Transactions on Robotics and Automation,
12(4):566-580, 1996.

%\bibitem{Lat92}
%J.C. Latombe, \emph{Robot Motion PLanning}. \hskip 1em plus 0.5em
%minus 0.4em\relax Kluwer Academic Publishers, 1992.

%\bibitem{MT04}
%R.J. Milgram and J.C. Trinkle, \emph{The Geometry of Configuration
%Spaces for Closed Chains in Two and Three Dimensions}. \hskip 1em
%plus 0.5em minus 0.4em\relax Homology Homotopy and Applications,
%6(1):237-267, 2004.

\bibitem{TM02}
J.C. Trinkle and R.J. Milgram, \emph{Complete Path Planning for
Closed Kinematic Chains with Spherical Joints}. \hskip 1em plus
0.5em minus 0.4em\relax International Jounral of Robotics Research,
21(9):773-789, December, 2002.

%\bibitem{LTMwafr04}
%G.F. Liu, J.C. Trinkle and R.J. Milgram, \emph{Toward Complete Path Planning
%for Planar 3R-Manipulators Among Point Obstacles}. \hskip 1em plus
%0.5em minus 0.4em\relax In Algorithmic Foundations of Robotics VI, STAR 17, Springer-Verlag,
%329-344, 2005.

\bibitem{LTrss05}
G.F. Liu and J.C. Trinkle, \emph{Complete Path Planning for Planar Closed Chains Among Point Obstacles}.
\hskip 1em plus 0.5em minus 0.4em
\relax In Robotics: Science and Systems, MIT Press, 2005.

\bibitem{YLK01}
J. Yakey, S. M. LaValle, and L. E. Kavraki, \emph{Randomized path
planning for linkages with closed kinematic chains}. \hskip 1em plus
0.5em minus 0.4em \relax IEEE Transactions on Robotics and
Automation, 17(6):951--958, December 2001.

%
%%\bibitem{SHS87}
%%J. Schwartz, J. Hopcroft, and M. Sharir, \emph{Planning, Geometry,
%%and Complexity of Robot Motion}. \hskip 1em plus 0.5em minus
%%0.4em\relax Ablex, 1987.
%
%
%\bibitem{RRT}
%S. M. LaValle and J. J. Kuffner, \emph{Rapidly-exploring random
%trees: Progress and prospects}. \hskip 1em plus 0.5em minus
%0.4em\relax In B. R. Donald, K. M. Lynch, and D. Rus, editors,
%Algorithmic and Computational Robotics: New Directions, pages
%293-308, A K Peters, Wellesley, MA, 2001.
%
%\bibitem{Lin04}
%F. Lingelbach, \emph{Path Planning using Probabilistic Cell
%Decomposition}. \hskip 1em plus 0.5em minus 0.4em\relax IEEE
%International Conference on Robotics and Automation, PP. 467-472,
%2004.
%
%\bibitem{LBL04}
%S.M. LaValle, M.S. Branicky and S.R. Lindemann, \emph{On the
%Relationship between Classical Grid Search and Probabilistic
%Roadmaps}. \hskip 1em plus 0.5em minus 0.4em\relax
% The International Journal of Robotics Research, Number 23, Volume 7/8
% (2004):673-692.
%
%\bibitem{LSB}
%S. R. Lindemann and S. M. LaValle, \emph{Current issues in
%sampling-based motion planning}. \hskip 1em plus 0.5em minus 0.4em
%\relax In P. Dario and R. Chatila, editors, Proc. Eighth Int'l Symp.
%on Robotics Research. Springer-Verlag, Berlin, 2004. To appear.
%
%\bibitem{Mer00}
%J.P. Merlet, \emph{Parallel Robots}. \hskip 1em plus 0.5em minus
%0.4em \relax Kluwer Academic Publishers, 2000.
%
%\bibitem{SS83}
%J.T. Schwartz and M. Sharir, \emph{On the piano movers II. General
%techniques for computing topological properties on real algebraic
%manifolds}. \hskip 1em plus 0.5em minus 0.4em \relax Adv. Appl.
%Math., vol.4, PP. 298-351, 1983.

%\bibitem{HA01}
%L. Han and N.M. Amato, \emph{A kinematics-based probabilistic
%roadmap method for closed chain systems}. \hskip 1em plus 0.5em
%minus 0.4em \relax in Algorithmic and Computational Robotics: New
%Directions, B.R. Donald, K.M. Lynch, and D. Rus, Eds. AK Peters,
%Wellesley, PP. 233-246, 2001.
%
%\bibitem{ABDJV98}
%N. Amato, B. Bayazit, L. Dale, C. Jones, and D. Vallejo,
%\emph{OBPRM: An obstacle-based PRM for 3d workspaces}. \hskip 1em
%plus 0.5em minus 0.4em \relax in Robotics: The Algorithmic
%Perspective, P. Agarwal, L. Kavraki, and M. Mason, Eds. Natick, MA:
%A.K. Peters, 1998, PP. 156-168.
%
%\bibitem{BOS99}
%V. Boor, M. Overmas, and A.F. van der Stappen, \emph{The Gaussian
%sampling strategy for probabilistic roadmap planners}. \hskip 1em
%plus 0.5em minus 0.4em \relax IEEE International Conference on
%Robotics and Automation, 1999.
%
%\bibitem{BK00}
%R. Bohlin and L. Kavraki, \emph{Path planning using lazy PRM}.
%\hskip 1em plus 0.5em minus 0.4em \relax IEEE International
%Conference on Robotics and Automation, PP. 521-528, 2000.
%
%\bibitem{KL00}
%J.J. Kuffner and S.M. LaValle, \emph{RRT-Connect: An Efficient
%Approach to Single-Query Path Planning}. \hskip 1em plus 0.5em minus
%0.4em \relax IEEE International Conference on Robotics and
%Automation, PP. 995-1001, 2000.

%\bibitem{HKLR02}
%D. Hsu, R. Kindel, J.-C. Latombe, \emph{Randomized Kinodynamic
%Motion Planning with Moving Obstacles}. \hskip 1em plus 0.5em minus
%0.4em \relax International Journal of Robotics Research, Vol 21, No.
%3, PP. 233-255, 2002.
%
%\bibitem{KKL98}
%L. Kavraki, M.N. Kolountzakis, and J.-C. Latombe, \emph{Analysis of
%Probabilistic Roadmaps for Path Planning}. \hskip 1em plus 0.5em
%minus 0.4em \relax IEEE Transactions on Robotics and Automation,
%Vol. 14, PP. 166-171, 1998.
%
%\bibitem{LK04}
%A.M. Ladd and L.E. Kavraki, \emph{Measure Theoretic Analysis of
%Probabilistic Path Planning}. \hskip 1em plus 0.5em minus 0.4em
%\relax IEEE Transactions on Robotics and Automation, Vol. 20, No. 2,
%PP. 229-242, 2004.
%
%\bibitem{LK99}
%S.M. LaValle and J.J. Kuffner, \emph{Randomized kinodynamic
%planning}. \hskip 1em plus 0.5em minus 0.4em \relax IEEE
%International Conference on Robotics and Automation, 1999.

%\bibitem{BLOY01}
%M.S. Branicky, S.M. LaValle, K. Olson, and L.B. Yang,
%\emph{Quasi-Randomized Path Planning}. \hskip 1em plus 0.5em minus
%0.4em \relax IEEE International Conference on Robotics and
%Automation, PP. 1481-1487, 2001.

%\bibitem{icra}
%G.F. Liu, J.C. Trinkle, R.J. Milgram,  \emph{Complete Path
%Planning for Planar 2-R Manipulators With Point Obstacles}. \hskip
%1em plus 0.5em minus 0.4em\relax Proceedings of IEEE International
%Conference on Robotics and Automation, 3263-3269, 2004.

%\bibitem{LTMwafr-2004}
%G.F. Liu, J.C. Trinkle, R.J. Milgram, \emph{Complete Path Planning
%for Planar 3R-Manipulators Among Point Obstacles}. \hskip 1em plus
%0.5em minus 0.4em\relax To appear in Proceedings of WAFR 2004.

%
%\bibitem{SSB05}
%N. Shvalb, M. Shoham, D. Blanc \emph{The Configuration
%Space of an Arachnoid Mechanism}. \hskip 1em plus 0.5em minus
%0.4em\relax To appear in Forum Mathemaicum 2005.
%
%%\bibitem{HK98}
%%J.-C. Hausmann and A. Knutson, \emph{The cohomology ring of polygon
%%spaces}. \hskip 1em plus 0.5em minus 0.4em\relax Ann. Inst. Fourier,
%%Vol. 48, PP. 281-321, 1998.
%%
%%\bibitem{KTT98}
%%Y. Kamiyama, M. Tezuka, and T. Toma, \emph{Homology of the
%%Configuration Spaces of Quasi-Equilateral Polygon Linkages}. \hskip
%%1em plus 0.5em minus 0.4em\relax Transactions of the American
%%Mathematical Society, Vol. 350, No. 12, PP. 4869-4896, 1998.
%%
%%\bibitem{BL91}
%%J. Barraquand and J.-C. Latombe, \emph{Nonholonomic Multibody Mobile
%%Robots: Controllability and motion planning in the presence of
%%obstacles}. \hskip 1em plus 0.5em minus 0.4em\relax IEEE
%%Interantional Conference on Robotics and Automation, PP. 2328-2335,
%%1991.
%%
%%\bibitem{NS83}
%%C. Nash and S. Sen, \emph{Topology and Geometry for Physicists}.
%%\hskip 1em plus 0.5em minus 0.4em\relax Academic Press, 1983.
%%
%%\bibitem{GP74}
%%V. Guillemin and A. Pollack \emph{Defferential Topology}. \hskip 1em
%%plus 0.5em minus 0.4em\relax Publisher:Prentice Hall PTR, 1974.
%%
%%\bibitem{WH86}
%%J. Hopcroft and G. Wilfong, \emph{On the motion of objects in
%%contact}. \hskip 1em plus 0.5em minus 0.4em \relax Int. J. Robotics
%%Research, Vol. 4, 1986, PP. 32-46.
%
%\bibitem{CG99}
%M. Cherif and K.K. Gupta, \emph{Planning Quasi-Static Fingertip
%Manipulation For Reconfiguring Objects}. \hskip 1em plus 0.5em minus
%0.4em\relax IEEE Transactions on Robotics and Automation, Vol. 15,
%No. 5, PP. 837-848, 1999.
%
% \bibitem{Koditschek87}
%D. E. Koditschek, \emph{Exact Robot Navigation by Means of Potential
%Functions: Some Topological Considerations}. \hskip 1em plus 0.5em
%minus 0.4em\relax IEEE International Conference on Robotics and
%Automation, PP. 1-6, 1987.
%
%\bibitem{Lozano-Perez83}
%Lozano-Perez83, \emph{Spatial Planning: A Configuration Space
%Approach}. \hskip 1em plus 0.5em minus 0.4em \relax IEEE
%Transactions on Computers, Vol. C-32, No. 2, PP. 108-119, 1983.
%
%\bibitem{RK88}
%E. Rimon and D. E. Koditschek, \emph{Exact Robot Navigation Using
%Cost Functions: The Case of Distinct Spherical Boundaries in $E^n$}.
%\hskip 1em plus 0.5em minus 0.4em\relax IEEE International
%Conference on Robotics and Automation, PP. 1791-1796, 1988.

%\bibitem{HLM97}
%D. Hsu, J.C. Latombe, and R. Motwani, \emph{Path Planning in
%Expansive Configuration Spaces}. \hskip 1em plus 0.5em minus 0.4em
%\relax IEEE International Conference on Robotics and Automation,
%1997.

%\bibitem{RK89}
%E. Rimon and D. E. Koditschek, \emph{The Construction of Analytic
%Diffeomorphisms for Exact Robot Navigation on Star Worlds}. \hskip
%1em plus 0.5em minus 0.4em\relax IEEE International Conference on
%Robotics and Automation, PP. 21-26, 1989.
%
\end{thebibliography}

% that's all folks
\end{document}
