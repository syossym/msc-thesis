%======================================================================
%                               Dissertation
%                                Nir Shvalb
%
%                                  2.5.06
%========================== Including Packages ===========================
\documentclass[12pt,final,a4paper,oneside]{report}
\usepackage[centertags]{amsmath}
\usepackage{amsfonts}
\usepackage{multicol}
\usepackage{hebfont}
\usepackage[all]{xypic}
\usepackage{epsf}
\usepackage{amssymb}
\usepackage{lscape}
\usepackage{amsthm}
\usepackage{newlfont}
\usepackage{graphics}
\usepackage{epsfig}
\usepackage{proposal} %DAL Thesis Style
\usepackage{hyperref}
\usepackage{latexsym}
\usepackage{amstext}
\usepackage{verbatim}
\usepackage{subfigure}
\usepackage{algorithm}
\usepackage{algorithmic}
\usepackage{makeidx}         % allows index generation
\usepackage{graphicx}        % standard LaTeX graphics tool when including figure files
%\usepackage{palatino, url, multicol}
%\usepackage{xtocinc} %Include Table of Contents as the first entry in TOC
%\usepackage{hebrew}
%\usepackage{hebcal}
%\usepackage[bottom]{footmisc}% places footnotes at page bottom

%========================================================================
\hfuzz2pt % Don't bother to report over-full boxes if over-edge is < 2pt

% Line spacing ====================================================================
\newlength{\defbaselineskip}
\setlength{\defbaselineskip}{\baselineskip}
\newcommand{\setlinespacing}[1]%
           {\setlength{\baselineskip}{#1 \defbaselineskip}}
\newcommand{\doublespacing}{\setlength{\baselineskip}%
                           {2.0 \defbaselineskip}}
\newcommand{\singlespacing}{\setlength{\baselineskip}{\defbaselineskip}}
\newcommand{\hsp}{\hspace{10 mm}}
\newcommand{\hs}{\hspace{5 mm}}
\newcommand{\hsm}{\hspace{2 mm}}
\newcommand{\vs}{\vspace{7 mm}}
\newcommand{\vsm}{\vspace{2 mm}}
% MATH ====================================================================
\def\cspace{configuration space}
%\def\RR{\mathbb{R}}
\def\TT{\mathbb{T}}
\def\R{\mathcal{R}}
\def\C{\mathcal{C}}
\def\L{\mathcal{L}}
% LIST OF NOTATIONS ====================================================================
\def\listofnotations{\input symbols.tex \clearpage}
\def\addnotation #1#2{#1 \dotfill \pageref{#2}}
\def\newnot#1{\label{#1}}

% ====================== Definitions CHAPTER 0 ======================
\def\N{\mathcal{N}}
% ====================== Definitions CHAPTER 1 ======================
\newtheorem{thm}{Theorem}[section]
\newtheorem{lemma}[thm]{Lemma}
\newtheorem{prop}[thm]{Proposition}
\newtheorem{cor}[thm]{Corollary}
\newtheorem{subsec}[thm]{}
\theoremstyle{definition}
\newtheorem{defn}[thm]{Definition}
\newtheorem{example}[thm]{Example}
\newtheorem{fact}[thm]{Fact}
\theoremstyle{definition}
\newtheorem*{org}{Organization}
\theoremstyle{remark}
\newtheorem{remark}[thm]{Remark}
\newtheorem{assume}[thm]{Assumptions}
\newtheorem{conj}[thm]{Conjecture}
\numberwithin{equation}{section} \numberwithin{figure}{section}
\def\sect{\setcounter{thm}{0}\section}
\newcommand{\eee}{\hfill$\Box$}
\newcommand{\col}{\operatorname{Col}}
\newcommand{\conv}{\operatorname{conv}}
\newcommand{\Edge}{\operatorname{Edge}}
\newcommand{\In}{\mathfrak{In}} \newcommand{\Int}{\operatorname{Int}}
\newcommand{\Out}{\mathfrak{Out}} \newcommand{\pt}{\operatorname{pt}}
\newcommand{\rank}{\operatorname{Rank}}
\newcommand{\spn}{\operatorname{Span}}
\newcommand{\Vertex}{\operatorname{Vertex}}
\newtheorem*{exm}{Example}
\def\cspace{configuration space}
\def\ggraph{$G$-graph}
\def\smech{arachnoid mechanism}
\def\Re{\mathbb{R}}
\def\Ze{\mathbb{Z}}
\newcommand{\Sb}[1]{{\mathbf{S}^{#1}}}
\def\D{\mathcal{D}}
\def\A{\mathcal{A}}
\def\M{\mathcal{M}}
\def\H{\overline{\mathcal{D}}}
\def\PP{\mathcal{P}}
\def\U{\mathcal{U}}
\newcommand{\T}[1]{{\mathbf{T}^{#1}}}
\def\V{\mathcal{V}}
\def\W{\mathcal{W}}
\def\Wc{\mathcal{W}}
\def\Vc{\mathcal{V}}
\def\Xc{\mathcal{X}}
\def\Lc{\mathcal{L}}
\newcommand{\rmtt}[1]{\textit{\rm{#1}}}
\newcommand{\up}[1]{\sp{({#1})}}
\newcommand{\Lj}[1]{L\up{#1}}
\newcommand{\lj}[1]{\ell\up{#1}}
\newcommand{\nj}[1]{n\up{#1}}
\newcommand{\Vj}[1]{V\up{#1}}
\newcommand{\vj}[1]{\bv\up{#1}}
\newcommand{\Wi}[1]{W\up{#1}}
\newcommand{\xj}[1]{\bx\up{#1}}
\newcommand{\dsc}[1]{\D({#1})}
\newcommand{\hle}[1]{\overline{\dsc{#1}^{c}}}
\newcommand{\anl}[1]{\A({#1})}
\newcommand{\bu}{\mathbf{u}}
\newcommand{\bv}{\mathbf{v}}
\newcommand{\bx}{\mathbf{x}}
\newcommand{\bw}{\mathbf{w}}
\newcommand{\by}{\mathbf{y}}
% ====================== Definitions CHAPTER 2 ======================
\newtheorem{assumption}{{\bf Assumption.}}
\newtheorem{Proposition}{{\bf Proposition.}}
\newtheorem{Lemma}{{\bf Lemma}}
\newtheorem{Theorem}{{\bf Theorem}}
\newtheorem{Corollary}{{\bf Corollary}}
\newtheorem{Remark}{{\bf Remark}}
\newtheorem{Algorithm}{{\bf Algorithm}}
\newcommand{\reals}{{\Bbb R}}
\def\V{\mathcal{V}}
\def\Vc{\mathcal{V}}
\newcommand{\dif}[2]{\partial_{{#2}}\,{#1}}
\newcommand{\diff}[2]{\partial_{{#2}}\,{#1}}
\newcommand{\vp}[1]{(\bv\up{#1})^{\perp}}
\newcommand{\Ind}{\operatorname{Ind}}
\newcommand{\wspace}{work space}
\newcommand{\ap}[1]{(\ba\up{1,{#1}})^{\perp}}
\newcommand{\hj}[1]{h\up{#1}}
\newcommand{\vpp}[1]{(\vj{{#1}}_{1})^{\perp}\cdot\vj{{#1}}_{2}}
\newcommand{\vw}{\vec{\bw}}
% ====================== Definitions CHAPTER 3 ======================
\newenvironment{myeq}[1][]
{\stepcounter{thm}\begin{equation}\tag{\thethm}{#1}}
{\end{equation}}
%
\newcommand{\mydiag}[2][]{\myeq[#1]{\xymatrix{#2}}}
\newcommand{\mydiagram}[2][]
{\stepcounter{thm}\begin{equation}
     \tag{\thethm}{#1}\xymatrix{#2}\end{equation}}
%use:  \mydiagram[\label{``label''}]{``xy-pic syntax''}
%
\newenvironment{mysubsection}[2][]
{\begin{subsec}\begin{upshape}\begin{bfseries}{#2.}
\end{bfseries}{#1}}
{\end{upshape}\end{subsec}}
%
\newenvironment{mysubsect}[2][]
{\begin{subsec}\begin{upshape}\begin{bfseries}{#2\vsm.}
\end{bfseries}{#1}}
{\end{upshape}\end{subsec}}
%
\newcommand{\w}[2][ ]{\ \ensuremath{#2}{#1}\ }

\newcommand{\dd}{\operatorname{d}}
\newcommand{\df}{\operatorname{df}}
\newcommand{\dF}{\operatorname{dF}}
\newcommand{\dhh}{\operatorname{dh}}
\newcommand{\Id}{\operatorname{Id}}
\newcommand{\Image}{\operatorname{Im}}
\newcommand{\Line}{\operatorname{Line}}
\newcommand{\NN}{{\mathbb N}}
\newcommand{\RR}[1]{{\mathbb R}^{#1}}
\newcommand{\ZZ}{{\mathbb Z}}
\newcommand{\hC}{\hat{\C}}
\newcommand{\hd}{\hat{\bd}}
\newcommand{\hF}{\hat{F}}
\newcommand{\hI}{\hat{\Ic}}
\newcommand{\hV}{\hat{\Vc}}
\newcommand{\hi}{\hat{\text{\textit{\i}}}}
\newcommand{\Ck}{\C(k)}
\newcommand{\hP}{\hat{\Phi}}
\newcommand{\Ec}{\mathcal{E}}
\newcommand{\Gc}{\mathcal{G}}
\newcommand{\Ic}{\mathcal{I}}
\def\Lc{\mathcal{L}}
\newcommand{\Pc}{\mathcal{P}}
\newcommand{\Sc}{\mathcal{S}}
\def\Xc{\mathcal{X}}
\newcommand{\aj}[1]{\ba\up{#1}}
\newcommand{\bj}[2]{\vec{\bb}\sp{({#1},{#2})}}
\newcommand{\cj}[1]{\bc\up{#1}}
\newcommand{\gj}[1]{g\up{#1}}
\newcommand{\mj}[1]{m\up{#1}}
\newcommand{\ppj}[1]{\bp\up{#1}}
\newcommand{\uj}[2]{\vec{\mathbf u}\sp{({#1})}\sb{#2}}
\newcommand{\oj}[1]{\bo_{#1}}
\newcommand{\wj}[2]{\vec{\mathbf w}\sp{({#1})}\sb{#2}}
\newcommand{\yj}[1]{\by\up{#1}}
\newcommand{\ba}{\mathbf{a}}
\newcommand{\bb}{\mathbf{b}}
\newcommand{\bc}{\mathbf{c}}
\newcommand{\bd}{\mathbf{d}}
\newcommand{\be}{\mathbf{e}}
\newcommand{\bX}{\mathbf{X}}
\newcommand{\bp}{\mathbf{p}}
\newcommand{\bz}{\mathbf{z}}
\newcommand{\vy}[1]{\vec{\by}\up{#1}}
\newcommand{\bo}{\mathbf{\omega}}
\newcommand{\vz}{\mathbf{0}}
% ====================== Definitions CHAPTER 4 ======================
\newcommand{\alignd}{\operatorname{aligned}}
\newcommand{\rest}{\operatorname{rest}}
\newcommand{\iin}{\operatorname{in}}
\newcommand{\out}{\operatorname{out}}
\newcommand{\RRR}{\mathbb{R}}
\newcommand{\Ac}{\mathcal{A}}
\newcommand{\Aj}[1]{\Ac\up{#1}}
\newcommand{\hAj}[1]{\hat{\Ac}\up{#1}}
\newcommand{\laj}[1]{\lambda\up{#1}}
\newcommand{\sj}[1]{\$\up{#1}}
\newcommand{\Tj}[1]{T\up{#1}}
\newcommand{\sjp}[1]{\$^{({#1})\perp}}
\newtheorem{examples}[thm]{Examples}

% ====================== start document ======================
\begin{document}
\pagestyle{empty} % No headers or page numbers
% ---------------------  Title page --------------------------------
\begin{center}
\vspace*{2.0cm} \Huge {\bf \Huge The Configuration Space of
Parallel Mechanisms}


\vspace*{6.0cm} \Large
Nir Shvalb\\
\end{center}
\newpage
% ---------------------  inner title page --------------------------------
\begin{center}
\Huge {\bf \Huge The Configuration Space of Parallel Mechanisms}

\vspace*{1.5cm} \Large
 Research thesis

\vspace*{1.5cm} \Large
Submitted in Partial Fulfilment of the \\
Requirement for the degree of \\
Doctor of Philosophy \\
\vspace*{3.0cm} \Large
Nir Shvalb\\



\vspace*{2.0cm} \Large
Submitted to Senate of\\
the Technion - Israel Institute of Technology\\

\end{center}
\begin{center} \vspace*{.5cm} Haifa, Israel, Nov 2006, Kislev 5776
\end{center}
\newpage

%%%%%%%%%%%%%%%%%%%%%%%%%%%%%%%%%%%%%%%%%%%%%%%%%%%%%%%%%%%%%
% PRELIMINARY PAGES

\pagestyle{plain} % No headers, just page numbers
\pagenumbering{roman} % Roman numerals

%
\prefacesection{Acknowledgments}
The Research Thesis Was Done Under The Supervision of Prof. Moshe
Shoham in the Department of mechanical Engineering - The Technion
and Dr. David Blanc from the Department of mathematics - Haifa
University.

%\setcounter{page}{1}
\tableofcontents
%\listoftables
% --------------- list of notations ---------------
\prefacesection{List of Notations}
\listofnotations
% -------------------list of figures --------------
\listoffigures
\addcontentsline{toc}{chapter}{List of figures}
\newpage


\prefacesection{Acknowledgments}
is the longest word in this dissertation. Well, Under such a
formal title. I would like to express my love and gratitude to
some of the people that convinced me that torturing myself is good , so here goes\\
\textbf{Mom and Dad} without your love (and evidently a lucky day
for me 34 years ago) I wouldn't be at all, which is not a good
start, thanks for that and for your endless love and care. I love you both very much.\\
\textbf{Prof. David Blanc} - My advisor and my math teacher thank
you, without your advices and ruthless dedication I would probably
be still wondering with a dumb smile somewhere between Homotopy and Morse Theory .\\
\textbf{Oded Salomon} thanks for your friendship, thanks for
hearing my mathematic nonsense for three years, and for all the
wonderfull advices you gave me.\\ \textbf{Micha, Rani} thanks for
making me laugh.\\ \textbf{Naama, Gabriela, Meital, Pnina, Noga,
Tzahi, Daniel,Yonatan} for being real friends. Lastly I would like
to thank \textbf{Prof. Moshe Shoham} - My advisor, thanks for your
guidance, support and encouragement throughout the research.
thanks for the freedom you gave me and for your faith in me.\\
$\heartsuit$ you all\\
Nir.


\prefacesection{Abstract}

We study the \cspace \ of some parallel mechanism that are in use.
starting with the \cspace s of arachnoid mechanisms which consist
of $k$\ branches each of which has an arbitrary number of links
and a fixed initial point, while all branches end at one common
end-point. We show that generically, the \cspace s of such
mechanisms are manifolds, and determine the conditions for the
exceptional cases. The \cspace\  of planar arachnoid mechanisms
having $k$ branches, each with two links is fully characterized
for both the non-singular and the singular cases.

Applying these results we study the motion planning problem for
such planar manipulators. A topological analysis is used to
understand the global structure of the configuration space so that
motion planning problem can be solved exactly. The worst-case
complexity of our algorithm is $O(k^3N^3)$, where $N$ is the
maximum number of links in a leg. Examples illustrating our method
are given.

Next we study the  \cspace\ $\C$ of a parallel polygonal
mechanism, having a moving polygonal platform, and a flexible leg
(consisting of concatenated rods) attached to each vertex, with
the other end fixed in  \ $\RR{d}$. \ We give necessary conditions
for the existence of uncertainty singularities; and show that
generically $\C$ is a smooth manifold. In the planar case, we
construct an explicit Morse function on $\C$, and show how
geometric information about the mechanism can be used to identify
the critical points. Finally we describe how topological
singularities give rise to instantaneous kinematic and explicit
example is provided.




%===================================================
\prefacesection{Methods and preliminaries}
We now introduce some definitions, theorems and methods which are
in extensive use in this dissertation.\\
\emph{\textbf{Smooth Manifolds:}} An $n$-dimensional
\emph{manifold} is a \emph{topological space} which "near" each of
its points is locally like $\Re^n$ i.e. there is a neighborhood
which is topologically "the same" as an open unit ball in $\Re^n$.
Formally, two additional properties are needed to form a manifold,
being \emph{Hausdorff}
and \emph{Second Countable} (see for example \cite{C}).\\
A differential manifold $\M$ \ is a manifold for which overlapping
\emph{charts} "relate smoothly" to each other, where a chart
$\alpha:U\to\Re^n$ is a continuous injection from an open subset
$U$\ of $\M$\ to $\Re^n$.\\
A manifold is \emph{connected} if, roughly speaking, it is all in one piece.\\
A \emph{submanifold} is a subset of a manifold which is itself a
manifold, but has smaller dimension. For example, the equator of a sphere is a submanifold.\\
If $\M$\ is an orientable surface, its genus $g(\M)$ is the number
of ``handles'' it possesses. Furthermore, any orientable surface
is a sphere, or the \emph{connected sum} of $n$\ tori. We say the
sphere has genus $0$, and that the connected sum of $n$\ tori has
genus $n$. Also, $g(\M)=1-\chi(\M)/2$\ where $\chi(\M)$\ is the
Euler characteristic of $\M$.

\emph{\textbf{Smooth Mappings and Singularity}}: The following
enabled us to prove smoothness of generic \smech , and used also
in finding necessary conditions for uncertainty singularities in
polygonal mechanisms:\\
Let $f:\N^n\to\M^m$\ be a smooth map on smooth manifolds. A
critical point of $f$\ is a point $x\in\N$\ such that the
differential $df$\ considered as a linear transformation of real
vector spaces has rank $<m$. A critical value of $f$\ is the image
$f(x)$\ of a critical point. A regular value of $f$\ is a point
which is not the image of any critical point.
\begin{thm}(Regular value theorem) Let $f:\N^n\to\M^m$\ be a smooth mapping. if $y$\ is a regular
value of $f$\ then $f^{-1}(y)$\ is a smooth $n-m$ manifold.
\end{thm}


The notion of objects \emph{intersecting transversally} is
fundamental to singularity theory. The simplest situation to look
at is two subspaces of a vector space $V$: We say that they
intersect transversally when their vector sum is $V$, formally:

\begin{defn}
Let $f: \N \to \M$\ be a smooth mapping, and let $Q$ be a smooth
submanifold of $\M$. $f$\ and $Q$\ are said to intersect
transversally (denoted $f\pitchfork Q$) iff for all $x\in\N$\ with
$y=f(x)\in Q$\ we have $d_xf(T_x\N)+T_yQ=T_y\M$.
\end{defn}

where $T_x\M$\ is the space spanned by the set of all vectors
tangent to $\M$\ at point $x\in\M$, and $d_xf$\ the jacobian of
the map $f$\ calculated at point $x$. The next proposition (see
also \cite[pp31]{AGV}) provides us a tool for examining
uncertainty singularities in the special case where the regular
value theorem fail.
\begin{prop}
Let $f:\N\to \M$\ be a smooth mapping, and let $Q$\ be a smooth
submanifold of $\M$ with $f\pitchfork Q$: then $f^{-1}(Q)$\ is a
smooth submanifold of $\N$, having the same codimension as $Q$ or
is empty.
\end{prop}



%===================================================

\prefacesection{Introduction}
Generally speaking a \emph{configuration} is a set of independent
parameters uniquely specifying the position of each of its
mechanical components relative to a fixed frame of reference. A
\emph{\cspace} of a given mechanism is therefore the totality of
all its admissible positions in the Euclidean space. For example,
a simple $n$-linked branch mechanism with revolute joints have the
$n$-torus \ $\TT^n$ \ as its configuration space.

Obviously knowing the configuration space is a powerful tool for
motion planning problems, singularities characterization and
global understanding of a system. Main research have been set on
the \cspace s of a type of mechanism called \emph{polygonal
linkage}, which is simply a concatenation of links and hinged
joints forming a closed chain in \ $\RR{2}$ and \ $\RR{3}$. A
substantial amount of mathematical literature on polygonal
linkage's \cspace\ has has been written using various methods like
computation of Euler characteristics, homology groups and
handle-body surgery;

Worth special note is the work of J. C. Hausmann, and A. Knutson
\cite{HK} who determine the cohomology rings for these spaces and
the co-work of R.J. Milgram and J. Trinkle \cite{MT1} who
presented full surgery analysis for the same kind of mechanism.
The results of \cite{KM}, provide a good review of previous work
and some interesting results on the structure of these \cspace s.

However, to the best of the author's knowledge \cspace s of more
common parallel mechanisms, i.e. mechanisms with multiple branches
having a \emph{tree} topology or a spider like mechanisms where
all branches are attached in one rigid platform (\emph{polygonal
mechanism}) or one point (\emph{Arachnoid mechanism}), have never
been dealt with before. Such mechanisms are of great importance in
the "real life" due to their accuracy and stiffness properties.

In this dissertation we deal with mechanisms which are constructed
by rigid rods and spherical (rotational)  joints, called
\emph{graph mechanisms}, which mathematically speaking are simply
graphs \ $(G(v,e),\mathcal{L})$ \ having s ets of vertices and
edges such that \ $e(G)=v(g)\times v(g)$ \ and a set of fixed
lengths $\mathcal{L}$.

The \cspace \ is sometimes non-accurately defined in literature.
Here we define the \cspace \ as the quotient \
$\mathcal{C}=\mathcal{R}(G)/\Lambda$ \ where
%
$$
\R(G)=\{\textbf{x}_i \in \RR{v(G)}|i\in v(G),
\|\textbf{x}_i-\textbf{x}_j\|=\L(i,j) \}
$$
%
and $\Lambda$ \ is the group of isometries of translations and
rotations in \ $\RR{2}$, or in \ $\RR{3}$. Thus \ $\Lambda$ \
"fixes" the the mechanism to a fixed frame. The difference between
these two definitions for the \cspace \ can be realized in the
following: A subgraph $H$ \ of graph \ $G$ \ is a graph such that
\ $v(H) \ \subset \ v(G)$ \ and \ $e(H) \ \subset e(G)$, which
leads to the definition:
\begin{defn}
A \emph{sub-mechanism} \ $(H,\mathcal{L}|_{H})\subset (G,\L)$ \ is
a subgraph \ $H$ \ of \ $G$\ together with the restricted length
subset \ $\L|_{H}$, and we denote such a sub-mechanism simply by \
$H$. \ A \emph{branch} is a sub-mechanism such that all vertices
in $H$\ have valence $1$\ or $2$.
\end{defn}
In light of these definitions one can define a feasible
configuration of a parallel mechanism on condition that all of its
branches "agree" upon one (or more) branch configuration. One
could claim that for  arachnoid mechanisms the \cspace \ is simply
the fibered product of the \cspace \ of its branches, but then he
would be wrong. Alternatively we know that:
\begin{thm}
Given a mechanism \ $(G,\L)$ \ with a set \ $H_{1},H_{2},H \subset
G$ \ of its sub-mechanisms such that \ $H_{1}\cap H_{2}=H$ \ then
\ $\R(\L)$ \ is the pullback of the the natural projections:
%
$$
\R(H_{1}) \xrightarrow{\pi_{H_{1}}} \R(H) \xleftarrow{\pi_{H_{2}}}
\R(H_{2}).
$$
\end{thm}
Note that this does not extend to the \cspace \ $\C$ \ since the
induced projections \ $\{\R(H_{i})/\Lambda \rightarrow
\R(H)/\Lambda\}$ \ are determined up to isometry \ $\Lambda$ \ and
thus do not determine a unique point (A proof for this theorem and
a detailed discussion can be found in the Appendix \ref{appendix}).\\

We first analyze a simple class of mechanism called
\emph{arachnoid} which consists of multiple branches each having
arbitrary number of links a fixed initial point, and all end at a
common point (this type of mechanism resembles some parallel
robots which are in practical use \cite{LLL}: parallel
manipulators, walking robots, and dexterous manipulation). It is
shown (see Chapter \ref{chap1}) that generically, the \cspace s of
such mechanisms are manifolds, and the conditions for the
exceptional cases are then determined. We analyze the \cspace\ of
planar arachnoid mechanisms having $k$ branches, each with two
links. Such mechanisms have a $2$ dimensional \cspace \ which
enables a full classification. The non-manifold cases where
singularities emerges are calculated.

As mentioned above, knowing the \cspace \ often enables problems
solving, like the existence of continuous motion planners (Farber
cf. \cite{F}) or the motion planning problem (Note that this kind
of approach completely discards the nature of the mechanism in
issue). Due to the computational complexity and difficulty of
implementing general exact motion planning algorithms, such as
Canny's \cite{Can88}, today sample-base algorithms, such as
Kavraki's \cite{KSLO96} dominate motion planning research.
However, there are important classes of problems for which these
algorithms do not perform well. These arise in systems whose
\cspace \ cannot effectively be represented as a set of parameters
with simple bounds, but rather is most naturally represented as a
variety of co-dimension one or greater embedded in a
higher-dimensional ambient space \cite{YLK01}. Examples of these
systems include manipulators with one or multiple closed loops,
whose configuration space is defined by loop closure constraints.
The \emph{Random Loop Generator} method \cite{Cortes02,CS03}
improves the sampling techniques through estimating the regions of
sampling parameters. However, its efficiency relies on the
accuracy of the estimation, which is often case-variant. Moreover,
it ignores the global structure of \cspace \, and may fail to
sample globally important regions. For which specialized exact
methods have not been developed. This provides motivation to try
to develop effective exact algorithms. For most mechanisms \cspace
\ is of dimension greater than two which makes them impossible to
be fully characterized. Yet, global properties like connectivity
and smoothness are attainable. Recent advances (especially
connectivity analysis methods) in the understanding of the global
structure of \cspace \ of single-loop closed chains
\cite{MT2,SSB05} allow us (Chapter \ref{chap2} here) to develop a
polynomial-time exact algorithm for arachnoid manipulators (which
actually can be extended to any planar mechanism with an arbitrary
tree topology).

Next we focus attention on parallel robots (in any dimension)
comprised of arbitrary number of branches each having arbitrary
number of links concatenated by rotational joints and attached to
a moving platform, which we shall refer to as \emph{polygonal
mechanisms}, these are the mathematical parallel of most parallel
mechanisms which are increasingly in use. Still their \cspace \
had never been analyzed. We examine closely the case of parallel
robots singularity called uncertainty singularity which is simply
a topological singularity of the configuration space:

Kinematic singularities of parallel mechanisms have been studied
extensively in the literature (cf.\ \cite{Hu2}, and the example of
open kinematic chains in \cite{Hu1,WW}). On the other hand, the
\cspace\ of such a mechanism may have topological singularities.
Our goal here is to try to relate these two kinds of singularity,
for that certain class of mechanisms.

Choosing appropriate local coordinates for a given mechanism, we
try to endow its \cspace \ with the structure of a differentiable
manifold. The points where this cannot be done constitute the
\emph{topological singularities} of the mechanism. We shall be
concerned only with \emph{uncertainty singularities} (where  the
\emph{instantaneous mobility} is greater than the \emph{full cycle
mobility} \ -- \ cf.\ \cite{Hu2}); In such a position the
mechanism as a whole gains an additional degree of freedom. Such
points may sometimes separate \cspace \ into distinct regions,
allowing transition between different operating modes. (An
extensive examination of such remarkable transitions in a special
case (a \emph{constraint} singularity for $3$-URU parallel
mechanism with three degrees of freedom) is given in \cite{ZBC}).
In Chapter \ref{chap3} we give a necessary geometric condition for
uncertainty singularities in said mechanisms, and explain how such
topological singularities give rise to instantaneous kinematic
singularities in Chapter \ref{chap4}. Finally, the occurrence of
an uncertainty singularity is illustrated visually in an explicit
example.

% Change page numbering back to Arabic numerals
%\pagenumbering{arabic}

%\setcounter{page}{1}

% Pages which are generated automatically
%\setcounter{page}{6} % Set this counter to get correct page numbers



%%%%%%%%%%%%%%%%%%%%%%%%%%%%%%%%%%%%%%%%%%%%%%%%%%%%%%%%%%%%%
 % heading, ack, abstract
\pagenumbering{arabic}
\chapter{Arachnoid mechanisms}
\label{chap1}

\textsf{A paper published in \textsl{Forum Mathematica} Vol 17/6
(Nov '05),
1033-1042.}\\
\textsf{Co-authors: David Blanc and Moshe Shoham.}\\

The \cspace s of arachnoid mechanisms are analyzed in this paper.
These mechanisms consist of $k$\ branches each of which has an
arbitrary number of links and a fixed initial point, while all
branches end at one common end-point. It is shown that
generically, the \cspace s of such mechanisms are manifolds, and
the conditions for the exceptional cases are determined. The
\cspace\  of planar arachnoid mechanisms having $k$ branches, each
with two links is analyzed for both the non-singular and the
singular cases.


\sect{Introduction}

Mechanisms and robots consist of links and joints, the actuation
of which causes them to move. The type of a mechanism is described
by an abstract graph which corresponds to its links and joints,
and a specific embedding of this graph in the plane or in
$3$-space is called a \emph{configuration} of the mechanism. The
collection of all such embeddings forms a topological space,
called the \emph{\cspace} of the mechanism. For example, the
\cspace\ of a planar mechanism with revolute joints consisting of
$n$ rods arranged serially is the $n$-torus.

In recent years, there has been interest among mathematicians in
the study of such spaces, which are of importance in motion
planning \ -- \ that is, moving a mechanism from one given
position to another, taking into account various constraints (see
for example \cite{MT2}). The topological properties of the
\cspace\ provide insight into practical questions in planning such
motions (see \cite{F}) and analysis of some mechanical
singularities (see \cite{NM}, and \cite{ZFB}).

The main focus had been set on the \cspace s of a type of
mechanism called \emph{polygonal linkage}, which is simply a
concatenation of links and hinged joints forming a closed chain. A
substantial amount of mathematical literature on polygonal
linkage's \cspace\ has accumulated: Kamiyama , Tezuka and Toma
studied Euler characteristics in \cite{K}, and homology groups in
\cite {KT,KTT}; Trinkle and Milgram constructed a handle-body
surgery in \cite{MT1}; and in \cite{Ho}, Holcomb studied a special
parallel graph mechanism called multi-polygonal linkages, which
are three free branches identified at their initial and terminal
vertices.

In this paper we analyze a type of mechanism called
\emph{arachnoid} which, to the best of our knowledge, has never
been dealt with in the literature. This kind of mechanism consists
of multiple branches each of which has an arbitrary number of
links and a fixed initial point, while all branches end at a
common end-point (this type of mechanism resembles some parallel
robots which are in practical use). It is shown that generically,
the \cspace s of such mechanisms are manifolds, and the conditions
for the exceptional cases are then determined. The \cspace\  of
planar arachnoid mechanisms having $k$ branches, each with two
links is fully analyzed, while for the non-manifold cases we
analyze the singular configurations.

We now introduce some notation and terminology to describe such
mechanism types, and in particular the \smech s which are the
subject of this note:

\begin{defn}
%
For a mechanism $M$ in \ $\Re^{d}$, \ a \emph{branch} \ $(L,\bx)$
\ of multiplicity $n$ is a sequence \
$L=(\ell_{1},\dotsc,\ell_{n})$ \ of $n$ positive numbers, together
with a point \ $\bx\in\Re^{d}$. \ We think of $L$ as the lengths
of $n$ concatenated rods, having revolute (i.e., rotational)
joints at the meeting point of every two consecutive rods, and at
the fixed initial point $\bx$, called the \emph{base point} of the
branch.

A \emph{branch configuration} \ $V=(\bv_{1},\dotsc,\bv_{n})$ \ for
a branch \ $(L,\bx)$ \ then consists of $n$ vectors in \ $\Re^{d}$
\ with the given norms \ $\|\bv_{i}\|=\ell_{i}$ \ ($i=1,\dotsc
n$).
\end{defn}
\newnot{1-3}


Since the \cspace\ of a branch \ $(L,\bx)$ \ \newnot{1-1} is
independent of the order of the set \ $\ell_1,..,\ell_n$ \ (up to
homeomorphism), we can (and shall) assume \ $\ell_1,..,\ell_n$ \
to be in descending order.

\begin{defn}
%
An \emph{\smech} consists of $k$ branches \ $$
(\Lc,\Xc)=((\Lj{1},\xj{1}),\dotsc (\Lj{k},\xj{k})) $$
\newnot{1-2} with multiplicities \ $\nj{1},...,\nj{k}$. \ We think
of this as a linkage of branches connected by a single revolute
joint at their common end point (whose location is not fixed).
\newnot{1-4}

An \smech\ configuration for \ $(\Lc,\Xc)$ \ thus consists of a
set $$ \Vc=(\Vj{1},..,\Vj{k}) $$ of branch confiurations for $\Lc$
having a common \emph{end point}  \
$\by=\xj{i}+\sum_{j=1}^{\nj{i}}~\vj{i}_{j}$ \ $(i=1,..,k)$. \
  \end{defn}

\begin{figure}[h]
\begin{center}
\epsfysize=5cm %\epsfxsize=4.4cm
\leavevmode \epsffile{fig_1/mechanism.eps} \caption{An \smech\
with $k=3$, \ multiplicities $2$.} \label{fig:mechanism}
\end{center}
\end{figure}

\begin{defn}
\label{def:aligned}

 For an \smech \ $(\Lc,\Xc)$:

\begin{enumerate}
\item A branch configuration $\V=(\bv_1,\dotsc,\bv_n)$ \ is
\emph{aligned} (with direction $\bw$) if each vector \
$\bv_1,\dotsc,\bv_n$ \ is a scalar multiple of $\bw$. \item A
configuration \ $\Vc=(\Vj{1},..,\Vj{k})$ \ of \ $(\Lc,\Xc)$ \ is
called  a $t$-\emph{node} if it has $t$ aligned  branch
configurations with directions \ $\bw_{i_{1}},\dots,\bw_{i_{t}}$ \
respectively, which are linearly dependent; otherwise $\Vc$ is
called \emph{generic}.
\end{enumerate}
\end{defn}

\newnot{1-6}

\begin{defn}
The collection \ $\C=\C(\Lc,\Xc)$ \ of all configurations \ $\Vc$
\ for \ $(\Lc,\Xc)$ \ is called its \emph{\cspace}. It is
topologized as a subspace of the appropriate Euclidean space. The
space of all such common endpoints $\by$ will be called the
\emph{work space} \ $\Wc=\Wc(\Lc,\Xc)$ \  for \ $(\Lc,\Xc)$. \ The
\emph{work map} \ $\Phi:\C\to\W$ \ assigns to each configuration
$\Vc$ its common endpoint $\by$.
\end{defn}

\newnot{1-5}

\begin{org} \label{sorg} In section \ref{cmain} we show that the \cspace\ of a generic
 \smech\ \ $(\Lc,\Xc)$ \ is a manifold. In section \ref{cplan} we study planar
  \smech s for which each branch has $2$ joints, and give an explicit formula
  for the toplogical type of \ $\C=\C(\Lc,Xc)$ \ in the generic case. Finally,
   in section \ref{csingc}, we analyze the singularities of $\C$ for such planar
    \smech s in the non-manifold case.
\end{org}
%
%c2   Generic \smech s in $\Re^{d}$}
%
\sect{Generic \smech s in $\Re^{d}$} \label{cmain}

First, we show that, generically, the \cspace\ of an \smech\ is a
manifold:

\begin{thm}\label{thm:main}\stepcounter{subsection}
%
Let \ $(\Lc,\Xc)$ \ be an \smech\ in \ $\Re^{d}$ \ with $k$
branches of multiplicities \ $\nj{1},...,\nj{k}$, \ respectively.
If all configurations of \ $(\Lc,\Xc)$ \ are generic, then its
\cspace\ $\C$ is a smooth closed orientable manifold of dimension
\ $d(N-k+1)-N$, \ where \ $N=\sum_{i=1}^{k}\nj{i}$.
%
\end{thm}

\begin{proof}
%
We can identify \ $\C=\C(\Lc,\Xc)$ \ as the pre-image of a certain
function \ $G:\Re^{d(N-k+1)}\to\Re^{N}$, \ where $G$ is defined as
follows:

For each \ $n\geq 1$ \ let \ $g_{n}:(\Re^{d})^{n}\to\Re^{n-1}$ \
be defined \ $$ g_{n}(\bv_{1},\dotsc,\bv_{n}):=
(|\bv_{2}-\bv_{1}|^{2},\dotsc,|\bv_{n}-\bv_{n-1}|^{2}),
$$
where \ $|\bu|:=(\sum_{i=1}^{d}t_{i}^{2})^{1/2}$ \ is the length
of a vector \ $\bu=(t_{1},\dotsc,t_{d})\in\Re^{d}$. \ Now for each
branch \ $\Lj{i}=(\lj{i}_1,..,\lj{i}_{\nj{i}})$ \ of $\Lc$, \ let
\ $(\vj{i}_{1},\dotsc,\vj{i}_{\nj{i}})$ \ be position vectors of
the ends of the \ $\nj{i}$ \ links of a branch configuration,
where \ $\vj{i}_{0}=\xj{i}$ \ (the given base point for this
branch). Since in an \smech\ all branches end at the same point \
$\bu\in\Re^{d}$, \ we have \ $\vj{i}_{\nj{i}}=\bu$ \ for all \
$1\leq i\leq k$. \ Thus we have \ $N-k+1$ \ different vectors \
$\{\vj{i}_{1},\dotsc,\vj{i}_{\nj{i}}\}_{i=1}^{k}$, \ and we define
$$ G(\vj{1}_{1},\dotsc,\vj{k}_{\nj{k}}):=
(g_{n_{1}}(\vj{1}_{0},\dotsc,\vj{1}_{n_{1}}),
g_{n_{2}}(\vj{2}_{0},\dotsc,\vj{2}_{n_{2}}),\dotsc
g_{n_{k}}(\vj{k}_{0},\dotsc,\vj{k}_{n_{k}})),
$$
so that \ $\C=G^{-1}(\vec{\ell})$ \ for \
$\vec{\ell}:=(\lj{1}_{1})^{2},..,(\lj{1}_{\nj{1}})^{2},\dotsc
(\lj{k}_{1})^{2},..,(\lj{k}_{\nj{k}})^{2})$. \ By the Regular
Value Theorem (see \cite[I, Thm.~3.2]{Hi}), \ $\C$ will be a
smooth manifold if \ $\vec{\ell}$ \ is a regular value of $G$ \ --
\ that is, \ $dG$ \ has maximal rank.

Note that (after applying elementary row and column operations), \
$dG$ \ has the following \ $N \times dN$ \ Jacobian matrix:

$$
dG=2\left(
\begin{array}{ccccc}
    A^{(1)} &B^{(1)}& & 0&\\
    A^{(2)} & &B^{(2)}& &\\
     \vdots &  & & \ddots & \\
      A^{(k)} &0 &  & & B^{(k)} \vspace{2mm}\\ \end{array}
\right)
$$

\noindent where each \ $(\nj{i}\times d)$-block \ $A^{(i)}$ \ is:

$$
A^{(i)}=\left(
\begin{array}{c}
      \vj{i}_{\nj{i}}-\vj{i}_{\nj{i}-1}\\
       0    \\
     \vdots \\
        0   \\
\end{array}
\right)
$$

\noindent and \ $B^{(i)}$ \ is:

$$
\left(
\begin{array}{cccccc}
\vj{i}_{\nj{i}-1}-\vj{i}_{\nj{i}} &  0 & 0 &\dotsc & 0 & 0 \\
\vj{i}_{\nj{i}-1}-\vj{i}_{\nj{i}-2} &
\vj{i}_{\nj{i}-2}-\vj{i}_{\nj{i}-1} &
0 &\dotsc & 0 & 0 \\ 0 &
\vj{i}_{\nj{i}-2}-\vj{i}_{\nj{i}-3} &
\vj{i}_{\nj{i}-3}-\vj{i}_{\nj{i}-2}
  & \dotsc & 0 & 0 \\
0  & 0 & 0 & \dotsc & 0 & 0  \\
\vdots  & \vdots & \vdots & \ddots & \vdots & \vdots  \\ 0  & 0 &
0 & \dotsc & \vj{i}_{2}-\vj{i}_{1}  & \vj{i}_{1}-\vj{i}_{2} \\ 0
& 0 & 0 & \dotsc & 0 & \vj{i}_{1}-\vj{i}_{0} \end{array} \right)
$$

\noindent an \ $\nj{i}\times d(\nj{i}-1)$ matrix which can be
thought of as the Jacobian matrix for a corresponding closed
$\nj{i}$-branch. Note that \ $\rank(B^{(i)})\leq\nj{i}$, \ and \
$B^{(i)}$ \ has less than full rank only when all vectors \
$\vj{i}_{\nj{j}-1}-\vj{i}_{\nj{j}}$ \ are collinear for \ $1\leq
j\leq \nj{i}$ \ -- \ so that the $i$-th branch is aligned. In this
case \ $\rank(B^{(i)})=\nj{i}-1$, \ and the non-zero row \
$\bw^{(i)}:=\vj{i}_{\nj{i}}-\vj{i}_{\nj{i}-1}$ \ of \ $A^{(i)}$ \
is precisely the direction of the branch.

We can thus divide the matrix \ $dG$ \ horizontally into two
blocks: \ $(A,B)$, where $$  A:=\left( \begin{array}{c}
  A^{(1)} \\
  \vdots \\
  A^{(k)} \\
\end{array}\right)
 \ \ \ \ \text{and} \ \ \ \ B:=\left(
\begin{array}{cccc}
    B^{(1)}& & 0&\\
     &B^{(2)}& &\\
      & & \ddots & \\
      0 &  & & B^{(k)} \\
\end{array}
\right),
$$
and the rank of \ $dG$ \ is then given by:
\begin{equation}\label{eone} \rank(A,B)=\rank(A)
+\rank(B)-\dim(\col(A) \cap\col(B) ). \end{equation}

Denote by $I$ the set of all indices $i$ for which the $i^{(th)}$
branch is aligned, so that \ $\rank(B)=N-|I|$. \ Thus if \
$I=\emptyset$, \ then \ $dG$ \ has maximal rank. If \ $|I|\neq0$,
\ let \ $A_I$ \ be the sub-matrix of $A$ consisting of the blocks
\ $A^{(i)}$ \ with \ $i \in I$. \ Its rows are therefore spanned
by the directions \ $\{\bw^{(i)}\}_{i\in I}$ \ of the aligned
branches. Observe that \ $\rank(A)-\rank(A_I)$ \ is the dimension
of the subspace of \ $\col(A)$ \ consisting of columns whose
entries vanish in the rows indexed by \ $i\in I$. \ Since the
blocks of $B$ indexed by \ $i\not\in I$ \ have full rank, we see
that $$ \dim(\col(A)\cap\col(B))~\leq~\rank(A)-\rank(A_I)
$$
\noindent (in fact, equality holds). By \eqref{eone}:
$$
\rank(dG)\geq\rank(A_I)+\rank(B)=N-|I|+\rank(A_I),
$$
which means that \ $dG$ \ has full rank unless \ $|I|>\rank(A_I)$.
\ The latter implies that the directions of the aligned branches
are linearly dependent \ -- \ that is, we have a $k$-node
configuration.

Note that \ $\C=G^{-1}(\vec{\ell})$ \ is in fact orientable when \
$dG$ \ has maximal rank, since in that case it induces an
isomorphism between the normal bundle $\nu$ to $\C$ in \
$\Re^{d(N-k+1)}$ \ at any point and the ``normal bundle'' to \
$\{\ell\}\hookrightarrow\Re^{N}$. \ Thus $\nu$ (the complement to
the tangent bundle \ $T\C$ \ in \ $\Re^{d(N-k+1)}$) \ is
orientable, so  \ $T\C$ \  is, too. Finally, $\C$ is compact since
it is a closed subset of the free configuration space, which is a
$N$-torus. \end{proof}
\newnot{1-8}
\begin{remark}
%
For an \smech \ in $\Re^{3}$, \ the matrix \ $dG$ \ will be
singular for a $2$-node configuration (two aligned branches along
one line); a $3$-node configuration (three aligned branches
contained in one plane); or a $4$-node configuration (four aligned
branches).
%
\end{remark}

%
%c3   Planar mechanisms
%
\sect{Planar mechanisms} \label{cplan}

>From now on we restrict attention to \smech s \ $(\Lc,\Xc)$ \ in the
>plane
(that is, \ $d=2$).

\subsection{The work space}
\label{sws}\stepcounter{thm}

In this case, each vector \ $v_j$ \ in a branch configuration $V$
(of multiplicity $n$) is determined by its argument $\theta_j$ \
(since \ $\|v_j\|=\ell_j$), \  and $V$ can thus be identified with
a point \ ($\theta_1,..,\theta_n$) in the $n$-torus \ $$
\T{n}=\underbrace{\Sb{1}\times \dotsc\times\Sb{1}}_{n}. $$
\newnot{1-9} Thus if \ $\nu=\nj{1}+\dotsc+\nj{k}$, \ then \
$\C(\Lc,\Xc)\subseteq\T{\nu}=\prod_{i=1}^{k}\T{\nj{i}}$.


Given such an \smech, we can describe the work space $\Wc$ as
follows: for any branch \ \ $L=(\ell_{1},\dotsc,\ell_{n})$, \ let
\ $\beta(L)_{\min}$ \ denote the minimal value of \
$|\sum_{j=1}^{n}\varepsilon_{j}\ell_{j}|$, \ where \
$\varepsilon_{j}=\pm 1$ \ for each \ $1\leq j\leq n$; \ and let \
$\beta(L)_{\max}:=\sum_{j=1}^{n}\ell_{j}$ \ (the maximal value). \
The work space \ $W=\Wc(L,\bx)$ \ for the branch $L$ with base
point $\bx$ \ (without any constraint on the end point) \ is then
an annulus bounded by circles of radius \ $\beta(L)_{\min}$ \ and
\ $\beta(L)_{\max}$, \ respectively.

If  \ $\Lc=(\Lj{1},...,\Lj{k})$, \ with multiplicities \
$\nj{1},...,\nj{k}$, \ and \ $\Xc=(\xj{1},...,\xj{k})$, \ the work
space for \ $(\Lc,\Xc)$ \ is \ $\Wc=\bigcap_{i=1}^k\Wi{i}$, \
where \ $\Wi{i}=W(\Lj{i},\xj{i})$. \ Each component of $\Wc$ is a
curvilinear polygon $P$ (not necessarily convex), whose edges \
$\Edge(P)$ \ are arcs of the annuli boundary circles \
$\partial\Wi{i}$, \ and whose vertices \ $\Vertex(P)$ \ are
intersection points of such arcs.

\subsection{The \cspace}
\label{scs}\stepcounter{thm}
\newnot{1-7}
The \cspace\ for any branch \ \ $L=(\ell_{1},\dotsc,\ell_{n})$ and
base-point $\bx$ \ is an $n$-torus \ $\T{n}$, \ with work map \
$\phi:\T{n}\to W$.

Note that the fiber \ $\phi^{-1}(z)$ \ over any point \ $z\in\Int
W$ \ is the \cspace\ for the \emph{closed} chain with links \
$\ell_{0},\ell_{1},\dotsc,\ell_{n}$, \ where \ $\ell_{0}:=z-\bx$.
\ This \cspace\ has been analyzed in \cite{HR}. On the other hand,
if $z$ is on the boundary of the annulus $W$, then \
$\phi^{-1}(z)$ \ is evidently discrete, and in fact consists of a
single point (unless it is on the inner circle, and \
$\beta(L)_{\min}$ \ can be written as \
$|\sum_{j=1}^{n}\varepsilon_{j}\ell_{j}|$ \ in more than one way).

If \ $\Lc=(\Lj{1},...,\Lj{k})$, \ with multiplicities \
$\nj{1},...,\nj{k}$, \ and \ $\Xc=(\xj{1},...,\xj{k})$, \ its
\cspace\ is the pullback $$
\C(\Lc,\Xc)~=~\{(\tau_{1},\dotsc,\tau_{k})\in\prod_{i=1}^{k}~\T{\nj{i}}~
|\ \phi_{1}(\tau_{1})=\dotsc=\phi_{k}(\tau_{k})\in\Wc\}.
$$

\subsection{Example}\label{exa:lens}\stepcounter{thm}
Consider an \smech\ consisting of three branches, each of
multiplicity $2$, as in Figure \ref{fig:mechanism}.

The workspace for each free branch is an annulus; let us assume
that the three annuli intersect in the shaded lens-shaped
component $P$ in Figure \ref{fig:annulus intersection}.

\begin{figure}[h]
\begin{center}
\epsfysize=5cm \leavevmode \epsffile{fig_1/annuli.eps}
\caption{Polygonal intersection} \label{fig:annulus intersection}
\end{center}
\end{figure}

Now consider an interior point \ $y\in\Int(P)$: \ in the
corresponding configurations in the fiber \ $\Phi^{-1}(y)$, \ each
of the three branches can be in one of two positions (branch
configurations), usually termed ``elbow up'' (\textbf{u}) and
``elbow down '' (\textbf{d}), so for each branch we have a copy of
\ $\Sb{0}=\{\mathbf{u},\mathbf{d}\}$. \ Thus the fiber consists of
eight points \ $\mathbf{uuu},\mathbf{uud},\dotsc,\mathbf{ddd}$, \
thought of as the product \ $\Sb{0}\times\Sb{0}\times\Sb{0}$. \
Thus \ $\Phi^{-1}(\Int(P)$ \ is simply an eight-fold cover of the
interior of the lens.

On the other hand, if $y$ is on the edge $\alpha$ of $P$, which is
in the outer boundary of the first annulus, the first branch is
completely extended, identifying its \textbf{u} and \textbf{d}
positions, thus collapsing the first \ $\Sb{0}$ \ to a single
point, and generally identifying the copy of $\alpha$ in $\C$
indexed by \ $\mathbf{u\ast\ast}$ \ with the copy indexed by
$\mathbf{d\ast\ast}$, \ for \
$\mathbf{\ast\ast}\in\{\mathbf{uu},\mathbf{ud},\mathbf{du},\mathbf{dd}\}$.

Similarly, the copy of $\beta$ in $\C$ indexed by \
$\mathbf{\ast\ast u}$ \ is identified with that indexed by
$\mathbf{\ast\ast d}$. \ Therefore, the fiber of \
$y\in\alpha\cap\beta$ \ consists of two points. Note that the
second \ $\Sb{0}$-factor never collapses, so $\C$ is of the form \
$\Sb{0}\times\M$ \ -- \ i.e., $\C$ has two components, each
isomorphic to $\M$.

To evaluate the Euler characteristic of $\M$, note that it is
obtained from four $2$-gons (the lens-shaped intersection in $\W$)
by identifying their $8$ edges pairwise (as explained above),
identifying the ``top'' vertex in all the $2$-gons to a single
point, and similarly for the ``bottom'' vertex. Thus \
$\chi(\M):=V-E+F=4-4+2=2$, \ so $\M$ is a $2$-sphere, and \
$\C\cong\Sb{2}\sqcup\Sb{2}$.

\subsection{Invariants of annulus arrangements} \label{sica}\stepcounter{thm}

An annulus (i.e., pair of concentric circles) in the plane is
determined by \ $(\bx,\beta_{\min},\beta_{\max})$, \ where \
$\bx\in\Re^{2}$ \ is the center and \
$0<\beta_{\min}<\beta_{\max}$ \ are the radii. Consider a system
$$ \langle(\xj{1},\beta(1)_{\min},\beta(1)_{\max});
\dotsc;(\xj{k},\beta(k)_{\min},\beta(k)_{\max})\rangle
$$
of $k$ such pairs (with distinct centers), and let $\W$ denote the
intersection of all the corresponding anulli; this may have
several connected components \ $V_{1},\dotsc,V_{t}$. \ What we
have in mind, of course, is the collection of boundary circles for
the work space of branches of an \smech.

The boundary \ $\partial V$ \ of each component $V$ of $\W$ is a
curvilinear planar polygon, not necessarily connected; \ let \
$\alpha:=\alpha(V)$ \ denote the number of components of \
$\partial V$ \ contained in the interior of its convex hull \
$\conv(V)$. \ We set \ $$ \gamma:=\gamma(V)=\begin{cases}
1 & \text{if \ $\conv(V)$ \ is a disc}\\ 0 & \text{otherwise.}\end{cases} $$ % $V$ may
be wholely contained in the interior of some of the annuli; denote the number
of such annuli by \ $\beta:=\beta(V)$ \ ($0\leq \beta(V)\leq k$), \ and let the
\emph{$c$-invariant} of $V$ be \ $c(V):=2^{\beta}$. \ Finally, the \emph{$g$-invariant} of $V$ is:
$$
g(V):=1-2^{k-\beta-3}(|\Vertex(V)|-2|\Edge(V)|+4+2\gamma-4\alpha).
$$

\begin{thm}
\label{thm:sss}\stepcounter{subsection}
%
Let \ $(\Lc,\Xc)$  \ be a planar \smech\ with $k$ branches, each
of multiplicity $2$,  and assume that the \cspace\ \
$\C=\C(\Lc,\Xc)$ \ has no node configrations;  then $\C$
decomposes as the disjoint union of the pre-image under $\Phi$ of
the components of the workspace $\W$, and for each such component
$V$, \ $\Phi^{-1}(V)$ \ consists of \ $c(V)$ closed orientable
surfaces of genus \ $g(V)$.
%
\end{thm}

A special case of this Theorem appears in \cite{E}.

\begin{proof}
%
As in example \ref{exa:lens} above, \ $\Phi^{-1}(V)$ \ is obtained
from the \ $2^{k}$ \ curvilinear polygonal ``tiles'' (i.e., copies
of $V$, corresponding to the ``elbow up/elbow down'' position of
each branch), by identifications of those edges which correspond
to the \ $k-\beta$ \ ``relevant'' branches. Since we know from
Theorem \ref{thm:main} that (each component of) $\C$ is a closed
orientable $2$-manifold, its type (genus) is determined by the
Euler characteristic, which may be computed by calculating how
many identifications we have for each vertex or edge of  \
$\Phi^{-1}(V)$.

To orient \ $\Phi^{-1}(V)$, \ choose an orientation for some
(fixed) tile \ $H_{0}$. \ Every other tile $H$ of \ $\Phi^{-1}(V)$
\ differs from \ $H_{0}$ \ in exactly $\tau$ of the $k$ possible
``elbow up/elbow down'' positions, and we reverse its orientation
(relative to that of $H_{0}$) \ if and only if $\tau$ is odd.
%
\end{proof}

\begin{figure}[htbp]
\begin{center}
\epsfysize=3cm %\epsfxsize=4.4cm
\leavevmode \epsffile{fig_1/embeddings.eps}  \caption{Work and
configuration space intersections of two
branches}\label{fig:embeddings}
\end{center}
\end{figure}

\begin{remark}
\label{rem:conntor}

As noted in \S \ref{sws}, the workspace $\W$ was obtained by
repeatedly intersecting annuli, which are the workspaces of the
individual branches. If we concentrate on the annulus $A$ of the
first branch, say, then generically the annuli for each of the
remaining branches will interesect with $A$ in one of the six
shaded patterns $V$ in the first row of Figure
\ref{fig:embeddings}. In each case we obtain a certain subset \
$\phi^{-1}(V)$ \ of the $2$-torus \ $\T{2}$ \ which is the
\cspace\ for the first branch, where \ $\phi:\T{2}\to A$ \ is the
work map for the first branch.

Note that when we further intersect $V$ with a third annulus, the
pattern may be more complicated; in particular, three-fold
intersections need not be connected, as illustrated by Figure
\ref{fig:punctured_torus}, which shows the subset of the $2$-torus
associated with the workspace of Figure~\ref{fig:mechanism}
(without indicating the identifications).

\begin{figure}[h] \epsfysize=3cm%\epsfxsize=4.4cm
\begin{center}
\leavevmode \epsffile{fig_1/punctured_torus.eps}   \caption{Subset
of $\T{2}$}\label{fig:punctured_torus}
\end{center}
\end{figure}
\end{remark}

%c4   Singular \cspaces
\sect{Singular \cspace s} \label{csingc}

While the analysis of the \cspace \ $\C=\C(\Lc,\Xc)$ \ of a
mechanism in the non-manifold case is in general difficult, for a
planar \smech\ with branch multiplicity $2$ the description of
Theorem \ref{thm:sss} can actually be extended to singular points,
corresponding to the node configurations.

\begin{prop}\label{prop:2node}\stepcounter{subsection}
%
Let \ $(\Lc,\Xc)$  \ be a planar \smech\ with $k$ branches, each
of multiplicity $2$,  and let $\Vc$ be a node configuration in \
$\C=\C(\Lc,\Xc)$. \ Then $\Vc$ has an open neighborhood in $\C$
which is a wedge of  \ $2^{q-\varepsilon}$ \ $2$-dimensional discs
with common center $\Vc$, where $q$ is the number of aligned
branches, \ $\varepsilon=1$ \ if the aligned branches have a
common direction, and \ $\varepsilon=2$ \ otherwise.

\end{prop}

\noindent\emph{Proof.} \ Let \ $v:=\Phi(\Vc)$ \ be the common
end-point of the $k$-branches in $\Vc$, and let $U$ be an small
disc containing $v$ in the workspace $\Wc$. Since $v$ is
necessarily in the boundary of a curvilnear polygonal component of
$\Wc$, then $P$, the boundary of $U$ near $v$, consists of arcs of
the boundary circles of the annuli.

\begin{enumerate}
\item When all of the $q$ aligned branches have a common
direction, the centers of the corresponding annuli must be
collinear, and $P$ is an arc $e$ of a single boundary circle of
such an annulus. Note that \ $\Phi^{-1}(\Int(U))$ \ intersects the
component of $\Vc$ in \ $2^{q}$ \ disjoint discs, which are
identified pairwise along \ $\Phi^{-1}(e)$ \ so as to yield \
$2^{q-1}$ \ discs whose only common point is \
$\Vc\in\Phi^{-1}(v)$.

\item Otherwise, $P$ must consist of two arcs \ $e_{1}$, \ $e_{2}$
\ of distinct boundary circles (whose centers are not collinear
with $v$). Again \ $\Phi^{-1}(\Int(U))$ \ intersects the component
of $\Vc$ in \ $2^{q}$ \ disjoint discs, but now every four of them
(corresponding to the four ``elbow up/elbow down'' positions of
the two branches associated to \ $e_{1}$ \ and \ $e_{2}$, \
respectively) are identified in \ $\Phi^{-1}(U)\subset\C$ \ along
\ $\Phi^{-1}(e_{1})$ \ and \ $\Phi^{-1}(e_{2})$, \ forming the
four quadrants of a new disc \ -- \ where again $\Vc$ is the only
point in common. This yields a total of \ $2^{q-1}$ \ discs.\eee

\end{enumerate}

\chapter{Motion Planning for a Class of Planar Closed-Chain Manipulators}
\label{chap2}

\textsf{Accepted to \textsl{International Journal of Robotic Research}, 2006.} \\
\textsf{Co-authors: Guanfeng Liu, Jeff Trinkle and Moshe
Shoham.}\\
\textsf{Presented in \textsl{IEEE International
Conference on Robotics and Automation}, 2006.} \\

We study the motion planning problem for planar {\sl star-shaped}
manipulators. These manipulators are formed by joining $k$ ``legs"
to a common point (like the thorax of an insect) and then fixing
the ``feet" to the ground.  The result is a planar parallel
manipulator with $k-1$ independent closed loops.  A topological
analysis is used to understand the global structure the
configuration space so that planning problem can be solved
exactly. The worst-case complexity of our algorithm is
$O(k^3N^3)$, where $N$ is the maximum number of links in a leg.
Examples illustrating our method are given.

\sect{Introduction} The canonical robot motion planning problem is
known as the ``piano movers'" problem.  In this problem, one is
given initial and goal configurations of a ``piano" (a rigid body
that is free to move in an environment with fixed rigid obstacles)
and geometric models of the piano and obstacles.  The goal is to
find a continuous motion of the piano connecting the initial and
goal configurations. Lozano-Perez studied this problem in
configuration space, or \cspace, a space in which a configuration
of the piano maps to a point, a motion maps to a continuous curve,
and the obstacles map to the C-obstacle, {\em i.e.,} the set
corresponding to overlap between the piano and an obstacle
\cite{Lozano-Perez83}.  The dimension of \cspace \ is equal to the
number of degrees of freedom of the system. The free space, or
C-free, is what remains after removing the C-obstacle from
\cspace. In \cspace, the motion planning problem becomes a path
planning problem. That is, one must construct a continuous path
connecting the initial and goal configurations that lies entirely
within C-free. Theoretical results for the piano movers' problem
were first obtained by Schwartz, Sharir, and Hopcroft
\cite{SS83,SHS87}. They found that the problem is PSPACE hard, and
proposed an algorithm based on Collins' decomposition to find a
path. Since the worst-case running time of Collins' decomposition
algorithm is doubly exponential in the dimension of \cspace, it is
impractical.

The more complex generalized movers' problem, is the problem in
which there are multiple rigid bodies moving simultaneously in a
workspace. The bodies are the links of one or more robots, and
thus may be required to obey constraints corresponding to their
kinematic structures and joint limits. Given the importance of
motion planning problem in robotics, researchers worked to find
more efficient algorithms despite the depressing complexity
results found earlier. The most efficient exact method known is
Canny's algorithm, which has time complexity that is only singly
exponential in the dimension of \cspace \cite{Can88}. He also made
the important observation that this bound is worst-case optimal,
since the worst-case number of components in \cspace \ is
exponential in its dimension. Canny's algorithm is very difficult
to implement - to date no full implementation exists.

In the 1990's, the intractability of exact motion planning for
general problems stimulated a paradigm shift to randomized
methods. The method of Barraquand and Latombe combined potential
field methods with random walk \cite{BL91}.  In essence, a
potential field method defines an artificial potential field on
\cspace \ such that the goal configuration is the global minimum
of the potential function and no saddle points or other local
minima exist. When the function has this property, motion planning
can be done by any gradient following algorithm. An important
class of such functions are navigation functions
\cite{Koditschek87,RK88,RK89}. Ideally, the potential function
will be a function of the goal configuration, and the global
minimum property will hold for all possible goal configurations.
Since such potential functions can be difficult to design,
Barraquand and Latombe suggested the use of random walks to escape
local minima \cite{BL91}. This modification yielded a method that
is practically effective and probabilistically complete.

When possibly many motion planning queries must be handled for a
single static environment, a different type of randomized method
has been found to be more efficient than rerunning the
Barraquand-Latombe algorithm for each query. The probabilistic
roadmap method (PRM) of Kavraki {\em et. al} \cite{KSLO96}, is an
easy-to-implement randomized version of Canny's \cite{Can88}.

Because PRMs have been successful in solving problems in \cspace s
with dimension approaching 100, many researchers have worked to
make the method more efficient ({\em e.g.,}
\cite{ABDJV98,BOS99,BK00}) and to modify it to solve more
challenging types of problems, such as those with closed kinematic
loops, nonholonomic constraints, dynamics, and intermittent
contact ({\em e.g.,} \cite{YLK01,HA01,CG99,KL00,LK99,RRT}).

In this paper, we are particularly interested in planar {\sl
star-shaped manipulators}. These manipulators are formed by
joining $k$ planar ``legs" to a common point (like the thorax of
an insect) and then fixing the ``feet" to the ground.  The result
is a planar parallel manipulator with $k-1$ independent closed
loops. They are important because they arise in parallel
manipulators, walking robots, and dexterous manipulation, and
motion plans are difficult to obtain using PRMs. In such systems,
\cspace \ is often most naturally viewed as a lower-dimensional
space embedded in an ambient space (typically the joint space).
The embedding results from equality constraints corresponding to
kinematic loop closure. In such settings, it is difficult to
obtain an explicit description of \cspace \ with minimal number of
parameters and a suitable metric to guide sample generation. These
problems make it difficult to construct a roadmap with the
requisite properties, and hence difficult  to solve motion
planning problems for systems with kinematic loops using PRMs. The
RLG (random loop generator) method \cite{Cortes02,CS03} improves
the sampling techniques through estimating the regions of sampling
parameters. However, its efficiency relies on the accuracy of the
estimation, which often varies case by case. Moreover, it ignores
the global structure of \cspace, and may fail to sample globally
important regions.
%
\begin{figure}
  \centering
  \includegraphics[width=3in]{fig_2/x-shape.eps}
  \caption{Star-shaped manipulator with $k=4$.}
  \label{X-shape}
\end{figure}
%
The difficulties associated with applying randomized motion
planning methods to manipulators with closed chains and the
availability of new results in topology \cite{KTT,KM,MT1,SSB05}
have recently led to renewed interest in exact planning
algorithms. Trinkle and Milgram derived some topological
properties of the \cspace s (the number of components and the
structures of the components) of single-loop closed chains with
spherical joints in a workspace {\em without} obstacles
\cite{MT2,MT1}.  These properties drove the design of a complete,
polynomial-time motion planning algorithm that works roughly as
follows.
%
\begin{enumerate}
\item Choose a subset ${\cal A}$ of the links that can be
positioned arbitrarily, and yet the remaining links can close the
loop; \item Move the links in ${\cal A}$ to their goal
orientations along an arbitrary path while maintaining loop
closure; \item Permanently fix the orientations of the links in
${\cal A}$; \item Repeat until all link orientations are fixed.
\end{enumerate}
%
The main result that guided the algorithm's design is Theorem~2 in
\cite{MT1}. In generic cases, the \cspace \ is the union of
manifolds that are products of spheres and intervals. The joint
coordinates corresponding to the spheres are those that can
contribute to the subset ${\cal A}$ mentioned above and the
structure of the \cspace \ suggests a local parametrization for
each step.

Here, the previous methods for \cspace \ connectivity analysis are
extended to planar star-shaped manipulators with revolute joints.
These manipulators have a common junction point and $k$ $(k>0)$
legs connecting the junction to the fixed base. Following a
topological analysis of the global structure of \cspace \ (i.e.,
the structure of the inverse kinematics of the manipulator), the
motion planning problem is solved completely in polynomial time.
Furthermore since we consider only a point end-effector, the
direct kinematics is straightforward while the inverse kinematics
is more complex. Thus, while for most parallel mechanisms using
\cspace \ as a mean for path planning should be carefully
considered, here our approach is natural.

Note that this paper is aimed at possibly more complex
macromolecules with non-covalent bonds (not just manipulators) for
which the number of legs and links in each leg may be very large
\cite{JRKT01}. So such a polynomial-complex algorithm would play a
key role in several issues in structural biology, such as
structure prediction in protein folding and binding, and study of
protein mobility in folded states. In these applications, we are
interested in the robot motion planning problem without
considering the control and sensor issues.

The main contributions of this paper are two folds. First, we
establish the global connectivity of the \cspace \ of star-shaped
manipulators via a combination of the cell decomposition of
workspace and the structure of the \cspace \ at points in all
cells. Although the results seem obvious, the proof detail is
highly non-trivial. Second, the global connectivity result of
\cspace \ only suggests an exponential algorithm for path
existence, which makes the motion planning formidable for
macromolecules with very large number of links. In this paper we
propose novel techniques for path existence that avoids the
exponential complexity.
%
\begin{figure}[h]
  \centering
  \includegraphics[width=3.5in]{fig_2/IK.eps}
  \caption{Inverse kinematics of a three-link serial chain.}
 \label{IK}
\end{figure}
%
This paper is organized as follows. In Section \ref{section-1},
kinematics and singularities of the manipulator are analyzed. In
Section \ref{section-2}, necessary and sufficient conditions for
\cspace \ connectivity and path existence are derived, based on
which a complete polynomial-time algorithm is developed in Section
\ref{section-3}. Section \ref{section-4} addresses path
optimization and robustness issues. Section \ref{section-5} shows
simulation results that tests the effectiveness of our algorithm.
Finally \ref{section-6} ends this paper with a brief conclusion.

%\sect{Notation}
%\begin{center}
%\begin{tabular}{rl}
%\hline \hline \multicolumn{2}{c}{Manipulator Notation} \\ \hline
%${M}$ & - Manipulator \\
%$A$ & - Root junction or thorax of ${M}$ \\
%$o_i$ & - Grounding point of foot $i$ of ${M}$ \\
%${M}_j$ & - Leg $j$ of ${M}$ with foot fixed at $o_j$ \\
%               & and other end free, $j=1,...,k$ \\
%$n_j$ & - Number of links in ${M}_j$ \\
%$l_{j,i}$ & - Length of link $i$ of ${M}_j$; $i=1,...,n_j$ \\
%$\theta_{j,i}$ & - Angle of link $i$ relative to link $i-1$ \\
%$\tilde {M}_j(p)$ & - Leg $j$ of ${M}$ with foot fixed at $o_j$\\
%               & and other end fixed at $p$ \\
%$\tilde {M}(p)$ & - Manipulator with $A$ fixed at $p$\\
%$L_j$ & - Sum of lengths of links of $\tilde {M}_j(p)$ \\
%$L_{j,0}$ & - Sum of lengths of links of $M_j$\\
%${\cal L}_j(p)$ & - A set of long links of $\tilde {M}_j(p)$ \\
%$|{\cal L}_j^*(p)|$ & - Number of long links of $\tilde {M}_j(p)$ \\
%               \hline \hline
%  \multicolumn{2}{c}{Workspace Notation} \\ \hline
%$W_A$ & - Workspace of $A$ \\
%$^dU_i$ & - Cell of dimension $d$ of $W_A$ \\
%$p$ & - Point in the plane of ${M}$ \\
%$\gamma = p(t)$ & - Curve in the plane of ${M}$ \\
%$f$ & - Kinematic map of $A$ \\
%$f_j$ & - Kinematic map of endpoint of $M_j$ \\
%$\Sigma$ & - Critical set of $f$ in $W_A$  \\
%$\Sigma_j$ & - Critical set of $f_j$ \\
%                 \hline \hline
%  \multicolumn{2}{c}{Configuration Space (\cspace) Notation} \\
%\hline
%${\cal C}$ & - \cspace \ of ${M}$ \\
%$\tilde {\cal C}(p)$ & - \cspace \ of $\tilde {M}(p)$ \\
%${\cal C}_j$ & - \cspace \ of ${M}_j$ \\
%$\tilde {\cal C}_j(p)$ & - \cspace \ of $\tilde {M}_j(p)$ \\
%$c$ & - Point in \cspace \\ \hline \hline
%\end{tabular}
%\end{center}

\sect{Preliminaries} \label{section-1}
\newnot{2-1}
A star-shaped manipulator is composed of $k$ serial chains with
all revolute joints (see Fig.~\ref{X-shape}).  Leg $M_j$ is
composed of $n_j$ links of lengths $l_{j,i}, i=1,...,n_j$
%jct 1/16/06 - small change
and joint angles $\theta_{j,i}, i=1,...,n_j$. At one end (the
foot), $M_j$ is connected to ground by a revolute joint fixed at
the point $o_j$. At the other end, it is connected by another
revolute joint to a junction point denoted by $A$. Note that when
$k$ is one, a star-shaped manipulator is an open serial chain.
When $k$ is two, it is a single-loop closed chain.

Assuming that the foot of $M_j$ is fixed at $o_j$, let
$f_j(\Theta_j)=p$ denote the kinematic map of $M_j$, where
$\Theta_j = (\theta_{j,1}, \cdots, \theta_{j,n_j})$ is the tuple
of joint angles, and $p$ is the location of the endpoint of the
leg (the thorax end). When $M_j$ is detached from the junction
$A$, the image of its joint space is the reachable set of
positions of the free end of the leg, called the workspace $W_j$.
%jct 1/16/06 - small change
In the absence of joint limits, the workspace $W_j$ is an annulus
if and only if there exists one link with length strictly greater
than the sum of all the other link lengths. Otherwise it is a
disk. Clearly, the workspace $W_A$ of $A$ when all the legs are
connected to $A$ is given by:
\begin{equation}
\label{eq:defW}
   W_A = \bigcap_{j=1}^k W_j.
\end{equation}
\newnot{2-4}

In our study of ${\cal C}$, it will be convenient to refer to
several other \cspace s. The \cspace \ of leg $M_j$ when detached
from the rest of the manipulator will be denoted by ${\cal C}_j$.
When the endpoint is fixed at the point $p$, leg $j$ will be
denoted by $\tilde M_j(p)$, where the tilde is used to emphasize
the fact that the endpoint has been fixed. Note that $\tilde
M_j(p)$ is a single-loop planar closed chain, about which much is
known (see \cite{MT2}), including global structural properties of
its \cspace, denoted by $\tilde {\cal C}_j(p) = f_j^{-1}(p)$. When
the junction $A$ of a star-shaped manipulator is fixed at point
$p$, its \cspace \ will be denoted by $\tilde {\cal C}(p)$. Since
collisions are ignored, the motions of the legs are independent,
and therefore the \cspace \ of the manipulator (with fixed
junction) is the product of the \cspace s of the legs with all
endpoints fixed at $p$:
%
\newnot{2-2}
\begin{equation}
\label{eq:def_tildeCp}
   \left. \begin{array}{rcl}
   \tilde {\cal C}(p) & = & \tilde {\cal C}_1(p) \ \times
           \cdots \times \ \tilde {\cal C}_k(p) \\ [5pt]
          & = & f_1^{-1}(p) \ \times \cdots \times \ f_k^{-1}(p) \\ [5pt]
          & = & f^{-1}(p)
   \end{array} \right\}
\end{equation}
%
where by analogy, $f$ is a total kinematic map of the star-shaped
manipulator. Loosely speaking, the union of the \cspace s $\tilde
{\cal C}(p)$ at each point $p$ in $W_A$ gives the \cspace \ of a
star-shaped manipulator: \newnot{2-7}
%
\begin{equation}
\label{eq:def_C}
   {\cal C} = \bigcup_{p \in W_A} \tilde {\cal C}(p).
\end{equation}
%
Several properties of the \cspace s ${\cal C}_j$ and $\tilde {\cal
C}_j(p)$ are highly relevant and so are reviewed here before
analyzing the \cspace \ of star-shaped manipulators.  It is well
known that the \cspace \ of $M_j$ is a product of circles ({\em
i.e.,} ${\cal C}_j = (S^1)^{n_j}$)
%jct 1/16/06 - footnote added
\footnote{Recall the assumption of no joint limits.}. The
workspace $W_j$ contains a critical set $\Sigma_j$ which is
composed of all points $p$ in $W_j$ for which the Jacobian of the
kinematic map $Df_j(\Theta_j)$ drops rank for some $\Theta_j \in
f_j^{-1}(p)$. These points form concentric circles of radii
$|l_{j,1}\pm l_{j,2} \pm \cdots \pm l_{j,n_j}|$, as shown in
Fig~\ref{fig:single-leg}. When $A$ coincides with a point in
$\Sigma_j$, the links can be arranged such that they are all
colinear, in which case the number of instantaneous degrees of
freedom of the endpoint of the leg is reduced from two to one.
%
\begin{figure}
  \centering
  \includegraphics[width=3in]{fig_2/nir-paper-1b.eps}
  \caption{{\bf Left:} The workspace $W_j$ of a three-link open chain $M_j$
    based at $o_j$. The critical set $\Sigma_j$ of the kinematic map $f_j$ is four
    concentric circles.  The small circles, figure eights, and points at 12 o'clock
    show the topology of the \cspace \ $\tilde {\cal C}_j(p)$ of the
    leg when its endpoint is fixed at a point in one of the seven regions
    delineated by the critical circles (one of the four circles or one of
    the three open annular regions between them).  {\bf Right:} The inverse
    image of the curve $\gamma$ - a ``pair of pants."}
 \label{fig:single-leg}
\end{figure}\newnot{2-8}
%
Now consider the case where the endpoint of leg $j$ is fixed to
the point $p$.  In other words, we are interested in the \cspace \
$\tilde {\cal C}_j(p)$ of $\tilde M_j(p)$, which amounts to
calculating the inverse kinematics (IK) $f_j^{-1}(p)$. The
structure of $f_j^{-1}(p)$ has been established in
\cite{Burdick89} for four-link single-loop closed chains, and in
\cite{MT2,MT1} for chains with an arbitrary number of links.
\newnot{2-9}

Next, we compute $f_j^{-1}(p)$ for a three-link serial chain to
explain the basic idea used in \cite{MT2, MT1} for analyzing the
IK map of a closed chain with an arbitrary number of links. As
shown in Fig. \ref{IK}, we begin with the IK of a two-link serial
chain. Each point in the workspace could have 0,1,or 2 IK
solutions, and the set of points with constant number of IK
solutions forms annular regions separated by the critical set of
this chain. The IK of a three-link serial chain is then deduced by
breaking the chain into a two-link serial chain and a one-link
serial chain, and taking the union of the IK solutions of the
two-link chain for all points in the workspace of the one-link
chain. It is easy to check that for points in the outer most
annular region in the workspace of the three-link chain, the
workspace of the one-link chain (a circle) always intersects with
the outer-most critical circle of the two-link chain at two
points, indicating that the IK of this point is the union of two
curve segments with a pair of endpoints identified, respectively,
i.e., a circle. The inverse kinematics for points in other regions
can be derived similarly. The results are shown in
Fig.~\ref{fig:single-leg}. In the 12 o'clock position, points,
circles, and figure eights are drawn to represent the global
structures of $\tilde {\cal C}_j(p)$ in the seven regions of
$W_j$. Specifically, when $A$ is fixed to a point $p$ on the
outer-most critical circle, $\tilde {\cal C}_j(p)$ is a single
point. For $p$ fixed to any point in the largest open annular
region, \cspace \ is a single circle. Continuing inward, the
possible \cspace \ types are a figure eight (on the second largest
critical circle), two disconnected circles, a figure eight again,
a single circle, and a single point (on the inner-most critical
circle).

First, the connectivity of $\tilde {\cal C}_j(p)$ is uniquely
determined by the number of ``long links."  Consider the augmented
link set composed of the links of $M_j$ and $\overline {o_jp}$,
which will be called the fixed base link with length denoted by
$l_{j,0}$. Let $L_j$ be the sum of all the link lengths including
the fixed base link ({\em i.e.,} $L_j = \sum_{i=0}^{n_j}
l_{j,i}$). Further, let ${\cal L}_j(p)$ be a subset of
$\{0,1,...,n_j\}$ such that $l_{j,\alpha}+l_{j,\beta} > L_j / 2; \
\alpha,\beta \in {\cal L}_j(p), \ \alpha \neq \beta$. Over all
such sets, let ${\cal L}_j^*(p)$ be a set of maximal cardinality.
Then the number of long links of $\tilde M_j(p)$ is defined as
$|{\cal L}_j^*(p)|$, where $| \cdot |$ denotes set cardinality.
\newnot{2-3}

\medskip
\begin{Lemma} {\bf Kapovich and Milson \cite{KM}, Trinkle and
Milgram \cite{MT2}}\\
\label{lem-02} \rm The \cspace \ $\tilde {\cal C}_j(p) =
f_j^{-1}(p)$ has two components if and only if $|{\cal
L}_j^*(p)|=3$, and is connected if and only if $|{\cal
L}_j^*(p)|=2$ or $0$. No other cardinality is possible.
\end{Lemma}

\medskip

Let us return to the discussion of Fig.~\ref{fig:single-leg}.
Viewing $W_j$ as a base manifold and the \cspace \ corresponding
to each end point location as a fibre, it is apparent that the
critical set $\Sigma_j$ partitions $W_j$ into regions over which
the \cspace s $\tilde {\cal C}_j(p)$ form a trivial fibration. The
implications of this observation are useful in determining the
\cspace \ of more complicated mechanisms.  Consider a modification
to $\tilde M_j(p)$ that allows the endpoint to move along a
one-dimensional curve segment $\gamma$ within $W_j$.  Then as long
as $\gamma$ is entirely contained in one of the regions defined by
the critical circles, $\tilde {\cal C}_j(\gamma) = \tilde {\cal
C}_j(p) \times I$, where $I$ is the interval.  If $\gamma$ crosses
a critical circle transversally, then $\tilde {\cal C}_j(\gamma) =
(\tilde {\cal C}_j(p_1) \times I) \bigcup \tilde {\cal C}_j(p_3)
\bigcup (\tilde {\cal C}_j(p_2) \times I)$, where $p_1$ is a point
in one of the two open annular regions containing $\gamma$, $p_2$
is a point in the other, and $p_3$ is a point on the critical
circle crossed by $\gamma$, and $\bigcup$ denotes the standard
``gluing" operation. In Fig.~\ref{fig:single-leg}, an example
$\gamma$ and the corresponding \cspace \ $\tilde {\cal
C}_j(\gamma)$ are shown.

\sect{Analysis of Star-Shaped Manipulators} \label{section-2} For
star-shaped manipulators with one or two legs, the global
topological properties of the \cspace \ ${\cal C}$ are fully
understood (for one, see \cite{Lat92}; for two, see
\cite{MT2,MT1}). The goals of this section are to study the global
properties of ${\cal C}$ when $M$ has more than two legs and to
derive necessary and sufficient conditions for solution existence
to the motion planning problem.

\subsubsection{Local Analysis}
As a direct generalization of the critical set of a single leg, we
define the critical set of a star-shaped manipulator as a subset
$\Sigma$ of $W_A$ such that for every $p \in \Sigma$, there exists
a configuration $c$ such that at least one of the Jacobians
$\{Df_1(c),\cdots,Df_k(c)\}$ drops rank.  By definition we have:
%
\begin{eqnarray}
\label{eqn-01}
 \Sigma=\left(\bigcup_{i=1}^k \Sigma_i \right)\bigcap W_A.
\end{eqnarray}
%
An advantage of this definition is that $\Sigma$ can be used to
stratify $W_A$ such that each stratum is trivially fibred.
Figure~\ref{fig:double-leg} shows a star-shaped manipulator with
two legs.  The critical set $\Sigma$ is the boundary of the lune
formed by the intersection of the outer critical circles of their
individual workspaces.  For every point interior to the lune, the
fibre is two circles (the direct product of two points with one
circle). The fibres associated to the vertices of the lune are
single points, which correspond to simultaneous full extension of
the two legs.
%
\begin{figure}
  \centering
  \includegraphics[width=3in]{fig_2/nir-paper-2a.eps}
 \caption{The workspace $W_A$ of $A$ for a star-shaped manipulator with $k=2$
    is the intersection of the workspaces of $A$ for each leg considered
    separately.  The critical set $\Sigma$ is composed of the black
    circular arcs where they bound or intersect the gray area.
    }
 \label{fig:double-leg}
\end{figure}
%
Fig.~\ref{fig:chambers} shows a possible workspace for a
star-shaped manipulator with three legs. The critical set defines
65 distinct sets $^dU_i$ of varying dimension $d$, where $i$ is an
arbitrarily assigned index that simply counts components. We will
refer to these sets as {\sl chambers}. There are 12
two-dimensional, 32 one-dimensional, and 21 zero-dimensional
chambers, each of which is trivially fibred. Removing the
$^0\!U_i$ from $\Sigma$ partitions it into open one-dimensional
chambers $^1\!U_i$, $i=1,\cdots,^1\!m$. Removing $^0\!U_i$ and
$^1\!U_i$ from $W_A$ yields open two-dimensional sets $^2\!U_i, \
i=1,\cdots,^2\!m$, for which the following relationships hold:
%
\newnot{2-5}

\begin{eqnarray}
\Sigma & = & \left(\bigcup_{i=1}^{^0\!m} {}^0\!U_i \right) \bigcup
\left(\bigcup_{i=1}^{^1\!m} {}^1\!U_i \right) \\[5pt]
W_A - \Sigma & = & \bigcup_{i=1}^{^2\!m} {}^2\!U_i.
\end{eqnarray}
\begin{figure}
 \vbox{
      \centering {\epsfysize=2.3in \epsffile{fig_2/nir-paper-3a.eps}}
      }
 \caption{Workspace (shaded gray) of a star-shaped manipulator with
 three legs. The critical set partitions $W_A$ into $12$ two-dimensional,
 32 one-dimensional, and 21 zero-dimensional chambers.}
 \label{fig:chambers}
\end{figure}

\smallskip

\begin{Proposition}
\label{prop-01} \rm For all $d=0,1,2$ and $i$, $f^{-1}(^d\!U_i) \
= \ ^d\!U_i \times f^{-1}(p)$, where $p$ is any point in $^d\!U_i$
and the operator $\times$ denotes the direct product. Gluing the
$f^{-1}(^d\!U_i)$ for all $i$ and $d$ gives the total \cspace \
${\cal C}$.
\end{Proposition}

\medskip

{\bf Proof:} When $d=0$, ${}^0\!U_i$ contains a single point, the
result follows. When $d=1$, ${}^1\!U_i$ belongs to one critical
circle of one leg, say $\tilde M_j$. Any two points $p_1,p_2 \in
{}^1\!U_i$ are related by a Euclidean rotation $p_2-o_j =
R(p_1-o_j)$, indicating that $\tilde {\cal C}_j(p_1)$ and $\tilde
{\cal C}_j(p_2)$ are  homotopic. %homeomorphic.
Thus $\tilde {\cal C}_j(p)$ for all $p \in {}^1\!U_i$ have
equivalent topological structure. For the other legs $\tilde M_l$,
$l \neq j$, according to \cite{MT1} (Lemma $6.1$ and
Corollary~$6.5$) $\tilde {\cal C}_l(p)$ for all $p \in {}^1\!U_i$
have equivalent topological structures as ${}^1\!U_i$ is free of
critical points of $\tilde M_l(p)$. Thus $f^{-1}(p) = \tilde {\cal
C}_1(p)\times \cdots \times \tilde {\cal C}_k(p)$ for all $p \in
{}^1\!U_i$ have equivalent topological structures. The case when
$d=2$ can be proved by applying Lemma $6.1$ and Corollary $6.5$ of
\cite{MT1} to all legs. \hfill$\blacksquare$

\medskip

Proposition \ref{prop-01} and the fact that ${}^d\!U_i$ is a
simply connected set, reveal that each component of
$f^{-1}({}^d\!U_i)$ is a direct product of one component of
${\tilde {\cal C}}_j(p)$, $j=1,\cdots,k$, with a $d$-dimensional
disk. Using $|{\cal L}_j^*(p)|$, $j=1,\cdots,k$ and Lemma
\ref{lem-02}, one can show that the number of components of
$f^{-1}({}^d\!U_i)$ is $2^{k_0}$, where $k_0 \leq k$ is the number
of legs for which $|{\cal L}_j^*(p)| = 3$.

\smallskip

\subsubsection{Local Path Existence}
Before considering the global path existence problem, consider
motion planning between two valid configurations $c_{\rmtt[init]}$
and $c_{\rmtt[goal]}$ for which the junction $A$ lies in the same
chamber. Since the fibre over every point in ${}^d\!U_i$ is
equivalent, path existence amounts to checking the component
memberships of the configurations $c_{\rmtt[init]}$ and
$c_{\rmtt[goal]}$.

For a single leg $\tilde{M}_j(p)$, if the number of long links
$|{\cal L}_j^*(p)|$ is not three, then any two configurations of
$\tilde{M}_j(p)$ are in the same component. When $|{\cal
L}_j^*(p)|=3$, choose any two long links and test the sign of the
angle between them (with full extension taken as zero).  There are
two possible signs, one corresponding to {\sl elbow-up} and the
other to {\sl elbow-down}. If for two distinct configurations of
$\tilde{M}_j$, $A$ lies in the same chamber, there is a continuous
motion between them while keeping $A$ in this chamber, if and only
if the elbow sign is the same at both configurations (naturally,
one must perform the sign test with the same two links and in the
same order for both configurations). Considering all the legs
together, a continuous motion of $A$ in ${}^d\!U_i$ exists if and
only if a motion exists for each leg individually. The previous
discussion serves to prove the following result.

\medskip

\begin{Proposition}
\label{prop-2} \rm Restricted to $f^{-1}({}^d\!U_i)$, two
configurations $c_1,c_2 \in f^{-1}({}^d\!U_i)$ are path connected
if and only if for each leg $\tilde M_j$ with $|{\cal L}_j^*|=3$
in ${}^d\!U_i$, the elbow angle of $\tilde M_j$ has the same sign
at $c_1$ and $c_2$.
\end{Proposition}

\medskip

Proposition \ref{prop-2} completely solves the path existence
problem if $W_A$ consists of a single chamber. However, things
become complex when $W_A$ has more than one chamber.

\subsubsection{Singular Set and Global \cspace \ Analysis}

Recall that the \cspace \ ${\cal C}$ is a union of
$f^{-1}({}^d\!U_i)$, $d \in \{0,1,2\}$, $i=1,\cdots,{}^d\!m$ and
that $f^{-1}(p)$, $p\in {}^d\!U_i$ for $d\neq 2$ and all $i$ is a
set containing at least a singularity of $f$. Combining the local
\cspace \ and singular set analysis yields the global structure of
\cspace.

\medskip

\begin{Proposition}
\label{prop-sing} \rm For all $p \in \Sigma_j$, $f_j^{-1}(p)$ is a
singular set containing isolated singularities. If a singularity
separates its neighborhood $V$ in $f_j^{-1}(p)$, then it is these
singularities which glue the two separated components in
$f_j^{-1}(q)$ where $q \in W_A-\Sigma_j$ is a point sufficiently
close to $p$.
\end{Proposition}

\medskip

{\bf Proof:} First it is obvious that $f_j^{-1}(p)$ contains
isolated singularities for there are finite ways to colinearize
all the links of a close chain. Second, let
\[
   \gamma: (-\varepsilon,\varepsilon) \rightarrow W_A,\, \gamma(0)=p
\]
\newnot{2-6}
be a curve that is transverse to $\Sigma_j$. According to
Corollary 6.6 of \cite{MT1}, the distance function
$s(\gamma(t))=\int_{0}^t |{\dot \gamma}|dt$ defines a Morse
function on $f_j^{-1}(\gamma)$
\[
   s \circ f_j: f_j^{-1}(\gamma) \rightarrow \reals.
\]
Note that $0$ is a singular value of $s \circ f_j$ and the
isolated singularities of $f_j^{-1}(p)$ are also singularities of
$s \circ f_j$. The result of Morse theory applying to $s \circ
f_j$ yields that $(s \circ f_j)^{-1}(0)=f_j^{-1}(p)$ is given by
attaching a handle to $(s \circ
f_j)^{-1}(\varepsilon_0)=f_j^{-1}(q)$ for a sufficiently small
$\varepsilon_0$ and $q$ a point sufficiently close to $p$. The
Proposition follows.
  \hfill$\blacksquare$\medskip

%jct 1/16/06 - changed all {\cal J} to J based on reviewer question - 4 places.
Next, we establish necessary and sufficient conditions for the
connectivity of ${\cal C}$.  Let ${J}$ be the index set such that
for all $j \in {J}$, $|{\cal L}_j^*|=3$ for at least one chamber
${}^d\!U_i$. We prove the following theorem.

\medskip

\begin{Theorem}
\label{them-1} \rm Suppose $W_A = \bigcup_{d=0}^2 \left(
\bigcup_{i=1}^{{}^d\!m} {}^d\!U_i \right)$.  Then ${\cal C} =
f^{-1}(W_A)$ is connected if and only if:
\begin{enumerate}
\item
    $W_A$ is connected;
\item
    $\Sigma_j \bigcap W_A \ne \emptyset$ for all $j \in {J}$.
\end{enumerate}
\end{Theorem}

\medskip

{\bf Proof:} (i) ``Necessity:" Since ${\cal C}$ is a fibration of
the base manifold $W_A$, it can have one component only when $W_A$
has one component.  Thus item $1$ of Theorem~\ref{them-1} is
required.
%
Second, in order that ${\cal C}$ be connected, for each leg $M_j$
restricted to $W_A$, the \cspace \ ${\tilde {\cal C}}_j(W_A) =
f_j^{-1}(W_A)$ must be connected. By definition, for all $j \in
{J}$, there exists a chamber ${}^d\!U_i$ such that $|{\cal
L}_j^*|=3$. The result of Proposition \ref{prop-sing} means that
${\tilde {\cal C}}_j(W_A)$ is connected only if $W_A \bigcap
\Sigma_j \ne \emptyset$.

(ii) ``Sufficiency:" Item 1 and 2 imply that ${\tilde {\cal
C}}_j(W_A)$ are path connected for all $j$. Moreover, ${\cal C}$
is a fibration over $W_A$. The result follows.
 \hfill$\blacksquare$\medskip

Fig.~\ref{fibre} illustrates the global connectivity for an
example $W_A$ corresponding to a star-shaped manipulator with two
legs and a workspace for which there are two chambers $^2U_1$ and
$^2U_3$ where leg~1 has three long links and another chamber
$^2U_4$ where both legs have three long links.  Among these
chambers, ${}^1\!U_1$ and ${}^1\!U_2$ belong to $\Sigma_1$, and
${}^1\!U_3$ belongs to $\Sigma_2$. According to
Theorem~\ref{them-1}, the \cspace \ is path connected.  In this
example, the ${\cal C}$ is the product of the two structures
shown.
%
\begin{figure}
 \vbox{
      \centering {\epsfysize=3.2in \epsffile{fig_2/fibre.eps}}
      }
 \caption{\cspace \ of a star-shape manipulator with two legs.
 For simplicity, only the portion of $f^{-1}(\gamma)$
 is shown, where $\gamma$ is a continuous curve in $W_A$
 that visits all chambers. }
 \label{fibre}
\end{figure}

\smallskip

\begin{Corollary}
\label{cor-1} \rm Two configurations $c_1$ and $c_2$ of a
star-shaped manipulator are in the same component if and only if
\begin{enumerate}
\item $f(c_1)$ and $f(c_2)$ are in the same component of $W_A$;
\item For each leg $j$ with $|{\cal L}_j^*|=3$ for all chambers
${}^d\!U_i$ in the component of $W_A$ which contains $f(c_1)$ and
$f(c_2)$, the elbow sign is same at both $c_1$ and $c_2$.
\end{enumerate}
\end{Corollary}

\medskip

\begin{Remark}
\rm As a matter of fact, $\Sigma$ completely determines the
connectivity of \cspace. When computing a path between two given
configurations, often motions of the junction to points on
$\Sigma$ are incorporated to allow adjustment of leg-angle signs.
However, inevitable deviations of the junction from $\Sigma$
caused by numerical errors, make it impossible to adjust the sign
of legs while fixing its end point. For these reasons, points in
2D chambers are preferred for sign adjustment.
\end{Remark}

\sect{A Polynomial-Time, Exact, Complete Algorithm}
\label{section-3} Our algorithm consists of two main routines,
{\tt PathExists} and {\tt ConstructPath}. {\tt PathExists} solves
the path existence problem, i.e., determining if an initial and an
goal configuration are path-connected, and {\tt ConstructPath}
constructs a path between them if there exists a path.

Notice that the \cspace \ of a star-shaped manipulator could be
very complex. Even the simple planar five-link single-loop closed
chain, its \cspace \ could be as complex as the connected sum of
four torii \cite{MT1}. So determining the path existence without
using the \cspace \ information is difficult. Our strategy is to
solve the path existence based on the set of critical circles
$\Sigma_j$ in the workspace, and then construct the path combining
our knowledge of the workspace and the structure of the \cspace \
of single-loop closed chains. We emphasize here that the problem
is not just moving the junction point between an initial and a
goal position, but moving the manipulator along with all its legs
from an initial configuration to a goal configuration. So, the
workspace information will be insufficient for path construction.
In {\tt ConstructPath}, we employ a move that changes the shape of
a leg with its endpoint fixed in the workspace. This move, called
the sign-adjust move, uses the knowledge of the \cspace \ of a
single-loop closed chain. Below we will show that the overall
complexity of {\tt PathExists} and {\tt ConstructPath} is
$O(k^3N^3)$, where $N$ is the maximum number of links in a leg and
$k$ is the number of legs. The polynomial complexity is key to the
applications like folding of macromolecules, which can can be
modeled as a closed chain with large $k$ and $N$.


The logical flow of {\tt PathExists} is illustrated in
Figure~\ref{PathExists}. Its input is the topology and link
lengths of a star-shaped manipulator and two valid configurations,
$c_{\rmtt[init]}$ and $c_{\rmtt[goal]}$.  The output is the answer
to the path existence question.
%
\begin{figure}
  \centering
  \includegraphics[width=2.8in]{fig_2/star-shaped-flow-chart6_1.eps}
  \caption{Logical flow and complexity of the major steps
  of {\tt PathExists}.}
  \label{PathExists}
\end{figure}
%
The approach taken is to compute $W_A$ and then, for each leg with
its end point constrained to lie in $W_A$, to determine if its
initial and goal configurations are path connected. Notice that
$W_j$ is either a disk or an annular region,  $W_A$ can be
constructed by calculating the intersections between no more than
$2n$ circles. So constructing $W_A$ is polynomial-complex.  The
most difficult issue is to check the path existence. Since the
\cspace \ of a leg is guaranteed to be connected if one of its
critical circles $\Sigma_j$ intersects $W_A$, the most straight
forward way to test connectivity is to explicitly perform the
intersections. However, since there are as many as $2^{n_j-1}$
critical circles, any algorithm based on this approach will have
worst-case complexity that is at least exponential in $N$. The key
contribution of {\tt PathExists} is a polynomial-time algorithm
for checking the existence of an intersection between $W_A$ and a
critical circles - even though there is an exponential number of
these circles.  Recall that if a leg has three long links, then it
is impossible to move the leg so that the three long links change
from an elbow-up configuration to an elbow-down configuration. The
following algorithm constructs a novel polynomial-complex method
that can determine if there is point in $W_A$ in which the number
of long links of a leg is not $3$ (see Step 4 in {\tt
PathExists}).

\medskip

\noindent \framebox{1. Construct $W_A$}
We compute $W_A$ in three steps.\\
Step 1: Compute the boundary circles of $W_j$. In general, $W_j$
is an
%jct 1/16/06 - typo
annulus. The radius of its outer boundary circle is
$r_{\rmtt[max]}=\sum_{i=1}^{n_j} l_{j,i}$, while that of its inner
boundary circle, $r_{\rmtt[min]}$, can be determined by comparing
$l_{\rmtt[max]}:=\max_i l_{j,i}$ and
$r_{\rmtt[max]}-l_{\rmtt[max]}$.

If $l_{\rmtt[max]}>r_{\rmtt[max]}-l_{\rmtt[max]}$, then
$r_{\rmtt[min]}=2l_{\rmtt[max]}-r_{\rmtt[max]}$, else,
$r_{\rmtt[min]}=0$;\\
Step 2: Decompose the whole plane into cells using all boundary
circles of all legs (e.g., the line sweeping algorithm can do
this), and construct the cell adjacency graph;\\
Step 3: Pick a point from the interior of each cell, compute its
distance from each base point, and compare the distance with the
radii of the two boundary circles of $W_j$. The set of cells which
can be reached by all legs constitute $W_A$.\\
The complexity of this 2-D cell decomposition algorithm is
$O(k^2+kN)$.
\medskip \\
\framebox{2. Are $p_{\rmtt[init]}$ and $p_{\rmtt[goal]}$ in same
component of $W_A$?} As an immediate consequence of the cell
decomposition, this can be answered directly by searching the cell
graph.
\medskip  \\
\framebox{3. Compute $J$} This step is used to filter out easy
solution existence checks, based on the cardinality and members of
the sets ${\cal L}_j^*(p_{\rmtt[init]})$ and ${\cal
L}_j^*(p_{\rmtt[goal]})$. For each leg $\tilde
M_j(p_{\rmtt[init]})$, compute $L_j$ (see Section~III) and find
the three longest links of the set $\{l_{j,0},...,l_{j_{n_j}} \}$.
Denote these links by $(p_{\rmtt[init]};
\lambda_{j,1},\lambda_{j,2},\lambda_{j,3})$. Do the same for
$(p_{\rmtt[goal]})$ and define $(p_{\rmtt[goal]};
\lambda_{j,1},\lambda_{j,2},\lambda_{j,3})$. This requires $O(N)$
work. Finally, $|{\cal L}_j^*(p_{(\cdot)})| = 3$ if and only if
$\lambda_{j,2} + \lambda_{j,3} > L_j/2$.  If ${\cal
L}_j^*(p_{\rmtt[init]}) = {\cal L}_j^*(p_{\rmtt[goal]})$ and
$|{\cal L}_j^*(p_{\rmtt[init]})| = 3$, and if the signs of the
long links are different at $c_{\rmtt[init]}$ and
$c_{\rmtt[goal]}$, then add $j$ into $J$.
Computing $J$ is $O(kN)$. \medskip  \\

\framebox{4. Does the set of long links vary for all $j \in J$?}
If and only if $q \in W_A$ exists such that ${\cal L}_j^*(q) \neq
{\cal L}_j^*(p_{\rmtt[init]})$, then it is possible to make the
long links colinear and thus change the signs of their relative
angles. This can be done by computing a point $q \in W_A$ on the
boundary of the cell that contains $p_{\rmtt[goal]}$ and keeps the
same set ${\cal L}_j^*(p)$ for all $p$ in this cell. This boundary
is characterized by $\lambda_{j,2}+\lambda_{j,3}=L_j/2$. Since
$l_{j,0}$ is the only link whose length varies along with $p$,
this boundary must be one or two circles (called inner and outer
circles, respectively) whose radii, denoted $d_{\rmtt[max]}$ and
$d_{\rmtt[min]}$, depend on the link lengths of the leg. Let
$L_{j,0}=\sum_{i=1}^{n_j} l_{j,i}$ and suppose the four longest
links at $p_{\rmtt[goal]}$ are
$(\lambda_{j,1}>\lambda_{j,2}>\lambda_{j,3}>\lambda_{j,4})$ with
$\lambda_{j,2}+\lambda_{j,3}>L_j/2$, we deduce the radii of the
boundary circles for
four different cases:\\
Case 1: if $l_{j,0}(p_{\rmtt[goal]})=\lambda_{j,1}$, then
$d_{\rmtt[max]}= 2(\lambda_{j,2}+\lambda_{j,3})-L_{j,0}$, and
$d_{\rmtt[min]}=\max \{L_{j,0}-2\lambda_{j,3},
2(\lambda_{j,3}+\lambda_{j,4})-L_{j,0}\}$.\\
Case 2: if $l_{j,0}(p_{\rmtt[goal]})=\lambda_{j,2}$,
$d_{\rmtt[max]}=2(\lambda_{j,1}+\lambda_{j,3})-L_{j,0}$, and
$d_{\rmtt[min]}=\max\{L_{j,0}-2\lambda_{j,3},
2(\lambda_{j,3}+\lambda_{j,4})-L_{j,0}\}$.\\
Case 3: if $l_{j,0}(p_{\rmtt[goal]})=\lambda_{j,3}$,
$d_{\rmtt[max]}=2(\lambda_{j,1}+\lambda_{j,2})-L_{j,0}$, and
$d_{\rmtt[min]}=\max\{L_{j,0}-2\lambda_{j,2},2(\lambda_{j,2}+\lambda_{j,4})-L_{j,0}\}
$. \\
Case 4: Otherwise,
$d_{\rmtt[max]}=\min\{2(\lambda_{j,2}+\lambda_{j,3})-L_{j,0},L_{j,0}-2\lambda_{j,2}\}$,
and $d_{\rmtt[min]}=0$.\\
If there is no overlap between the two boundary circles and the
component of $W_A$ that contains $p_{\rmtt[init]}$ and
$p_{\rmtt[goal]}$, then no path exists between $c_{\rmtt[init]}$
and $c_{\rmtt[goal]}$. Otherwise, path exists and we obtain way
points $p_j$ for all legs $j\in J$. Computing $d_{\rmtt[max]}$,
$d_{\rmtt[min]}$, and the way points $p_j$ is $O(kN)$.

\medskip

The basic idea of {\tt ConstructPath} is that when moving from
$c_{\rmtt[init]}$ to $c_{\rmtt[goal]}$, those legs $j \in J$ may
require a change in the signs of relative angles between long
links, which is always possible at the way point $p_j$ or other
critical points of the corresponding leg. A natural approach then
is to use two motion generation primitives: {\sl accordion move}
and {\sl sign-adjust move}.  The former moves the thorax endpoint
(at $A$) along a specified path segment with all legs moving
compliantly so that all loop closures are maintained. The latter
keeps the endpoint fixed at a way point $q_j \in \Sigma_j$ (e.g.,
$q_j=p_j$ or other critical points) while moving leg $j$ into a
singular configuration and then to a nearby configuration with the
sign of the relative angle between a pair of long links in this
leg chosen to match those of $c_{\rmtt[goal]}$.

Note that though the guard points $q_j$ are in the critical set
$\Sigma_j$, the configuration at which leg $j$ approaches these
points need not to be aligned. Without considering the control
issue, a leg can be moved to a colinear configuration with the
thorax fixed. Even if control is considered for the sign-adjust
move, the thorax can still be maintained to be fixed by keeping
all other legs not aligned in the vicinity of $q_j$.

The input of {\tt ConstructPath} is $W_A$ and its cell graph,
$c_{\rmtt[init]}$, $c_{\rmtt[goal]}$, and the set of way points
$p_{j}\, \in W_A, \, j\in J$ computed during the execution of {\tt
PathExist}.

\noindent \framebox{1. Construct an initial path} {\tt
ConstructPath} explores the cell graph of $W_A$, and constructs a
path in $W_A$ connecting $p_{\rmtt[init]}$ to $p_{\rmtt[goal]}$
and visiting all of the way points. Since there are at most $k$
way points, this can be done in $O(k^3)$ time (the path has $k+1$
segments each with $O(k^2)$ arcs).

\noindent \framebox{2. Construct {\em guards} and insert the {\rm
guards} into the path} Notice that when one accordion moves a leg
in a cell in which the number of long links is not $3$ (called
one-component cell), neither the signs of concatenating angles,
nor the sign between any pair of links in this leg will be kept
invariant. Thus even if the desired sign between a pair of long
links is adjusted at a way point, it still could change if the leg
keeps moving in a one-component cell. For this reason, we set {\em
guards} for legs which have three long links at $p_{\rmtt[goal]}$.
These are the set of points $q_j$, each of which is the last
intersection point between the above constructed path in $W_A$ and
the boundary of the two-component cell of leg $j$ containing
$p_{\rmtt[goal]}$. Thus the number of guards ($q_j$'s) may be more
than the number of way points since the number of legs that have
three long links at $p_{\rmtt[goal]}$ may be more than the
cardinality of $J$. Next the {\em guards} are inserted into the
path. Later when we construct the path in ${\cal C}$, sign-adjust
moves are only performed at {\em guards} $q_j$ (but not $p_j$) for
after that the thorax endpoint gets into the two-component cell
and the sign between a pair of long links will not change during
accordion moves, i.e., the leg will always remain in the right
component of its \cspace.  Assuming each arc in the path is
approximated by a fixed number of line segments, finding guards is
$O(k^3)$.

\noindent \framebox{3. Accordion moves and sign-adjust moves} The
path in ${\cal C}$ then is produced by using accordion moves along
the path and sign-adjust moves at the {\em guards}. At each {\em
guard}, one checks the sign between a pair of long links of the
corresponding leg. If it does not match the goal one, then the
junction point is fixed while a sign-adjust move is executed,
otherwise, the accordion move continues. Once $A$ is coincident
with $p_{\rmtt[goal]}$, one is assured by the previous steps, that
with $A$ fixed at $p_{\rmtt[goal]}$, the configuration of each leg
is in the same component of its current \cspace \ $\tilde
C_j(p_{\rmtt[goal]})$ as $c_{\rmtt[goal]}$. The final move can be
accomplished using a special accordion move algorithm found in
\cite{MT1}. At this stage, we remark that finding the set of way
points $p_j$ and planning an initial path visiting all $p_j$ is
necessary for otherwise, an arbitrary path between
$p_{\rmtt[init]}$ and $p_{\rmtt[goal]}$ may not intersect the
boundary of the two-component cell of a leg that contains
$p_{\rmtt[goal]}$.

The complexity of the accordion move algorithms reported in
\cite{MT2} are $O(N^3)$. Since the path has $O(k^3)$ line segments
the complexity of {\tt ConstructPath} is $O(k^3N^3)$. Note that
accordion move algorithms with the required behavior can be
designed to be $O(N^2)$, so the complexity of {\tt ConstructPath}
could be reduced.

Overall, our path planning algorithm is  $O(k^3N^3)$.

\sect{Path Optimization and Robustness} \label{section-4} If a
path between two given configurations exists, it is obvious that
in our algorithm the choice of way points, and thus the path
between the two configurations, is not unique. So a natural
problem is path optimization with respect to meaningful metrics
such as path length and singularity avoidance. Basically we say
that many possible optimization objectives are potentially useful,
but we consider the shortest path in this section. Notice that the
way points are necessary for successively constructing a path from
$c_{\rmtt[init]}$ to $c_{\rmtt[goal]}$ since an arbitrary chosen
path of the thorax from $p_{\rmtt[init]}$ to $p_{\rmtt[goal]}$
(like the line connecting them) will either go out of $W_A$, or
have no intersection with the critical set $\Sigma_j$, in which a
sign-adjust move is required for a leg. Thus the optimization
problem arises in the choice of way points, the order of the way
points, and the path between two consecutive way points.

One may also take into consideration parallel singularities (cf.
\cite{SG95}) and try to avoid them as much as possible by
minimizing the number of singularities crossings. More precisely
this may be done for example by first clustering singularity
regions (see \cite{DCSY03}) and then choosing way points with the
corresponding path having minimum number of crossings of these
regions, finally their connecting path or trajectories (i.e. a
path with temporal relations as well as the geometrical ones)
should be chosen in the proper way (cf. \cite{NTU00}).

Since ${\cal C}$ is a fibration over $W_A$, a meaningful metric
for ${\cal C}$ is
\begin{eqnarray}
\label{rieman-metric}
  ds^2=a_1 dp^Tdp+ a_2\sum_{j=1^k} du_j^TV_j^TV_jdu_j,\, a_1, a_2>0,
\end{eqnarray}
where $a_1$ and $a_2$ are two weights assigned to $dp^Tdp$ and
$\sum_{j=1^k} du_j^TV_j^TV_jdu_j$, respectively for they are
quantities with different physical meaning. The column vectors of
$V_j \in \reals^{n_j \times (n_j-2)}$ forms a basis for the null
space of the Jacobian $J_j=\frac{\partial f_j}{\partial \Theta_j}$
of leg $j$, i.e.,
\[
    J_jV_j=0.
\]
$du_j$ denotes the incremental changes of the local coordinates on
$\tilde {\cal C}_j(p)$. $dp$ and $V_j du_j$ stands for the
infinitesimal motion along the base manifold and the fibre,
respectively. The shortest path problem is to find a path
$(p(t),\Theta_1(t),\cdots,\Theta_k(t))$ such that
\begin{eqnarray}
\label{obj}
   \int_{t=0}^1 ds
\end{eqnarray}
is minimal. The optimal solution to (\ref{obj}) satisfies the
geodesic equation
\begin{eqnarray}
\label{geodesic-eqn}
   {\ddot v}_k+\Gamma_{ij}^k {\dot v}_i {\dot v}_j=0
\end{eqnarray}
where $v=[p^T,u_1^T,\cdots,u_k^T]^T$, and $\Gamma_{ij}^k$ denotes
the Christoffel symbol of the metric (\ref{rieman-metric}).
Solving (\ref{geodesic-eqn}) exactly is difficult. However, an
approximation solution can be derived. Since a path from
$c_{\rmtt[init]}$ to $c_{\rmtt[goal]}$ is globally optimal if and
only if this path is also locally optimal, we construct an
approximate shortest path in a way so that (i) $p(t)$ connects
$p_{\rmtt[init]}$ and $p_{\rmtt[goal]}$ and visits all $p_i$.
Moreover, $\int_{t=0}^1 \sqrt{dp^Tdp}$ is minimal; (ii) There is
minimal number of accordion moves, and each accordion move is
minimal; (iii) Except for the accordion moves, there is no other
motions along the fibre.
\begin{Remark}
\rm The path constructed in this way is shortest if no accordion
moves are needed for we always achieve minimal
$ds=\sqrt{a_1dp^Tdp}$ infinitesimally, while it is only an
approximation if there is at least one accordion move.
\end{Remark}
Mathematically, this problem can be described as
%
\[
    \begin{array}{c}
     \min  \int_0^1 \|dp\|   \\
      p(t) \in W_A, \forall t \in [0,1] \\
      p_i \in \{p(t)\}, \forall i \\
      p_i \in {\cal A}_{\delta(i)}
    \end{array}
\]
%
where ${\cal A}_j, j\in J$ is the boundary arcs of the
two-component cell of leg $j$ that contains $p_{\rmtt[goal]}$.
$\delta(J)$ is a permutation of $J$ with $\delta(J) =J$. Solving
this problem exactly is extremely hard, but a random search method
(for example, the Controlled Random Search Method \cite{BS96}) can
be used to quickly find a good approximate solution. Using the
minimal number of accordion moves has been solved in {\tt
ConstructPath}, and the minimal accordion move problem has been
solved in \cite{MT1}. Combining these two, (ii) is solved.

To solve (iii), we notice that for a local motion $dp=[dx,dy]$ of
the thorax endpoint, $d\Theta_j^Td\Theta_j$ is minimized if and
only if.
%
\[
   d\Theta_j=J_j^+ dp
\]
%
where $J_j^+$=$J_j^T(J_jJ_j^T)^{-1}$.

Another important issue about our planning algorithm is
robustness. The sign-adjust move of leg $j$ performed at a guard
$q_j$ is only feasible when ${\cal C}_j(q_j)$ is connected. Since
$q_j \in \Sigma_j$ which is only 1-D, a small perturbation of the
junction point in $W_A$ (e.g., due to numerical errors) will
violate the condition $p \in \Sigma_j$. When $p$ moves into a
two-component cell, then the sign-adjust move may fail. A remedy
to this is to modify the path of the thorax in the neighborhood of
$q_j \in \Sigma_j$ so that a point $q_j'$ in the interior of a
one-component cell is reached. After the sign of leg $j$ is
adjusted to the desired one with $p$ fixing at $q_j'$, we apply a
constrained accordion move algorithm to ensure that the leg $j$
stays in the right component of ${\cal C}_j(p)$ just before its
thorax endpoint enters the two-component cell containing
$p_{\rmtt[goal]}$.  This resulting algorithm will also be robust
to other errors such as the control and sensor errors if they are
taken into account.


\sect{Examples} \label{section-5}
%jct 1/16/06 - efficiency -> complexity? correctness?
In this section, we demonstrate the correctness and complexity of
our algorithm through two examples: a manipulator with three
three-link legs, and a manipulator with three five-link legs.
Movies of the motion plans are very helpful in understanding the
figures.  They can be found at
\verb$http://www.cs.rpi.edu/~trink/ccwo.html$.
%
\begin{figure}
  \centering
  \includegraphics[width=3in]{fig_2/initial_goal_config.eps}
  \caption{Manipulator's initial configuration (junction on the right, drawn red)
  and goal configuration (junction just below the top left, drawn blue.) The boundary
  circles of $W_j$ are drawn as dashed green lines.}
  \label{start-goal}
\end{figure}
%
\begin{figure}
  \centering
  \includegraphics[width=3in]{fig_2/workspace_path_2.eps}
  \caption{A path between $p_{\rmtt[init]}$ and $p_{\rmtt[goal]}$ that is completely
  contained in $W_A$.}
  \label{workspace-path}
\end{figure}
%
\begin{figure}
  \centering
  \includegraphics[width=3in]{fig_2/guards.eps}
  \caption{The two guard points are the last intersection points between the
  path of $A$ and the boundary circles of two two-component cells.}
  \label{guard}
\end{figure}
%
\begin{figure}
  \centering
  \includegraphics[width=3.5in]{fig_2/first_interval_2.eps}
  \caption{All legs use an accordion move to move the junction $A$
  to the first guard $q_1$ of leg $1$.}
  \label{accordion-1}
\end{figure}
%
\begin{figure}
  \centering
  \includegraphics[width=3.5in]{fig_2/second_interval_2.eps}
  \caption{With the junction $A$ at $q_1$, the joint angles of leg $1$
can be adjusted to achieve the signs required at the goal
configuration. All other legs are fixed in place.}
  \label{adjust_sign-1}
\end{figure}
%
\begin{figure}
  \centering
  \includegraphics[width=3.5in]{fig_2/third_interval_2.eps}
  \caption{All legs use an accordion move to move the junction $A$ to
  the second guard $q_2$ of leg $2$.}
  \label{accordion-2}
\end{figure}
%
\begin{figure}
  \centering
  \includegraphics[width=3.5in]{fig_2/fourth_interval_2.eps}
  \caption{With the junction $A$ at $q_2$, the joint angles of leg $2$
can be adjusted to achieve the signs required at the goal
configuration. All other legs are fixed in place.}
  \label{adjust_sign-2}
\end{figure}
%
\begin{figure}
  \centering
  \includegraphics[width=3.5in]{fig_2/five_interval_2.eps}
  \caption{All legs use an accordion move to move the junction $A$ to its
  goal location.  The signs of the joint angles are preserved guaranteeing
  that legs~1 and~2 will be in the correct \cspace \ component once $A$ is fixed at
  the goal position.}
  \label{accordion_goal}
\end{figure}
%
\begin{figure}
  \centering
  \includegraphics[width=3.5in]{fig_2/six_interval_2.eps}
  \caption{All legs use the Trinkle-Milgram algorithm to achieve their
  goal configurations with the junction $A$ fixed.}
  \label{TMalg}
\end{figure}

In the first example, two of the three legs of the manipulator
have three long links when $A$ is fixed at $p_{\rmtt[goal]}$.
Figure~\ref{start-goal} shows the manipulator in its starting and
goal configurations. Our algorithm predicts $J=\emptyset$.

Then the algorithm constructs a path in $W_A$ from
$p_{\rmtt[init]}$ to $p_{\rmtt[goal]}$, drawn as the dark solid
lines in Fig. \ref{workspace-path}.  This path intersects

the boundary of the two-component annular region of leg $j$ that
contains

$p_{\rmtt[goal]}$ several times, among which $q_j$, $j=1,2$ are
the last ones. These two points are the guards (drawn as diamonds)
where sign-adjust moves are performed.

At $q_j$, $j=1,2$, we check the sign of a pair of long links of
leg~$j$ and see if it matches its sign at the goal. If not, we fix
the other two legs and adjust the sign of the chosen long links in
leg $j$. In this particular example, we chose the two longest
links as the pair of long links,and we find that at $q_1$ the sign
of leg $1$ does not match that at the goal (while at $q_2$, leg
$2$ has the same sign as the goal). Before leaving $q_j$ via the
next accordion move, the pair of long links of leg $j$ was moved
to the elbow-opposite configuration (recall that there are two
configurations for these two links, one is ``elbow up", the other
is ``elbow down"), which has exactly the same sign as the goal
configuration. The Trinkle-Milgram algorithm \cite{MT2} is used to
plan such a motion between the two elbow-opposite configurations.
Figures~\ref{accordion-1} \--\ \ref{TMalg} show the progress of
the manipulation plan as the steps of the complete planning
algorithm are carried out.

A bit more complex example in which the star-shaped manipulator
has three five-link legs is shown in
\verb$http://www.cs.rpi.edu/~trink/ccwo.html$. The computation
time for path existence for star-shaped manipulators with less
than 10 legs, and legs of less than 10 links is typically from
less than 1 second to a few seconds when run in a Matlab, P4,
WindowsXP system.



\sect{Discussion} Star-shaped manipulators are closed chain
manipulators subject to multiple loop closure constraints. The
\cspace \ of these manipulators is often a lower-dimensional
submanifold with high genus \footnote{The genus of a surface is
defined as the largest number of nonintersecting simple closed
curves that can be drawn on the surface without separating it.}
embedded in the ambient space. Computing the silhouette of this
manifold requires solving the extreme points of the manifold
either in the ambient space whose dimension is much higher than
that of the manifold itself, or in a set of local neighborhoods
(local coordinate charts) whose number grows exponentially along
with the genus of the submanifold. Although Canny's algorithm is
very efficient in general, there is difficulty in implementation
for star-shaped manipulators. Second, the classical cylindrical
decomposition of \cspace \ (e.g. collin's decomposition) is a
partition into simple connected subsets of \cspace \ called cells.
However, this algorithm requires a description of the \cspace \ in
terms of a set of polynomials over its ambient space. Again
because the dimension of the ambient space could be very high, the
computation time of this algorithm could become formidable.

Our algorithm employs the special structural properties (fibration
over the workspace) of the \cspace \ of star-shaped manipulators.
It avoids using the coordinates of the ambient space as well as
the local coordinate charts that covers the \cspace. In our
algorithm the path existence and path construction are solved in
polynomial time by combining the cell decomposition of the
workspace (which is two dimensional and with simple shape) and the
structure of the \cspace \ of single-loop closed chains. The
critical set $\Sigma_j$, which marks the change of the topology of
the \cspace \ of each leg, plays a key role in this algorithm.

\sect{Conclusion} \label{section-6} In this paper, we studied the
global structural properties of planar star-shaped manipulators.
Via the analysis of the critical set $\Sigma$, we derived the
global connectivity of the \cspace, and necessary and sufficient
conditions for path existence. Based on these results, we devised
a complete polynomial algorithm for motion planning. Simulation
examples were used to illustrate the key ideas behind the motion
planning problem of planar star-shaped manipulators.

\chapter{The configuration space of a parallel polygonal mechanism}
\label{chap3}

\textsf{Submitted to \textsl{Homology Homotopy and Applications}, 2007.}\\
\textsf{Co-authors: David Blanc and Moshe Shoham.}\\

 We study the  \cspace\ $\C$ of a parallel polygonal
mechanism, and give necessary conditions for the existence of
singularities; this shows that generically $\C$ is a smooth
manifold. In the planar case, we construct an explicit Morse
function on $\C$, and show how geometric information about the
mechanism can be used to identify the critical points.

%
%s1  Introduction}
%
\sect{Introduction} \label{cint}

The mathematical theory of robotics is based on the notion of a
mechanism, consisting of links, joints, and rigid parts known
as platforms.  The \emph{type} of a mechanism is defined by a
$q$-dimensional polyhedral complex, where the parts of dimension
$\geq 2$ \ correspond to the platforms, and the complementary
one-dimensional graph corresponds to the links and joints.
Here we consider only the polygonal case  \ ($q=2$). \
A specific embedding of this complex in the ambient Euclidean space  \
$\RR{d}$ \ is called a \emph{configuration} of the mechanism. The
collection of all such embeddings forms a topological space, called the
\emph{\cspace} of the mechanism (see \cite{Ha}).

\begin{figure}[htb]
\begin{center}
\epsfysize=4cm %\epsfxsize=6.4cm
\leavevmode \epsffile{fig_3/k_gon_mechanism.eps} \caption{A
pentagonal planar mechanism} \label{fig:kgon}
\end{center}
\end{figure}

The goal of this note is to study the \cspace\ of a mechanism
consisting of a moving polygonal platform, having a
flexible leg (consisting of concatenated rods) attached to each
vertex, with the other end fixed in  \ $\RR{d}$. \ We may think of the
latter as forming the \emph{fixed polygon} of the mechanism, ``parallel''
to the \emph{moving polygon} inside.
The spatial version of such a mechanism, consisting of a
two-dimensional platform free to move in three dimensions, has been
studied extensively, but even the planar version, to which we
later specialize, has practical applications \ -- \ for example, in
micro-electro-mechanical systems (MEMS)\vs.

Our main results are\vsm :

\begin{enumerate}
\renewcommand{\labelenumi}{(\alph{enumi})}
%
\item The \cspace\ of a parallel polygonal mechanism is a smooth
  manifold, except perhaps in some explicitly described  singular
  cases (Theorem \ref{tone}).
%
\item The topology of this manifold can be described for a triangular
  planar mechanism by means of an explicit Morse function (Theorem \ref{ttwo}),
  whose critical points can be identified geometrically (\S \ref{scrit})\vs.
%
\end{enumerate}

We start with some terminology and notation:

\begin{defn}\label{dbranch}
%
A \emph{branch} (of \emph{multiplicity} $n$) is a sequence \
$L=(\ell_{1},\dotsc,\ell_{n})$ \ of $n$ positive numbers, which we
think of as the lengths of $n$ concatenated rods, having revolute
(i.e., rotational) joints at the consecutive meeting points.
%
\end{defn}

\begin{defn}\label{dbconfiguration}
%
A \emph{configuration} in \ $\RR{d}$ \ for a branch \
$L=(\ell_{1},\dotsc,\ell_{n})$ \ consists of $n$ vectors \
$V=(\bv_{1},\dotsc,\bv_{n})$ \ in \ $\RR{d}$ \ with lengths \
$\|\bv_{i}\|=\ell_{i}$ \ ($i=1,\dotsc n$). \ We write \
$\sigma_{V}:=\sum_{i=1}^{n}\,\bv_{i}$ \ for their vector sum.

A branch configuration \ $V=(\bv_{1},\dotsc,\bv_{n})$ \ is
\emph{aligned} with a vector \ $\bv\in\RR{d}$ \ if all the vectors
\ $\bv_{i}$ \ are scalar multiples of $\bv$, which is called the
\emph{direction vector} of $V$. \newnot{3-3}
%
\end{defn}

The \emph{\cspace} \ $\C(L)$ \ of a branch $L$ is the product of $n$
spheres in \ $\RR{d}$ \ of radii \ $(R_{i}=\ell_{i})_{i=1}^{n}$. \
Up to homeomorphism, this is independent of the order on $L$, so we can (and
shall) assume \ $\ell_{1},\dotsc,\ell_{n}$ \ to be in descending order.

\begin{defn}\label{dkmechanism}
%
A \emph{polygonal mechanism} \ $(\Lc,\Xc,\Pc)$ \ in \ $\RR{d}$ \
consists of: \newnot{3-2}

\begin{enumerate}
\renewcommand{\labelenumi}{(\alph{enumi})}
%
\item $k$ branches \ $\Lc=\{\Lj{i}\}_{i=1}^{k}$ \ of multiplicity \
$\{n^{(i)}\}_{i=1}^{k}$, \ respectively;
%
\item $k$ distinct \emph{base points} \ $\Xc=\{\xj{i}\}_{i=1}^{k}$ \ in \
  $\RR{d}$, \ to which the initial points of the corresponding
  branches are attached.
%
\item An abstract $k$-polygon $\Pc$ in \ $\RR{d}$.
%
\end{enumerate}

Think of this mechanism as a linkage of $k$ branches, starting at
the base points (which form a polygon (not necessarily planar) in \
$\RR{d}$, \ called the  \emph{fixed platform}), and ending at the
vertices of a rigid planar polygon congruent to $\Pc$, called the
\emph{moving platform} of the mechanism. There are revolute joints
at both ends of each branch, too.

We use parenthesized superscripts to
indicate the branch number, and plain subscripts to indicate the rod
number \ -- \ e.g., \ $\ell_{j}^{(i)}$ \ denotes the length of the
$j$-th rod of the $i$-th branch.
%
\end{defn}

\begin{remark}\label{rpolygon}
%
Observe that a planar polygon $\Pc$ in \ $\RR{d}$ \ ($d>2$) \ with
vertices \ $\ppj{1}$, $\dotsc$, $\ppj{k}$ \ is determined up to
isometry by the sequence of triangles \
$\triangle(\ppj{i},\ppj{i+1},\ppj{i+2})$ \ and \
$\triangle(\ppj{i},\ppj{i+1},\ppj{i+3})$, \ which are determined
in turn by the lengths of their sides \
$\Gc~:=~(\gj{i,j})_{(i,j)\in\Ic}$. \ Here \
$\gj{i,j}:=\|\ppj{i}-\ppj{j}\|$,  \ and the index set $\Ic$
consists of the \ $3k-6$ \ ordered pairs: \newnot{3-4}
$$
(1,2),(1,3),(2,3),(1,4),(2,4),(3,4),\dotsc,(k-3,k),(k-2,k),(k-1,k)
$$
%
(in this order). Note that these diagonals force $\Pc$ to be planar, since
if we assume that the polygon is contained in the affine plane $\Ec$
determined by the first three vertices in \ $\RR{d}$, \ then $\Pc$ is
constructed inductively by adding one vertex at a time to an existing
edge to form a new triangle.


When \ $d=2$, \ the \ $2k-3$ \ pairs:
$$
\Ic=\{(1,2),(1,3),(2,3),(2,4),(3,4),\dotsc,(k-2,k-1),(k-1,k)\}
$$
%
suffice to determine $\Pc$ completely, if it is convex; otherwise, the
only additional data needed is the discrete information as to which
half-plane the new vertex is to be placed in.
%
\end{remark}

\begin{defn}\label{dconfiguration}
%
A \emph{configuration} for a polygonal mechanism \ $(\Lc,\Xc,\Pc)$
\ in \ $\RR{d}$ \ consists of a set \ $\Vc=(\Vj{1},\dotsc,\Vj{k})$
\ of $k$ branch configurations for $\Lc$ (Definition
\ref{dbconfiguration}), satisfying the condition that the
endpoints \ $\ppj{i}:=\xj{i}+\sigma_{\Vj{i}}$ \ ($i=1,\dotsc,k$) \
of the corresponding branch configurations (attached to the given
basepoints) form a planar polygon congruent to $\Pc$ in \
$\RR{d}$. \ If the branch configuration \ $\Vj{i}$ \ is aligned,
with direction vector \ $\vj{i}$ \ (Definition
\ref{dbconfiguration}), then the line \
$\Line\up{i}:=\{\xj{i}+t\vj{i}~|\ t\in\RR{}\}$ \ is called the
\emph{direction line} for \ $\Vj{i}$ \ (with \
$\ppj{i}\in\Line\up{i}$).\newnot{3-5}

The set of all configurations for the given mechanism \
$(\Lc,\Xc,\Pc)$ \ (as a subspace of the product of
the appropriate branch \cspace s), is its \emph{\cspace} \
$\C=\C(\Lc,\Xc,\Pc)$.
%
\end{defn}

\begin{defn}\label{dwork}
%
Note that the moving platform $\Pc$ can be translated and rotated
in \ $\RR{d}$ \ (subject to the constraints imposed by the branches
and the locations of the fixed vertices). The space of all allowable
positions for $\Pc$, denoted by \ $\Wc=\Wc(\Lc,\Xc,\Pc)$, \ is called
the \emph{\wspace} for \ $(\Lc,\Xc,\Pc)$. \ The \emph{work map} \
$\Phi:\C\to\Wc$ \ assigns to each configuration $\Vc$ the
resulting position of $\Pc$.
%
\end{defn}

\begin{mysubsection}[\label{sorg}]{Organization}
%
In section \ref{cmain} we show that the \cspace s $\C$ we consider here, in
any ambient dimension, are manifolds (generically). In section
\ref{cplan} we describe a Morse function for the \cspace\ of a generic planar
mechanism, analyze its critical points geometrically, and give a
simple example showing how this analysis may be used to recover $\C$.
%
\end{mysubsection}

%
%s2   Generic polygonal mechanisms in $\RR{d}$}
%
\sect{Generic polygonal mechanisms} \label{cmain}

We now show that, generically, the \cspace\ of a polygonal
mechanism is a manifold. Of course, it may be smooth even when  \
$(\Lc,\Xc,\Pc)$ \ is not generic, but in such cases singularities
can occur (cf.\ \cite{FSchuH} and \cite{KT}), and their analysis
is of interest in relation to the kinematic singularities.

\begin{defn}\label{dgeneric}
%
A configuration \ $\Vc=(\Vj{1},\dotsc,\Vj{k})$ \ of a polygonal
mechanism \ $(\Lc,\Xc,\Pc)$ \ is called \emph{singular} if:
%
\begin{enumerate}
\renewcommand{\labelenumi}{(\alph{enumi})~}
%
\item Two of its branch configurations \ $\Vj{i_{1}}$ \ and \
 $\Vj{i_{2}}$ \ are aligned, with coinciding direction lines: \
$\Line\up{i_{1}}=\Line\up{i_{2}}$ \ (see Figure \ref{fsingular}).

\begin{figure}[htb]
\begin{center}
\epsfysize=6cm \leavevmode
\epsffile{fig_3/critical_point_triangle.eps} \caption{Singular
configuration of type (a)} \label{fsingular}
\end{center}
\end{figure}

\item Three of its branch configurations are aligned, with direction lines
in the same plane meeting in a single point.
%
\item Four of its branch configurations are aligned, with
  direction lines in the same plane.
%
\end{enumerate}

The mechanism \ $(\Lc,\Xc,\Pc)$ \ is
called \emph{generic} if none of its configurations are singular
(compare \cite{Hau}).
%
\end{defn}

\begin{thm}\label{tone}
%
The \cspace\ \  \ $\C=\C(\Lc,\Xc,\Pc)$ \ of a generic polygonal mechanism
in \ $\RR{d}$ is a smooth closed orientable manifold of dimension \
$N(d-1)-|\Ic|$, \ where $k$ is the number of vertices of $\Pc$ \ and \
$N:=\sum_{i=1}^{k}\,{\nj{i}}$.
%
\end{thm}
\newnot{3-1}
\noindent\textit{Proof.} \ For any $n$, define a map \
$f_{n}:(\RR{d})^{n}\to\RR{n}$ \ by:
$$
f_{n}(\bv_{1},\dotsc,\bv_{n})~:=~(|\bv_{1}|^{2},\dotsc,|\bv_{n}|^{2})~.
$$

Consider the \emph{constraint map} \ $F:\RR{dN}\to\RR{N+|\Ic|}$, \ defined:
%
\begin{equation}\label{econstr}
%
F(\Vc)~=~F(\Vj{1},\dotsc,\Vj{i})~:=~
(f_{\nj{1}}(\Vj{1}),\dotsc,f_{\nj{k}}(\Vj{k}),
\|\aj{1,2}\|^{2},\dotsc,\|\aj{k-1,k}\|^{2})~,
%
\end{equation}
%
where \ $\ppj{i}:=\xj{i}+\sum_{t=1}^{\nj{i}}\,\vj{i}_{t}$ \ is the
endpoint of the $i$-th branch for the configuration \
$\Vj{i}=(\vj{i}_{1},\dotsc,\vj{i}_{\nj{i}})$ \ attached to the
basepoint \ $\xj{i}\in\RR{d}$, \ and \ $\aj{i,j}:=\ppj{i}-\ppj{j}$ \
is the \ $(i,j)$-diagonal of the polygon spanned by these endpoints.

Recall from \S \ref{rpolygon} that $\Pc$ determines the set of
diagonals \ $\Gc=(\gj{i,j})_{(i,j)\in\Ic}$, \ where \
$$
|\Ic|~=~\begin{cases} 3k-6 & \text{if~}d>2\\
2k-3 & \text{if~}d=2
\end{cases}
$$
%
Let:
%
\begin{equation*}
%
Z_{(\Lc,\Gc)}~:=~((\lj{1}_{1})^{2},\dotsc,(\lj{1}_{\nj{1}})^{2},\dotsc
((\lj{k}_{1})^{2},\dotsc,(\lj{k}_{\nj{k}})^{2},\
(\gj{1,2})^{2},\dotsc,(\gj{k-1,k})^{2})
%
\end{equation*}

The \cspace\ \ $\C=\C(\Lc,\Xc,\Pc)$ \ is the pre-image
of \ $Z_{(\Lc,\Gc)}\in \RR{N+|\Ic|}$ \ under the function $F$; \
if \ $d=2$, \ $\C$ is one connected component of this pre-image\vsm.

$\C$ will be a smooth manifold if \ $Z_{(\Lc,\Gc)}$ \ is a regular
value of $F$ \ -- \ i.e., if  \ $\dF_{\Vc}$ \ is of rank \ $N+|\Ic|$ \
(see \cite[I, Theorem 3.2]{Hi}). \ We calculate:
%
\begin{equation}\label{edf}
%
\dF_{\Vc}~=~2
\begin{pmatrix}
%
A\up{1}    & 0         &  0           &  0          & \dotsc  & 0  \\
 0         &A\up{2}    &  0           &  0          & \dotsc  & 0  \\
 0         & 0         & A\up{3}      &             & \dotsc  & 0  \\
 0         & 0         &  0           & A\up{4}     & \dotsc  & 0  \\
 \vdots    & \vdots    &  \vdots      & \vdots      & \ddots  & \vdots \\
 0         & 0         &  0           & 0           & \dotsc  & A\up{k} \\
& & & & & \\
\bj{1}{2}  & \bj{2}{1} & 0            & 0 &  \ldots     & 0          \\
\bj{1}{3}  & 0         & \bj{3}{1}    & 0 &  \ldots     & 0          \\
 0         &\bj{2}{3}  & \bj{3}{2}    & 0 &  \ldots     & 0          \\
\vdots     & \vdots    &  \vdots      &  \vdots     & \ddots  & \vdots \\
 0         & \dotsc    & 0 & \bj{k-2}{k}  & 0           & \bj{k}{k-2}\\
 0         & 0    & \dotsc & 0             & \bj{k-1}{k} & \bj{k}{k-1}
\end{pmatrix}\vspace{4mm},
%
\end{equation}
%
where \ $A\up{i}$ \ is the  \ $\nj{i}\times d\nj{i}$ \ matrix:
%
\begin{equation}\label{eai}
\begin{pmatrix}
%
\vj{i}_{1} & \dotsc & 0     \\
\vdots     & \ddots & \vdots \\
0          & \dotsc & \vj{i}_{\nj{i}}
\end{pmatrix}
%
\end{equation}
%
\noindent with rows denoted by: \ $\uj{i}{1},\dotsc,\uj{i}{\nj{i}}$.

The \ $|\Ic|$ \ bottom rows \ $(\wj{i}{j})$ \ of \ $\frac{1}{2}\dF$ \
are indexed by \ $(i,j)\in\Ic$, \ where \
$$
\wj{i}{j}:=
(\underbrace{0,\dotsc,0}_{d(\nj{1}+\dotsc+\nj{i-1})},
\bj{i}{j},\underbrace{0,\dotsc,0}_{d(\nj{i+1}+\dotsc+\nj{j-1})},
\bj{j}{i},\underbrace{0,\dotsc,0}_{d(\nj{j+1}+\dotsc+\nj{k})})~,
$$
%
each edge \ $\aj{i,j}\in\RR{d}$ \ appearing \ $\nj{i}$ \ times in
\newnot{3-6}
$$
\bj{i}{j}~:=~\underbrace{(\aj{i,j},\aj{i,j},\dotsc,\aj{i,j})}_{\nj{i}}~.
$$

If we think of the bottom rows as forming $N$
blocks of \ $|\Ic|\times d$ \ matrices \ $B_{1},\dotsc, B_{N}$, \
then:
%
\begin{equation}\label{eblock}
\sum_{r=1}^{N}\ B_{r} ~=~ 0\vsm.
\end{equation}

Let \ $\Vc\in\C=F^{-1}(Z_{(\Lc,\Gc)})$, \ and consider a vanishing
linear combination of the rows of \ $\dF_{\Vc}$:
$$
\sum_{i=1}^{k}~\left(\sum_{j=1}^{\nj{i}}\,\lambda\up{i}_{j}\uj{i}{j}\right)~+~
\sum_{(i,j)\in\Ic}\,\gamma\up{i,j}\wj{i}{j}~=~\vz~.
$$

For each \ $1\leq i\leq k$, \ let:
$$
\vj{i}~:=~\sum_{\substack{(s,i)\in\Ic\\ s<i}}\,\gamma\up{s,i}\cdot\aj{s,i}
~-~\sum_{\substack{(i,t)\in\Ic\\ i<t}}\,\gamma\up{i,t}\cdot\aj{i,t}~,
$$
%
and note that if \ $\lambda\up{i}_{j'}\neq 0$ \ for \emph{some} \
$1\leq j'\leq\nj{i}$, \ then this holds for all \ $1\leq j\leq\nj{i}$, \
so that \ $\vj{i}_{j}~=~\frac{1}{\lambda\up{i}_{j}}\cdot\vj{i}$ \
(the $i$-th branch is aligned with direction vector \ $\vj{i}$).

By \eqref{eblock}, \ $\sum_{i=1}^{k}\,\vj{i}=\vz$, \ so if \
$\vj{i}=\vz$ \ for \ $i\neq i_{0},i_{1}$, \ then \
$\vj{i_{0}}+\vj{i_{1}}=\vz$, \ and thus branches \ $i_{0}$ \ and \
$i_{1}$ \ are co-aligned with direction vector \ $\aj{i_{0},i_{1}}$. \
This cannot happen if \ $(\Lc,\Xc,\Pc)$ \ is generic
(cf.\ \S \ref{dgeneric}(a)). \ Similarly, no more than three
branches can be aligned (\S \ref{dgeneric}(c)).

Unfortunately, even for generic mechanisms \ $\dF$ \ need not be of maximal
rank. Thus, to complete the proof of the Theorem we need the
following:
%
%\end{proof}

\begin{prop}\label{pzero}
%
For a generic polygonal mechanism, any configuration $\Vc$ having
at most three aligned branches is smooth.
%
\end{prop}

\begin{proof}
%
Without loss of generality we may assume that the three aligned
branches are \ $1,2$ \ and $k$, \ with direction vectors \
$\oj{1}$,\ $\oj{2}$, \ and \ $\oj{k}$, \ respectively.

Let $\hC$ denote the configuration space of the mechanism obtained
from \ $(\Lc,\Xc,\Pc)$ \ by omitting the last branch, and \ $\Ck$
\ the configuration space for this branch (attached to \
$\xj{k}\in\RR{d}$). \newnot{3-2}

The \emph{\wspace} of both mechanisms (i.e., the set of possible
locations for the $k$-th vertex of $\Pc$) is contained in \ $\RR{d}$, \
and we have \emph{work maps} \ $\psi:\hC\to\RR{d}$ \ and \
$\phi:\Ck\to\RR{d}$ \ which associate to each configuration the
location of this vertex.

Note that the manifold \ $\Ck$ \ (an $n$-torus for \ $n=\nj{k}$) \
is the preimage of \ $Z_{\lj{k}}$ \ under the map \
$f=f_{n}:\RR{nd}\to\RR{n}$, \ and we write \ $i:\Ck\to\RR{nd}$ \ for
the inclusion. Similarly, $\hC$ is determined by a smooth
constraint map:
$$
\hF:\RR{M}\times\Sc\to\RR{N-n+|\hI|}
$$
%
for \ $M=(N-n)d$, \ where $\Sc$ is a point if \ $k\geq 4$ \ and \
$\Sc=S^{d-2}$ \ if \ $k=3$. \ Here
%
\begin{equation}\label{econstn}
%
\begin{split}
%
\hF(\hV)~=&~\hF(\Vj{1},\dotsc,\Vj{k-1},\bw)~\\
:=&~(f_{\nj{1}}(\Vj{1}),\dotsc,f_{\nj{k-1}}(\Vj{k-1}),\,
\|\aj{1,2}\|^{2},\,\dotsc,\|\aj{k-2,k-1}\|^{2})~
%
\end{split}
%
\end{equation}
%
(with the functions \ $\|\aj{i,j}\|^{2}$ \ indexed by \
$(i,j)\in\hI$). \ The vector \ $\bw\in S^{\Delta}$ \ is needed for \
$k=3$ \ and \ $d>2$, \ since in that case the location of the first \
$k-1$ \ vertices of $\Pc$ determines the position of the moving
platform only up to rotation of \ $\aj{1,3}$ \ around the given edge \
$\aj{1,2}$ \ (relative to the plane \
$\Ec=\Ec(\vj{1}_{\nj{1}},\aj{1,2})$ \ spanned by \
$\vj{1}_{\nj{1}}$ \ and \ $\aj{1,2}$).

Thus $\hC$ is \ $\hF^{-1}(Z_{(\hat{\Lc},\hat{\Gc})})$ \
for the obvious \ $(\hat{\Lc},\hat{\Gc})$. \
Moreover, \ $(\hat{\Lc},\hat{\Gc})$ \ is a regular value of $\hF$,
as in the proof of Theorem \ref{tone}, because \ $(\Lc,\Xc,\Pc)$ \
is generic, so no branches but \ $1,2,k$ \ are aligned.
Thus $\hC$ is a smooth submanifold, and we write \
$\hi:\hC\to\RR{M}$ \ for the inclusion.

Let \ $X:=\Ck\times\RR{M}\times\Sc$, \
$Y:=\RR{d}\times\RR{M}\times\Sc$, \ and define \
$h:X\to Y$ \ to be the product map \
$\phi\times\Id_{\RR{M}}\times\Id_{\Sc}$ \ and \
$g:\hC\to Y$ \ to be \ $(\psi,\hi)$, \ so that $g$ is an embedding of
$\hC$ as a submanifold in $Y$. Since \ $\C=\C(\Lc,\Xc,\Pc)$ \ is
simply the pullback of:
$$
\Ck~\xrightarrow{\phi}~\RR{d}~\xleftarrow{\psi}~\hC~,
$$
%
it may be identified with the preimage of the submanifold \
$\hC\subseteq Y$ \ under $h$\vsm.

Let \ $\hV\in\hC$ \ be a configuration where the first two branches
are aligned (but not co-aligned), with direction vectors in the plane
$\Ec$ determined by the polygon $\Pc$ \ -- \ and let \ $\Vj{k}\in\Ck$ \ be an
aligned configuration  with direction vector \ $\vj{k}$. \ Assume
that \ $\psi(\hV)=\phi(\Vj{k})$, \ and let \ $\bx\in X$ \ be the
configuration \ $(\Vj{k},\hi(\hV))$, \ so that \ $h(\bx)=g(\hV)$. \
We must therefore show that the point \ $\Vc\in\C$ \
defined by \ $(\hV,\Vj{k})$ \ is smooth:

$$
\xymatrix@R=25pt{
\hspace*{3.5mm}\bx\in X\ar@<2.7ex>[d]_{h} & = &
\Ck\ar@<-2.8ex>[d]_{\phi}\ni\Vj{k} & \times &
\RR{M}\times\Sc\ar@<-3.7ex>[d]_{\Id}\ni\hi(\hV)\\
h(\bx)\in Y           & = & \hspace*{2mm}\RR{d}\ni\phi(\Vj{k})  & \times &
\RR{M}\times\Sc\ni\hi(\hV) \\
\hspace*{0.5mm}\hV\in\hC
\ar@<-2.7ex>[u]^{g}\ar[urr]^{\psi}\ar[urrrr]_{\hi} & & & &
}
$$

By \cite[I, Theorem 3.3]{Hi}), it suffices to show that \
$h\pitchfork\hC$  \ -- \ i.e., $h$ is locally transverse
to $\hC$ at the points \ $\bx\in X$ \ and \ $\hV\in\hC$. \
In other words:
%
\begin{equation}\label{etransv}
%
\Image\dhh_{\bx}~+~T_{\hV}(\hC)~=~T_{\hV}(Y)~=~\RR{d}\times\RR{M+\Delta}~.
%
\end{equation}

First note that since \ $\Ck=f^{-1}(Z_{\lj{k}})\subseteq\RR{nd}$, \
the tangent space \ $T_{\Vj{k}}(\Ck)$ \ may be identified with the
kernel of \ $\df_{\Vj{k}}:\RR{nd}\to\RR{n}$, \ which
is the null space of the matrix \ $A\up{k}$ \ of \eqref{eai}. Since \
$\Vj{k}$ \ is aligned, by assumption, with direction vector \
$\oj{k}$, \ we see that:
%
\begin{equation*}
%
\begin{split}
%
T_{\Vj{k}}(\Ck)~\cong~&
\{(\yj{k}_{1},\dotsc,\yj{k}_{n})\in\RR{nd}~|\
\yj{k}_{1}\cdot\oj{k}=0,\ \dotsc,
\yj{k}_{n}\cdot\oj{k}=0\}\\
=&~\underbrace{\oj{k}^{\perp}\times\dotsc\times\oj{k}^{\perp}}_{n}
%
\end{split}
%
\end{equation*}

Furthermore, \ $\phi:\Ck\to\RR{d}$ \ extends to \
$\hat{\phi}:\RR{nd}\to\RR{d}$ \ (so that \ $\hat{\phi}\circ i=\phi$), \
with
$$
\hat{\phi}(\bv_{1},\dotsc,\bv_{n})~=~\xj{k}+\bv_{1}+\dotsb+\bv_{n}~.
$$

Since \ $\hat{\phi}$ \ is linear, its differential \ $\dd\hat{\phi}$ \
is represented by the \ $d\times nd$ \ matrix \
$(I_{d},I_{d},\dotsc,I_{d})$ \ ($n$ blocks).

Thus:
%
$$
\Image\dhh_{\bx}~=~\oj{k}^{\perp}\times\RR{M+\Delta}~,
$$
%
\noindent so that in order for \eqref{etransv} to hold it suffices to
prove that:
%
\begin{equation}\label{eomega}
%
\oj{k}\in T_{\hV}(\hC) \vsm.
%
\end{equation}

Now the tangent space \ $T_{\hV}(\hC)$ \ may be identified with the
kernel of:
$$
\dd\hF_{\hV}:\RR{M+\Delta}\to\RR{N-n+2k-5}~,
$$
%
where \ $\dd\hF_{\hV}$ \ is described as in \eqref{edf} by the matrix:

\begin{equation}\label{edfn}
%
\dd\hF_{\hV}~=~
2\begin{pmatrix}
%
A\up{1}   & 0         &  \dotsc  &      0      & 0 \dotsc 0 \\
 \vdots   & \vdots    &  \ddots  &    \vdots   & \vdots     \\
 0        & 0         &  \dotsc  & A\up{k}     & 0 \dotsc 0 \\
          &           &          &             &            \\
\bj{1}{2} & \bj{2}{1} &  \dotsc  &       0     & 0 \dotsc 0 \\
\vdots    & \vdots    &  \ddots  &  \vdots     & 0 \dotsc 0 \\
 0        & 0         & \dotsc   & \bj{k}{k-1} & 0 \dotsc 0
\end{pmatrix}\vspace{4mm}\quad
%
\end{equation}

Since the first two branches are aligned with
direction vectors \ $\oj{1}$,\ $\oj{2}$, \ $T_{\hV}(\hC)$ \
may be identified with the set of \ $N-n$ \ $d$-dimensional vectors \
$$
\yj{1}_{1},\dotsc,\yj{1}_{\nj{1}};
\yj{2}_{1},\dotsc,\yj{2}_{\nj{2}};\dotsc;
\yj{k-1}_{1},\dotsc,\yj{k-1}_{\nj{k-1}}~,
$$
%
together with \ $\bz\in\RR{\Delta}$, \ where the first \ $\nj{1}$ \
vectors are all in \ $\oj{1}^{\perp}$, \ the next \ $\nj{2}$ \ are all
in \ $\oj{2}^{\perp}$, \ and the remainder are in individual orthogonal
complements:
$$
(\vj{3}_{1})^{\perp}\times\dotsb\times(\vj{3}_{\nj{3}})^{\perp}\times
(\vj{k-1}_{1})^{\perp}\times\dotsb\times(\vj{k-1}_{\nj{k-1}})^{\perp}~.
$$

If we let \ $\vy{i}:=\sum_{t=1}^{\nj{i}}\,\yj{i}_{t}$ \
($i=1,\dotsc,k-1$), \ these must satisfy:
$$
\aj{i,j}\cdot(\vy{i}-\vy{j})~=~0
$$
%
for each \ $(i,j)\in\hI$.

Likewise, \ $\psi:\hC\to\RR{d}$ \ extends to \
$\hat{\psi}:\RR{M}\times\Sc\to\RR{d}$, \  with
$$
\hat{\psi}(\Vj{1},\dotsc,\Vj{k-1},\bw)~=~
\xj{1}+\vj{1}_{1}+\dotsb+\vj{1}_{\nj{1}}~+~R_{\bw}\cdot\aj{1,2}~,
$$
%
where \ $R_{\bw}$ \ is \ $\lambda\cdot B_{\bw}\cdot A_{\theta}$, \
for \ $\lambda:=\lj{1,k}/\lj{1,2}$, \ $B_{\bw}$ \ the rotation matrix
about \ $\aj{1,2}$ \ determined by \ $\bw\in\Sc$ \ (if \ $\Sc=S^{\Delta}$), \
and \ $A_{\theta}$ \ rotates \ $\aj{1,2}$ \ (in the plane $\Ec$ spanned by \
$\vj{1}_{\nj{1}}$ \ and \ $\aj{1,2}$) \ by the angle $\theta$ to the
side \ $\aj{1,3}$ \ in the given polygon $\Pc$.

Thus \ $\dd\hat{\psi}_{(\Vj{1},\dotsc,\Vj{k-1},\bw)}$ \ is represented
by the \ $(M+\Delta)\times d$ \ matrix:
%
\begin{equation}\label{epsi}
%
\left[~ \underbrace{I_{d}+R_{\bw} | \dotsc | I_{d} + R_{\bw}}_{\nj{1}}~|~
\underbrace{-R_{\bw} | \dotsc | -R_{\bw}}_{\nj{2}}~|~
\underbrace{ 0\dotsc 0 }_{d\nj{3}}~|~\dotsc~|~
\underbrace{ 0\dotsc 0 }_{d\nj{k-1}}~|~
\frac{\partial R_{\bw}}{\partial\bw}\right]~.
%
\end{equation}

Therefore, the image of \ $\dd\psi$ \ is obtained by applying
\eqref{epsi} to \ $T_{\hV}(\hC)\subseteq\RR{M+\Delta}$ \ described
above, and we see that \ $\Image(\dd\psi_{\hV})$ \ is the sum of
%
\begin{equation}\label{edpsi}
%
\{\vy{1}+ R_{\bw}(\vy{1}-\vy{2})~|\
\vy{1}\in\oj{1}^{\perp},\  \vy{2}\in\oj{2}^{\perp},\
(\vy{1}-\vy{2})\in\ap{2}\}
%
\end{equation}
%
\noindent and \ $\Image\frac{\partial R_{\bw}}{\partial\bw}$\vsm.

In the case we are interested in, the two direction vectors \ $\oj{1}$ \
and \ $\oj{2}$ \ are in the plane $\Ec$ spanned by $\Pc$, so \
$B_{\bw}$ \ is the identity matrix and thus \
$R_{\bw}=\lambda\cdot A_{\theta}$ \ takes \ $\aj{1,2}$ \ to \
$\aj{1,3}$ \ in $\Ec$. Clearly, \ $\Image(\dd\psi_{\hV})$ \ contains
the orthogonal complement \ $\Ec^{\perp}$, \ so to prove
\eqref{eomega} we may reduce to the case \ $d=2$ \ (all vectors
in the plane $\Ec$), \ so that in fact \eqref{edpsi} is all of \
$\Image(\dd\psi_{\hV})$.

Now choose unit vectors \ $\cj{i}$ \ spanning \ $\oj{i}^{\perp}$ \
($i=1,2$) \ in $\Ec$, let \ $\bd:=A_{\pi/2}(\aj{1,2})$ \ (so that it
spans \ $(\aj{1,2})^{\perp}$), \ and let \ $\hd:=R_{\bw}\bd$. \
Then \ $\Image(\dd\psi_{\hV})$ \  is the set of
vectors \ $s\cj{1}+t\hd$ \ such that \ $s\cj{1}-t\bd$ \ is a
multiple of \ $\cj{2}$.

For any three vectors $\bx$, $\by$, and $\bz$ in \ $\RR{2}$ \ we have:
%
\begin{equation}\label{elie}
%
(\bx\cdot A_{\pi/2}\by)\,\bz~+~(\by\cdot A_{\pi/2}\bz)\,\bx~+~
(\bz\cdot A_{\pi/2}\bx)\,\by~=~\vz~,
%
\end{equation}
%
\noindent so \ $\cj{2}$ \ is a multiple of \
$(\aj{1,2}\cdot\cj{2})\,\cj{1}~+~(\oj{1}\cdot\cj{2})\,\bd$, \ and thus \
$\Image(\dd\psi_{\hV})$ \ is spanned by:
$$
\be~:=(\aj{1,2}\cdot\cj{2})\,\cj{1}~-~(\oj{1}\cdot\cj{2})\,\hd~.
$$
%
Therefore, \eqref{eomega} fails only if \ $\oj{k}$ \ is perpendicular to
$\be$ \ -- \ in other words, if \ $\oj{k}$ \ is proportional to:
$$
(\aj{1,2}\cdot\cj{2})\,\oj{1}~-~(\oj{1}\cdot\cj{2})\,\aj{1,3}
$$
%
or equivalently, to:
$$
\bw~:=~\frac{(\bd\cdot\oj{2})}{(\oj{1}\cdot\cj{2})}\,\oj{1}~-~\aj{1,3}~,
$$
%
which by \eqref{elie} is precisely the vector connecting the meeting
point \ $P:=\ppj{1}+\frac{(\bd\cdot\oj{2})}{(\oj{1}\cdot\cj{2})}\,\oj{1}$ \ of \
$\Line\up{1}$ \ and \ $\Line\up{2}$ \ with the end point \
$\ppj{3}=\ppj{1}+\aj{1,3}$ \ of \ $\Vj{k}$, \ so that \ $\oj{k}$ \ is
proportional to $\bw$ if and only if the direction line \
$\Line\up{3}$ \ passes through $P$ \ -- \ which cannot happen in a
generic mechanism (see Figure \ref{fsingular} and
Definition \ref{dgeneric}(b)).

This completes the proof of Proposition \ref{pzero}, and thus of
Theorem \ref{tone}.
%
\end{proof}

%
%s3      Morse functions for planar mechanisms
%
\sect{Morse functions for planar mechanisms} \label{cplan}
%
From now on we shall concentrate on the simplest type of polygonal
mechanism \ -- \ namely, planar mechanisms \ ($d=2$) \ having
triangular platforms \ ($k=3$) \ and exactly two links per branch \
($\nj{1}=\nj{2}=\nj{3}=2$). \ These mechanism are known in the
robotics literature as \emph{$3$-RRR} (rotational) mechanisms.

Recall that a smooth real-valued function on a manifold is
called a \emph{Morse function} if all its critical points are
non-degenerate (cf.\ \cite[I, \S 2]{Mi}). Such functions may be used to
deduce the cellular structure of the manifold, and thus recover its
homotopy type  (see \cite[I, \S 3]{Mi}). Our goal is to describe a
Morse function for the \cspace\ of a $3$-RRR mechanism\vsm.

\begin{thm}\label{ttwo}
%
The function \ $f(\Vc)~:=~\sum_{j=1}^{3}\,\|\vj{j}\|^{2}$ \ is
generically a Morse function on \ $C(\Lc,\Xc,\Pc)$, \
where \ $\vj{j}:=\vj{j}_{1}+\vj{j}_{2}=\ppj{j}-\xj{j}$.
%
\end{thm}

\begin{proof}
%
In order to show that the critical points of $f$ are non-degenerate,
we must choose a local coordinate system near each such point.

\begin{figure}[htbp]
\begin{center}
\epsfysize=6cm %\epsfxsize=6.4cm
\leavevmode \epsffile{fig_3/mechanism.eps} \caption{Local
coordinates} \label{floccoord}
\end{center}
\end{figure}


Unfortunately, there is no uniform choice of such a system, so we must
distinguish three cases\vsm:

\noindent\textbf{Case I:} \ \
Let \ $\Phi:=(\phi_{1},\phi_{2},\phi_{3})$, \ where \
$\phi_{j}$ \ denotes the angle between the vectors \ $-\vj{j}_{1}$ \
and \ $\vj{j}_{2}$ \ for \ $j=1,2,3$. \ Then:
%
\begin{equation}\label{eqh}
%
\hj{j}(\phi_{j})~:=~\|\vj{j}\|~=~\|\vj{j}_{1}+\vj{j}_{2}\|~=~
\sqrt{(\lj{j}_{1})^{2}+(\lj{j}_{2})^{2}-2\lj{j}_{1}\,\lj{j}_{2}\cos\phi_{1}}
%
\end{equation}
%
and thus \ $f(\Phi)~=~\sum_{j=1}^{3}\ \hj{j}(\phi_{j})^{2}$, \ so that:
%
\begin{equation}\label{eqnabla}
%
\begin{split}
%
\nabla f~=~\nabla_{\Phi}\,f~=&~2\left(\lj{1}_{1}\lj{1}_{2}\,\sin(\phi_{1}),\
\lj{2}_{1}\lj{2}_{2}\,\sin(\phi_{2}),\
\lj{3}_{1}\lj{3}_{2}\,\sin(\phi_{3})\right)\\
=&~2\,(\vpp{1},~\vpp{2},~\vpp{3})
%
\end{split}
%
\end{equation}
%
where \ $\vw^{\perp}:=(b,-a)$ \ for \ $\vw=(a,b)$.

Thus $\Phi$ is a critical point if and only if:
%
\begin{equation}\label{eqsigma}
%
\Phi~=~\frac{\pi}{2}\,(1+\sigma_{1},\,1+\sigma_{2},\,1+\sigma_{3})
\hspace*{10mm}\text{for \ } \sigma_{1},\sigma_{2},\sigma_{3}\in\{\pm 1\}~.
%
\end{equation}

Computing the Hessian at a critical point $\Phi$ yields:
$$
H_{\Phi}~=~\begin{pmatrix}
        \sigma_{1}\,\lj{1}_{1}\lj{1}_{2} & 0  & 0 \\
0 &     \sigma_{2}\,\lj{2}_{1}\lj{2}_{2} & 0  \\
0 & 0 & \sigma_{3}\,\lj{3}_{1}\lj{3}_{2}
\end{pmatrix}~,
$$
%
which is non-degenerate, with index \ $\Ind_{\Phi}$ \ equal to the
number of negative values in \
$\{\sigma_{1},\,\sigma_{2},\,\sigma_{3}\}$. \ Such critical points
will be refered to as \emph{type I}.

\begin{figure}[htbp]
\begin{center}
\epsfysize=5cm %\epsfxsize=6.4cm
\leavevmode \epsffile{fig_3/convex_hull1.eps} \caption{Type I
critical point}\label{fcaseI}
\end{center}
\end{figure}

\noindent\textbf{Case II:} \ \ As we saw, critical points of $f$
appear when all three branches are aligned. However, for some
mechanisms this will never happen, because one or two
branches can never fully stretch or fold \ -- \
that is, \ $\phi_{3}$ \ (say) takes values in a proper subset \
$[a_{1},a_{2}]\cup[-a_{2},-a_{1}]$ \ of \ $[-\pi,\pi]$ \ (see
Example \ref{egcrit}). \ Clearly, \ $\phi_{3}$ \ cannot then serve as
a local coordinate at a point \ $(\phi_{1},\phi_{2},\pm a_{k})$.

However, if the first two branches can both be aligned, then in the vicinity
of doubly aligned configurations we take \
$\hP:=(\phi_{1},\phi_{2},\theta_{1})$, \ where \ $\phi_{j}$ \ ($j=1,2$)  \
as in Case I, and \ $\theta_{j}$ \ is the angle between  \
$\vj{j}:=\vj{j}_{1}+\vj{j}_{2}$ \ and the vector \
$\xj{2}$ \ (we assume for simplicity that \ $\xj{1}$ \ is at the origin).

Since
%
\begin{equation}\label{evj}
%
\vj{j}=\hj{j}(\cos\theta_{j},\sin\theta_{j})~,
%
\end{equation}
%
using \eqref{eqh} we have:
%
\begin{equation}\label{eeach}
%
(\dif{\vj{j}}{\phi_{j}},~\dif{\vj{j}}{\phi_{j'}},~\dif{\vj{j}}{\theta_{j}})~=~
(\frac{\vpp{j}}{\hj{j}(\phi_{j})^{2}}\,\vj{j},\,
0,\,-\vp{j}))~,
%
\end{equation}
%
for \ $\{j,j'\}=\{1,2\}$.

However, since \ $\theta_{2}$ \ is a dependent variable, we may
differentiate the norms in:
%
\begin{equation}\label{equad}
%
\vj{1}+\aj{1,2}-\vj{2}~=~\xj{2}
%
\end{equation}
%
implicitly and deduce that:
%
\begin{equation}\label{esecond}
%
\dif{\vj{2}}{\theta_{1}}~=~
-\dif{\theta_{2}}{\theta_{1}}\,\vp{2}~=~
-\frac{\aj{1,2}\cdot\vp{1}}{\aj{1,2}\cdot\vp{2}}\,\vp{2}~.
%
\end{equation}

Differentiating \eqref{equad} itself and using
\eqref{eeach}, \eqref{esecond} and \eqref{elie} yields:
%
\begin{equation}\label{ethird}
%
\nabla_{\hP}\aj{1,2}~=~
(-\frac{\vpp{1}}{\hj{1}(\phi_{1})^{2}}\,\vj{1},~
-\frac{\vpp{2}}{\hj{2}(\phi_{2})^{2}}\,\vj{2},~
\frac{\vj{1}\cdot\vp{2}}{\aj{1,2}\cdot\vp{2}}\,(\aj{1,2})^{\perp}\,)
%
\end{equation}

Since \ $\xj{1}=\vz$, \ we see \ $\vj{3}~=~\vj{1}+\aj{1,3}-\xj{3}$, \ so:
%
\begin{equation}\label{eqmorse}
\begin{cases}
\dif{f}{\phi_{1}}~=&~
2\,\vpp{1}~\frac{2\hj{1}(\phi_{1})^{2}+\vj{1}\cdot(\aj{1,3}-\xj{3})+
(\xj{3}-\vj{1})\cdot(B_{\alpha}\vj{1})}{\hj{1}(\phi_{1})^{2}}\\
%
\dif{f}{\phi_{2}}~=&~
2\,\vpp{2}~
\frac{\hj{2}(\phi_{2})^{2}~+(\xj{3}-\vj{1})\cdot(B_{\alpha}\vj{2})}
{\hj{2}(\phi_{2})^{2}}\\
%
\dif{f}{\theta_{1}}~=&~
2\,(\xj{3}-\aj{1,3})\cdot\vp{1}
~+~2\,\frac{[\vj{1}\cdot\vp{2}]\,[(\vj{1}-\xj{3})\cdot(\aj{1,3})^{\perp}]}
{\aj{1,2}\cdot\vp{2}}
\end{cases}
%
\end{equation}
%
where \ $B_{\alpha}$ \ is the rotation-and-dilitation matrix taking \
$\aj{1,2}$ \ to \ $\aj{1,3}$.

Note that we use the coordinates $\hP$ only at points where the first
two branches are aligned, so that \
$\vj{j}_{2}\cdot(\vj{j}_{1})^{\perp}=0$ \ for \ $j=1,2$, \ and thus
the first two entries of \ $\nabla_{\hP}(f)$ \ vanish at these
points. The vanishing of \ $\frac{\partial f}{\partial\theta_{1}}$ \
is equivalent to the condition:
%
\begin{equation}\label{elast}
%
\begin{split}
%
[&\xj{3}\cdot\vp{1}]\,[\aj{1,2}\cdot\vp{2}]
~-~[\aj{1,3})\cdot\vp{1}]\,[\aj{1,2}\cdot\vp{2}]\\
+&~[\vj{1}\cdot\vp{2}]\,[\vj{1}\cdot(\aj{1,3})^{\perp}]~-~
[\vj{1}\cdot\vp{2}]\,[\xj{3}\cdot(\aj{1,3})^{\perp}]~=~0
%
\end{split}
%
\end{equation}
%

Note that by \eqref{elie} again, the intersection of \ $\Line\up{1}$ \
with \ $\Line\up{2}$ \ is at the point:
$$
P:=\xj{3}+\vj{1}+\frac{\aj{1,2}\cdot\vp{2}}{\vj{1}\cdot\vp{2}}\,\vj{1}~,
$$
%
and \eqref{elast} is equivalent to the colinearity of \ $\xj{3}$, \
$P$, and \ $\vj{1}+\aj{1,3}$. \ Such critical points
will be refered to as \emph{type II}.

\begin{figure}[htbp]
\begin{center}
\epsfysize=5cm %\epsfxsize=6.4cm
\leavevmode \epsffile{fig_3/convex_hull.eps} \caption{Type II
critical point}\label{fcaseII}
\end{center}
\end{figure}

Calculating the Hessian matrix \ $H_{f}$ \ of $f$ at a critical point,
we find that it is diagonal, with:
%
\begin{equation*}
%
\begin{split}
%
\diff{f}{\phi_{1}\phi_{1}}~=&~
-2\,\vj{1}_{1}\cdot\vj{1}_{2}~
\frac{2\hj{1}(\phi_{1})^{2}+\vj{1}\cdot(\aj{1,3}-\xj{3})+
(\xj{3}-\vj{1})\cdot(B_{\alpha}\vj{1})}{\hj{1}(\phi_{1})^{2}}\\
%
\diff{f}{\phi_{2}\phi_{2}}~=&~
-2\,\vj{2}_{1}\cdot\vj{2}_{2}~
\frac{\hj{2}(\phi_{2})^{2}~+(\xj{3}-\vj{1})\cdot(B_{\alpha}\vj{2})}
{\hj{2}(\phi_{2})^{2}}\\
%
\diff{f}{\theta_{1}\theta_{1}}~=&~
~\frac{2}{[\aj{1,2}\cdot\vp{2}]^{2}}\
\left(\right.
-~2\,[\vj{1}\cdot\vp{2}]\,[\aj{1,3}\cdot\vj{1}]\,[\aj{1,2}\cdot\vp{2}]\\
&+~[(\xj{3}-\aj{1,3})\cdot\vj{1}]\,[\aj{1,2}\cdot\vp{2}]^{2}\\
&+~[\vj{1}\cdot\vj{2}]\,[\aj{1,2}\cdot(\vj{1}-\vj{2})^{\perp}]\,
[(\vj{1}-\xj{3})\cdot(\aj{1,3})^{\perp}]\\
&+~[\vj{1}\cdot\vp{2}]^{2}\,[(\vj{1}-\xj{3})\cdot(\aj{1,3})]\\
&+~[\aj{1,2}\cdot\vj{2}]\,[(\vj{2}-\aj{1,2})\cdot\vp{1}]\,
[(\xj{3}-\aj{1,3})\cdot\vp{1}]\left.\right)
%
\end{split}
%
\end{equation*}

If we solve \eqref{eqmorse} to find
explicitly the critical points of $f$ in the coordinates $\hP$, and
then substitute into the expression we have found for  \
$H_{f}$ \ at these points, we obtain a polynomial expression of
degree $6$ in the parameters \ $(\Lc,\Xc,\Gc)$ \ for the
mechanism. Thus the critical point we identified is degenerate only
when this polynomial vanishes, so generically $f$ is indeed a Morse
function\vsm.

\noindent\textbf{Case III:} \ \
Note that the \wspace\ $\Wc$ for each vertex of $\Pc$ is the intersection
of three annuli (so it is compact), and thus the boundary of $\Wc$
must intersect at least one of the bounding circles of the annuli.
Therefore, at least one of the three branches (say, the first)
\emph{can} be aligned.

Thus, at critical points of $f$ where neither $\Phi$ nor $\hP$ can be
used as local coordinates, the first branch is aligned, and we take \
$\Psi:=(\theta_{1},\phi_{1},\psi)$ \ as our local coordinates,
where \ $\theta_{1}$ \ and \ $\phi_{1}$ \ are as in Case II above, \ and
$\psi$ denotes the angle between \ $\aj{1,2}$ \ and $\xj{2}$ \ (see
Figure \ref{floccoord}). Note that this will not work when the second branch is
also aligned, since these coordinates only determine the length of \
$\vj{2}$, \ and not ``elbow up/down'' near \
$\phi_{2}=\frac{\pi}{2}\,(1+\sigma_{2})$.

Here:
%
$$
f(\Psi)~=~\|\vj{1}_{1}+\vj{1}_{2}\|^{2}
~+~\|\vj{1}+\aj{1,2}-\xj{2}\|^{2}~+~\|\vj{1}+\aj{1,3}-\xj{3}\|^{2}~.
$$
%
and since \ $\aj{1,2}=\gj{1,2}(\cos\psi,\sin\psi)$, \ we have \
$\nabla_{\Psi}(\vj{1})=
(-\vp{1},~\frac{\vpp{1}}{\hj{1}(\phi_{1})^{2}}\,\vj{1},\,0)$ \ and \
$\nabla_{\Psi}(\aj{1,j})=(0,\,0,\,-\ap{j}$ \ for \ $j=2,3$, \ so:
%
\begin{equation}
%
\begin{split}
%
\dif{f}{\theta_{1}}~=&~
2\vp{1}\cdot((\xj{2}+\xj{3})-(\aj{1,2}+\aj{1,3}))\\
%
\dif{f}{\phi_{1}}~=&~
-\vpp{1}\ \frac{\vj{1}\cdot(4\,\vj{1}+\aj{1,2}+\aj{1,3}-\xj{2}-\xj{3})}
{\|\vj{1}\|^{2}}\\
%
\dif{f}{\psi}~
=&~\ap{2}\cdot(\xj{2}-\vj{1})~+~\ap{3}\cdot(\xj{3}-\vj{1})
%
\end{split}
%
\end{equation}

We are using the coordinates $\Psi$ because the first leg is aligned,
so indeed \ $\dif{f}{\phi_{1}}=0$. \ In order for this to be a
critical point, we have two additional geometric conditions: the vanishing of \
$\dif{f}{\theta_{1}}$ \ implies that the vector
connecting the midpoints of sides of the fixed and moving platforms
opposite the first vertex \ -- \ that is, \
$A:=(\xj{2}+\xj{3})/2$ \ and \ $B:=(\aj{2}+\aj{3})/2$ \ -- \ is aligned
with \ $\vj{1}$ \ (see Figure \ref{fcaseIII}). \ On the other hand,
the vanishing (in addition) of \ $\dif{f}{\psi}$ \ is equivalent to:
%
\begin{equation}\label{eqcaseiii}
%
\vp{2}\cdot\xj{2}~+~\vp{3}\cdot\xj{3}~=~0~,
%
\end{equation}
%
which means that the areas of the triangles spanned by \ $\vj{j}$ \
and \ $\xj{j}$ \ ($j=2,3$) \ are equal. Such critical points
will be refered to as \emph{type III}.

\begin{figure}[htbp]
\begin{center}
\epsfysize=5cm %\epsfxsize=6.4cm
\leavevmode \epsffile{fig_3/convex_hull2.eps} \caption{Type III
critical point}\label{fcaseIII}
\end{center}
\end{figure}

Now, calculating the Hessian of $f$ at the critical points we have:
%
\begin{equation*}
%
\begin{split}
%
\diff{f}{\theta_{1}\theta_{1}}~=&~
-~2\vj{1}\cdot((\xj{2}+\xj{3})-(\aj{1,2}+\aj{1,3}))\\
%
\diff{f}{\phi_{1}\theta_{1}}~=&~\diff{f}{\phi_{1}\psi}~=~0\\
%
\diff{f}{\psi\theta_{1}}~=&~2\vj{1}\cdot(\aj{1,2}+\aj{1,3})\\
%
\diff{f}{\phi_{1}\phi_{1}}~=&~
\vj{1}_{1}\cdot\vj{1}_{2}\
\frac{\vj{1}\cdot(4\,\vj{1}+\aj{1,2}+\aj{1,3}-\xj{2}-\xj{3})}{\|\vj{1}\|^{2}}\\
%
\diff{f}{\psi\psi}~=&~\aj{1,2}\cdot(\xj{2}-\vj{1})~+~\aj{1,3}\cdot(\xj{3}-\vj{1})
%
\end{split}
%
\end{equation*}

Again, generically the critical point is non-degenerate\vsm.
%
\end{proof}

\begin{mysubsect}[\label{scrit}]{Identifying the critical points}

Since we usually have no explicit description of the \cspace\ $\C$ as
a manifold, it is hard to calculate the Morse function \ $f:\C\to\RR{}$ \
directly. However, in the course of proving Theorem \ref{tone} we gave
a geometric description of each of the possible critical points of
$f$ \ -- \ which are the main ingredient needed for analyzing the
topology of $\C$ \ -- \ in terms of the \wspace\ $\Wc$. We can use
this geometric information in order to identify all possible
candidates for critical points, and then we need only calculate \
$\df$ \ in local coordinates at these points (also provided in the
proof above) to check if they are indeed critical, and find their indices.

Recall that $\Wc$ (Definition \ref{dwork}) is the space of all possible
locations of the moving platform $\Pc$, whose vertices must be
situated in the respective \wspace s \ $\Wc_{i}$ \ ($i=1,2,3$) \ of
(the end points of) the three branches.  Each \ $\Wc_{i}$ \ is an
annulus centered at the $i$-the vertex \ $\xj{i}$ \ of the fixed triangle.

Also recall the concept of the \emph{coupler curve} $\gamma$ of a
planar four-bar linkage \ - \ that is, a degenerate polygonal
mechanism with \ $k=2$ \ linear branches \ ($\nj{1}=\nj{2}=1$), \
but having a triangular platform $\Pc$: \ the coupler curve is the
\wspace\ for the third (unattached) vertex of $\Pc$. See
\cite[Ch.\ 4]{Ha}. We consider the coupler curves for two vertices \ $\xj{i}$ \
(say \ $i=1,2$) \ of a triangular mechanism  \ $(\Lc,\Xc,\Pc)$ \ as
above, in which the two corresponding branches are aligned, so that
each can be replaced by a single linear branch of length \
$\lj{i}:=\lj{i}_{1}+\lj{i}_{2}$ \ or \ $\lj{i}_{1}-\lj{i}_{2}$, \
as the case may be.

\begin{enumerate}
\renewcommand{\labelenumi}{(\arabic{enumi})}
%
\item The critical points of type I (all three branches aligned)
correspond to placements of $\Pc$ with all three vertices on the
(inner or outer) boundary circles of these annuli. Determining these
is a straightforward geometric problem, which can be described as
intersecting the coupler curve for the first two vertices, say, with
the two boundary circles of \ $\Wc_{3}$.
%
\item For critical points of type II, we need also a line field $V$
along the coupler curve $\gamma$, where \ $V(\gamma(t))$ \ is the
line from \ $\gamma(t)$ \ to the intersection point \ $P(t)$ \
of  \ $\Line\up{1}$ \ with \ $\Line\up{2}$. \ This line field is
readily calculated from $\gamma$. The critical points are
then those configurations for which \ $V(\gamma(t))$ \ passes through \
$\xj{3}$.
%
\item For critical points of type III, the first vertex \ $\vj{1}$ \
of $\Pc$ must lie on one of the two boundary circles of \ $\Wc_{1}$. \
Given \ $\vj{1}$, \ the possible positions of $\Pc$ are determined by
its rotation angle $\theta$ around its first vertex, and at most two
values \ $\theta'$, \ $\theta''$ \ of $\theta$ satisfy condition
\eqref{eqcaseiii}. \ Thus we can define on \ $\partial \Wc_{1}$ \ two
line fields \ $V'$, \ $V''$ \  which associate to \ $\vj{1}$ \ the
line between the midpoints of the \ $(2,3)$-side of the fixed and
moving triangles in the positions corresponding to  \ $\theta'$, \
$\theta''$ \ respectively. The critical points are those for which the
vector \ $\vj{1}$ \ lies on one of these two lines\vsm.
%
\end{enumerate}
%
\end{mysubsect}

\begin{example}\label{egcrit}
%
In general, the critical points of a manifold do not determine its
topology, though they impose certain restrictions on its
homology, and thus on its homotopy type, via the Morse inequalities.
However, in the simplest cases the geometric considerations described
above limit the possible critical points so severely that the \cspace\
$\C$ can be recovered in full. Note that there are two connected
components in $\C$, determined by the orientation of the moving
platform\vsm .

For example, consider a triangular mechanism with one branch (say, \
$k=3$) \ having one very large link, so that the
\wspace\ for the vertex \ $\ppj{3}$ \ contains those for all points of
the moving platform, and thus imposes no restriction on the allowed
configurations.  We assume the moving platform is a small triangle,
and that the \wspace\ for (the vertex of) the first branch is a small
annulus, intersecting that of the second branch in a crescent-shaped
lune, which is the approximate ``\wspace'' for the moving platform
(i.e., for its barycenter). Finally,  assume that the fixed
vertex \ $\xj{3}$ \ is far to the left (see Figure \ref{fmechanism}).

\begin{figure}[htbp]
\begin{center}
\epsfysize=4cm %\epsfxsize=6.4cm
\leavevmode \epsffile{fig_3/fullpicture.eps} \caption{Work spaces
for the
 three moving vertices}\label{fmechanism}
\end{center}
\end{figure}

Now we may analyze the possible critical points as follows:

\begin{enumerate}
\renewcommand{\labelenumi}{(\arabic{enumi})}
%
\item Since the two small annuli above are wholly contained in the
  large one, and the moving platform is small, there are no
  critical points of type I.
%
\item Note that since the linkage is not Grashof (cf.\ \cite{KM}),
there are exactly two cases where \ $\Line\up{1}$ \
meets \ $\Line\up{2}$ \ on the inner boundary circle of the \wspace\ for
vertex $2$. Since \ $\gj{23}$ \ is very small, any critical points
of type II must occur nearby, so that the edges \ $\aj{12}$ \ and \
$\aj{13}$ \ (which nearly coincide) are aligned with \ $\vj{1}$ \
(see Figure \ref{fig4}).

\begin{figure}[htbp]
\begin{center}
\epsfysize=4cm %\epsfxsize=6.4cm
\leavevmode \epsffile{fig_3/crit2.eps} \caption{A critical point
of type II}\label{fig4}
\end{center}
\end{figure}

By choosing appropriate generic values for the parameters, we can
ensure that there are exactly two critical of type II in each
component of $\C$.

%
\item Consider the three dashed lines \ $\Lj{k}$ \ in Figure
  \ref{fig5}, each connecting \ $\xj{k}$ \ with the midpoint of
  opposite (fixed) edge (for \ $k=1,2,3$). \ Because the moving triangle
  is so small, the vector \ $\vj{k}$ \ must approximate the direction
  of \ $\Lj{k}$ \ in order to obtain a critical point of type III \ --
  \ but since these lines do not pass near the approximate \wspace\
  for the moving platform, no such critical points can occur.

\begin{figure}[htbp]
\begin{center}
\epsfysize=5cm %\epsfxsize=6.4cm
\leavevmode \epsffile{fig_3/twoannuli.eps} \caption{Potential
critical points of type III}\label{fig5}
\end{center}
\end{figure}

\end{enumerate}

Thus each component of the \cspace\ \ $\C(\Lc,\Xc,\Pc)$ \ has exactly two critical
points in this case (both of type II), so it is homeomorphic to \ $S^{3}$.
%
\end{example}

\chapter{Uncertainty singularities in parallel mechanisms}
\label{chap4}

\textsf{Submitted to \textsl{IEEE Transactions on Robotics}, 2006.} \\
\textsf{Co-authors: David Blanc and Moshe Shoham.}\\

We study singularities for a parallel mechanism with spherical
joints and moving planar platform, giving a necessary condition
for an uncertainty singularity (defined topologically, in terms of
the \cspace) to occur, and describe the corresponding
instantaneous kinematic singularity.  An explicit example is
provided.



\sect{Introduction} \label{cint}
%
Kinematic singularities of parallel mechanisms have been studied
extensively in the literature (cf.\ \cite{Hu2}, and \cite{Hu1,WW}
for the example of open kinematic chains). On the other hand, many
authors have investigated \emph{topological} singularities of the
\cspace s of such mechanisms (see, e.g., \cite{KM,KT,MT1,SSB05}).

In this note, we study parallel mechanisms with rotational joints and
planar moving platform, and give a necessary
geometric condition  for uncertainty singularities in such mechanisms
(Theorem \ref{thmain}). We then explain how such topological
singularities give rise to instantaneous kinematic singularities
(Proposition \ref{pkinsing}).
Finally, the occurence of an uncertainty singularity is illustrated
visually in an explicit example (Section \ref{cve}).

\subsection{Topological singularities}
\label{sts}
%
By choosing appropriate local coordinates for a given mechanism $\Gamma$, we
can think of the set of all its configurations as a topological space
$\C$, called its \emph{\cspace}, and try to endow it with the
structure of a differentiable manifold. The points of $\C$  where this
cannot be done constitute the \emph{topological singularities} of
$\Gamma$. See Section \ref{cts} below for further details.

These are refered to in the robotics literature as \emph{uncertainty
singularities}: they occur when the \emph{instantaneous} mobility
is greater than the \emph{full cycle} mobility  (cf.\ \cite{Hu2}).
In such a position the mechanism as a whole gains an additional
degree of freedom. At times, such singularities are the only point
where two disjoint regions of $\C$ meet, allowing transition between
different operating modes. See \cite{ZBC} for an examination of such
transitions in the case of a constraint singularity for a $3$-URU
$3$-DOF parallel mechanism.

\subsection{Kinematic singularities}
\label{sks}
%
In general, a configuration $\Vc$ is naturally described by the vector \
$\bx=(x_{1},\dotsc,x_{N})$ \ whose coordinates \ $x_{i}$ \ correspond
to the positions of the various joints and links of the mechanism.
In practice, we choose a subset \ $\bx_{\iin}$ \
of \emph{input} coordinates (corresponding to the actuated joints),
which can serve as local coordinates for $\C$ around $\Vc$. In
addition, we often focus on a subset \ $\bx_{\out}$ \ of \emph{output}
coordinates of interest (which may describe the position of the end
effector). The structure of the mechanism imposes relations which must
hold among the coordinate \ $x_{i}$; \ in particular, we may assume that:
%
\begin{equation}\label{eq:constr}
%
F(\bx_{\iin},\bx_{\out})=0
%
\end{equation}
%
identically in a neighborhood of $\Vc$ in $\C$. The Jacobian \
$J:=\frac{\partial F}{\partial\bx}$ \ consists of two blocs \
$\frac{\partial F}{\partial\bx_{\iin}}$ \ and \
$J_{\out}:=\frac{\partial F}{\partial\bx_{\out}}$, \ where \
$J_{\out}$ \ must be nonsingular if the \ $\bx_{\iin}$ \ are to serve
as local coordinates near $\Vc$.

The kinematics of the mechanism are described by a path \ $\bx(t)$
\ ($t\in\RRR$) \ in the \cspace. Differentiating \
\eqref{eq:constr} \ with respect to $t$ we get \ $\frac{\partial
F}{\partial\bx}\,\dot{\bx}=0$, \ which can be written in the form:
%
\begin{equation} \label{eq:gosselin_sing}
%
J_{\iin}\ \dot{\bx}_{\iin}~=~J_{\out}\ \dot{\bx}_{\out}
%
\end{equation}
%
where \ $J_{\iin}:=-\frac{\partial F}{\partial\bx_{\iin}}$. \ If \
$J_{\out}$ \ is of maximal rank and \ $J_{\iin}$ \ is not, \
$\Vc$ is called a kinematic singularity \emph{of type I}: this
means that we cannot obtain every (infinitesimal) change in output we
may want by an appropriate change in the actuated joints.
On the other hand, if \ $J_{\iin}$ \ is of maximal rank and  \
$J_{\out}$ \ is not, $\Vc$ is called singular of \emph{of type II}: in
this case the actuated joints do not determine uniquely the behavior
of the outputs.
Finally, if neither \ $J_{\out}$ \ nor \ $J_{\iin}$ \ is of maximal
rank, $\Vc$ is called singular \emph{of type III} (cf.\ \cite{GA}).
Gosselin and Angeles give examples of all three types of
singularities for a parallel \ $3$-RRR \ planar mechanism.

However, as Zlatanov, Fenton and Benhabib observe, the choice of \
$\bx_{\iin}$ \ and \ $\bx_{\out}$ \ is arbitrary; moreover, in
general it need not fully determine the mechanism, which may have
additional ''passive coordinates'' of interest. They therefore provide
a different approach to the singularity analysis of an arbitrary
kinematic chain (see \cite{ZFB1}).

%
%s2       Topological singularities
%
\sect{Topological singularities} \label{cts}

We consider \emph{polygonal} mechanism $\Gamma$ in \ $\RRR^{d}$ \
($d=2,3$), \ consisting of a moving platform $\Pc$ (planar
$k$-polygonal), with $k$ branches attached to its vertices. The
$i$-th branch is a sequence of \ $\nj{i}$ \ concatenated links of
lengths \ $\lj{i}_{j}$ \ ($j=1,\dotsc\nj{i}$), \ connected by
spherical joints. One end of the branch is attached to a vertex of
the moving platform, and the other is fixed  at \
$\xj{i}\in\RRR^{d}$.

\begin{defn}
%
A \emph{configuration} for an $n$-link branch consists of $n$
vectors \ $(\bv_{1},\dotsc,\bv_{n})$ \ in \ $\RRR^{d}$ \ of
specified lengths \ $\|\bv_{j}\|=\ell_{j}$ \ ($j=1,\dotsc n$). \

A branch configuration is said to be \emph{aligned} if all the
vectors \ $\bv_{i}$ \ are scalar multiples of $\bv$, which is
called the \emph{direction vector} of $V$. The \emph{direction
line} for $V$ is \ $\Line:=\{\bx+t\bv~|\ t\in\RRR^{}\}$.

A configuration for $\Gamma$ consists of a set \
$\Vc=(\Vj{1},\dotsc,\Vj{k})$ \ of configurations for each of the $k$
branches, such that the $k$ endpoints
%
$$
\ppj{i}:=\xj{i}+\sum_{i=1}^{n}\,\bv_{i} \ \ \ (i=1,\dotsc,k)
$$
of the corresponding branch configurations form a polygon congruent to the
given moving platform $\Pc$. The set \ $\C=\C_{\Gamma}$ \ of all such
configurations, topologized in the obvious way, is the \emph{\cspace}
of $\Gamma$.
%
\end{defn}

\textit{Convention:} \
%
Parenthesized superscripts indicate the branch
number, and subscripts indicate the link number. For example, \
$\ell_{j}^{(i)}$ \ denotes the length of the $j$-th link of the $i$-th
branch.

\begin{defn}
%
A configuration $\Vc$ for $\Gamma$ is called \emph{singular of type
  (a)} if two of its branch configurations \
$\Vj{i_{1}}$ \ and \ $\Vj{i_{2}}$ \ are aligned, with coinciding
direction lines: \ $\Line\up{i_{1}}=\Line\up{i_{2}}$. \ See Fig.
\ref{fig:sing1}.

\begin{figure}[ht]
\centering \epsfysize=5.5cm  \epsffile{fig_4/singular_point1.eps}
\caption{A singular configuration of type (a)} \label{fig:sing1}
\end{figure}

It is \emph{singular of type (b)} if three of its branch
configurations are aligned, with direction lines in the same
plane meeting in a single point (this is referred to in literature as
a \emph{flat pencil}). See Fig. \ref{fig:sing2}.

\begin{figure}[ht]
\centering \epsfysize=6cm  \epsffile{fig_4/singular_point2.eps}
\caption{A singular configuration of type (b)} \label{fig:sing2}
\end{figure}

It is \emph{singular of type (c)} if four of its branch configurations
are aligned, with direction lines in the same plane.
%
\end{defn}

We can now formulate our main result:

\begin{thm}\label{thmain}
%
A necessary condition for a configuration \ $\Vc=(\Vj{1},\dotsc,\Vj{k})$ \
of a polygonal mechanism $\Gamma$ to be an uncertainty singularity is that it
be singular of type (a), (b) or (c).
%
\end{thm}

\begin{examples}\label{eks}
%
Before sketching the proof, consider the following two examples\vsm:

\noindent (1) \ If $\Vc$ is singular of type (a), as in Fig.
\ref{fig:sing1}, we can construct an open neighborhood \ $U\subset
\C$ \ of $\Vc$ as a product \ $U_{\alignd} \times U_{\rest}$. \
Here \ $U_{\alignd}$ \ may be identified with an neighborhood of
$\Vc$ in the \cspace\ of a closed chain mechanism \ $\Gamma'$, \
consisting of the aligned branches of $\Gamma$. In the case where
the aligned branches have one and two links respectively,  \
$U_{\alignd}$ \ is a one-point union of two $2$-discs (see
\cite[Proposition 4.1]{SSB2}), so it is singular.

On the other hand, \ $U_{\rest}$ \ is a neighborhood of $\Vc$ in the
\cspace\ of the mechanism obtained from $\Gamma$ by omitting the
aligned branches, which is non-singular (generically)\vsm.

\noindent (2) \ Consider a mechanism with a triangular platform, and
two branches with one link each. In this case, the workspace
for the third vertex of the platform is the coupler curve $\gamma$ for the
corresponding $4$-chain mechanism (see \cite[Ch.\ 4]{Ha}), while the
workspace for the third branch is an annulus $A$.

For suitable parameters, the boundary \ $\partial A$ \ (where the
third branch is aligned) will be tangent to $\gamma$. The
corresponding configuration $\Vc$ for $\Gamma$ will be singular of
type (b), as in Fig. \ref{fig:second case intuitive}.

\begin{figure}[htb]
\centering \epsfysize=6cm \leavevmode
\epsffile{fig_4/intantanious_rotation.eps} \caption{Coupler curve
tangent to branch workspace boundary} \label{fig:second case
intuitive}
\end{figure}

Note that near $\Vc$ each point in $\gamma$ has two
corresponding configurations, associated to ``elbow up/down''
positions of the third branch, which coalesce at $\Vc$ itself; thus
$\Vc$ has a singular neighborhood consisting of two transverse
intervals\vsm.
%
\end{examples}


\textit{Sketch of proof of Theorem \ref{thmain}}\vsm:

A configuration $\Vc$ for the polygonal mechanism $\Gamma$ in \
$\RRR^{d}$ \ ($d=2,3$) \ is determined by the \ $Nd$ \ Cartesian
coordinates of the endpoints of the \ $N=\sum_{i=1}^{k}\,\nj{i}$ \
links of $\Gamma$. However, there are \ $M:=N+dk-3(d-1)$ \
relations among these imposed by the lengths of the links and a
certain minimal collection $\Ic$ of edges and diagonals of the
polygonal platform $\Pc$. Thus the \cspace\ $\C$ is the preimage
of a certain vector \ $Z\in\RRR^{M}$ \ under the \emph{constraint
map}  \ $F:\RRR^{dN}\to\RRR^{M}$, \ defined:
%
\begin{equation}\label{econstr}
\begin{split}
%
F(\Vc)~:=&(f_{\nj{1}}(\Vj{1}),\dotsc,f_{\nj{k}}(\Vj{k}),\\
&\|\aj{1,2}\|^{2},\dotsc,\|\aj{k-1,k}\|^{2})~,
%
\end{split}
\end{equation}
%
where \
$f_{n}(\bv_{1},\dotsc,\bv_{n}):=(|\bv_{1}|^{2},\dotsc,|\bv_{n}|^{2})$ \
and \ $\aj{i,j}$ \ is the \ $(i,j)$-diagonal (or edge) of $\Pc$
spanned by the endpoints of branches $i$ and $j$\vsm.

\noindent\textbf{Step I.}~~~By the Regular Value Theorem, $\C$ will be
a smooth manifold if \ $\dF_{\Vc}$ \ is of rank $M$, where \
$\dF_{\Vc}/2$ \ is:
%
%\setlength{\arraycolsep}{0pt}
\begin{equation}\label{edf}
%
\begin{pmatrix}
%
A\up{1}    & 0         &  0           & \dotsc  & 0        \\
 0         & A\up{2}   &  0           & \dotsc  & 0        \\
 \vdots    & \vdots    &  \vdots      & \ddots  & \vdots   \\
 0         & 0         &  0           & \dotsc  & A\up{k}  \\
\bj{1}{2}  & \bj{2}{1} & 0            & \ldots  & 0        \\
\bj{1}{3}  & 0         & \bj{3}{1}    & \ldots  & 0        \\
 0         & \bj{2}{3} & \bj{3}{2}    & \ldots  & 0        \\
\vdots     & \vdots    &  \vdots      & \ddots  & \vdots \\
 0         & \dotsc    & 0            & \ldots  & \bj{k}{k-1}
\end{pmatrix}\vspace{4mm}\quad
%
\end{equation}
%
and \ $A\up{i}$ \ is the \ $\nj{i}\times d\nj{i}$ \ matrix:
%
\begin{equation}\label{eai}
\begin{pmatrix}
%
\vj{i}_{1} & \dotsc    & 0              \\
\vdots     & \ddots    & \vdots         \\
0          & \dotsc    & \vj{i}_{\nj{i}}
\end{pmatrix}~.
%
\end{equation}

Each edge \ $\aj{i,j}\in\RRR^{d}$ \ appears \ $\nj{i}$ \ times in
the vector
%
\begin{equation}\label{ebj}
%
\bj{i}{j}~:=~\underbrace{(\aj{i,j},\aj{i,j},\dotsc,\aj{i,j})}_{\nj{i}}\vsm.
%
\end{equation}

\noindent\textbf{Step II.}~~~
%
Let \ $\Vc\in\C=F^{-1}(Z)$, \ and consider a vanishing linear
combination of the rows of \ $\dF_{\Vc}$. \ If some row of \ $A\up{i}$ \
has a non-zero coefficient, so does at least one of the \ $M-N$ \
bottom rows; but since these are all repeated, as in \ \eqref{ebj}, \
we see that all the direction vectors \ $\vj{i}_{j}$ \
for the $i$-th branch must be equal \ -- \ that is, this
branch is aligned, with the direction vector \ $\vj{i}$ \ a linear
combination of appropriate diagonals \ $\aj{k,\ell}$ \ of $\Pc$, and
thus lying in the plane $\Ec$ determined by $\Pc$.

Moreover, from \ \eqref{edf} \ we see that \ $\sum_{i=1}^{k}\,\vj{i}=\vz$. \
Therefore, if \ $\vj{i}=\vz$ \ for \ $i\neq i_{0},i_{1}$, \ then \
$\vj{i_{0}}+\vj{i_{1}}=\vz$, \ and thus branches \ $i_{0}$ \ and \
$i_{1}$ \ are co-aligned with direction vector \
$\aj{i_{0},i_{1}}$, \ yielding a singularity of type (a).

On the other hand, if at least four branches are aligned (with
direction vectors in $E$), we obtain a singularity of type (c)\vsm.

\noindent\textbf{Step III.}~~~
%
Now assume we have exactly three aligned branches, say \ $1,2$ \
and $k$. Let $\hC$ denote the \cspace\ of the mechanism obtained
by omitting the last ($k$-th) branch \ -- \ again given as the
fiber of an appropriate constraint map \
$\hF:\RRR^{(N-\nj{k})d}\times\Sc\to\RRR^{K}$ \ (where \
$\Sc=S^{d-2}$ \ if \ $k=3$, \ and a point otherwise), with \
$\hi:\hC\to\RRR^{(N-\nj{k})d}$ \ the inclusion. Similarly, let
$\Ck$ be the \cspace\ for the $k$-th branch (an \ $\nj{k}$-torus,
with the obvious embedding \ $i:\Ck\to\RRR^{\nj{k}d}$).

The \emph{workspace} of both mechanisms \ -- \ that is, the
possible locations for the $k$-th vertex of the moving platform \
-- \ are in  \ $\RRR^{d}$, \ and we denote the \emph{work maps} by
\ $\psi:\hC\to\RRR^{d}$ \ and \ $\phi:\Ck\to\RRR^{d}$.

Let \ $X:=\Ck\times\RRR^{(N-\nj{k})d}\times\Sc$, \
$Y:=\RRR^{d}\times\RRR^{(N-\nj{k})d}\times\Sc$, \ and define \
$h:X\to Y$ \ to be the product map \ $\phi\times\Id$, \ and \
$g:\hC\to Y$ \ to be \ $(\psi,\hi)$. \ Thus $\C$ (the \cspace\ for
$\Gamma$) is the preimage of the submanifold \ $\hC\subseteq Y$ \
under $h$.

Let \ $\hV\in\hC$ \ be a configuration where branches $1$ and $2$
are aligned, with distinct direction vectors \ $\oj{1}$, $\oj{2}$ \ in $\Ec$
(the plane of $\Pc$), and let \ $\Vj{k}\in\Ck$ \ be an aligned
configuration with direction vector \ $\oj{k}\in\Ec$. \ Assume that \
$\psi(\hV)=\phi(\Vj{k})$, \ and let \ $\bx\in X$ \ be the
configuration \ $(\Vj{k},\hi(\hV))$.

To prove Theorem \ref{thmain}, we must show that if the point \
$\Vc\in\C$ \ defined by \ $(\hV,\Vj{k})$ \ is not singular of type
(b), then it is smooth. This would follow if $h$ is
locally transverse to $\hC$ at the points \ $\bx\in X$ \ and \
$\hV\in\hC$, \ in other words:
%
\begin{equation}\label{etransv}
%
\Image\dhh_{\bx}~+~T_{\hV}(\hC)~=~T_{\hV}(Y)~.
%
\end{equation}

But \ $T_{\Vj{k}}(\Ck)$ \ is just the null space of \
$A\up{k}$ \ of \eqref{eai}, \ namely: \ $(\oj{k}^{\perp})^{n}$, \
so it suffices to prove that \ $\oj{k}\in T_{\hV}(\hC)$. \
Choose unit vectors \ $\cj{i}$ \ spanning \ $\oj{i}^{\perp}$ \
($i=1,2$) \ in $\Ec$; then \ $\oj{k}\not\in T_{\hV}(\hC)$ \
only if \ $\oj{k}$ \ is proportional to: \
$(\aj{1,2}\cdot\cj{2})\,\oj{1}~-~(\oj{1}\cdot\cj{2})\,\aj{1,3}$, \
which is the vector connecting the meeting point of \
$\Line\up{1}$ \ and \ $\Line\up{2}$ \ with the end point
of \ $\Vj{k}$. \ See \cite{SSB2} for the details.\hfill$\Box$

%
%s3      Kinematic singularities
%
\sect{Kinematic singularities} \label{crks}

Theorem \ref{thmain} gives a \emph{necessary} condition for a
configuration to be singular (topologically): namely, that some
subset $\{i_{1},\dotsc,i_{m}\}$ \ of its branches be aligned, with
direction lines \ $(\Line\up{i_{j}})_{j=1}^{m}$. \ Moreover, the
Pl\"{u}cker line coordinates of these lines must span certain
types of \emph{varieties}, of positive codimension in \ $\RRR^{6}$
\ (see \cite[\S 5]{M}).

We now show how uncertainty singularities give rise to kinematic
singularities, for the polygonal mechanism discussed in Section \ref{cts}.

\subsection{Mechanism architectures}
\label{sarch}

Recall from \S \ref{sks} that the latter require an
\emph{actuation input choice} \ -- \ that is, a choice of input
coordinates \ $\bx_{\iin}$ \ (corresponding to the actuated
joints), and output coordinates  \ $\bx_{\out}$ \ (corresponding
to the end effector of the mechanism: in our case, the position
and orientation of the moving platform).

For the spatial case (i.e., $\Gamma$ embedded in \ $\RRR^{3}$, and
having spherical joints), \ there are three architectures to
consider as a result of platform mobility considerations:
%
\begin{enumerate}
\renewcommand{\labelenumi}{(\alph{enumi})}
%
\item Six branches \
  $\{i_{1},\dotsc,i_{6}\}\subseteq\{1,\dotsc,k\}$, \ where the $j$-th
  branch has \ $\nj{i_{j}}-1$ \ spherical actuators (say, at all
  joints but the last two).
%
\item One branch \ $i_{1}$ \ having a total of \ $\nj{i_{1}}$ \
  spherical actuators (at each joint but the last); and three more
  branches \ $i_{2}$, \ $i_{3}$, \ and \ $i_{4}$, \ with \
  $\nj{i_{j}}-1$ \ spherical actuators in the $j$-th branch \
  ($j=2,3,4$).
%
\item Two branches \ $i_{1}$, \ $i_{2}$ \ with actuators at each joint
    but the last; and one more branch \ $i_{3}$ \ with \
  $\nj{i_{3}}-1$ \ spherical actuators.
%
\end{enumerate}

\subsection{Kinematic analysis through screw theory}
\label{ssth}

We use screw theory (see \cite{DH}) to describe the forces operating
at each joint of our mechanism:

A \emph{screw} $\$ $ is a Pl\"{u}cker vector in \ $\RRR P^{5}$, \
describing a line \ -- \ or equivalently, the position and
direction of vector \ -- \ in \  $\RRR^{3}$ \ (cf.\ \cite[\S
5]{M}). Thus  Fig. \ref{fig:screws} depicts the equivalent
kinematic chain of an arbitrary branch $i$ of a mechanism, where
the movement of each spherical joint $j$\ is described by three
unit screws \ $\sj{i}_{j,1}$, \ $\sj{i}_{j,2}$, \ and \
$\sj{i}_{j,3}$ \ attached to its center. Note that just before a
passive joint, we can make do with two screws.

Considering the $i$-th branch as an open chain, we can express the
instantaneous twist of the end-effector as:
%
\begin{equation}
\label{eq:vector_loop_equation}
%
 \$_{p}~=~\sum_{j=0}^{\nj{i}}\,\sum_{k=1}^{3}\,
\sj{i}_{j,k} \cdot \dot{\theta}\up{i}_{j,k}
%
\end{equation}
%
where \ $\theta\up{i}_{j,k}$ \ is the input coordinate for the \
$\sj{i}_{j,k}$ \ screw (cf.\ \cite[\S 5.6]{T}). \newnot{4-1}

\begin{figure}[htb]
\centering \epsfysize=7cm \leavevmode \epsffile{fig_4/screws.eps}
\caption{Equivalent kinematic structure of a branch}
\label{fig:screws}
\end{figure}

In order to rid \ \eqref{eq:vector_loop_equation} \ of the passive
joints, for the $i$-th branch, we must multiply both sides by
appropriate reciprocal screws. More precisely, choose a basis \
$\{\sjp{i}_{t}\}_{t=1}^{\laj{i}}$ \ for the space \ $\Vj{i}$ \ of common
reciprocals of the screws of all passive joints for this branch:

\begin{enumerate}
\renewcommand{\labelenumi}{(\alph{enumi})}
%
\item If the branch has \ $\nj{i}-1$ \ spherical actuators, at all
  joints but the last two, the reciprocal screw for these two joints
  corresponds to the line passing through them (i.e., \ $\Vj{i}$ \ is
  one-dimensional).
%
\item If the branch has \ $\nj{i}$ \ spherical actuators, at all but
  the last joint, \ $\Vj{i}$ \ is   $3$-dimensional, with the
  corresponding lines all passing through the last joint (see \cite[Ch. 5]{T}).
%
\end{enumerate}

For the $i$-th branch we obtain a system of \ $\laj{i}$ \ linear
equations for \ $\$_{p}$:
%
\begin{equation}
\label{eq:one_branch_screw}
%
\sjp{i}_{t} \cdot \$_p~=~\sum_{j=0}^{\nj{i}}\,\sum_{k=1}^{3}\,
\sjp{i}_{t}\cdot \sj{i}_{j,k}\cdot \dot{\theta}\up{i}_{j,k}
%
\end{equation}
%
($t=1,\dotsc,\lj{i}$), \ in which of course the unactuated inputs \
$\dot{\theta}\up{i}_{j,k}$ \ have zero coefficient.

Combining the \ $\lambda=\sum_{i=1}^{k}\,\laj{i}$ \ equations \
\eqref{eq:one_branch_screw} \ for all $k$ branches, we
obtain equation \ \eqref{eq:gosselin_sing} \ for the chosen
architecture \ ($\lambda=6$ \ for the first two, and \ $\lambda=7$ \
for the third).

The \ $\lambda\times 6$ \ matrix \ $J_{\out}$ \ will take the form:
%
\begin{equation}
\label{eq:six_branches_Jout}
%
J_{\out}=
\begin{pmatrix}
\sjp{1}_{1}\\
\vdots\\
\sjp{1}_{\laj{1}}\\
\vdots\\
\sjp{k}_{\laj{k}}
\end{pmatrix}
%
\end{equation}
%
whose rows are the reciprocal screws of all actuated branches.

The matrix \ $J_{\iin}$ \ is bloc-diagonal:
$$
J_{\iin}~=~
\begin{pmatrix}
A\up{i_{1}} & \dotsc & 0     \\
\vdots      & \ddots & \vdots \\
0           & \dotsc & A\up{i_{k}}
\end{pmatrix}~,
$$
%
with the $i$-th bloc (for the $i$-th actuated branch, with $d$ passive
joints) a \ $\laj{i}\times(\nj{i}-d)$ \ matrix of the form:
$$
A\up{i}~:=~\begin{pmatrix}
\sjp{i}_{1}\cdot\sj{i}_{0,1} & \dotsc & \sjp{i}_{1}\cdot\sj{i}_{\nj{i}-d,3}\\
\vdots & & \vdots \\
\sjp{i}_{\laj{i}}\cdot\sj{i}_{0,1} & \dotsc &
\sjp{i}_{\laj{i}}\cdot\sj{i}_{\nj{i}-d,3}
\end{pmatrix}~.
$$

\begin{prop}\label{pkinsing}
%
For a polygonal mechanism $\Gamma$ (with no unactuated branches), there
is an instantaneous kinematic singularity of type I or III at any
uncertainty singularity.
%
\end{prop}

\begin{proof}
%
At an uncertainty singularity at least two branches are aligned,
so the reciprocals to the passive joint(s) are reciprocal to all
screws of these branches, and thus \ $A\up{i}=0$ \ for these branches. Since \
$\lambda\leq 7$, \ $J_{\iin}$ \ is singular.

Now by Theorem \ref{thmain}, an uncertainty singularity can have:
%
\begin{enumerate}
%
\item Two coaligned branches, each with a pair of unactuated joints:
  they have a common reciprocal (and \ $\lambda=6$), \ so \
  $J_{\out}$ \ has rank \ $\leq 5$. \ The same holds in the second
  architecture whenever two branches are coaligned.
%
\item Three aligned branches whose lines lie in a planar pencil, each with
  a pair of unactuated joints: in this case the corresponding screws
  are linearly dependent, so again \ $J_{\out}$ \ has rank \ $\leq 5$.
%
\item Four aligned branches whose lines are in one plane: the lines of
  those with a pair of unactuated joints each lie in a
  planar pencil (rank $3$). In the first architecture, the last two
  lines each add at most $1$ to the rank; in the second, adding the last
  line form a degenerate congruence (total rank $4$). Thus in any case \
  $J_{\out}$ \ has rank \ $\leq 5$.
%
\end{enumerate}
%
\end{proof}

%
%s4      An illustrative example
%
\sect{An illustrative example} \label{cve}

We now give an visual presentation of an uncertainty singularity in an
explicit $3$-URU $3$-DOF mechanism $\Gamma$, introduced
in \cite{ZFB}.

\begin{figure}[htb]
\centering \epsfysize=5.6cm \leavevmode \epsffile{fig_4/robot.eps}
\caption{Singular config. for a 3-DOF 3-URU mechanism}
\label{fig:triangle2} \end{figure}

The centers of the three U-joints at the base (platform) form an
equilateral triangle. The three R-joint axes fixed in the base
(platform) meet (not at the center of the base(platform) triangle) in
a "Y" shape. The three intermediate R joints in each branch are all
parallel.

\begin{remark}\label{rem:planarity}
%
To enable visualization near the singular configuration \ $\Vc\in\C$, \
we need a mechanism with instantaneous mobility near $\Vc$ equal to
$3$. We therefore make sure that the intersection point of the three
R-joint axes avoids the center of the base platform (otherwise
mobility would be increased by the additional spin dexterity of the
extended branch).
%
\end{remark}

In the region in $\C$ where $\Gamma$ acts as a planar mechanism
(see \cite{ZFB2}), the pose of the moving platform $\Pc$ (an equilateral
triangle) is determined by two coordinates (say, $x$ and $y$) of its
barycenter $\bp$, and its planar rotation $\phi$. Denote by $r$
the common distance \ $d(\bp,\ppj{i})$ \ ($i=1,2,3$).

The \wspace\ for each vertex \ $\ppj{i}$ \ of $\Pc$ is an annulus \
$\Aj{i}$ \ centered at the fixed endpoint \ $\xj{i}$ \ of the $i$-th
branch, with radii \ $\lj{i}_{1}\pm\lj{i}_{2}$ \ respectively.
Thus if we fix the orientation $\phi$ of $\Pc$, the resulting
constrained \wspace\ \ $\Wc_{\phi}$ \ for $\bp$ (the shaded area in
Fig. \ref{annuli}) is the intersection of three annuli (with centers
at \ $\Tj{i}$): \ namely, the displacements \ $\hAj{i}$ \ ($i=1,2,3$) \
of \ $\Aj{i}$ \ by a vector \ $\ppj{i}\bp=\xj{i}\Tj{i}$ \ of length $r$.

\begin{figure}[ht]
\centering
\epsfysize=7cm %\epsfxsize=6.4cm\leavevmode
\epsffile{fig_4/annuli.eps} \caption{constrained \wspace\ \
$\Wc_{\phi}$\ for $\bp$ with fixed $\phi$} \label{annuli}
\end{figure}

Thus $\C$ is described in the vicinity of $\Vc$ by the location of
$\bp$ in \ $\Wc_{\phi}$ and the orientation \ $\phi\in I$. \
Generically, the \wspace s $\Wc_{\phi}$ \ will all be homeomorphic to
a fixed space \ $\Wc'$ \ for nearby values of $\phi$, so that a
neighborhood of $\Vc$ will be a topological product \ $I\times\Wc'$. \
However, for some values of $\phi$, \ $\Wc_{\phi}$ \ may vanish, so
a larger neighborhood $U$ of $\Vc$ need not be connected (see Fig.
\ref{fig:2 boxes}).

\begin{figure}[ht]
\centering
\epsfysize=5.8cm %\epsfxsize=6.4cm
\leavevmode \epsffile{fig_4/before_boxes_kiss.eps}
\caption{\cspace\ for $3$-RRR mechanism} \label{fig:2 boxes}
\end{figure}

In addition, we need discrete data on the elbow up/down position of
each branch at $\Vc$, so the relevant region of $\C$ will actually
contain eight identical copies of $U$, each consisting of one or more
product ``boxes'' as above. Since the boundaries of \ $\Wc_{\phi}$ \
are at the boundaries of the annuli \ $\hAj{i}$, \ where the links of
the $i$-th branch are aligned, the faces of each box are identified
with a corresponding face in another box.
The aligned poses can be calculated analytically using the algorithm
in Gosselin and Merlet (cf.\ \cite{GM}), since each of the extreme
situations can be treated as an equivalent $3$-RPR robot, whose
link lengths are fixed and known.

For example, gluing faces for the situation depicted in Fig.
\ref{fig:2 boxes} yields two $3$-tori. (Of course, this is only
true in the region where the mechanism platform motion is planar \
-- \ see Remark \ref{rem:planarity}; so all we can conclude about
$\C$ near $\Vc$ is that it has two connected components, locally
isomorphic to \ $\RRR^{3}$.)

As the parameters vary, we find that the two connected components
of $U$ approach each other, and for an appropriate value they
touch at one point (see Fig. \ref{fig:box_kiss}).  As a result,
$\C$ is now locally homeomorphic to \ $\RRR^{3}
\vee_{\Vc}\RRR^{3}$ that is a singular point where two regions
meet.

\begin{figure}[ht]
\centering
\epsfysize=5.8cm %\epsfxsize=6.4cm
\leavevmode \epsffile{fig_4/box_kiss.eps} \caption{\cspace\ is
locally \ $\textbf{T}^3\vee_c \textbf{T}^3$} \label{fig:box_kiss}
\end{figure}


% conclusions

\section{Conclusions}

In this paper we derive necessary conditions for the existence of
topological (uncertainty) singularities in a class of spatial and
planar robots having arbitrary number of legs and arbitrary number
of spherical joints in each leg, all attached to a moving planar
platform. We conclude that a topological singularity emerges when
one of the following holds: two legs are aligned and have common
direction, three legs are aligned  and form a flat pencil or when
four legs are aligned and lie in the same plane.

We then relate these topological singular points with the well
known kinematics ones.

Lastly, these results are used in an illustrative example for a
topological singular point, where intrinsically the robot
actuation is not sufficient \--\ regardless of the actuators
placements.

\chapter{Discussion}
\label{Disc}

Recently a growing interest has arise applying toplogy theory to
Robotics in general and to parallel robots configuration spaces in
particular. To this end, main focus had been set on the \cspace s
of a type of robot called \emph{polygonal linkage}, which is
simply a concatenation of links and hinged joints forming a closed
chain, obviously these are of limited practical use. Generally
speaking for a given robot the \cspace \ dimension equals its
mobility, thus for most robots in use the \cspace \ is of
dimension three or more. Furthermore since investigation focuses
usually on global properties of the \cspace,
common mathematical tools are useless. \\

In what follows we discuss our investigation generally without
getting into detailed mathematics and formal defintions.  In order
to keep things simple we have started our investigation studying a
robot we named \emph{arachnoid mechanism} which admits a two
dimensional \cspace \ if as depicted in Figure
\ref{fig:mechanism}. The robot is constructed of $k$ branches
fixed at their one end and a common other end for all
branches,(the figure depicts a planar robot with branches having
only two links while our discussion holds also for the spatial
case where each branch may have an arbitrary number of links).
Note that this type of mechanism resembles some parallel robots
which are in practical use. Our first task was to examine if its
\cspace\ is a manifold, this could be easily done by applying the
regular value theorem (see \cite[I, Thm.~3.2]{Hi}): By identifying
the \cspace\ as the pre-image of a certain map $G :\Re^{d(N-k+1)}
\to R^N$, that sends each two successive joint loci to their
distance and finding the conditions for the jacobian $dG$ to have
full rank amounts to Theorem \ref{thm:main}. Thus \cspace\ of an
arachnoid mechanism is generically a closed orientable manifold of
dimension $d(N - k + 1)-N$, where $N$ is the total number of links
in the given mechanism and $d=2,3$ for the planar and spatial
cases respectively. Furthermore the cases in which this is not
true (i.e. the \cspace \ possess at least one topological
singularity) are when two branches (or more) are co-aligned (see
Definition
\ref{def:aligned},1) or when $d+1$ branches are aligned (Definition \ref{def:aligned},2).\\

We continue our investigation by explicitly characterizing the
topology of a certain kind of arachnoid mechanism: The \cspace \
for any branch and base-point $x$ is an $n$-torus $T^n$, with
$\phi : T^n \to W$ which maps into the associated work space. Note
that the fiber $\phi^{-1}(z)$ over any point $z \in \Int W$ is the
\cspace\ for the closed chain with links lengths
$\ell_{0},\ell_{1},\cdots,\ell_{n}$, where $\ell_{0} := z - x$. If
$z$ is on the boundary of the work space, then $\phi^{-1}(z)$ is
evidently discrete. Thus the \cspace \ is the pullback:
\begin{equation}
\label{eq:cspace}
\C~=~\{(\tau_{1},\dotsc,\tau_{k})\in\prod_{i=1}^{k}~\T{\nj{i}}~ |\
\phi_{1}(\tau_{1})=\dotsc=\phi_{k}(\tau_{k})\in\Wc\}.
\end{equation}

For the special planar case where all branches have exactly two
links each point within the work space $\textrm{Int}(W)$
corresponds (within the \cspace) to a discrete number of copies
(see Figure \ref{fig:annulus intersection}), thus
$$\textrm{Int}(W)\times \{0,1,2,\cdots,k\} \in \C$$ while a point
within the work space boundary $\partial W$ corresponds to certain
suitable gluing of the $\textrm{Int}(W)$ replicas, this way we can
explicitly calculate the Euler characteristics from the work space
as stated in Theorem \ref{thm:sss}. In the non-generic case we can
carefully follow the singularities emerging by "traveling" through
the \emph{moduli space}: the latter is the space of geometric
characterization of the mechanism in hand, i.e. lengths of rods
and platforms geometry. This space is divided by \emph{walls} into
\emph{cells} where all points (mechanisms) within a given cell are
associated to the same \cspace. Hence in order to understand the
topological singularities within a wall (non-generic mechanisms)
we simply travel from cell to cell and track changes. We have
found that the singularities emerging are \emph{pinch points} of a
discrete number of $2$-discs which is given as a theorem in
\ref{prop:2node}. \\

In order to take advantage of the results discussed above we
continued our study with the problem of motion planning for a
\emph{star-shaped mechanisms} (see Figure \ref{X-shape})which are
simply planar arachnoid mechanisms. Since we consider only a point
end-effector, the direct kinematics is straightforward while the
inverse kinematics is more complex. Thus, while for most parallel
mechanisms using \cspace as a mean for path planning should be
carefully considered, here our approach is natural. Actually (a
slightly modified) motion planing strategy taken below can be
applied to any planar robot with a simple graph topology, though
we will not consider such here. In view of equation
\ref{eq:cspace} and the question of motion planning physibility we
would like to inquire if a fiber contains a given configuration
\--\ which later on we think of as the goal configuration:\\
We first follow some topological properties of the \cspace s (the
number of components and the structures of the components) of
single-loop closed chains with spherical joints (see Figure
\ref{fig:single-leg}). These properties drove (see Theorem~2 in
\cite{MT2}) the design of a complete, polynomial-time motion
planning algorithm, which we shall use here. "Adding up" more
branches (see Figure \ref{fig:double-leg}) we can define the work
space \emph{critical} parts $\Sigma=\left(\bigcup_{i=1}^k \Sigma_i
\right)\bigcap W_A$ as the regions where at least one of the
branches is aligned (see Figure \ref{fig:chambers}). \\
We proceed by introducing the conditions (Theorem \ref{prop-2} and
Corollary \ref{cor-1}) in which two given configurations are in
the same connected component of the \cspace \ which simply put is
true iff each branch \textbf{that cannot be aligned in any way}
need not change its \emph{Elbow UP/Down}
signs (of a set of links called \emph{large links}) in order to move from
initial configuration to goal (see Figure \ref{fibre}).\\

Our strategy is to solve the path existence based on the set of
critical regions in the workspace, and then construct the path
combining our knowledge of the workspace and the structure of the
\cspace \ of single-loop closed chains. Note that the problem is
not just moving the junction point between an initial and a goal
position, but moving the manipulator along with all its legs from
an initial configuration to a goal configuration. So, the
workspace information will be insufficient for path construction.
We employ a move that changes the shape of a leg with its endpoint
fixed in the workspace. This move, called the \emph{sign-adjust
move}, uses the knowledge of the \cspace \ of a single-loop closed
chain. We then showed that the overall complexity of the motion
planning algorithm is $O(k^3N^3)$, where $N$ is the maximum number
of links in a leg and $k$ is the number of branches. The
polynomial complexity is key to the applications like folding of
macro- molecules, which can be modeled as a closed chain with
large $k$
and N. \\

Our next goal was to study the \cspace\ of mechanism we call a
\emph{polygonal mechanism} which consists of a moving polygonal
platform, having branches attached to each vertex, with the other
end fixed in $\Re^d$, while each branch is a concatenation of rods
with revolute (i.e., rotational) joints at the consecutive meeting
points.(see Figure \ref{fig:kgon}). As done before, we use the
regular value theorem to find the conditions in which a polygonal
mechanism \cspace\ is a manifold. Proving the above involves a
careful examination of the geometry that uniquely determines the
mechanism (see Remark \ref{rpolygon}). Furthermore by
\emph{generic} we mean that there is a zero-measure of non-generic
cases, formally we define generic in Definition \ref{fsingular}.

The map we consider $F:\RR{dN}\to\RR{N+|\Ic|}$, \ (see proof of
Theorem \ref{tone}) is defined as:
%
\begin{equation}\label{econstr}
%
F(\Vc)~=~F(\Vj{1},\dotsc,\Vj{i})~:=~
(f_{\nj{1}}(\Vj{1}),\dotsc,f_{\nj{k}}(\Vj{k}),
\|\aj{1,2}\|^{2},\dotsc,\|\aj{k-1,k}\|^{2})~,
%
\end{equation}
%
Which admits a full ranked jacobian iff all participant branches
are aligned and  $\sum^{k}_{i=1}\vj{i} $, on the moving platform.
Concluding these results:
\begin{enumerate}
\item Two participants vectors is not a generic case (Figure
\ref{fig:sing1}).
\item Four or more participants vectors is not a generic case and we disallow
it.
\item Three participants vectors is a generic case - (coupler curve
intersection with a circle)
\end{enumerate}
So we need the �finer� \emph{Transversality theorem} handling the
three aligned legs case to find that for a generic polygonal
mechanism, any configuration having at most three aligned branches
is smooth (Proposition \ref{pzero}). Finally by redefining the
"three aligned branches case" as one where the three branches are
aligned, with direction lines in the same plane meeting in a
single point as depicted in Figure \ref{fig:sing3} (which is
referred to in literature as a flat pencil) we complete Theorem
\ref{tone} proof.\\

Thus we gave a necessary condition for an uncertainty singularity
to occur in a polygonal spatial/planar mechanism with arbitrary
number of branches and arbitrary number of links in each branch.
Naturally one can inquired about the corresponding instantaneous
kinematic singularities:\\
First (Section \ref{sarch}) we numerate all edundant/non-redundant
architecture possibilities for statically defining the moving
platform, then we use screw theory to describe (Proposition
\ref{pkinsing}) the kinematic singularities emerging due to the
topological ones. Finally we give a visual presentation (Figure
\ref{fig:box_kiss}) of an
uncertainty singularity in an explicit 3-URU 3-DOF mechanism.\\

We Conclude our study by constructing a morse function for the
simplest type of polygonal mechanism \ -- \ namely, planar
mechanisms \ ($d=2$) \ having triangular platforms \ ($k=3$) \ and
exactly two links per branch \ ($\nj{1}=\nj{2}=\nj{3}=2$). We
propose a morse function \ref{ttwo}, prove it is Morse and find
its critical points (Figures
\ref{fcaseI},\ref{fcaseII},\ref{fcaseIII}), and give a computation
for a mechanism having $S^3$ \cspace (Figure \ref{fmechanism}).

\section{conclusions}

Configuration space is hard thing to follow. Nevertheless for the
two dimensional case (see chapter \ref{chap2} - the
\emph{Arachnoid mechanism}) were actual visualization is possible
one can "find his way". Configuration space had been found
explicitly: we devised an easy way to "build up" this space as a
handle body, which if followed can be used as a guide for further
bigger dimensioned \cspace \ investigations. Singularities results
in this regard can be easily applicable either in design stage or
for motion planning and control in cases were the robot \cspace \
is intentionally designed to contain topology singularities. Thus
a further investigation of topological singularities may naturally
begin with these results. We have found that said singularities in
the two dimensional case, are simply a wedge of discs. Clearly,
generally this is not the case, and apart of presentations of
$3$-dimensional \cspace \ singularities for given mechanisms, we
haven't found any characterization, but it seems like a good
subject to deal with in the future.

The question of the manifold character of \cspace \ which arose at
the beginning of our research had popped again once we had begun
thinking of the \emph{polygonal mechanisms}, except in this case,
solution was not an easy task. We had found that polygonal
mechanisms have generically a manifold \cspace. Moreover we
introduced three necessary conditions for a polygonal mechanism to
have a topological singularity. We underline here that these are
not sufficient, for example consider a triangular mechanism having
$2,1,1$ links in its associated first, second and third branches.
If we (virtually) disconnect the first branch from the rest of the
mechanism we end with an open $2$-chain and a coupler mechanism.
Consequently the work space of the $2$-open chain is an annulus,
while the coupler's work space is a coupler curve. Note that for
the situation depicted below where the coupler curve is tangent to
the annulus, while the annulus intersects the coupler curve into
two segments, all branches are aligned and their lines of
alignment intersect in one point (the tangency point). Still the
configuration where both curves are tangent does not induces a
topological singular configuration. Thus there is a need to
continue the research, and come up with a sufficient condition for
topological singularities.
%
\begin{figure}[h]
  \centering
  \includegraphics[width=3in]{fig-conclusion/non-singular.eps}
  \caption{A counter example for singular criterion presented in last chapter.}
  \label{non-singular}
\end{figure}
%
The manifold character of the generic star-shaped mechanism,
generic arachnoid mechanism, generic polygonal mechanisms is
herein proved, while the manifold character for simple closed
chain mechanism (link loops) had been proven in literature.

On the other hand Kempe \cite{Kem} proved in 1876 that given an
arbitrary real algebraic curve there exist a linkage such that one
of its vertices will trace the curve. Jordan and Steiner \cite{JS}
construct an homeomorphism between any given algebraic variety and
some components of a mechanical linkage \cspace.\\

The above may indicate a more general truth. Thus we conclude this
dissertation with the conjecture:\\

\begin{conj}
All planar (spatial) graph mechanisms comprised of rigid rods and
rotational (spherical) joints have a smooth orientable manifold
\cspace s for a generic set of lengths
\end{conj}

By the term \emph{generic} we mean "almost every set of lengths".


\begin{center}Thats all folks...\end{center}

\chapter{Appendix - The \cspace \ as a fibred product.}
\label{appendix}

Configuration space can be defined as \emph{the set of all
possible embeddings of a mechanism into the ambient space}. While
this is sufficient for most discussions one should distinguish
between two possible definitions, one in which transformation of
the mechanism as a hole is "permitted" and one where those are
"forbidden". This two does not always endow equivalent results:

Let a \emph{graph mechanism}, be a graph \ $(G(v,e),\mathcal{L})$
\ having edge and vertex sets such that \ $e(G) \subseteq
v(g)\times v(g)$ \ and a set of fixed lengths
$\mathcal{L}=\{\ell_{(i,j)}|(i,j)\in e(G)\}$. Also denote by
$\textbf{x}_i$ the location of the $i$-th vertex. Under these
notations the \cspace \ we describe in this dissertation is the
quotient \ $\mathcal{C}=\mathcal{R}(G)/\Lambda$ \ where
%
$$
\R(G)=\{\textbf{x}_i \in \RR{v(G)}|i\in v(G),
\|\textbf{x}_i-\textbf{x}_j\|=\ell_{(i,j)} \}
$$
%
and $\Lambda$ \ the group of isometries of translations and
rotations in \ $\RR{2}$, or in \ $\RR{3}$. Thus \ $\Lambda$ \
"fixes" the the mechanism to a fixed frame. The difference between
these two definitions for the \cspace \ can be realized in the
following. A subgraph $H$ \ of graph \ $G$ \ is a graph such that
\ $v(H) \ \subset \ v(G)$ \ and \ $e(H) \ \subset e(G)$. Which
leads to the definition:
\begin{defn}
A \emph{sub-mechanism} \ $(H,\mathcal{L}|_{H})\subset (G,\L)$ \ is
a subgraph \ $H$ \ of \ $G$\ together with the restricted length
subset \ $\L|_{H}$, and we denote such a sub-mechanism simply by \
$H$. \ A \emph{leg} is a sub-mechanism such that all vertices in
$H$\ has valence $1$\ or $2$.
\end{defn}

In light of these definitions one can define a feasible
configuration of a parallel mechanism as such that all of its
branches "agree" upon one (or more) leg configuration. One could
naively claim that for a arachnoid mechanisms the \cspace \ is
simply the fibered product of the \cspace of its legs, but this is
not true. Alternatively we know that:
\begin{thm}
Given a mechanism \ $(G,\L)$ \ with a set \ $H_{1},H_{2},H \subset
G$ \ of its sub-mechanisms such that \ $H_{1}\cap H_{2}=H$ \ then
\ $\R(\L)$ \ is the pullback of the the natural projections:
%
$$
\R(H_{1}) \xrightarrow{\pi_{H_{1}}} \R(H) \xleftarrow{\pi_{H_{2}}}
\R(H_{2}).
$$
\end{thm}

Note that this does not extend to a the \cspace \ $\C$ \ since the
induced projections \ $\{\R(H_{i})/\Lambda \rightarrow
\R(H)/\Lambda\}$ \ are determined up to isometry \ $\Lambda$ \ and
thus do not determine a unique point. \\
%\textbf{Example 1:} take the spider mechanism build of 3 legs each
%having 2 links, $H_1,H_2,H_3$ \ then $\C(H_i)=S^1$ \ (for we
%should fix one link) for $i=1,2,3$ \ and $\C(H_1\cap H_2\cap
%H_3)=pt.$ while $\C$ of the entire spider mechanism should be a
%surface (of dimension two).\\
for example take the mechanism depicted in figure \ref{mech} and
denote the two rods $H_1,H_2$ \ then $\C(H_i)=pt$ (for we should
fix one link) and
$\C(H_1\cap H_2)=pt.$ $\C$ of the entire mechanism is obviously $S^1$.\\
\begin{figure}[h]
\label{mech} \centering
\epsfysize=6cm %\epsfxsize=6.4cm
\leavevmode \epsffile{fig_appendix/mech.eps} \caption{Example
mechanism}
\end{figure}
For convenience we restrict the proof to the case where there are
only $2$ sub-mechanisms, but this can be trivially extended to any
number of sub-mechanisms:
\begin{proof}


Let $H:=H_1 \cap H_2$ \ as before. in order to show that
$$\R(G)=\R(H_1)\times_{H} \R(H_2)$$
we should show that for every space $\R$ \ and maps
$\tilde{P}_{H_i}:\R \rightarrow \R(H_i)$  \ $(i=1,2)$ \ such that
$\tilde{P}_{H_1}\circ \pi_{H_1}=\tilde{P}_{H_2}\circ \pi_{H_2}$ \
then there exist a unique map $\mu:\R \rightarrow \R(G)$, such
that the following diagram commutes (that is for $i=1,2$ \ the
identity $P_{H_i}\circ\mu=\tilde{P}_{H_i}$ holds)
$$
\xymatrix@R=25pt{ \hspace*{3.5mm} & \ar[dl]_{\tilde{P}_{H_1}} \R \ar[dr]^{\tilde{P}_{H_2}}\ar[d]^\mu &  \\
\R(H_1)\ar[dr]_{\pi_{H_1}} & \ar[l]_{P_{H_1}}\R(G)\ar[r]^{P_{H_2}} & \R(H_2)\ar[dl]^{\pi_{H_2}}\\
&\R(H)&}
$$
So we assume the existence of such $\R$ \ and maps
$\tilde{P}_{H_i}$ such that the diagram commutes and use the known
fact that \emph{the \cspace\ of a cloud of points is a direct
product} to prove the uniqueness of $\mu$.\\ Denote the \cspace \
of $v(K)$ \ \emph{free} points in $\RR{2}$ by \ $\C_{f}(K)$ \ for
some submechanism $K$. (here "free" means that we require no
distance constraints).

The \cspace \ of the cloud of points $v(G)$ \ is a direct product
of  $\C_f(H_1)$ \ and $\C_f(H_2)$:
%
$$
\C_f(H_{1})\xleftarrow{P_1} \C_f(G) \xrightarrow{P_2} \C_f(H_{2})
$$
where $P_i$ \ are projections. If there is an $\R$ \ with the
corresponding maps $$ \C(H_{1})\xleftarrow{\tilde{P}_1} \R
\xrightarrow{\tilde{P}_2} \C(H_{2})$$ such that its pullback
diagram commutes, then there is a unique map $\mu:\R \rightarrow
\C_f(G)$. Note there are inclusion maps $j_K:\R(K) \hookrightarrow \C_f(K)$ for all submechanisms $K$.\\

Now, if we set $\tilde{P}_i=j_{H_i} \circ \tilde{P}_{H_i}$\ then
there is a unique map $\mu$\ such that $P_i \circ \mu = j_{H_i}
\circ \tilde{P}_i$ (and keeps the diagram commutative). But since
$j_{H_i} \circ \tilde{P}_i$ \ ''keeps distances'' so is $P_i \circ
\mu$ \ lastly since $P_i$ \ is just a projection we have proven
that $\mu$ ''keeps'' distances so we can deduce that $\mu(\R)
\subset \R(G)$ which completes the proof (and wonderful 3.5 years)
(since in the $\C_f$-diagram $\mu$ is unique).
%
\end{proof}


\begin{thebibliography}{9}

%---------start - into ---------
\bibitem{LLL} G. Liu, Y. Lou \& Z. Li, \emph{SSingularities of Parallel Manipulators: A Geometric Treatment},
\textit{IEEE Trans. on Robotics And Automation} \textbf{19,4}
(2003), 579-594.


\bibitem[C]{C} D.\ R.\ J.\ chillingworth,
``Differential Topology with a view to Applications'',
\textit{Research Notes in Mathematics, $\pi$}\ \textit{Vol. 9},
2001.
%---------end - intro ---------
%
%
%
%---------start - chapter 1 ---------
\bibitem{E} D.\ Eldar, \emph{Maps and Machines}, Hebrew Univ. M.Sc. project, in preparation,
{\tt http://www.math.toronto.edu/~drobn/People/Eldar}.

\bibitem{F} M.\ Farber, \emph{Instabilities of Robot Motion}, preprint, 2002 \ {\tt cs.RO/0205015}.

\bibitem{GS} S.\ Goldberger, M.\  Shoham and O.\ Ben-Porat, \emph{Robots in Surgery
- Reality or Vision}, \textit{Ob.\ Gyn.\ Update} \textbf{22}
(1997), 4-8.

\bibitem{GA} C.\ Gosselin \& J.\ Angeles, \emph{Singularity analysis of Closed Loop
Kinematic Chains}, \textit{IEEE Trans. on Robotics And Automation}
\textbf{6} (1990), 281-290.

\bibitem{HK} J.C.\ Hausman \& A.\ Knutson, \emph{The Cohomology Ring of Polygon
Spaces}, \textit{Ann.\ Inst.\ Fourier (Grenoble)} \textbf{48}
(1998), 281-321.

\bibitem{Hi} M.W. Hirsch, \textit{Differential Topology}, Springer-\-Verlag,
Berlin-\-New York, 1976.

\bibitem{Ho} M.\ Holcomb, \emph{On the Moduli Spaces of Multipolygonal Linkages in the Plane},
preprint, 2003, {\tt math.GT/0307001}.

\bibitem{HR} J.C.\ Hausman \& E.\ Rodriguez, \emph{The Space of Clouds in an
Euclidan Space}, preprint, 2002.

\bibitem{K} Y.~Kamiyama, \emph{Topology of equilateral polygon linkages in the
(Euclidean) plane modulo isometry group}, \textit{Oska J. Math.}
\textbf{36} (1999), 731-745.

\bibitem{KM} M.\ Kapovich \& J.\ Millson, \emph{On the moduli space of polygons in
the Euclidean plane}, \textit{J. Differential Geom.} \textbf{42}
(1995), 430-464.

\bibitem{KT} Y.\ Kamiyama \& M.\ Tezuka, \emph{Topology and Geometry of Equilateral
Polygon linkages in the Euclidean plane}, \textit{Quart. J.\ Math.
(Oxford)} \textbf{50}, 1999, 463-470.

\bibitem{KTT} Y.\ Kamiyama, M.\ Tezuka \& T.\ Toma, \emph{Homology Of the
Configuration spaces of Quasi-equilateral Polygonal Linkages},
\textit{Trans. Am. Math. Soc.} \textbf{350} (1998),  4869-4896.

\bibitem{MT1} R.J.~Milgram \& J.~Trinkle, \emph{The Geometry of Configuration spaces
of Closed Chains in Two and Three Dimensions}, to appear in
\textit{Homology, Homotopy \& Applications}.

\bibitem{MT2} R.J.~Milgram \& J.~Trinkle, \emph{Motion Planning for Planar $n$-Bar
Mechanisms with Revolute Joints}, in \textit{IEEE Int. Conf. on
Intelligent Robots and Systems} , IEEE, 2001, 1602-1608.

\bibitem{NM}
N.~Simaan \& M.~Shoham, \emph{Singularity analysis of a class of
Composite serial In-Parallel Robots}, \textit{IEEE Transactions On
Robotics and Automation} \textbf{17} (2001), 301-311.

\bibitem{ZFB} D.S.\ Zlatanov, R.G.\ Felton \& B.\ Benhabib, \emph{Classification and
Interpretation of Singularities of Redundant Mechanisms},
\textit{Proceedings of the ASME 24th Annual Design Automation
Conference, Atlanta, GA}, Sept.  1998, Paper $\#$DETC98/MECH-5896.
%---------end   - chapter 1 ---------
%
%
%
%
%---------start - chapter 2 ---------
\bibitem{CG99}
M. Cherif and K.K. Gupta, \emph{Planning Quasi-Static Fingertip
Manipulation For Reconfiguring Objects}. \hskip 1em plus 0.5em
minus 0.4em\relax IEEE Transactions on Robotics and Automation,
Vol. 15, No. 5, PP. 837-848, 1999.

 \bibitem{Koditschek87}
D. E. Koditschek, \emph{Exact Robot Navigation by Means of
Potential Functions: Some Topological Considerations}. \hskip 1em
plus 0.5em minus 0.4em\relax IEEE International Conference on
Robotics and Automation, PP. 1-6, 1987.

\bibitem{Lozano-Perez83}
Lozano-Perez83, \emph{Spatial Planning: A Configuration Space
Approach}. \hskip 1em plus 0.5em minus 0.4em \relax IEEE
Transactions on Computers, Vol. C-32, No. 2, PP. 108-119, 1983.

\bibitem{JRKT01}
D.J. Jacobs, A.J. Reider, L.A. Kuhn, and M.F. Thorpe,
\emph{Protein Flexibility Predictions Using Graph Theory}. \hskip
1em plus 0.5em minus 0.4em \relax PROTEINS: Structure, Function,
and Genetics, 44:150-165.

\bibitem{Cortes02}
J. Cort\'es, T. Sim\'eon, and J.P. Laumond, \emph{A Random Loop
Generator for Planning the Motions of Closed Kinematic Chains
using PRM Methods}. \hskip 1em plus 0.5em minus 0.4em\relax In
Proceedings of the 2002 IEEE International Conference on Robotics
and Automation, pages 2141-2146, 2002.

\bibitem{CS03}
J. Cortes and T. Sim\'eon, \emph{Probabilistic Motion Planning for
Parallel Mechanisms}. \hskip 1em plus 0.5em minus 0.4em\relax In
Proceedings of the 2003 IEEE International Conference on Robotics
and Automation, pages 4354-4359, 2003.

\bibitem{RK88}
E. Rimon and D. E. Koditschek, \emph{Exact Robot Navigation Using
Cost Functions: The Case of Distinct Spherical Boundaries in
$E^n$}. \hskip 1em plus 0.5em minus 0.4em\relax IEEE International
Conference on Robotics and Automation, PP. 1791-1796, 1988.

\bibitem{HLM97}
D. Hsu, J.C. Latombe, and R. Motwani, \emph{Path Planning in
Expansive Configuration Spaces}. \hskip 1em plus 0.5em minus 0.4em
\relax IEEE International Conference on Robotics and Automation,
1997.

\bibitem{RK89}
E. Rimon and D. E. Koditschek, \emph{The Construction of Analytic
Diffeomorphisms for Exact Robot Navigation on Star Worlds}. \hskip
1em plus 0.5em minus 0.4em\relax IEEE International Conference on
Robotics and Automation, PP. 21-26, 1989.

\bibitem{SSB05}
N. Shvalb, M. Shoham, D. Blanc \emph{The Configuration Space of an Arachnoid Mechanism}.
\hskip 1em plus 0.5em minus 0.4em\relax To appear in Forum
Mathemaicum 2005.

\bibitem{BL91}
J. Barraquand and J.-C. Latombe, \emph{Nonholonomic Multibody
Mobile Robots: Controllability and motion planning in the presence
of obstacles}. \hskip 1em plus 0.5em minus 0.4em\relax IEEE
Interantional Conference on Robotics and Automation, PP.
2328-2335, 1991.

\bibitem{Burdick89}
J.W. Burdick. On the inverse kinematics of redundant manipulators:
characterization of the self-motion manifolds. {\em IEEE Int.
Conf. on Robotics and Automation} (ICRA), pages 264$-$270, May
1989.

\bibitem{SHS87}
J. Schwartz, J. Hopcroft, and M. Sharir, \emph{Planning, Geometry,
and Complexity of Robot Motion}. \hskip 1em plus
  0.5em minus 0.4em\relax Ablex, 1987.

\bibitem{RRT}
S. M. LaValle and J. J. Kuffner, \emph{Rapidly-exploring random
trees: Progress and prospects}. \hskip 1em plus
 0.5em minus 0.4em\relax In B. R. Donald, K. M. Lynch, and D. Rus, editors, Algorithmic and
Computational Robotics: New Directions, pages 293-308, A K Peters,
Wellesley, MA, 2001.

\bibitem{SS83}
J.T. Schwartz and M. Sharir, \emph{On the piano movers II. General
techniques for computing topological properties on real algebraic
manifolds}. \hskip 1em plus 0.5em minus 0.4em \relax Adv. Appl.
Math., vol.4, PP. 298-351, 1983.

\bibitem{HA01}
L. Han and N.M. Amato, \emph{A kinematics-based probabilistic
roadmap method for closed chain systems}. \hskip 1em plus 0.5em
minus 0.4em \relax in Algorithmic and Computational Robotics: New
Directions, B.R. Donald, K.M. Lynch, and D. Rus, Eds. AK Peters,
Wellesley, PP. 233-246, 2001.

\bibitem{ABDJV98}
N. Amato, B. Bayazit, L. Dale, C. Jones, and D. Vallejo,
\emph{OBPRM: An obstacle-based PRM for 3d workspaces}. \hskip 1em
plus 0.5em minus 0.4em \relax in Robotics: The Algorithmic
Perspective, P. Agarwal, L. Kavraki, and M. Mason, Eds. Natick,
MA: A.K. Peters, 1998, PP. 156-168.

\bibitem{BOS99}
V. Boor, M. Overmas, and A.F. van der Stappen, \emph{The Gaussian
sampling strategy for probabilistic roadmap planners}. \hskip 1em
plus 0.5em minus 0.4em \relax IEEE International Conference on
Robotics and Automation, 1999.

\bibitem{BK00}
R. Bohlin and L. Kavraki, \emph{Path planning using lazy PRM}.
\hskip 1em plus 0.5em minus 0.4em \relax IEEE International
Conference on Robotics and Automation, PP. 521-528, 2000.

\bibitem{KL00}
J.J. Kuffner and S.M. LaValle, \emph{RRT-Connect: An Efficient
Approach to Single-Query Path Planning}. \hskip 1em plus 0.5em
minus 0.4em \relax IEEE International Conference on Robotics and
Automation, PP. 995-1001, 2000.

\bibitem{LK99}
S.M. LaValle and J.J. Kuffner, \emph{Randomized kinodynamic
planning}. \hskip 1em plus 0.5em minus 0.4em \relax IEEE
International Conference on Robotics and Automation, 1999.

\bibitem{Can88}
J.F. Canny, \emph{The Complexity of Robot Motion Planning}. \hskip
1em plus 0.5em minus 0.4em\relax Cambridge, MA: MIT Press, 1988.

\bibitem{KSLO96}
L.E. Kavraki, P. $\check{S}$vestka, J.C. Latombe,
and M.H. Overmars, \emph{Probablistic Roadmaps for path planning
in high-dimensional configuration space}. \hskip 1em plus 0.5em
minus 0.4em\relax IEEE Transactions on Robotics and Automation,
12(4):566-580, 1996.

\bibitem{Lat92}
J.C. Latombe, \emph{Robot Motion PLanning}. \hskip 1em plus 0.5em
minus 0.4em\relax Kluwer Academic Publishers, 1992.

\bibitem{YLK01}
J. Yakey, S. M. LaValle, and L. E. Kavraki, \emph{Randomized path
planning for linkages with closed kinematic chains}. \hskip 1em
plus 0.5em minus 0.4em \relax IEEE Transactions on Robotics and
Automation, 17(6):951--958, December 2001.

\bibitem{SG95}
J. Sefrioui and C.M. Gosselin, \emph{On the quadratic nature of
the singularity curves of planar three-degree-of-freedom parallel
manipulators}. \hskip 1em plus 0.5em minus 0.4em \relax Mechanism
and Machine Theory, 30(4):533--551, May 1995.

\bibitem{DCSY03}
A.K. Dash,   C. I-Ming,   Y. Song-Huat and Y. Guilin,
\emph{Singularity-free path planning of parallel manipulators
using clustering algorithm and line geometry}. \hskip 1em plus
0.5em minus 0.4em \relax Robotics and Automation, 1:761- 766,
September 2003.

\bibitem{NTU00}
D.N. Nenchev ,Y. Tsumaki and M. Uchiyama
\emph{Singularity-Consistent Parameterization of Robot Motion and
Control} \hskip 1em plus 0.5em minus 0.4em \relax The
International Journal of Robotics Research, 19(2):159-182, 2000.

\bibitem{BS96}
J.R. Banga and W.D. Seider, \emph{Global Optimization of Chemical
Processes Using Stachastic Algorithms}, \hskip 1em plus 0.5em
minus 0.4em \relax State of the Art in Global Optimization:
Computational Methods and Applications, C.A. Floudas and P.M.
Pardalos (Eds.), Kluwer Academic Publishers, pages 563-583, 1996.
%---------end   - chapter 2 ---------
%
%
%
%---------start - chapter 3---------
\bibitem{FSchuH} M.\v{S}.~Farber \& D.~Sch\"{u}tz, \emph{Homology of planar polygon
spaces}, preprint, 2006, \ {\tt math.AT/0609140}.
%
\bibitem{Ha} A.S.~Hall, Jr., \textit{Kinematics and Linkage Design},
Prentice-Hall, Englewood Cliffs, NJ, 1961.
%
\bibitem{Hau} J.-C.~Hausmann, \emph{Sur la topologie des bras articul\'{e}s}, in
S.~Jackowski, R.~Oliver, \& K.~Pawa{\l}owski, eds.,
\textit{Algebraic Topology - Pozn\'{a}n 1989}, Springer Lec. Notes
Math. \textbf{1474}, Berlin-\-New York, 1991, 146-159.
%
\bibitem{Hi} M.W.~Hirsch, \textit{Differential Topology}, Springer-\-Verlag,
Berlin-\-New York, 1976.

\bibitem{Mi} J.~Milnor, \textit{Morse Theory}, Princeton University Press,
Princeton, NJ, 1963.
%---------end - chapter 3---------
%
%
%
%
%---------start - chapter 4-------
\bibitem{Hu2}
K.H.\ Hunt, ``Kinematic Geometry of Mechanisms'', Oxford Universty
Press, Oxford, UK, 1978.
%
\bibitem{Hu1}
K.H.\ Hunt, ``Special configurations of robot-arms via screw
theory, Part 1'', \textit{Robotica} \textbf{4} (1986), 171-179.
%
\bibitem{WW}
K.J.\ Waldron, S.L.\ Wang \& S.J.\ Bolin, ''A study of the
Jacobian matrix of serial manipulator'', \textit{Trans.\ of ASME,
J.\ Mech.\ Trans.\ Aut. in Des.} \textbf{107} (1985), 230-238.

\bibitem{ZBC}
D.S.\ Zlatanov, I.\ Bonev \& C. Gosselin, ``Constraint
Singularities as Configuration Space Singularities'',
\textit{Advances in Robot Kinematics: Theory and Applications},
Kluwer Academic Press, 2002, 183-192.

\bibitem{ZFB1}
D.S.\ Zlatanov, R.G.\ Felton \& B.\ Benhabib, ``Singularity
analysis of Mechanisms and Robots via a Motion--Space Model of the
Instantaneous Kenmatics'', \textit{Proceedings IEEE Conf.\ in
Robotics \& Automation}, 1994, 980-985.
%
\bibitem[Hal]{Ha}
A.S.~Hall, Jr., \textit{Kinematics and Linkage Design},
Prentice-Hall, Englewood Cliffs, NJ, 1961.
%
\bibitem{SSB2}
N.\ Shvalb, M.\ Shoham \& D.\ Blanc, ``The Conifguration space of
parallel mechanisms'', preprint, 2006.
%
\bibitem{M}
J.P.~Merlet, \textit{Parallel Robots}, Kluwer Academic Publishers,
Dordrecht, 2000.
%
\bibitem{T}
L.W.~Tsai, \textit{Robot Analysis - The mechanics of serial and
parallel manipulators}, Wiley interscience Publication - John
Wiley \& Sons, New York, 1999.

\bibitem{GM}
C.M.\ Gosselin \& J.P.~Merlet, ''The Direct Kinematics of Planar
Parallel Manipulators: Special Architectures and Number of
Solutions'', \textit{Mechanism \& Machine Theory} \textbf{29}
(1994), 1083-1097.
%
\bibitem{ZFB2}
D.S.\ Zlatanov, R.G.\ Fenton \& B.\ Benhabib, ``Classification and
Interpertation of Singularities of Redundant Mechanisms'',
\textit{ASME Design Engineering Technical Conference}, 1998, 1-10.
%
\bibitem{DH}
J.K.\ Davidson \& K.H.\ Hunt \textit{Robots and Screw Theory \--\
Applications of kinematics and robotics}, Oxford University Press,
Oxford, UK, 2004.
%
\bibitem{AGV}
V.I.\ Arnold, S.M.\ Gusein-Zade \& A.N.\ Varchencko,
\textit{Singularities of Differentiable Maps}, Vol.\ I, Monographs
in Mathematics, Academic Press, New York, 1985.
%---------end - chapter 4 ---------

%---------start- discussion---------
\bibitem[Kem]{Kem} A.\ B.\ Kempe,
``On a General Method of Describing Plane Curves of the $n$--th
Degree by Linkwork'', \textit{Proc. London Math. Soc.},
\textit{Vol. 7}, 1875.
%
\bibitem[JS]{JS} D. Jordan and M. Steiner,
``Configuration Spaces of Mechanical Linkages'' \textit{Descrete
and Computational Geometry} \textit{Vol. 22} 1999, 297--315.
%---------end- discussion---------

\end{thebibliography}

\chapter{Hebrew introduction}
\newpage
1
\newpage
1


\end{document}
