\chapter{The configuration space of a parallel polygonal mechanism}
\label{chap3}

\textsf{Submitted to \textsl{Homology Homotopy and Applications}, 2007.}\\
\textsf{Co-authors: David Blanc and Moshe Shoham.}\\

 We study the  \cspace\ $\C$ of a parallel polygonal
mechanism, and give necessary conditions for the existence of
singularities; this shows that generically $\C$ is a smooth
manifold. In the planar case, we construct an explicit Morse
function on $\C$, and show how geometric information about the
mechanism can be used to identify the critical points.

%
%s1  Introduction}
%
\sect{Introduction} \label{cint}

The mathematical theory of robotics is based on the notion of a
mechanism, consisting of links, joints, and rigid parts known
as platforms.  The \emph{type} of a mechanism is defined by a
$q$-dimensional polyhedral complex, where the parts of dimension
$\geq 2$ \ correspond to the platforms, and the complementary
one-dimensional graph corresponds to the links and joints.
Here we consider only the polygonal case  \ ($q=2$). \
A specific embedding of this complex in the ambient Euclidean space  \
$\RR{d}$ \ is called a \emph{configuration} of the mechanism. The
collection of all such embeddings forms a topological space, called the
\emph{\cspace} of the mechanism (see \cite{Ha}).

\begin{figure}[htb]
\begin{center}
\epsfysize=4cm %\epsfxsize=6.4cm
\leavevmode \epsffile{fig_3/k_gon_mechanism.eps} \caption{A
pentagonal planar mechanism} \label{fig:kgon}
\end{center}
\end{figure}

The goal of this note is to study the \cspace\ of a mechanism
consisting of a moving polygonal platform, having a
flexible leg (consisting of concatenated rods) attached to each
vertex, with the other end fixed in  \ $\RR{d}$. \ We may think of the
latter as forming the \emph{fixed polygon} of the mechanism, ``parallel''
to the \emph{moving polygon} inside.
The spatial version of such a mechanism, consisting of a
two-dimensional platform free to move in three dimensions, has been
studied extensively, but even the planar version, to which we
later specialize, has practical applications \ -- \ for example, in
micro-electro-mechanical systems (MEMS)\vs.

Our main results are\vsm :

\begin{enumerate}
\renewcommand{\labelenumi}{(\alph{enumi})}
%
\item The \cspace\ of a parallel polygonal mechanism is a smooth
  manifold, except perhaps in some explicitly described  singular
  cases (Theorem \ref{tone}).
%
\item The topology of this manifold can be described for a triangular
  planar mechanism by means of an explicit Morse function (Theorem \ref{ttwo}),
  whose critical points can be identified geometrically (\S \ref{scrit})\vs.
%
\end{enumerate}

We start with some terminology and notation:

\begin{defn}\label{dbranch}
%
A \emph{branch} (of \emph{multiplicity} $n$) is a sequence \
$L=(\ell_{1},\dotsc,\ell_{n})$ \ of $n$ positive numbers, which we
think of as the lengths of $n$ concatenated rods, having revolute
(i.e., rotational) joints at the consecutive meeting points.
%
\end{defn}

\begin{defn}\label{dbconfiguration}
%
A \emph{configuration} in \ $\RR{d}$ \ for a branch \
$L=(\ell_{1},\dotsc,\ell_{n})$ \ consists of $n$ vectors \
$V=(\bv_{1},\dotsc,\bv_{n})$ \ in \ $\RR{d}$ \ with lengths \
$\|\bv_{i}\|=\ell_{i}$ \ ($i=1,\dotsc n$). \ We write \
$\sigma_{V}:=\sum_{i=1}^{n}\,\bv_{i}$ \ for their vector sum.

A branch configuration \ $V=(\bv_{1},\dotsc,\bv_{n})$ \ is
\emph{aligned} with a vector \ $\bv\in\RR{d}$ \ if all the vectors
\ $\bv_{i}$ \ are scalar multiples of $\bv$, which is called the
\emph{direction vector} of $V$. \newnot{3-3}
%
\end{defn}

The \emph{\cspace} \ $\C(L)$ \ of a branch $L$ is the product of $n$
spheres in \ $\RR{d}$ \ of radii \ $(R_{i}=\ell_{i})_{i=1}^{n}$. \
Up to homeomorphism, this is independent of the order on $L$, so we can (and
shall) assume \ $\ell_{1},\dotsc,\ell_{n}$ \ to be in descending order.

\begin{defn}\label{dkmechanism}
%
A \emph{polygonal mechanism} \ $(\Lc,\Xc,\Pc)$ \ in \ $\RR{d}$ \
consists of: \newnot{3-2}

\begin{enumerate}
\renewcommand{\labelenumi}{(\alph{enumi})}
%
\item $k$ branches \ $\Lc=\{\Lj{i}\}_{i=1}^{k}$ \ of multiplicity \
$\{n^{(i)}\}_{i=1}^{k}$, \ respectively;
%
\item $k$ distinct \emph{base points} \ $\Xc=\{\xj{i}\}_{i=1}^{k}$ \ in \
  $\RR{d}$, \ to which the initial points of the corresponding
  branches are attached.
%
\item An abstract $k$-polygon $\Pc$ in \ $\RR{d}$.
%
\end{enumerate}

Think of this mechanism as a linkage of $k$ branches, starting at
the base points (which form a polygon (not necessarily planar) in \
$\RR{d}$, \ called the  \emph{fixed platform}), and ending at the
vertices of a rigid planar polygon congruent to $\Pc$, called the
\emph{moving platform} of the mechanism. There are revolute joints
at both ends of each branch, too.

We use parenthesized superscripts to
indicate the branch number, and plain subscripts to indicate the rod
number \ -- \ e.g., \ $\ell_{j}^{(i)}$ \ denotes the length of the
$j$-th rod of the $i$-th branch.
%
\end{defn}

\begin{remark}\label{rpolygon}
%
Observe that a planar polygon $\Pc$ in \ $\RR{d}$ \ ($d>2$) \ with
vertices \ $\ppj{1}$, $\dotsc$, $\ppj{k}$ \ is determined up to
isometry by the sequence of triangles \
$\triangle(\ppj{i},\ppj{i+1},\ppj{i+2})$ \ and \
$\triangle(\ppj{i},\ppj{i+1},\ppj{i+3})$, \ which are determined
in turn by the lengths of their sides \
$\Gc~:=~(\gj{i,j})_{(i,j)\in\Ic}$. \ Here \
$\gj{i,j}:=\|\ppj{i}-\ppj{j}\|$,  \ and the index set $\Ic$
consists of the \ $3k-6$ \ ordered pairs: \newnot{3-4}
$$
(1,2),(1,3),(2,3),(1,4),(2,4),(3,4),\dotsc,(k-3,k),(k-2,k),(k-1,k)
$$
%
(in this order). Note that these diagonals force $\Pc$ to be planar, since
if we assume that the polygon is contained in the affine plane $\Ec$
determined by the first three vertices in \ $\RR{d}$, \ then $\Pc$ is
constructed inductively by adding one vertex at a time to an existing
edge to form a new triangle.


When \ $d=2$, \ the \ $2k-3$ \ pairs:
$$
\Ic=\{(1,2),(1,3),(2,3),(2,4),(3,4),\dotsc,(k-2,k-1),(k-1,k)\}
$$
%
suffice to determine $\Pc$ completely, if it is convex; otherwise, the
only additional data needed is the discrete information as to which
half-plane the new vertex is to be placed in.
%
\end{remark}

\begin{defn}\label{dconfiguration}
%
A \emph{configuration} for a polygonal mechanism \ $(\Lc,\Xc,\Pc)$
\ in \ $\RR{d}$ \ consists of a set \ $\Vc=(\Vj{1},\dotsc,\Vj{k})$
\ of $k$ branch configurations for $\Lc$ (Definition
\ref{dbconfiguration}), satisfying the condition that the
endpoints \ $\ppj{i}:=\xj{i}+\sigma_{\Vj{i}}$ \ ($i=1,\dotsc,k$) \
of the corresponding branch configurations (attached to the given
basepoints) form a planar polygon congruent to $\Pc$ in \
$\RR{d}$. \ If the branch configuration \ $\Vj{i}$ \ is aligned,
with direction vector \ $\vj{i}$ \ (Definition
\ref{dbconfiguration}), then the line \
$\Line\up{i}:=\{\xj{i}+t\vj{i}~|\ t\in\RR{}\}$ \ is called the
\emph{direction line} for \ $\Vj{i}$ \ (with \
$\ppj{i}\in\Line\up{i}$).\newnot{3-5}

The set of all configurations for the given mechanism \
$(\Lc,\Xc,\Pc)$ \ (as a subspace of the product of
the appropriate branch \cspace s), is its \emph{\cspace} \
$\C=\C(\Lc,\Xc,\Pc)$.
%
\end{defn}

\begin{defn}\label{dwork}
%
Note that the moving platform $\Pc$ can be translated and rotated
in \ $\RR{d}$ \ (subject to the constraints imposed by the branches
and the locations of the fixed vertices). The space of all allowable
positions for $\Pc$, denoted by \ $\Wc=\Wc(\Lc,\Xc,\Pc)$, \ is called
the \emph{\wspace} for \ $(\Lc,\Xc,\Pc)$. \ The \emph{work map} \
$\Phi:\C\to\Wc$ \ assigns to each configuration $\Vc$ the
resulting position of $\Pc$.
%
\end{defn}

\begin{mysubsection}[\label{sorg}]{Organization}
%
In section \ref{cmain} we show that the \cspace s $\C$ we consider here, in
any ambient dimension, are manifolds (generically). In section
\ref{cplan} we describe a Morse function for the \cspace\ of a generic planar
mechanism, analyze its critical points geometrically, and give a
simple example showing how this analysis may be used to recover $\C$.
%
\end{mysubsection}

%
%s2   Generic polygonal mechanisms in $\RR{d}$}
%
\sect{Generic polygonal mechanisms} \label{cmain}

We now show that, generically, the \cspace\ of a polygonal
mechanism is a manifold. Of course, it may be smooth even when  \
$(\Lc,\Xc,\Pc)$ \ is not generic, but in such cases singularities
can occur (cf.\ \cite{FSchuH} and \cite{KT}), and their analysis
is of interest in relation to the kinematic singularities.

\begin{defn}\label{dgeneric}
%
A configuration \ $\Vc=(\Vj{1},\dotsc,\Vj{k})$ \ of a polygonal
mechanism \ $(\Lc,\Xc,\Pc)$ \ is called \emph{singular} if:
%
\begin{enumerate}
\renewcommand{\labelenumi}{(\alph{enumi})~}
%
\item Two of its branch configurations \ $\Vj{i_{1}}$ \ and \
 $\Vj{i_{2}}$ \ are aligned, with coinciding direction lines: \
$\Line\up{i_{1}}=\Line\up{i_{2}}$ \ (see Figure \ref{fsingular}).

\begin{figure}[htb]
\begin{center}
\epsfysize=6cm \leavevmode
\epsffile{fig_3/critical_point_triangle.eps} \caption{Singular
configuration of type (a)} \label{fsingular}
\end{center}
\end{figure}

\item Three of its branch configurations are aligned, with direction lines
in the same plane meeting in a single point.
%
\item Four of its branch configurations are aligned, with
  direction lines in the same plane.
%
\end{enumerate}

The mechanism \ $(\Lc,\Xc,\Pc)$ \ is
called \emph{generic} if none of its configurations are singular
(compare \cite{Hau}).
%
\end{defn}

\begin{thm}\label{tone}
%
The \cspace\ \  \ $\C=\C(\Lc,\Xc,\Pc)$ \ of a generic polygonal mechanism
in \ $\RR{d}$ is a smooth closed orientable manifold of dimension \
$N(d-1)-|\Ic|$, \ where $k$ is the number of vertices of $\Pc$ \ and \
$N:=\sum_{i=1}^{k}\,{\nj{i}}$.
%
\end{thm}
\newnot{3-1}
\noindent\textit{Proof.} \ For any $n$, define a map \
$f_{n}:(\RR{d})^{n}\to\RR{n}$ \ by:
$$
f_{n}(\bv_{1},\dotsc,\bv_{n})~:=~(|\bv_{1}|^{2},\dotsc,|\bv_{n}|^{2})~.
$$

Consider the \emph{constraint map} \ $F:\RR{dN}\to\RR{N+|\Ic|}$, \ defined:
%
\begin{equation}\label{econstr}
%
F(\Vc)~=~F(\Vj{1},\dotsc,\Vj{i})~:=~
(f_{\nj{1}}(\Vj{1}),\dotsc,f_{\nj{k}}(\Vj{k}),
\|\aj{1,2}\|^{2},\dotsc,\|\aj{k-1,k}\|^{2})~,
%
\end{equation}
%
where \ $\ppj{i}:=\xj{i}+\sum_{t=1}^{\nj{i}}\,\vj{i}_{t}$ \ is the
endpoint of the $i$-th branch for the configuration \
$\Vj{i}=(\vj{i}_{1},\dotsc,\vj{i}_{\nj{i}})$ \ attached to the
basepoint \ $\xj{i}\in\RR{d}$, \ and \ $\aj{i,j}:=\ppj{i}-\ppj{j}$ \
is the \ $(i,j)$-diagonal of the polygon spanned by these endpoints.

Recall from \S \ref{rpolygon} that $\Pc$ determines the set of
diagonals \ $\Gc=(\gj{i,j})_{(i,j)\in\Ic}$, \ where \
$$
|\Ic|~=~\begin{cases} 3k-6 & \text{if~}d>2\\
2k-3 & \text{if~}d=2
\end{cases}
$$
%
Let:
%
\begin{equation*}
%
Z_{(\Lc,\Gc)}~:=~((\lj{1}_{1})^{2},\dotsc,(\lj{1}_{\nj{1}})^{2},\dotsc
((\lj{k}_{1})^{2},\dotsc,(\lj{k}_{\nj{k}})^{2},\
(\gj{1,2})^{2},\dotsc,(\gj{k-1,k})^{2})
%
\end{equation*}

The \cspace\ \ $\C=\C(\Lc,\Xc,\Pc)$ \ is the pre-image
of \ $Z_{(\Lc,\Gc)}\in \RR{N+|\Ic|}$ \ under the function $F$; \
if \ $d=2$, \ $\C$ is one connected component of this pre-image\vsm.

$\C$ will be a smooth manifold if \ $Z_{(\Lc,\Gc)}$ \ is a regular
value of $F$ \ -- \ i.e., if  \ $\dF_{\Vc}$ \ is of rank \ $N+|\Ic|$ \
(see \cite[I, Theorem 3.2]{Hi}). \ We calculate:
%
\begin{equation}\label{edf}
%
\dF_{\Vc}~=~2
\begin{pmatrix}
%
A\up{1}    & 0         &  0           &  0          & \dotsc  & 0  \\
 0         &A\up{2}    &  0           &  0          & \dotsc  & 0  \\
 0         & 0         & A\up{3}      &             & \dotsc  & 0  \\
 0         & 0         &  0           & A\up{4}     & \dotsc  & 0  \\
 \vdots    & \vdots    &  \vdots      & \vdots      & \ddots  & \vdots \\
 0         & 0         &  0           & 0           & \dotsc  & A\up{k} \\
& & & & & \\
\bj{1}{2}  & \bj{2}{1} & 0            & 0 &  \ldots     & 0          \\
\bj{1}{3}  & 0         & \bj{3}{1}    & 0 &  \ldots     & 0          \\
 0         &\bj{2}{3}  & \bj{3}{2}    & 0 &  \ldots     & 0          \\
\vdots     & \vdots    &  \vdots      &  \vdots     & \ddots  & \vdots \\
 0         & \dotsc    & 0 & \bj{k-2}{k}  & 0           & \bj{k}{k-2}\\
 0         & 0    & \dotsc & 0             & \bj{k-1}{k} & \bj{k}{k-1}
\end{pmatrix}\vspace{4mm},
%
\end{equation}
%
where \ $A\up{i}$ \ is the  \ $\nj{i}\times d\nj{i}$ \ matrix:
%
\begin{equation}\label{eai}
\begin{pmatrix}
%
\vj{i}_{1} & \dotsc & 0     \\
\vdots     & \ddots & \vdots \\
0          & \dotsc & \vj{i}_{\nj{i}}
\end{pmatrix}
%
\end{equation}
%
\noindent with rows denoted by: \ $\uj{i}{1},\dotsc,\uj{i}{\nj{i}}$.

The \ $|\Ic|$ \ bottom rows \ $(\wj{i}{j})$ \ of \ $\frac{1}{2}\dF$ \
are indexed by \ $(i,j)\in\Ic$, \ where \
$$
\wj{i}{j}:=
(\underbrace{0,\dotsc,0}_{d(\nj{1}+\dotsc+\nj{i-1})},
\bj{i}{j},\underbrace{0,\dotsc,0}_{d(\nj{i+1}+\dotsc+\nj{j-1})},
\bj{j}{i},\underbrace{0,\dotsc,0}_{d(\nj{j+1}+\dotsc+\nj{k})})~,
$$
%
each edge \ $\aj{i,j}\in\RR{d}$ \ appearing \ $\nj{i}$ \ times in
\newnot{3-6}
$$
\bj{i}{j}~:=~\underbrace{(\aj{i,j},\aj{i,j},\dotsc,\aj{i,j})}_{\nj{i}}~.
$$

If we think of the bottom rows as forming $N$
blocks of \ $|\Ic|\times d$ \ matrices \ $B_{1},\dotsc, B_{N}$, \
then:
%
\begin{equation}\label{eblock}
\sum_{r=1}^{N}\ B_{r} ~=~ 0\vsm.
\end{equation}

Let \ $\Vc\in\C=F^{-1}(Z_{(\Lc,\Gc)})$, \ and consider a vanishing
linear combination of the rows of \ $\dF_{\Vc}$:
$$
\sum_{i=1}^{k}~\left(\sum_{j=1}^{\nj{i}}\,\lambda\up{i}_{j}\uj{i}{j}\right)~+~
\sum_{(i,j)\in\Ic}\,\gamma\up{i,j}\wj{i}{j}~=~\vz~.
$$

For each \ $1\leq i\leq k$, \ let:
$$
\vj{i}~:=~\sum_{\substack{(s,i)\in\Ic\\ s<i}}\,\gamma\up{s,i}\cdot\aj{s,i}
~-~\sum_{\substack{(i,t)\in\Ic\\ i<t}}\,\gamma\up{i,t}\cdot\aj{i,t}~,
$$
%
and note that if \ $\lambda\up{i}_{j'}\neq 0$ \ for \emph{some} \
$1\leq j'\leq\nj{i}$, \ then this holds for all \ $1\leq j\leq\nj{i}$, \
so that \ $\vj{i}_{j}~=~\frac{1}{\lambda\up{i}_{j}}\cdot\vj{i}$ \
(the $i$-th branch is aligned with direction vector \ $\vj{i}$).

By \eqref{eblock}, \ $\sum_{i=1}^{k}\,\vj{i}=\vz$, \ so if \
$\vj{i}=\vz$ \ for \ $i\neq i_{0},i_{1}$, \ then \
$\vj{i_{0}}+\vj{i_{1}}=\vz$, \ and thus branches \ $i_{0}$ \ and \
$i_{1}$ \ are co-aligned with direction vector \ $\aj{i_{0},i_{1}}$. \
This cannot happen if \ $(\Lc,\Xc,\Pc)$ \ is generic
(cf.\ \S \ref{dgeneric}(a)). \ Similarly, no more than three
branches can be aligned (\S \ref{dgeneric}(c)).

Unfortunately, even for generic mechanisms \ $\dF$ \ need not be of maximal
rank. Thus, to complete the proof of the Theorem we need the
following:
%
%\end{proof}

\begin{prop}\label{pzero}
%
For a generic polygonal mechanism, any configuration $\Vc$ having
at most three aligned branches is smooth.
%
\end{prop}

\begin{proof}
%
Without loss of generality we may assume that the three aligned
branches are \ $1,2$ \ and $k$, \ with direction vectors \
$\oj{1}$,\ $\oj{2}$, \ and \ $\oj{k}$, \ respectively.

Let $\hC$ denote the configuration space of the mechanism obtained
from \ $(\Lc,\Xc,\Pc)$ \ by omitting the last branch, and \ $\Ck$
\ the configuration space for this branch (attached to \
$\xj{k}\in\RR{d}$). \newnot{3-2}

The \emph{\wspace} of both mechanisms (i.e., the set of possible
locations for the $k$-th vertex of $\Pc$) is contained in \ $\RR{d}$, \
and we have \emph{work maps} \ $\psi:\hC\to\RR{d}$ \ and \
$\phi:\Ck\to\RR{d}$ \ which associate to each configuration the
location of this vertex.

Note that the manifold \ $\Ck$ \ (an $n$-torus for \ $n=\nj{k}$) \
is the preimage of \ $Z_{\lj{k}}$ \ under the map \
$f=f_{n}:\RR{nd}\to\RR{n}$, \ and we write \ $i:\Ck\to\RR{nd}$ \ for
the inclusion. Similarly, $\hC$ is determined by a smooth
constraint map:
$$
\hF:\RR{M}\times\Sc\to\RR{N-n+|\hI|}
$$
%
for \ $M=(N-n)d$, \ where $\Sc$ is a point if \ $k\geq 4$ \ and \
$\Sc=S^{d-2}$ \ if \ $k=3$. \ Here
%
\begin{equation}\label{econstn}
%
\begin{split}
%
\hF(\hV)~=&~\hF(\Vj{1},\dotsc,\Vj{k-1},\bw)~\\
:=&~(f_{\nj{1}}(\Vj{1}),\dotsc,f_{\nj{k-1}}(\Vj{k-1}),\,
\|\aj{1,2}\|^{2},\,\dotsc,\|\aj{k-2,k-1}\|^{2})~
%
\end{split}
%
\end{equation}
%
(with the functions \ $\|\aj{i,j}\|^{2}$ \ indexed by \
$(i,j)\in\hI$). \ The vector \ $\bw\in S^{\Delta}$ \ is needed for \
$k=3$ \ and \ $d>2$, \ since in that case the location of the first \
$k-1$ \ vertices of $\Pc$ determines the position of the moving
platform only up to rotation of \ $\aj{1,3}$ \ around the given edge \
$\aj{1,2}$ \ (relative to the plane \
$\Ec=\Ec(\vj{1}_{\nj{1}},\aj{1,2})$ \ spanned by \
$\vj{1}_{\nj{1}}$ \ and \ $\aj{1,2}$).

Thus $\hC$ is \ $\hF^{-1}(Z_{(\hat{\Lc},\hat{\Gc})})$ \
for the obvious \ $(\hat{\Lc},\hat{\Gc})$. \
Moreover, \ $(\hat{\Lc},\hat{\Gc})$ \ is a regular value of $\hF$,
as in the proof of Theorem \ref{tone}, because \ $(\Lc,\Xc,\Pc)$ \
is generic, so no branches but \ $1,2,k$ \ are aligned.
Thus $\hC$ is a smooth submanifold, and we write \
$\hi:\hC\to\RR{M}$ \ for the inclusion.

Let \ $X:=\Ck\times\RR{M}\times\Sc$, \
$Y:=\RR{d}\times\RR{M}\times\Sc$, \ and define \
$h:X\to Y$ \ to be the product map \
$\phi\times\Id_{\RR{M}}\times\Id_{\Sc}$ \ and \
$g:\hC\to Y$ \ to be \ $(\psi,\hi)$, \ so that $g$ is an embedding of
$\hC$ as a submanifold in $Y$. Since \ $\C=\C(\Lc,\Xc,\Pc)$ \ is
simply the pullback of:
$$
\Ck~\xrightarrow{\phi}~\RR{d}~\xleftarrow{\psi}~\hC~,
$$
%
it may be identified with the preimage of the submanifold \
$\hC\subseteq Y$ \ under $h$\vsm.

Let \ $\hV\in\hC$ \ be a configuration where the first two branches
are aligned (but not co-aligned), with direction vectors in the plane
$\Ec$ determined by the polygon $\Pc$ \ -- \ and let \ $\Vj{k}\in\Ck$ \ be an
aligned configuration  with direction vector \ $\vj{k}$. \ Assume
that \ $\psi(\hV)=\phi(\Vj{k})$, \ and let \ $\bx\in X$ \ be the
configuration \ $(\Vj{k},\hi(\hV))$, \ so that \ $h(\bx)=g(\hV)$. \
We must therefore show that the point \ $\Vc\in\C$ \
defined by \ $(\hV,\Vj{k})$ \ is smooth:

$$
\xymatrix@R=25pt{
\hspace*{3.5mm}\bx\in X\ar@<2.7ex>[d]_{h} & = &
\Ck\ar@<-2.8ex>[d]_{\phi}\ni\Vj{k} & \times &
\RR{M}\times\Sc\ar@<-3.7ex>[d]_{\Id}\ni\hi(\hV)\\
h(\bx)\in Y           & = & \hspace*{2mm}\RR{d}\ni\phi(\Vj{k})  & \times &
\RR{M}\times\Sc\ni\hi(\hV) \\
\hspace*{0.5mm}\hV\in\hC
\ar@<-2.7ex>[u]^{g}\ar[urr]^{\psi}\ar[urrrr]_{\hi} & & & &
}
$$

By \cite[I, Theorem 3.3]{Hi}), it suffices to show that \
$h\pitchfork\hC$  \ -- \ i.e., $h$ is locally transverse
to $\hC$ at the points \ $\bx\in X$ \ and \ $\hV\in\hC$. \
In other words:
%
\begin{equation}\label{etransv}
%
\Image\dhh_{\bx}~+~T_{\hV}(\hC)~=~T_{\hV}(Y)~=~\RR{d}\times\RR{M+\Delta}~.
%
\end{equation}

First note that since \ $\Ck=f^{-1}(Z_{\lj{k}})\subseteq\RR{nd}$, \
the tangent space \ $T_{\Vj{k}}(\Ck)$ \ may be identified with the
kernel of \ $\df_{\Vj{k}}:\RR{nd}\to\RR{n}$, \ which
is the null space of the matrix \ $A\up{k}$ \ of \eqref{eai}. Since \
$\Vj{k}$ \ is aligned, by assumption, with direction vector \
$\oj{k}$, \ we see that:
%
\begin{equation*}
%
\begin{split}
%
T_{\Vj{k}}(\Ck)~\cong~&
\{(\yj{k}_{1},\dotsc,\yj{k}_{n})\in\RR{nd}~|\
\yj{k}_{1}\cdot\oj{k}=0,\ \dotsc,
\yj{k}_{n}\cdot\oj{k}=0\}\\
=&~\underbrace{\oj{k}^{\perp}\times\dotsc\times\oj{k}^{\perp}}_{n}
%
\end{split}
%
\end{equation*}

Furthermore, \ $\phi:\Ck\to\RR{d}$ \ extends to \
$\hat{\phi}:\RR{nd}\to\RR{d}$ \ (so that \ $\hat{\phi}\circ i=\phi$), \
with
$$
\hat{\phi}(\bv_{1},\dotsc,\bv_{n})~=~\xj{k}+\bv_{1}+\dotsb+\bv_{n}~.
$$

Since \ $\hat{\phi}$ \ is linear, its differential \ $\dd\hat{\phi}$ \
is represented by the \ $d\times nd$ \ matrix \
$(I_{d},I_{d},\dotsc,I_{d})$ \ ($n$ blocks).

Thus:
%
$$
\Image\dhh_{\bx}~=~\oj{k}^{\perp}\times\RR{M+\Delta}~,
$$
%
\noindent so that in order for \eqref{etransv} to hold it suffices to
prove that:
%
\begin{equation}\label{eomega}
%
\oj{k}\in T_{\hV}(\hC) \vsm.
%
\end{equation}

Now the tangent space \ $T_{\hV}(\hC)$ \ may be identified with the
kernel of:
$$
\dd\hF_{\hV}:\RR{M+\Delta}\to\RR{N-n+2k-5}~,
$$
%
where \ $\dd\hF_{\hV}$ \ is described as in \eqref{edf} by the matrix:

\begin{equation}\label{edfn}
%
\dd\hF_{\hV}~=~
2\begin{pmatrix}
%
A\up{1}   & 0         &  \dotsc  &      0      & 0 \dotsc 0 \\
 \vdots   & \vdots    &  \ddots  &    \vdots   & \vdots     \\
 0        & 0         &  \dotsc  & A\up{k}     & 0 \dotsc 0 \\
          &           &          &             &            \\
\bj{1}{2} & \bj{2}{1} &  \dotsc  &       0     & 0 \dotsc 0 \\
\vdots    & \vdots    &  \ddots  &  \vdots     & 0 \dotsc 0 \\
 0        & 0         & \dotsc   & \bj{k}{k-1} & 0 \dotsc 0
\end{pmatrix}\vspace{4mm}\quad
%
\end{equation}

Since the first two branches are aligned with
direction vectors \ $\oj{1}$,\ $\oj{2}$, \ $T_{\hV}(\hC)$ \
may be identified with the set of \ $N-n$ \ $d$-dimensional vectors \
$$
\yj{1}_{1},\dotsc,\yj{1}_{\nj{1}};
\yj{2}_{1},\dotsc,\yj{2}_{\nj{2}};\dotsc;
\yj{k-1}_{1},\dotsc,\yj{k-1}_{\nj{k-1}}~,
$$
%
together with \ $\bz\in\RR{\Delta}$, \ where the first \ $\nj{1}$ \
vectors are all in \ $\oj{1}^{\perp}$, \ the next \ $\nj{2}$ \ are all
in \ $\oj{2}^{\perp}$, \ and the remainder are in individual orthogonal
complements:
$$
(\vj{3}_{1})^{\perp}\times\dotsb\times(\vj{3}_{\nj{3}})^{\perp}\times
(\vj{k-1}_{1})^{\perp}\times\dotsb\times(\vj{k-1}_{\nj{k-1}})^{\perp}~.
$$

If we let \ $\vy{i}:=\sum_{t=1}^{\nj{i}}\,\yj{i}_{t}$ \
($i=1,\dotsc,k-1$), \ these must satisfy:
$$
\aj{i,j}\cdot(\vy{i}-\vy{j})~=~0
$$
%
for each \ $(i,j)\in\hI$.

Likewise, \ $\psi:\hC\to\RR{d}$ \ extends to \
$\hat{\psi}:\RR{M}\times\Sc\to\RR{d}$, \  with
$$
\hat{\psi}(\Vj{1},\dotsc,\Vj{k-1},\bw)~=~
\xj{1}+\vj{1}_{1}+\dotsb+\vj{1}_{\nj{1}}~+~R_{\bw}\cdot\aj{1,2}~,
$$
%
where \ $R_{\bw}$ \ is \ $\lambda\cdot B_{\bw}\cdot A_{\theta}$, \
for \ $\lambda:=\lj{1,k}/\lj{1,2}$, \ $B_{\bw}$ \ the rotation matrix
about \ $\aj{1,2}$ \ determined by \ $\bw\in\Sc$ \ (if \ $\Sc=S^{\Delta}$), \
and \ $A_{\theta}$ \ rotates \ $\aj{1,2}$ \ (in the plane $\Ec$ spanned by \
$\vj{1}_{\nj{1}}$ \ and \ $\aj{1,2}$) \ by the angle $\theta$ to the
side \ $\aj{1,3}$ \ in the given polygon $\Pc$.

Thus \ $\dd\hat{\psi}_{(\Vj{1},\dotsc,\Vj{k-1},\bw)}$ \ is represented
by the \ $(M+\Delta)\times d$ \ matrix:
%
\begin{equation}\label{epsi}
%
\left[~ \underbrace{I_{d}+R_{\bw} | \dotsc | I_{d} + R_{\bw}}_{\nj{1}}~|~
\underbrace{-R_{\bw} | \dotsc | -R_{\bw}}_{\nj{2}}~|~
\underbrace{ 0\dotsc 0 }_{d\nj{3}}~|~\dotsc~|~
\underbrace{ 0\dotsc 0 }_{d\nj{k-1}}~|~
\frac{\partial R_{\bw}}{\partial\bw}\right]~.
%
\end{equation}

Therefore, the image of \ $\dd\psi$ \ is obtained by applying
\eqref{epsi} to \ $T_{\hV}(\hC)\subseteq\RR{M+\Delta}$ \ described
above, and we see that \ $\Image(\dd\psi_{\hV})$ \ is the sum of
%
\begin{equation}\label{edpsi}
%
\{\vy{1}+ R_{\bw}(\vy{1}-\vy{2})~|\
\vy{1}\in\oj{1}^{\perp},\  \vy{2}\in\oj{2}^{\perp},\
(\vy{1}-\vy{2})\in\ap{2}\}
%
\end{equation}
%
\noindent and \ $\Image\frac{\partial R_{\bw}}{\partial\bw}$\vsm.

In the case we are interested in, the two direction vectors \ $\oj{1}$ \
and \ $\oj{2}$ \ are in the plane $\Ec$ spanned by $\Pc$, so \
$B_{\bw}$ \ is the identity matrix and thus \
$R_{\bw}=\lambda\cdot A_{\theta}$ \ takes \ $\aj{1,2}$ \ to \
$\aj{1,3}$ \ in $\Ec$. Clearly, \ $\Image(\dd\psi_{\hV})$ \ contains
the orthogonal complement \ $\Ec^{\perp}$, \ so to prove
\eqref{eomega} we may reduce to the case \ $d=2$ \ (all vectors
in the plane $\Ec$), \ so that in fact \eqref{edpsi} is all of \
$\Image(\dd\psi_{\hV})$.

Now choose unit vectors \ $\cj{i}$ \ spanning \ $\oj{i}^{\perp}$ \
($i=1,2$) \ in $\Ec$, let \ $\bd:=A_{\pi/2}(\aj{1,2})$ \ (so that it
spans \ $(\aj{1,2})^{\perp}$), \ and let \ $\hd:=R_{\bw}\bd$. \
Then \ $\Image(\dd\psi_{\hV})$ \  is the set of
vectors \ $s\cj{1}+t\hd$ \ such that \ $s\cj{1}-t\bd$ \ is a
multiple of \ $\cj{2}$.

For any three vectors $\bx$, $\by$, and $\bz$ in \ $\RR{2}$ \ we have:
%
\begin{equation}\label{elie}
%
(\bx\cdot A_{\pi/2}\by)\,\bz~+~(\by\cdot A_{\pi/2}\bz)\,\bx~+~
(\bz\cdot A_{\pi/2}\bx)\,\by~=~\vz~,
%
\end{equation}
%
\noindent so \ $\cj{2}$ \ is a multiple of \
$(\aj{1,2}\cdot\cj{2})\,\cj{1}~+~(\oj{1}\cdot\cj{2})\,\bd$, \ and thus \
$\Image(\dd\psi_{\hV})$ \ is spanned by:
$$
\be~:=(\aj{1,2}\cdot\cj{2})\,\cj{1}~-~(\oj{1}\cdot\cj{2})\,\hd~.
$$
%
Therefore, \eqref{eomega} fails only if \ $\oj{k}$ \ is perpendicular to
$\be$ \ -- \ in other words, if \ $\oj{k}$ \ is proportional to:
$$
(\aj{1,2}\cdot\cj{2})\,\oj{1}~-~(\oj{1}\cdot\cj{2})\,\aj{1,3}
$$
%
or equivalently, to:
$$
\bw~:=~\frac{(\bd\cdot\oj{2})}{(\oj{1}\cdot\cj{2})}\,\oj{1}~-~\aj{1,3}~,
$$
%
which by \eqref{elie} is precisely the vector connecting the meeting
point \ $P:=\ppj{1}+\frac{(\bd\cdot\oj{2})}{(\oj{1}\cdot\cj{2})}\,\oj{1}$ \ of \
$\Line\up{1}$ \ and \ $\Line\up{2}$ \ with the end point \
$\ppj{3}=\ppj{1}+\aj{1,3}$ \ of \ $\Vj{k}$, \ so that \ $\oj{k}$ \ is
proportional to $\bw$ if and only if the direction line \
$\Line\up{3}$ \ passes through $P$ \ -- \ which cannot happen in a
generic mechanism (see Figure \ref{fsingular} and
Definition \ref{dgeneric}(b)).

This completes the proof of Proposition \ref{pzero}, and thus of
Theorem \ref{tone}.
%
\end{proof}

%
%s3      Morse functions for planar mechanisms
%
\sect{Morse functions for planar mechanisms} \label{cplan}
%
From now on we shall concentrate on the simplest type of polygonal
mechanism \ -- \ namely, planar mechanisms \ ($d=2$) \ having
triangular platforms \ ($k=3$) \ and exactly two links per branch \
($\nj{1}=\nj{2}=\nj{3}=2$). \ These mechanism are known in the
robotics literature as \emph{$3$-RRR} (rotational) mechanisms.

Recall that a smooth real-valued function on a manifold is
called a \emph{Morse function} if all its critical points are
non-degenerate (cf.\ \cite[I, \S 2]{Mi}). Such functions may be used to
deduce the cellular structure of the manifold, and thus recover its
homotopy type  (see \cite[I, \S 3]{Mi}). Our goal is to describe a
Morse function for the \cspace\ of a $3$-RRR mechanism\vsm.

\begin{thm}\label{ttwo}
%
The function \ $f(\Vc)~:=~\sum_{j=1}^{3}\,\|\vj{j}\|^{2}$ \ is
generically a Morse function on \ $C(\Lc,\Xc,\Pc)$, \
where \ $\vj{j}:=\vj{j}_{1}+\vj{j}_{2}=\ppj{j}-\xj{j}$.
%
\end{thm}

\begin{proof}
%
In order to show that the critical points of $f$ are non-degenerate,
we must choose a local coordinate system near each such point.

\begin{figure}[htbp]
\begin{center}
\epsfysize=6cm %\epsfxsize=6.4cm
\leavevmode \epsffile{fig_3/mechanism.eps} \caption{Local
coordinates} \label{floccoord}
\end{center}
\end{figure}


Unfortunately, there is no uniform choice of such a system, so we must
distinguish three cases\vsm:

\noindent\textbf{Case I:} \ \
Let \ $\Phi:=(\phi_{1},\phi_{2},\phi_{3})$, \ where \
$\phi_{j}$ \ denotes the angle between the vectors \ $-\vj{j}_{1}$ \
and \ $\vj{j}_{2}$ \ for \ $j=1,2,3$. \ Then:
%
\begin{equation}\label{eqh}
%
\hj{j}(\phi_{j})~:=~\|\vj{j}\|~=~\|\vj{j}_{1}+\vj{j}_{2}\|~=~
\sqrt{(\lj{j}_{1})^{2}+(\lj{j}_{2})^{2}-2\lj{j}_{1}\,\lj{j}_{2}\cos\phi_{1}}
%
\end{equation}
%
and thus \ $f(\Phi)~=~\sum_{j=1}^{3}\ \hj{j}(\phi_{j})^{2}$, \ so that:
%
\begin{equation}\label{eqnabla}
%
\begin{split}
%
\nabla f~=~\nabla_{\Phi}\,f~=&~2\left(\lj{1}_{1}\lj{1}_{2}\,\sin(\phi_{1}),\
\lj{2}_{1}\lj{2}_{2}\,\sin(\phi_{2}),\
\lj{3}_{1}\lj{3}_{2}\,\sin(\phi_{3})\right)\\
=&~2\,(\vpp{1},~\vpp{2},~\vpp{3})
%
\end{split}
%
\end{equation}
%
where \ $\vw^{\perp}:=(b,-a)$ \ for \ $\vw=(a,b)$.

Thus $\Phi$ is a critical point if and only if:
%
\begin{equation}\label{eqsigma}
%
\Phi~=~\frac{\pi}{2}\,(1+\sigma_{1},\,1+\sigma_{2},\,1+\sigma_{3})
\hspace*{10mm}\text{for \ } \sigma_{1},\sigma_{2},\sigma_{3}\in\{\pm 1\}~.
%
\end{equation}

Computing the Hessian at a critical point $\Phi$ yields:
$$
H_{\Phi}~=~\begin{pmatrix}
        \sigma_{1}\,\lj{1}_{1}\lj{1}_{2} & 0  & 0 \\
0 &     \sigma_{2}\,\lj{2}_{1}\lj{2}_{2} & 0  \\
0 & 0 & \sigma_{3}\,\lj{3}_{1}\lj{3}_{2}
\end{pmatrix}~,
$$
%
which is non-degenerate, with index \ $\Ind_{\Phi}$ \ equal to the
number of negative values in \
$\{\sigma_{1},\,\sigma_{2},\,\sigma_{3}\}$. \ Such critical points
will be refered to as \emph{type I}.

\begin{figure}[htbp]
\begin{center}
\epsfysize=5cm %\epsfxsize=6.4cm
\leavevmode \epsffile{fig_3/convex_hull1.eps} \caption{Type I
critical point}\label{fcaseI}
\end{center}
\end{figure}

\noindent\textbf{Case II:} \ \ As we saw, critical points of $f$
appear when all three branches are aligned. However, for some
mechanisms this will never happen, because one or two
branches can never fully stretch or fold \ -- \
that is, \ $\phi_{3}$ \ (say) takes values in a proper subset \
$[a_{1},a_{2}]\cup[-a_{2},-a_{1}]$ \ of \ $[-\pi,\pi]$ \ (see
Example \ref{egcrit}). \ Clearly, \ $\phi_{3}$ \ cannot then serve as
a local coordinate at a point \ $(\phi_{1},\phi_{2},\pm a_{k})$.

However, if the first two branches can both be aligned, then in the vicinity
of doubly aligned configurations we take \
$\hP:=(\phi_{1},\phi_{2},\theta_{1})$, \ where \ $\phi_{j}$ \ ($j=1,2$)  \
as in Case I, and \ $\theta_{j}$ \ is the angle between  \
$\vj{j}:=\vj{j}_{1}+\vj{j}_{2}$ \ and the vector \
$\xj{2}$ \ (we assume for simplicity that \ $\xj{1}$ \ is at the origin).

Since
%
\begin{equation}\label{evj}
%
\vj{j}=\hj{j}(\cos\theta_{j},\sin\theta_{j})~,
%
\end{equation}
%
using \eqref{eqh} we have:
%
\begin{equation}\label{eeach}
%
(\dif{\vj{j}}{\phi_{j}},~\dif{\vj{j}}{\phi_{j'}},~\dif{\vj{j}}{\theta_{j}})~=~
(\frac{\vpp{j}}{\hj{j}(\phi_{j})^{2}}\,\vj{j},\,
0,\,-\vp{j}))~,
%
\end{equation}
%
for \ $\{j,j'\}=\{1,2\}$.

However, since \ $\theta_{2}$ \ is a dependent variable, we may
differentiate the norms in:
%
\begin{equation}\label{equad}
%
\vj{1}+\aj{1,2}-\vj{2}~=~\xj{2}
%
\end{equation}
%
implicitly and deduce that:
%
\begin{equation}\label{esecond}
%
\dif{\vj{2}}{\theta_{1}}~=~
-\dif{\theta_{2}}{\theta_{1}}\,\vp{2}~=~
-\frac{\aj{1,2}\cdot\vp{1}}{\aj{1,2}\cdot\vp{2}}\,\vp{2}~.
%
\end{equation}

Differentiating \eqref{equad} itself and using
\eqref{eeach}, \eqref{esecond} and \eqref{elie} yields:
%
\begin{equation}\label{ethird}
%
\nabla_{\hP}\aj{1,2}~=~
(-\frac{\vpp{1}}{\hj{1}(\phi_{1})^{2}}\,\vj{1},~
-\frac{\vpp{2}}{\hj{2}(\phi_{2})^{2}}\,\vj{2},~
\frac{\vj{1}\cdot\vp{2}}{\aj{1,2}\cdot\vp{2}}\,(\aj{1,2})^{\perp}\,)
%
\end{equation}

Since \ $\xj{1}=\vz$, \ we see \ $\vj{3}~=~\vj{1}+\aj{1,3}-\xj{3}$, \ so:
%
\begin{equation}\label{eqmorse}
\begin{cases}
\dif{f}{\phi_{1}}~=&~
2\,\vpp{1}~\frac{2\hj{1}(\phi_{1})^{2}+\vj{1}\cdot(\aj{1,3}-\xj{3})+
(\xj{3}-\vj{1})\cdot(B_{\alpha}\vj{1})}{\hj{1}(\phi_{1})^{2}}\\
%
\dif{f}{\phi_{2}}~=&~
2\,\vpp{2}~
\frac{\hj{2}(\phi_{2})^{2}~+(\xj{3}-\vj{1})\cdot(B_{\alpha}\vj{2})}
{\hj{2}(\phi_{2})^{2}}\\
%
\dif{f}{\theta_{1}}~=&~
2\,(\xj{3}-\aj{1,3})\cdot\vp{1}
~+~2\,\frac{[\vj{1}\cdot\vp{2}]\,[(\vj{1}-\xj{3})\cdot(\aj{1,3})^{\perp}]}
{\aj{1,2}\cdot\vp{2}}
\end{cases}
%
\end{equation}
%
where \ $B_{\alpha}$ \ is the rotation-and-dilitation matrix taking \
$\aj{1,2}$ \ to \ $\aj{1,3}$.

Note that we use the coordinates $\hP$ only at points where the first
two branches are aligned, so that \
$\vj{j}_{2}\cdot(\vj{j}_{1})^{\perp}=0$ \ for \ $j=1,2$, \ and thus
the first two entries of \ $\nabla_{\hP}(f)$ \ vanish at these
points. The vanishing of \ $\frac{\partial f}{\partial\theta_{1}}$ \
is equivalent to the condition:
%
\begin{equation}\label{elast}
%
\begin{split}
%
[&\xj{3}\cdot\vp{1}]\,[\aj{1,2}\cdot\vp{2}]
~-~[\aj{1,3})\cdot\vp{1}]\,[\aj{1,2}\cdot\vp{2}]\\
+&~[\vj{1}\cdot\vp{2}]\,[\vj{1}\cdot(\aj{1,3})^{\perp}]~-~
[\vj{1}\cdot\vp{2}]\,[\xj{3}\cdot(\aj{1,3})^{\perp}]~=~0
%
\end{split}
%
\end{equation}
%

Note that by \eqref{elie} again, the intersection of \ $\Line\up{1}$ \
with \ $\Line\up{2}$ \ is at the point:
$$
P:=\xj{3}+\vj{1}+\frac{\aj{1,2}\cdot\vp{2}}{\vj{1}\cdot\vp{2}}\,\vj{1}~,
$$
%
and \eqref{elast} is equivalent to the colinearity of \ $\xj{3}$, \
$P$, and \ $\vj{1}+\aj{1,3}$. \ Such critical points
will be refered to as \emph{type II}.

\begin{figure}[htbp]
\begin{center}
\epsfysize=5cm %\epsfxsize=6.4cm
\leavevmode \epsffile{fig_3/convex_hull.eps} \caption{Type II
critical point}\label{fcaseII}
\end{center}
\end{figure}

Calculating the Hessian matrix \ $H_{f}$ \ of $f$ at a critical point,
we find that it is diagonal, with:
%
\begin{equation*}
%
\begin{split}
%
\diff{f}{\phi_{1}\phi_{1}}~=&~
-2\,\vj{1}_{1}\cdot\vj{1}_{2}~
\frac{2\hj{1}(\phi_{1})^{2}+\vj{1}\cdot(\aj{1,3}-\xj{3})+
(\xj{3}-\vj{1})\cdot(B_{\alpha}\vj{1})}{\hj{1}(\phi_{1})^{2}}\\
%
\diff{f}{\phi_{2}\phi_{2}}~=&~
-2\,\vj{2}_{1}\cdot\vj{2}_{2}~
\frac{\hj{2}(\phi_{2})^{2}~+(\xj{3}-\vj{1})\cdot(B_{\alpha}\vj{2})}
{\hj{2}(\phi_{2})^{2}}\\
%
\diff{f}{\theta_{1}\theta_{1}}~=&~
~\frac{2}{[\aj{1,2}\cdot\vp{2}]^{2}}\
\left(\right.
-~2\,[\vj{1}\cdot\vp{2}]\,[\aj{1,3}\cdot\vj{1}]\,[\aj{1,2}\cdot\vp{2}]\\
&+~[(\xj{3}-\aj{1,3})\cdot\vj{1}]\,[\aj{1,2}\cdot\vp{2}]^{2}\\
&+~[\vj{1}\cdot\vj{2}]\,[\aj{1,2}\cdot(\vj{1}-\vj{2})^{\perp}]\,
[(\vj{1}-\xj{3})\cdot(\aj{1,3})^{\perp}]\\
&+~[\vj{1}\cdot\vp{2}]^{2}\,[(\vj{1}-\xj{3})\cdot(\aj{1,3})]\\
&+~[\aj{1,2}\cdot\vj{2}]\,[(\vj{2}-\aj{1,2})\cdot\vp{1}]\,
[(\xj{3}-\aj{1,3})\cdot\vp{1}]\left.\right)
%
\end{split}
%
\end{equation*}

If we solve \eqref{eqmorse} to find
explicitly the critical points of $f$ in the coordinates $\hP$, and
then substitute into the expression we have found for  \
$H_{f}$ \ at these points, we obtain a polynomial expression of
degree $6$ in the parameters \ $(\Lc,\Xc,\Gc)$ \ for the
mechanism. Thus the critical point we identified is degenerate only
when this polynomial vanishes, so generically $f$ is indeed a Morse
function\vsm.

\noindent\textbf{Case III:} \ \
Note that the \wspace\ $\Wc$ for each vertex of $\Pc$ is the intersection
of three annuli (so it is compact), and thus the boundary of $\Wc$
must intersect at least one of the bounding circles of the annuli.
Therefore, at least one of the three branches (say, the first)
\emph{can} be aligned.

Thus, at critical points of $f$ where neither $\Phi$ nor $\hP$ can be
used as local coordinates, the first branch is aligned, and we take \
$\Psi:=(\theta_{1},\phi_{1},\psi)$ \ as our local coordinates,
where \ $\theta_{1}$ \ and \ $\phi_{1}$ \ are as in Case II above, \ and
$\psi$ denotes the angle between \ $\aj{1,2}$ \ and $\xj{2}$ \ (see
Figure \ref{floccoord}). Note that this will not work when the second branch is
also aligned, since these coordinates only determine the length of \
$\vj{2}$, \ and not ``elbow up/down'' near \
$\phi_{2}=\frac{\pi}{2}\,(1+\sigma_{2})$.

Here:
%
$$
f(\Psi)~=~\|\vj{1}_{1}+\vj{1}_{2}\|^{2}
~+~\|\vj{1}+\aj{1,2}-\xj{2}\|^{2}~+~\|\vj{1}+\aj{1,3}-\xj{3}\|^{2}~.
$$
%
and since \ $\aj{1,2}=\gj{1,2}(\cos\psi,\sin\psi)$, \ we have \
$\nabla_{\Psi}(\vj{1})=
(-\vp{1},~\frac{\vpp{1}}{\hj{1}(\phi_{1})^{2}}\,\vj{1},\,0)$ \ and \
$\nabla_{\Psi}(\aj{1,j})=(0,\,0,\,-\ap{j}$ \ for \ $j=2,3$, \ so:
%
\begin{equation}
%
\begin{split}
%
\dif{f}{\theta_{1}}~=&~
2\vp{1}\cdot((\xj{2}+\xj{3})-(\aj{1,2}+\aj{1,3}))\\
%
\dif{f}{\phi_{1}}~=&~
-\vpp{1}\ \frac{\vj{1}\cdot(4\,\vj{1}+\aj{1,2}+\aj{1,3}-\xj{2}-\xj{3})}
{\|\vj{1}\|^{2}}\\
%
\dif{f}{\psi}~
=&~\ap{2}\cdot(\xj{2}-\vj{1})~+~\ap{3}\cdot(\xj{3}-\vj{1})
%
\end{split}
%
\end{equation}

We are using the coordinates $\Psi$ because the first leg is aligned,
so indeed \ $\dif{f}{\phi_{1}}=0$. \ In order for this to be a
critical point, we have two additional geometric conditions: the vanishing of \
$\dif{f}{\theta_{1}}$ \ implies that the vector
connecting the midpoints of sides of the fixed and moving platforms
opposite the first vertex \ -- \ that is, \
$A:=(\xj{2}+\xj{3})/2$ \ and \ $B:=(\aj{2}+\aj{3})/2$ \ -- \ is aligned
with \ $\vj{1}$ \ (see Figure \ref{fcaseIII}). \ On the other hand,
the vanishing (in addition) of \ $\dif{f}{\psi}$ \ is equivalent to:
%
\begin{equation}\label{eqcaseiii}
%
\vp{2}\cdot\xj{2}~+~\vp{3}\cdot\xj{3}~=~0~,
%
\end{equation}
%
which means that the areas of the triangles spanned by \ $\vj{j}$ \
and \ $\xj{j}$ \ ($j=2,3$) \ are equal. Such critical points
will be refered to as \emph{type III}.

\begin{figure}[htbp]
\begin{center}
\epsfysize=5cm %\epsfxsize=6.4cm
\leavevmode \epsffile{fig_3/convex_hull2.eps} \caption{Type III
critical point}\label{fcaseIII}
\end{center}
\end{figure}

Now, calculating the Hessian of $f$ at the critical points we have:
%
\begin{equation*}
%
\begin{split}
%
\diff{f}{\theta_{1}\theta_{1}}~=&~
-~2\vj{1}\cdot((\xj{2}+\xj{3})-(\aj{1,2}+\aj{1,3}))\\
%
\diff{f}{\phi_{1}\theta_{1}}~=&~\diff{f}{\phi_{1}\psi}~=~0\\
%
\diff{f}{\psi\theta_{1}}~=&~2\vj{1}\cdot(\aj{1,2}+\aj{1,3})\\
%
\diff{f}{\phi_{1}\phi_{1}}~=&~
\vj{1}_{1}\cdot\vj{1}_{2}\
\frac{\vj{1}\cdot(4\,\vj{1}+\aj{1,2}+\aj{1,3}-\xj{2}-\xj{3})}{\|\vj{1}\|^{2}}\\
%
\diff{f}{\psi\psi}~=&~\aj{1,2}\cdot(\xj{2}-\vj{1})~+~\aj{1,3}\cdot(\xj{3}-\vj{1})
%
\end{split}
%
\end{equation*}

Again, generically the critical point is non-degenerate\vsm.
%
\end{proof}

\begin{mysubsect}[\label{scrit}]{Identifying the critical points}

Since we usually have no explicit description of the \cspace\ $\C$ as
a manifold, it is hard to calculate the Morse function \ $f:\C\to\RR{}$ \
directly. However, in the course of proving Theorem \ref{tone} we gave
a geometric description of each of the possible critical points of
$f$ \ -- \ which are the main ingredient needed for analyzing the
topology of $\C$ \ -- \ in terms of the \wspace\ $\Wc$. We can use
this geometric information in order to identify all possible
candidates for critical points, and then we need only calculate \
$\df$ \ in local coordinates at these points (also provided in the
proof above) to check if they are indeed critical, and find their indices.

Recall that $\Wc$ (Definition \ref{dwork}) is the space of all possible
locations of the moving platform $\Pc$, whose vertices must be
situated in the respective \wspace s \ $\Wc_{i}$ \ ($i=1,2,3$) \ of
(the end points of) the three branches.  Each \ $\Wc_{i}$ \ is an
annulus centered at the $i$-the vertex \ $\xj{i}$ \ of the fixed triangle.

Also recall the concept of the \emph{coupler curve} $\gamma$ of a
planar four-bar linkage \ - \ that is, a degenerate polygonal
mechanism with \ $k=2$ \ linear branches \ ($\nj{1}=\nj{2}=1$), \
but having a triangular platform $\Pc$: \ the coupler curve is the
\wspace\ for the third (unattached) vertex of $\Pc$. See
\cite[Ch.\ 4]{Ha}. We consider the coupler curves for two vertices \ $\xj{i}$ \
(say \ $i=1,2$) \ of a triangular mechanism  \ $(\Lc,\Xc,\Pc)$ \ as
above, in which the two corresponding branches are aligned, so that
each can be replaced by a single linear branch of length \
$\lj{i}:=\lj{i}_{1}+\lj{i}_{2}$ \ or \ $\lj{i}_{1}-\lj{i}_{2}$, \
as the case may be.

\begin{enumerate}
\renewcommand{\labelenumi}{(\arabic{enumi})}
%
\item The critical points of type I (all three branches aligned)
correspond to placements of $\Pc$ with all three vertices on the
(inner or outer) boundary circles of these annuli. Determining these
is a straightforward geometric problem, which can be described as
intersecting the coupler curve for the first two vertices, say, with
the two boundary circles of \ $\Wc_{3}$.
%
\item For critical points of type II, we need also a line field $V$
along the coupler curve $\gamma$, where \ $V(\gamma(t))$ \ is the
line from \ $\gamma(t)$ \ to the intersection point \ $P(t)$ \
of  \ $\Line\up{1}$ \ with \ $\Line\up{2}$. \ This line field is
readily calculated from $\gamma$. The critical points are
then those configurations for which \ $V(\gamma(t))$ \ passes through \
$\xj{3}$.
%
\item For critical points of type III, the first vertex \ $\vj{1}$ \
of $\Pc$ must lie on one of the two boundary circles of \ $\Wc_{1}$. \
Given \ $\vj{1}$, \ the possible positions of $\Pc$ are determined by
its rotation angle $\theta$ around its first vertex, and at most two
values \ $\theta'$, \ $\theta''$ \ of $\theta$ satisfy condition
\eqref{eqcaseiii}. \ Thus we can define on \ $\partial \Wc_{1}$ \ two
line fields \ $V'$, \ $V''$ \  which associate to \ $\vj{1}$ \ the
line between the midpoints of the \ $(2,3)$-side of the fixed and
moving triangles in the positions corresponding to  \ $\theta'$, \
$\theta''$ \ respectively. The critical points are those for which the
vector \ $\vj{1}$ \ lies on one of these two lines\vsm.
%
\end{enumerate}
%
\end{mysubsect}

\begin{example}\label{egcrit}
%
In general, the critical points of a manifold do not determine its
topology, though they impose certain restrictions on its
homology, and thus on its homotopy type, via the Morse inequalities.
However, in the simplest cases the geometric considerations described
above limit the possible critical points so severely that the \cspace\
$\C$ can be recovered in full. Note that there are two connected
components in $\C$, determined by the orientation of the moving
platform\vsm .

For example, consider a triangular mechanism with one branch (say, \
$k=3$) \ having one very large link, so that the
\wspace\ for the vertex \ $\ppj{3}$ \ contains those for all points of
the moving platform, and thus imposes no restriction on the allowed
configurations.  We assume the moving platform is a small triangle,
and that the \wspace\ for (the vertex of) the first branch is a small
annulus, intersecting that of the second branch in a crescent-shaped
lune, which is the approximate ``\wspace'' for the moving platform
(i.e., for its barycenter). Finally,  assume that the fixed
vertex \ $\xj{3}$ \ is far to the left (see Figure \ref{fmechanism}).

\begin{figure}[htbp]
\begin{center}
\epsfysize=4cm %\epsfxsize=6.4cm
\leavevmode \epsffile{fig_3/fullpicture.eps} \caption{Work spaces
for the
 three moving vertices}\label{fmechanism}
\end{center}
\end{figure}

Now we may analyze the possible critical points as follows:

\begin{enumerate}
\renewcommand{\labelenumi}{(\arabic{enumi})}
%
\item Since the two small annuli above are wholly contained in the
  large one, and the moving platform is small, there are no
  critical points of type I.
%
\item Note that since the linkage is not Grashof (cf.\ \cite{KM}),
there are exactly two cases where \ $\Line\up{1}$ \
meets \ $\Line\up{2}$ \ on the inner boundary circle of the \wspace\ for
vertex $2$. Since \ $\gj{23}$ \ is very small, any critical points
of type II must occur nearby, so that the edges \ $\aj{12}$ \ and \
$\aj{13}$ \ (which nearly coincide) are aligned with \ $\vj{1}$ \
(see Figure \ref{fig4}).

\begin{figure}[htbp]
\begin{center}
\epsfysize=4cm %\epsfxsize=6.4cm
\leavevmode \epsffile{fig_3/crit2.eps} \caption{A critical point
of type II}\label{fig4}
\end{center}
\end{figure}

By choosing appropriate generic values for the parameters, we can
ensure that there are exactly two critical of type II in each
component of $\C$.

%
\item Consider the three dashed lines \ $\Lj{k}$ \ in Figure
  \ref{fig5}, each connecting \ $\xj{k}$ \ with the midpoint of
  opposite (fixed) edge (for \ $k=1,2,3$). \ Because the moving triangle
  is so small, the vector \ $\vj{k}$ \ must approximate the direction
  of \ $\Lj{k}$ \ in order to obtain a critical point of type III \ --
  \ but since these lines do not pass near the approximate \wspace\
  for the moving platform, no such critical points can occur.

\begin{figure}[htbp]
\begin{center}
\epsfysize=5cm %\epsfxsize=6.4cm
\leavevmode \epsffile{fig_3/twoannuli.eps} \caption{Potential
critical points of type III}\label{fig5}
\end{center}
\end{figure}

\end{enumerate}

Thus each component of the \cspace\ \ $\C(\Lc,\Xc,\Pc)$ \ has exactly two critical
points in this case (both of type II), so it is homeomorphic to \ $S^{3}$.
%
\end{example}
