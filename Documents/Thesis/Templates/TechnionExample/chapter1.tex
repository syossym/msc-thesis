\chapter{Arachnoid mechanisms}
\label{chap1}

\textsf{A paper published in \textsl{Forum Mathematica} Vol 17/6
(Nov '05),
1033-1042.}\\
\textsf{Co-authors: David Blanc and Moshe Shoham.}\\

The \cspace s of arachnoid mechanisms are analyzed in this paper.
These mechanisms consist of $k$\ branches each of which has an
arbitrary number of links and a fixed initial point, while all
branches end at one common end-point. It is shown that
generically, the \cspace s of such mechanisms are manifolds, and
the conditions for the exceptional cases are determined. The
\cspace\  of planar arachnoid mechanisms having $k$ branches, each
with two links is analyzed for both the non-singular and the
singular cases.


\sect{Introduction}

Mechanisms and robots consist of links and joints, the actuation
of which causes them to move. The type of a mechanism is described
by an abstract graph which corresponds to its links and joints,
and a specific embedding of this graph in the plane or in
$3$-space is called a \emph{configuration} of the mechanism. The
collection of all such embeddings forms a topological space,
called the \emph{\cspace} of the mechanism. For example, the
\cspace\ of a planar mechanism with revolute joints consisting of
$n$ rods arranged serially is the $n$-torus.

In recent years, there has been interest among mathematicians in
the study of such spaces, which are of importance in motion
planning \ -- \ that is, moving a mechanism from one given
position to another, taking into account various constraints (see
for example \cite{MT2}). The topological properties of the
\cspace\ provide insight into practical questions in planning such
motions (see \cite{F}) and analysis of some mechanical
singularities (see \cite{NM}, and \cite{ZFB}).

The main focus had been set on the \cspace s of a type of
mechanism called \emph{polygonal linkage}, which is simply a
concatenation of links and hinged joints forming a closed chain. A
substantial amount of mathematical literature on polygonal
linkage's \cspace\ has accumulated: Kamiyama , Tezuka and Toma
studied Euler characteristics in \cite{K}, and homology groups in
\cite {KT,KTT}; Trinkle and Milgram constructed a handle-body
surgery in \cite{MT1}; and in \cite{Ho}, Holcomb studied a special
parallel graph mechanism called multi-polygonal linkages, which
are three free branches identified at their initial and terminal
vertices.

In this paper we analyze a type of mechanism called
\emph{arachnoid} which, to the best of our knowledge, has never
been dealt with in the literature. This kind of mechanism consists
of multiple branches each of which has an arbitrary number of
links and a fixed initial point, while all branches end at a
common end-point (this type of mechanism resembles some parallel
robots which are in practical use). It is shown that generically,
the \cspace s of such mechanisms are manifolds, and the conditions
for the exceptional cases are then determined. The \cspace\  of
planar arachnoid mechanisms having $k$ branches, each with two
links is fully analyzed, while for the non-manifold cases we
analyze the singular configurations.

We now introduce some notation and terminology to describe such
mechanism types, and in particular the \smech s which are the
subject of this note:

\begin{defn}
%
For a mechanism $M$ in \ $\Re^{d}$, \ a \emph{branch} \ $(L,\bx)$
\ of multiplicity $n$ is a sequence \
$L=(\ell_{1},\dotsc,\ell_{n})$ \ of $n$ positive numbers, together
with a point \ $\bx\in\Re^{d}$. \ We think of $L$ as the lengths
of $n$ concatenated rods, having revolute (i.e., rotational)
joints at the meeting point of every two consecutive rods, and at
the fixed initial point $\bx$, called the \emph{base point} of the
branch.

A \emph{branch configuration} \ $V=(\bv_{1},\dotsc,\bv_{n})$ \ for
a branch \ $(L,\bx)$ \ then consists of $n$ vectors in \ $\Re^{d}$
\ with the given norms \ $\|\bv_{i}\|=\ell_{i}$ \ ($i=1,\dotsc
n$).
\end{defn}
\newnot{1-3}


Since the \cspace\ of a branch \ $(L,\bx)$ \ \newnot{1-1} is
independent of the order of the set \ $\ell_1,..,\ell_n$ \ (up to
homeomorphism), we can (and shall) assume \ $\ell_1,..,\ell_n$ \
to be in descending order.

\begin{defn}
%
An \emph{\smech} consists of $k$ branches \ $$
(\Lc,\Xc)=((\Lj{1},\xj{1}),\dotsc (\Lj{k},\xj{k})) $$
\newnot{1-2} with multiplicities \ $\nj{1},...,\nj{k}$. \ We think
of this as a linkage of branches connected by a single revolute
joint at their common end point (whose location is not fixed).
\newnot{1-4}

An \smech\ configuration for \ $(\Lc,\Xc)$ \ thus consists of a
set $$ \Vc=(\Vj{1},..,\Vj{k}) $$ of branch confiurations for $\Lc$
having a common \emph{end point}  \
$\by=\xj{i}+\sum_{j=1}^{\nj{i}}~\vj{i}_{j}$ \ $(i=1,..,k)$. \
  \end{defn}

\begin{figure}[h]
\begin{center}
\epsfysize=5cm %\epsfxsize=4.4cm
\leavevmode \epsffile{fig_1/mechanism.eps} \caption{An \smech\
with $k=3$, \ multiplicities $2$.} \label{fig:mechanism}
\end{center}
\end{figure}

\begin{defn}
\label{def:aligned}

 For an \smech \ $(\Lc,\Xc)$:

\begin{enumerate}
\item A branch configuration $\V=(\bv_1,\dotsc,\bv_n)$ \ is
\emph{aligned} (with direction $\bw$) if each vector \
$\bv_1,\dotsc,\bv_n$ \ is a scalar multiple of $\bw$. \item A
configuration \ $\Vc=(\Vj{1},..,\Vj{k})$ \ of \ $(\Lc,\Xc)$ \ is
called  a $t$-\emph{node} if it has $t$ aligned  branch
configurations with directions \ $\bw_{i_{1}},\dots,\bw_{i_{t}}$ \
respectively, which are linearly dependent; otherwise $\Vc$ is
called \emph{generic}.
\end{enumerate}
\end{defn}

\newnot{1-6}

\begin{defn}
The collection \ $\C=\C(\Lc,\Xc)$ \ of all configurations \ $\Vc$
\ for \ $(\Lc,\Xc)$ \ is called its \emph{\cspace}. It is
topologized as a subspace of the appropriate Euclidean space. The
space of all such common endpoints $\by$ will be called the
\emph{work space} \ $\Wc=\Wc(\Lc,\Xc)$ \  for \ $(\Lc,\Xc)$. \ The
\emph{work map} \ $\Phi:\C\to\W$ \ assigns to each configuration
$\Vc$ its common endpoint $\by$.
\end{defn}

\newnot{1-5}

\begin{org} \label{sorg} In section \ref{cmain} we show that the \cspace\ of a generic
 \smech\ \ $(\Lc,\Xc)$ \ is a manifold. In section \ref{cplan} we study planar
  \smech s for which each branch has $2$ joints, and give an explicit formula
  for the toplogical type of \ $\C=\C(\Lc,Xc)$ \ in the generic case. Finally,
   in section \ref{csingc}, we analyze the singularities of $\C$ for such planar
    \smech s in the non-manifold case.
\end{org}
%
%c2   Generic \smech s in $\Re^{d}$}
%
\sect{Generic \smech s in $\Re^{d}$} \label{cmain}

First, we show that, generically, the \cspace\ of an \smech\ is a
manifold:

\begin{thm}\label{thm:main}\stepcounter{subsection}
%
Let \ $(\Lc,\Xc)$ \ be an \smech\ in \ $\Re^{d}$ \ with $k$
branches of multiplicities \ $\nj{1},...,\nj{k}$, \ respectively.
If all configurations of \ $(\Lc,\Xc)$ \ are generic, then its
\cspace\ $\C$ is a smooth closed orientable manifold of dimension
\ $d(N-k+1)-N$, \ where \ $N=\sum_{i=1}^{k}\nj{i}$.
%
\end{thm}

\begin{proof}
%
We can identify \ $\C=\C(\Lc,\Xc)$ \ as the pre-image of a certain
function \ $G:\Re^{d(N-k+1)}\to\Re^{N}$, \ where $G$ is defined as
follows:

For each \ $n\geq 1$ \ let \ $g_{n}:(\Re^{d})^{n}\to\Re^{n-1}$ \
be defined \ $$ g_{n}(\bv_{1},\dotsc,\bv_{n}):=
(|\bv_{2}-\bv_{1}|^{2},\dotsc,|\bv_{n}-\bv_{n-1}|^{2}),
$$
where \ $|\bu|:=(\sum_{i=1}^{d}t_{i}^{2})^{1/2}$ \ is the length
of a vector \ $\bu=(t_{1},\dotsc,t_{d})\in\Re^{d}$. \ Now for each
branch \ $\Lj{i}=(\lj{i}_1,..,\lj{i}_{\nj{i}})$ \ of $\Lc$, \ let
\ $(\vj{i}_{1},\dotsc,\vj{i}_{\nj{i}})$ \ be position vectors of
the ends of the \ $\nj{i}$ \ links of a branch configuration,
where \ $\vj{i}_{0}=\xj{i}$ \ (the given base point for this
branch). Since in an \smech\ all branches end at the same point \
$\bu\in\Re^{d}$, \ we have \ $\vj{i}_{\nj{i}}=\bu$ \ for all \
$1\leq i\leq k$. \ Thus we have \ $N-k+1$ \ different vectors \
$\{\vj{i}_{1},\dotsc,\vj{i}_{\nj{i}}\}_{i=1}^{k}$, \ and we define
$$ G(\vj{1}_{1},\dotsc,\vj{k}_{\nj{k}}):=
(g_{n_{1}}(\vj{1}_{0},\dotsc,\vj{1}_{n_{1}}),
g_{n_{2}}(\vj{2}_{0},\dotsc,\vj{2}_{n_{2}}),\dotsc
g_{n_{k}}(\vj{k}_{0},\dotsc,\vj{k}_{n_{k}})),
$$
so that \ $\C=G^{-1}(\vec{\ell})$ \ for \
$\vec{\ell}:=(\lj{1}_{1})^{2},..,(\lj{1}_{\nj{1}})^{2},\dotsc
(\lj{k}_{1})^{2},..,(\lj{k}_{\nj{k}})^{2})$. \ By the Regular
Value Theorem (see \cite[I, Thm.~3.2]{Hi}), \ $\C$ will be a
smooth manifold if \ $\vec{\ell}$ \ is a regular value of $G$ \ --
\ that is, \ $dG$ \ has maximal rank.

Note that (after applying elementary row and column operations), \
$dG$ \ has the following \ $N \times dN$ \ Jacobian matrix:

$$
dG=2\left(
\begin{array}{ccccc}
    A^{(1)} &B^{(1)}& & 0&\\
    A^{(2)} & &B^{(2)}& &\\
     \vdots &  & & \ddots & \\
      A^{(k)} &0 &  & & B^{(k)} \vspace{2mm}\\ \end{array}
\right)
$$

\noindent where each \ $(\nj{i}\times d)$-block \ $A^{(i)}$ \ is:

$$
A^{(i)}=\left(
\begin{array}{c}
      \vj{i}_{\nj{i}}-\vj{i}_{\nj{i}-1}\\
       0    \\
     \vdots \\
        0   \\
\end{array}
\right)
$$

\noindent and \ $B^{(i)}$ \ is:

$$
\left(
\begin{array}{cccccc}
\vj{i}_{\nj{i}-1}-\vj{i}_{\nj{i}} &  0 & 0 &\dotsc & 0 & 0 \\
\vj{i}_{\nj{i}-1}-\vj{i}_{\nj{i}-2} &
\vj{i}_{\nj{i}-2}-\vj{i}_{\nj{i}-1} &
0 &\dotsc & 0 & 0 \\ 0 &
\vj{i}_{\nj{i}-2}-\vj{i}_{\nj{i}-3} &
\vj{i}_{\nj{i}-3}-\vj{i}_{\nj{i}-2}
  & \dotsc & 0 & 0 \\
0  & 0 & 0 & \dotsc & 0 & 0  \\
\vdots  & \vdots & \vdots & \ddots & \vdots & \vdots  \\ 0  & 0 &
0 & \dotsc & \vj{i}_{2}-\vj{i}_{1}  & \vj{i}_{1}-\vj{i}_{2} \\ 0
& 0 & 0 & \dotsc & 0 & \vj{i}_{1}-\vj{i}_{0} \end{array} \right)
$$

\noindent an \ $\nj{i}\times d(\nj{i}-1)$ matrix which can be
thought of as the Jacobian matrix for a corresponding closed
$\nj{i}$-branch. Note that \ $\rank(B^{(i)})\leq\nj{i}$, \ and \
$B^{(i)}$ \ has less than full rank only when all vectors \
$\vj{i}_{\nj{j}-1}-\vj{i}_{\nj{j}}$ \ are collinear for \ $1\leq
j\leq \nj{i}$ \ -- \ so that the $i$-th branch is aligned. In this
case \ $\rank(B^{(i)})=\nj{i}-1$, \ and the non-zero row \
$\bw^{(i)}:=\vj{i}_{\nj{i}}-\vj{i}_{\nj{i}-1}$ \ of \ $A^{(i)}$ \
is precisely the direction of the branch.

We can thus divide the matrix \ $dG$ \ horizontally into two
blocks: \ $(A,B)$, where $$  A:=\left( \begin{array}{c}
  A^{(1)} \\
  \vdots \\
  A^{(k)} \\
\end{array}\right)
 \ \ \ \ \text{and} \ \ \ \ B:=\left(
\begin{array}{cccc}
    B^{(1)}& & 0&\\
     &B^{(2)}& &\\
      & & \ddots & \\
      0 &  & & B^{(k)} \\
\end{array}
\right),
$$
and the rank of \ $dG$ \ is then given by:
\begin{equation}\label{eone} \rank(A,B)=\rank(A)
+\rank(B)-\dim(\col(A) \cap\col(B) ). \end{equation}

Denote by $I$ the set of all indices $i$ for which the $i^{(th)}$
branch is aligned, so that \ $\rank(B)=N-|I|$. \ Thus if \
$I=\emptyset$, \ then \ $dG$ \ has maximal rank. If \ $|I|\neq0$,
\ let \ $A_I$ \ be the sub-matrix of $A$ consisting of the blocks
\ $A^{(i)}$ \ with \ $i \in I$. \ Its rows are therefore spanned
by the directions \ $\{\bw^{(i)}\}_{i\in I}$ \ of the aligned
branches. Observe that \ $\rank(A)-\rank(A_I)$ \ is the dimension
of the subspace of \ $\col(A)$ \ consisting of columns whose
entries vanish in the rows indexed by \ $i\in I$. \ Since the
blocks of $B$ indexed by \ $i\not\in I$ \ have full rank, we see
that $$ \dim(\col(A)\cap\col(B))~\leq~\rank(A)-\rank(A_I)
$$
\noindent (in fact, equality holds). By \eqref{eone}:
$$
\rank(dG)\geq\rank(A_I)+\rank(B)=N-|I|+\rank(A_I),
$$
which means that \ $dG$ \ has full rank unless \ $|I|>\rank(A_I)$.
\ The latter implies that the directions of the aligned branches
are linearly dependent \ -- \ that is, we have a $k$-node
configuration.

Note that \ $\C=G^{-1}(\vec{\ell})$ \ is in fact orientable when \
$dG$ \ has maximal rank, since in that case it induces an
isomorphism between the normal bundle $\nu$ to $\C$ in \
$\Re^{d(N-k+1)}$ \ at any point and the ``normal bundle'' to \
$\{\ell\}\hookrightarrow\Re^{N}$. \ Thus $\nu$ (the complement to
the tangent bundle \ $T\C$ \ in \ $\Re^{d(N-k+1)}$) \ is
orientable, so  \ $T\C$ \  is, too. Finally, $\C$ is compact since
it is a closed subset of the free configuration space, which is a
$N$-torus. \end{proof}
\newnot{1-8}
\begin{remark}
%
For an \smech \ in $\Re^{3}$, \ the matrix \ $dG$ \ will be
singular for a $2$-node configuration (two aligned branches along
one line); a $3$-node configuration (three aligned branches
contained in one plane); or a $4$-node configuration (four aligned
branches).
%
\end{remark}

%
%c3   Planar mechanisms
%
\sect{Planar mechanisms} \label{cplan}

>From now on we restrict attention to \smech s \ $(\Lc,\Xc)$ \ in the
>plane
(that is, \ $d=2$).

\subsection{The work space}
\label{sws}\stepcounter{thm}

In this case, each vector \ $v_j$ \ in a branch configuration $V$
(of multiplicity $n$) is determined by its argument $\theta_j$ \
(since \ $\|v_j\|=\ell_j$), \  and $V$ can thus be identified with
a point \ ($\theta_1,..,\theta_n$) in the $n$-torus \ $$
\T{n}=\underbrace{\Sb{1}\times \dotsc\times\Sb{1}}_{n}. $$
\newnot{1-9} Thus if \ $\nu=\nj{1}+\dotsc+\nj{k}$, \ then \
$\C(\Lc,\Xc)\subseteq\T{\nu}=\prod_{i=1}^{k}\T{\nj{i}}$.


Given such an \smech, we can describe the work space $\Wc$ as
follows: for any branch \ \ $L=(\ell_{1},\dotsc,\ell_{n})$, \ let
\ $\beta(L)_{\min}$ \ denote the minimal value of \
$|\sum_{j=1}^{n}\varepsilon_{j}\ell_{j}|$, \ where \
$\varepsilon_{j}=\pm 1$ \ for each \ $1\leq j\leq n$; \ and let \
$\beta(L)_{\max}:=\sum_{j=1}^{n}\ell_{j}$ \ (the maximal value). \
The work space \ $W=\Wc(L,\bx)$ \ for the branch $L$ with base
point $\bx$ \ (without any constraint on the end point) \ is then
an annulus bounded by circles of radius \ $\beta(L)_{\min}$ \ and
\ $\beta(L)_{\max}$, \ respectively.

If  \ $\Lc=(\Lj{1},...,\Lj{k})$, \ with multiplicities \
$\nj{1},...,\nj{k}$, \ and \ $\Xc=(\xj{1},...,\xj{k})$, \ the work
space for \ $(\Lc,\Xc)$ \ is \ $\Wc=\bigcap_{i=1}^k\Wi{i}$, \
where \ $\Wi{i}=W(\Lj{i},\xj{i})$. \ Each component of $\Wc$ is a
curvilinear polygon $P$ (not necessarily convex), whose edges \
$\Edge(P)$ \ are arcs of the annuli boundary circles \
$\partial\Wi{i}$, \ and whose vertices \ $\Vertex(P)$ \ are
intersection points of such arcs.

\subsection{The \cspace}
\label{scs}\stepcounter{thm}
\newnot{1-7}
The \cspace\ for any branch \ \ $L=(\ell_{1},\dotsc,\ell_{n})$ and
base-point $\bx$ \ is an $n$-torus \ $\T{n}$, \ with work map \
$\phi:\T{n}\to W$.

Note that the fiber \ $\phi^{-1}(z)$ \ over any point \ $z\in\Int
W$ \ is the \cspace\ for the \emph{closed} chain with links \
$\ell_{0},\ell_{1},\dotsc,\ell_{n}$, \ where \ $\ell_{0}:=z-\bx$.
\ This \cspace\ has been analyzed in \cite{HR}. On the other hand,
if $z$ is on the boundary of the annulus $W$, then \
$\phi^{-1}(z)$ \ is evidently discrete, and in fact consists of a
single point (unless it is on the inner circle, and \
$\beta(L)_{\min}$ \ can be written as \
$|\sum_{j=1}^{n}\varepsilon_{j}\ell_{j}|$ \ in more than one way).

If \ $\Lc=(\Lj{1},...,\Lj{k})$, \ with multiplicities \
$\nj{1},...,\nj{k}$, \ and \ $\Xc=(\xj{1},...,\xj{k})$, \ its
\cspace\ is the pullback $$
\C(\Lc,\Xc)~=~\{(\tau_{1},\dotsc,\tau_{k})\in\prod_{i=1}^{k}~\T{\nj{i}}~
|\ \phi_{1}(\tau_{1})=\dotsc=\phi_{k}(\tau_{k})\in\Wc\}.
$$

\subsection{Example}\label{exa:lens}\stepcounter{thm}
Consider an \smech\ consisting of three branches, each of
multiplicity $2$, as in Figure \ref{fig:mechanism}.

The workspace for each free branch is an annulus; let us assume
that the three annuli intersect in the shaded lens-shaped
component $P$ in Figure \ref{fig:annulus intersection}.

\begin{figure}[h]
\begin{center}
\epsfysize=5cm \leavevmode \epsffile{fig_1/annuli.eps}
\caption{Polygonal intersection} \label{fig:annulus intersection}
\end{center}
\end{figure}

Now consider an interior point \ $y\in\Int(P)$: \ in the
corresponding configurations in the fiber \ $\Phi^{-1}(y)$, \ each
of the three branches can be in one of two positions (branch
configurations), usually termed ``elbow up'' (\textbf{u}) and
``elbow down '' (\textbf{d}), so for each branch we have a copy of
\ $\Sb{0}=\{\mathbf{u},\mathbf{d}\}$. \ Thus the fiber consists of
eight points \ $\mathbf{uuu},\mathbf{uud},\dotsc,\mathbf{ddd}$, \
thought of as the product \ $\Sb{0}\times\Sb{0}\times\Sb{0}$. \
Thus \ $\Phi^{-1}(\Int(P)$ \ is simply an eight-fold cover of the
interior of the lens.

On the other hand, if $y$ is on the edge $\alpha$ of $P$, which is
in the outer boundary of the first annulus, the first branch is
completely extended, identifying its \textbf{u} and \textbf{d}
positions, thus collapsing the first \ $\Sb{0}$ \ to a single
point, and generally identifying the copy of $\alpha$ in $\C$
indexed by \ $\mathbf{u\ast\ast}$ \ with the copy indexed by
$\mathbf{d\ast\ast}$, \ for \
$\mathbf{\ast\ast}\in\{\mathbf{uu},\mathbf{ud},\mathbf{du},\mathbf{dd}\}$.

Similarly, the copy of $\beta$ in $\C$ indexed by \
$\mathbf{\ast\ast u}$ \ is identified with that indexed by
$\mathbf{\ast\ast d}$. \ Therefore, the fiber of \
$y\in\alpha\cap\beta$ \ consists of two points. Note that the
second \ $\Sb{0}$-factor never collapses, so $\C$ is of the form \
$\Sb{0}\times\M$ \ -- \ i.e., $\C$ has two components, each
isomorphic to $\M$.

To evaluate the Euler characteristic of $\M$, note that it is
obtained from four $2$-gons (the lens-shaped intersection in $\W$)
by identifying their $8$ edges pairwise (as explained above),
identifying the ``top'' vertex in all the $2$-gons to a single
point, and similarly for the ``bottom'' vertex. Thus \
$\chi(\M):=V-E+F=4-4+2=2$, \ so $\M$ is a $2$-sphere, and \
$\C\cong\Sb{2}\sqcup\Sb{2}$.

\subsection{Invariants of annulus arrangements} \label{sica}\stepcounter{thm}

An annulus (i.e., pair of concentric circles) in the plane is
determined by \ $(\bx,\beta_{\min},\beta_{\max})$, \ where \
$\bx\in\Re^{2}$ \ is the center and \
$0<\beta_{\min}<\beta_{\max}$ \ are the radii. Consider a system
$$ \langle(\xj{1},\beta(1)_{\min},\beta(1)_{\max});
\dotsc;(\xj{k},\beta(k)_{\min},\beta(k)_{\max})\rangle
$$
of $k$ such pairs (with distinct centers), and let $\W$ denote the
intersection of all the corresponding anulli; this may have
several connected components \ $V_{1},\dotsc,V_{t}$. \ What we
have in mind, of course, is the collection of boundary circles for
the work space of branches of an \smech.

The boundary \ $\partial V$ \ of each component $V$ of $\W$ is a
curvilinear planar polygon, not necessarily connected; \ let \
$\alpha:=\alpha(V)$ \ denote the number of components of \
$\partial V$ \ contained in the interior of its convex hull \
$\conv(V)$. \ We set \ $$ \gamma:=\gamma(V)=\begin{cases}
1 & \text{if \ $\conv(V)$ \ is a disc}\\ 0 & \text{otherwise.}\end{cases} $$ % $V$ may
be wholely contained in the interior of some of the annuli; denote the number
of such annuli by \ $\beta:=\beta(V)$ \ ($0\leq \beta(V)\leq k$), \ and let the
\emph{$c$-invariant} of $V$ be \ $c(V):=2^{\beta}$. \ Finally, the \emph{$g$-invariant} of $V$ is:
$$
g(V):=1-2^{k-\beta-3}(|\Vertex(V)|-2|\Edge(V)|+4+2\gamma-4\alpha).
$$

\begin{thm}
\label{thm:sss}\stepcounter{subsection}
%
Let \ $(\Lc,\Xc)$  \ be a planar \smech\ with $k$ branches, each
of multiplicity $2$,  and assume that the \cspace\ \
$\C=\C(\Lc,\Xc)$ \ has no node configrations;  then $\C$
decomposes as the disjoint union of the pre-image under $\Phi$ of
the components of the workspace $\W$, and for each such component
$V$, \ $\Phi^{-1}(V)$ \ consists of \ $c(V)$ closed orientable
surfaces of genus \ $g(V)$.
%
\end{thm}

A special case of this Theorem appears in \cite{E}.

\begin{proof}
%
As in example \ref{exa:lens} above, \ $\Phi^{-1}(V)$ \ is obtained
from the \ $2^{k}$ \ curvilinear polygonal ``tiles'' (i.e., copies
of $V$, corresponding to the ``elbow up/elbow down'' position of
each branch), by identifications of those edges which correspond
to the \ $k-\beta$ \ ``relevant'' branches. Since we know from
Theorem \ref{thm:main} that (each component of) $\C$ is a closed
orientable $2$-manifold, its type (genus) is determined by the
Euler characteristic, which may be computed by calculating how
many identifications we have for each vertex or edge of  \
$\Phi^{-1}(V)$.

To orient \ $\Phi^{-1}(V)$, \ choose an orientation for some
(fixed) tile \ $H_{0}$. \ Every other tile $H$ of \ $\Phi^{-1}(V)$
\ differs from \ $H_{0}$ \ in exactly $\tau$ of the $k$ possible
``elbow up/elbow down'' positions, and we reverse its orientation
(relative to that of $H_{0}$) \ if and only if $\tau$ is odd.
%
\end{proof}

\begin{figure}[htbp]
\begin{center}
\epsfysize=3cm %\epsfxsize=4.4cm
\leavevmode \epsffile{fig_1/embeddings.eps}  \caption{Work and
configuration space intersections of two
branches}\label{fig:embeddings}
\end{center}
\end{figure}

\begin{remark}
\label{rem:conntor}

As noted in \S \ref{sws}, the workspace $\W$ was obtained by
repeatedly intersecting annuli, which are the workspaces of the
individual branches. If we concentrate on the annulus $A$ of the
first branch, say, then generically the annuli for each of the
remaining branches will interesect with $A$ in one of the six
shaded patterns $V$ in the first row of Figure
\ref{fig:embeddings}. In each case we obtain a certain subset \
$\phi^{-1}(V)$ \ of the $2$-torus \ $\T{2}$ \ which is the
\cspace\ for the first branch, where \ $\phi:\T{2}\to A$ \ is the
work map for the first branch.

Note that when we further intersect $V$ with a third annulus, the
pattern may be more complicated; in particular, three-fold
intersections need not be connected, as illustrated by Figure
\ref{fig:punctured_torus}, which shows the subset of the $2$-torus
associated with the workspace of Figure~\ref{fig:mechanism}
(without indicating the identifications).

\begin{figure}[h] \epsfysize=3cm%\epsfxsize=4.4cm
\begin{center}
\leavevmode \epsffile{fig_1/punctured_torus.eps}   \caption{Subset
of $\T{2}$}\label{fig:punctured_torus}
\end{center}
\end{figure}
\end{remark}

%c4   Singular \cspaces
\sect{Singular \cspace s} \label{csingc}

While the analysis of the \cspace \ $\C=\C(\Lc,\Xc)$ \ of a
mechanism in the non-manifold case is in general difficult, for a
planar \smech\ with branch multiplicity $2$ the description of
Theorem \ref{thm:sss} can actually be extended to singular points,
corresponding to the node configurations.

\begin{prop}\label{prop:2node}\stepcounter{subsection}
%
Let \ $(\Lc,\Xc)$  \ be a planar \smech\ with $k$ branches, each
of multiplicity $2$,  and let $\Vc$ be a node configuration in \
$\C=\C(\Lc,\Xc)$. \ Then $\Vc$ has an open neighborhood in $\C$
which is a wedge of  \ $2^{q-\varepsilon}$ \ $2$-dimensional discs
with common center $\Vc$, where $q$ is the number of aligned
branches, \ $\varepsilon=1$ \ if the aligned branches have a
common direction, and \ $\varepsilon=2$ \ otherwise.

\end{prop}

\noindent\emph{Proof.} \ Let \ $v:=\Phi(\Vc)$ \ be the common
end-point of the $k$-branches in $\Vc$, and let $U$ be an small
disc containing $v$ in the workspace $\Wc$. Since $v$ is
necessarily in the boundary of a curvilnear polygonal component of
$\Wc$, then $P$, the boundary of $U$ near $v$, consists of arcs of
the boundary circles of the annuli.

\begin{enumerate}
\item When all of the $q$ aligned branches have a common
direction, the centers of the corresponding annuli must be
collinear, and $P$ is an arc $e$ of a single boundary circle of
such an annulus. Note that \ $\Phi^{-1}(\Int(U))$ \ intersects the
component of $\Vc$ in \ $2^{q}$ \ disjoint discs, which are
identified pairwise along \ $\Phi^{-1}(e)$ \ so as to yield \
$2^{q-1}$ \ discs whose only common point is \
$\Vc\in\Phi^{-1}(v)$.

\item Otherwise, $P$ must consist of two arcs \ $e_{1}$, \ $e_{2}$
\ of distinct boundary circles (whose centers are not collinear
with $v$). Again \ $\Phi^{-1}(\Int(U))$ \ intersects the component
of $\Vc$ in \ $2^{q}$ \ disjoint discs, but now every four of them
(corresponding to the four ``elbow up/elbow down'' positions of
the two branches associated to \ $e_{1}$ \ and \ $e_{2}$, \
respectively) are identified in \ $\Phi^{-1}(U)\subset\C$ \ along
\ $\Phi^{-1}(e_{1})$ \ and \ $\Phi^{-1}(e_{2})$, \ forming the
four quadrants of a new disc \ -- \ where again $\Vc$ is the only
point in common. This yields a total of \ $2^{q-1}$ \ discs.\eee

\end{enumerate}
