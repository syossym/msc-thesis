\chapter{Motion Planning for a Class of Planar Closed-Chain Manipulators}
\label{chap2}

\textsf{Accepted to \textsl{International Journal of Robotic Research}, 2006.} \\
\textsf{Co-authors: Guanfeng Liu, Jeff Trinkle and Moshe
Shoham.}\\
\textsf{Presented in \textsl{IEEE International
Conference on Robotics and Automation}, 2006.} \\

We study the motion planning problem for planar {\sl star-shaped}
manipulators. These manipulators are formed by joining $k$ ``legs"
to a common point (like the thorax of an insect) and then fixing
the ``feet" to the ground.  The result is a planar parallel
manipulator with $k-1$ independent closed loops.  A topological
analysis is used to understand the global structure the
configuration space so that planning problem can be solved
exactly. The worst-case complexity of our algorithm is
$O(k^3N^3)$, where $N$ is the maximum number of links in a leg.
Examples illustrating our method are given.

\sect{Introduction} The canonical robot motion planning problem is
known as the ``piano movers'" problem.  In this problem, one is
given initial and goal configurations of a ``piano" (a rigid body
that is free to move in an environment with fixed rigid obstacles)
and geometric models of the piano and obstacles.  The goal is to
find a continuous motion of the piano connecting the initial and
goal configurations. Lozano-Perez studied this problem in
configuration space, or \cspace, a space in which a configuration
of the piano maps to a point, a motion maps to a continuous curve,
and the obstacles map to the C-obstacle, {\em i.e.,} the set
corresponding to overlap between the piano and an obstacle
\cite{Lozano-Perez83}.  The dimension of \cspace \ is equal to the
number of degrees of freedom of the system. The free space, or
C-free, is what remains after removing the C-obstacle from
\cspace. In \cspace, the motion planning problem becomes a path
planning problem. That is, one must construct a continuous path
connecting the initial and goal configurations that lies entirely
within C-free. Theoretical results for the piano movers' problem
were first obtained by Schwartz, Sharir, and Hopcroft
\cite{SS83,SHS87}. They found that the problem is PSPACE hard, and
proposed an algorithm based on Collins' decomposition to find a
path. Since the worst-case running time of Collins' decomposition
algorithm is doubly exponential in the dimension of \cspace, it is
impractical.

The more complex generalized movers' problem, is the problem in
which there are multiple rigid bodies moving simultaneously in a
workspace. The bodies are the links of one or more robots, and
thus may be required to obey constraints corresponding to their
kinematic structures and joint limits. Given the importance of
motion planning problem in robotics, researchers worked to find
more efficient algorithms despite the depressing complexity
results found earlier. The most efficient exact method known is
Canny's algorithm, which has time complexity that is only singly
exponential in the dimension of \cspace \cite{Can88}. He also made
the important observation that this bound is worst-case optimal,
since the worst-case number of components in \cspace \ is
exponential in its dimension. Canny's algorithm is very difficult
to implement - to date no full implementation exists.

In the 1990's, the intractability of exact motion planning for
general problems stimulated a paradigm shift to randomized
methods. The method of Barraquand and Latombe combined potential
field methods with random walk \cite{BL91}.  In essence, a
potential field method defines an artificial potential field on
\cspace \ such that the goal configuration is the global minimum
of the potential function and no saddle points or other local
minima exist. When the function has this property, motion planning
can be done by any gradient following algorithm. An important
class of such functions are navigation functions
\cite{Koditschek87,RK88,RK89}. Ideally, the potential function
will be a function of the goal configuration, and the global
minimum property will hold for all possible goal configurations.
Since such potential functions can be difficult to design,
Barraquand and Latombe suggested the use of random walks to escape
local minima \cite{BL91}. This modification yielded a method that
is practically effective and probabilistically complete.

When possibly many motion planning queries must be handled for a
single static environment, a different type of randomized method
has been found to be more efficient than rerunning the
Barraquand-Latombe algorithm for each query. The probabilistic
roadmap method (PRM) of Kavraki {\em et. al} \cite{KSLO96}, is an
easy-to-implement randomized version of Canny's \cite{Can88}.

Because PRMs have been successful in solving problems in \cspace s
with dimension approaching 100, many researchers have worked to
make the method more efficient ({\em e.g.,}
\cite{ABDJV98,BOS99,BK00}) and to modify it to solve more
challenging types of problems, such as those with closed kinematic
loops, nonholonomic constraints, dynamics, and intermittent
contact ({\em e.g.,} \cite{YLK01,HA01,CG99,KL00,LK99,RRT}).

In this paper, we are particularly interested in planar {\sl
star-shaped manipulators}. These manipulators are formed by
joining $k$ planar ``legs" to a common point (like the thorax of
an insect) and then fixing the ``feet" to the ground.  The result
is a planar parallel manipulator with $k-1$ independent closed
loops. They are important because they arise in parallel
manipulators, walking robots, and dexterous manipulation, and
motion plans are difficult to obtain using PRMs. In such systems,
\cspace \ is often most naturally viewed as a lower-dimensional
space embedded in an ambient space (typically the joint space).
The embedding results from equality constraints corresponding to
kinematic loop closure. In such settings, it is difficult to
obtain an explicit description of \cspace \ with minimal number of
parameters and a suitable metric to guide sample generation. These
problems make it difficult to construct a roadmap with the
requisite properties, and hence difficult  to solve motion
planning problems for systems with kinematic loops using PRMs. The
RLG (random loop generator) method \cite{Cortes02,CS03} improves
the sampling techniques through estimating the regions of sampling
parameters. However, its efficiency relies on the accuracy of the
estimation, which often varies case by case. Moreover, it ignores
the global structure of \cspace, and may fail to sample globally
important regions.
%
\begin{figure}
  \centering
  \includegraphics[width=3in]{fig_2/x-shape.eps}
  \caption{Star-shaped manipulator with $k=4$.}
  \label{X-shape}
\end{figure}
%
The difficulties associated with applying randomized motion
planning methods to manipulators with closed chains and the
availability of new results in topology \cite{KTT,KM,MT1,SSB05}
have recently led to renewed interest in exact planning
algorithms. Trinkle and Milgram derived some topological
properties of the \cspace s (the number of components and the
structures of the components) of single-loop closed chains with
spherical joints in a workspace {\em without} obstacles
\cite{MT2,MT1}.  These properties drove the design of a complete,
polynomial-time motion planning algorithm that works roughly as
follows.
%
\begin{enumerate}
\item Choose a subset ${\cal A}$ of the links that can be
positioned arbitrarily, and yet the remaining links can close the
loop; \item Move the links in ${\cal A}$ to their goal
orientations along an arbitrary path while maintaining loop
closure; \item Permanently fix the orientations of the links in
${\cal A}$; \item Repeat until all link orientations are fixed.
\end{enumerate}
%
The main result that guided the algorithm's design is Theorem~2 in
\cite{MT1}. In generic cases, the \cspace \ is the union of
manifolds that are products of spheres and intervals. The joint
coordinates corresponding to the spheres are those that can
contribute to the subset ${\cal A}$ mentioned above and the
structure of the \cspace \ suggests a local parametrization for
each step.

Here, the previous methods for \cspace \ connectivity analysis are
extended to planar star-shaped manipulators with revolute joints.
These manipulators have a common junction point and $k$ $(k>0)$
legs connecting the junction to the fixed base. Following a
topological analysis of the global structure of \cspace \ (i.e.,
the structure of the inverse kinematics of the manipulator), the
motion planning problem is solved completely in polynomial time.
Furthermore since we consider only a point end-effector, the
direct kinematics is straightforward while the inverse kinematics
is more complex. Thus, while for most parallel mechanisms using
\cspace \ as a mean for path planning should be carefully
considered, here our approach is natural.

Note that this paper is aimed at possibly more complex
macromolecules with non-covalent bonds (not just manipulators) for
which the number of legs and links in each leg may be very large
\cite{JRKT01}. So such a polynomial-complex algorithm would play a
key role in several issues in structural biology, such as
structure prediction in protein folding and binding, and study of
protein mobility in folded states. In these applications, we are
interested in the robot motion planning problem without
considering the control and sensor issues.

The main contributions of this paper are two folds. First, we
establish the global connectivity of the \cspace \ of star-shaped
manipulators via a combination of the cell decomposition of
workspace and the structure of the \cspace \ at points in all
cells. Although the results seem obvious, the proof detail is
highly non-trivial. Second, the global connectivity result of
\cspace \ only suggests an exponential algorithm for path
existence, which makes the motion planning formidable for
macromolecules with very large number of links. In this paper we
propose novel techniques for path existence that avoids the
exponential complexity.
%
\begin{figure}[h]
  \centering
  \includegraphics[width=3.5in]{fig_2/IK.eps}
  \caption{Inverse kinematics of a three-link serial chain.}
 \label{IK}
\end{figure}
%
This paper is organized as follows. In Section \ref{section-1},
kinematics and singularities of the manipulator are analyzed. In
Section \ref{section-2}, necessary and sufficient conditions for
\cspace \ connectivity and path existence are derived, based on
which a complete polynomial-time algorithm is developed in Section
\ref{section-3}. Section \ref{section-4} addresses path
optimization and robustness issues. Section \ref{section-5} shows
simulation results that tests the effectiveness of our algorithm.
Finally \ref{section-6} ends this paper with a brief conclusion.

%\sect{Notation}
%\begin{center}
%\begin{tabular}{rl}
%\hline \hline \multicolumn{2}{c}{Manipulator Notation} \\ \hline
%${M}$ & - Manipulator \\
%$A$ & - Root junction or thorax of ${M}$ \\
%$o_i$ & - Grounding point of foot $i$ of ${M}$ \\
%${M}_j$ & - Leg $j$ of ${M}$ with foot fixed at $o_j$ \\
%               & and other end free, $j=1,...,k$ \\
%$n_j$ & - Number of links in ${M}_j$ \\
%$l_{j,i}$ & - Length of link $i$ of ${M}_j$; $i=1,...,n_j$ \\
%$\theta_{j,i}$ & - Angle of link $i$ relative to link $i-1$ \\
%$\tilde {M}_j(p)$ & - Leg $j$ of ${M}$ with foot fixed at $o_j$\\
%               & and other end fixed at $p$ \\
%$\tilde {M}(p)$ & - Manipulator with $A$ fixed at $p$\\
%$L_j$ & - Sum of lengths of links of $\tilde {M}_j(p)$ \\
%$L_{j,0}$ & - Sum of lengths of links of $M_j$\\
%${\cal L}_j(p)$ & - A set of long links of $\tilde {M}_j(p)$ \\
%$|{\cal L}_j^*(p)|$ & - Number of long links of $\tilde {M}_j(p)$ \\
%               \hline \hline
%  \multicolumn{2}{c}{Workspace Notation} \\ \hline
%$W_A$ & - Workspace of $A$ \\
%$^dU_i$ & - Cell of dimension $d$ of $W_A$ \\
%$p$ & - Point in the plane of ${M}$ \\
%$\gamma = p(t)$ & - Curve in the plane of ${M}$ \\
%$f$ & - Kinematic map of $A$ \\
%$f_j$ & - Kinematic map of endpoint of $M_j$ \\
%$\Sigma$ & - Critical set of $f$ in $W_A$  \\
%$\Sigma_j$ & - Critical set of $f_j$ \\
%                 \hline \hline
%  \multicolumn{2}{c}{Configuration Space (\cspace) Notation} \\
%\hline
%${\cal C}$ & - \cspace \ of ${M}$ \\
%$\tilde {\cal C}(p)$ & - \cspace \ of $\tilde {M}(p)$ \\
%${\cal C}_j$ & - \cspace \ of ${M}_j$ \\
%$\tilde {\cal C}_j(p)$ & - \cspace \ of $\tilde {M}_j(p)$ \\
%$c$ & - Point in \cspace \\ \hline \hline
%\end{tabular}
%\end{center}

\sect{Preliminaries} \label{section-1}
\newnot{2-1}
A star-shaped manipulator is composed of $k$ serial chains with
all revolute joints (see Fig.~\ref{X-shape}).  Leg $M_j$ is
composed of $n_j$ links of lengths $l_{j,i}, i=1,...,n_j$
%jct 1/16/06 - small change
and joint angles $\theta_{j,i}, i=1,...,n_j$. At one end (the
foot), $M_j$ is connected to ground by a revolute joint fixed at
the point $o_j$. At the other end, it is connected by another
revolute joint to a junction point denoted by $A$. Note that when
$k$ is one, a star-shaped manipulator is an open serial chain.
When $k$ is two, it is a single-loop closed chain.

Assuming that the foot of $M_j$ is fixed at $o_j$, let
$f_j(\Theta_j)=p$ denote the kinematic map of $M_j$, where
$\Theta_j = (\theta_{j,1}, \cdots, \theta_{j,n_j})$ is the tuple
of joint angles, and $p$ is the location of the endpoint of the
leg (the thorax end). When $M_j$ is detached from the junction
$A$, the image of its joint space is the reachable set of
positions of the free end of the leg, called the workspace $W_j$.
%jct 1/16/06 - small change
In the absence of joint limits, the workspace $W_j$ is an annulus
if and only if there exists one link with length strictly greater
than the sum of all the other link lengths. Otherwise it is a
disk. Clearly, the workspace $W_A$ of $A$ when all the legs are
connected to $A$ is given by:
\begin{equation}
\label{eq:defW}
   W_A = \bigcap_{j=1}^k W_j.
\end{equation}
\newnot{2-4}

In our study of ${\cal C}$, it will be convenient to refer to
several other \cspace s. The \cspace \ of leg $M_j$ when detached
from the rest of the manipulator will be denoted by ${\cal C}_j$.
When the endpoint is fixed at the point $p$, leg $j$ will be
denoted by $\tilde M_j(p)$, where the tilde is used to emphasize
the fact that the endpoint has been fixed. Note that $\tilde
M_j(p)$ is a single-loop planar closed chain, about which much is
known (see \cite{MT2}), including global structural properties of
its \cspace, denoted by $\tilde {\cal C}_j(p) = f_j^{-1}(p)$. When
the junction $A$ of a star-shaped manipulator is fixed at point
$p$, its \cspace \ will be denoted by $\tilde {\cal C}(p)$. Since
collisions are ignored, the motions of the legs are independent,
and therefore the \cspace \ of the manipulator (with fixed
junction) is the product of the \cspace s of the legs with all
endpoints fixed at $p$:
%
\newnot{2-2}
\begin{equation}
\label{eq:def_tildeCp}
   \left. \begin{array}{rcl}
   \tilde {\cal C}(p) & = & \tilde {\cal C}_1(p) \ \times
           \cdots \times \ \tilde {\cal C}_k(p) \\ [5pt]
          & = & f_1^{-1}(p) \ \times \cdots \times \ f_k^{-1}(p) \\ [5pt]
          & = & f^{-1}(p)
   \end{array} \right\}
\end{equation}
%
where by analogy, $f$ is a total kinematic map of the star-shaped
manipulator. Loosely speaking, the union of the \cspace s $\tilde
{\cal C}(p)$ at each point $p$ in $W_A$ gives the \cspace \ of a
star-shaped manipulator: \newnot{2-7}
%
\begin{equation}
\label{eq:def_C}
   {\cal C} = \bigcup_{p \in W_A} \tilde {\cal C}(p).
\end{equation}
%
Several properties of the \cspace s ${\cal C}_j$ and $\tilde {\cal
C}_j(p)$ are highly relevant and so are reviewed here before
analyzing the \cspace \ of star-shaped manipulators.  It is well
known that the \cspace \ of $M_j$ is a product of circles ({\em
i.e.,} ${\cal C}_j = (S^1)^{n_j}$)
%jct 1/16/06 - footnote added
\footnote{Recall the assumption of no joint limits.}. The
workspace $W_j$ contains a critical set $\Sigma_j$ which is
composed of all points $p$ in $W_j$ for which the Jacobian of the
kinematic map $Df_j(\Theta_j)$ drops rank for some $\Theta_j \in
f_j^{-1}(p)$. These points form concentric circles of radii
$|l_{j,1}\pm l_{j,2} \pm \cdots \pm l_{j,n_j}|$, as shown in
Fig~\ref{fig:single-leg}. When $A$ coincides with a point in
$\Sigma_j$, the links can be arranged such that they are all
colinear, in which case the number of instantaneous degrees of
freedom of the endpoint of the leg is reduced from two to one.
%
\begin{figure}
  \centering
  \includegraphics[width=3in]{fig_2/nir-paper-1b.eps}
  \caption{{\bf Left:} The workspace $W_j$ of a three-link open chain $M_j$
    based at $o_j$. The critical set $\Sigma_j$ of the kinematic map $f_j$ is four
    concentric circles.  The small circles, figure eights, and points at 12 o'clock
    show the topology of the \cspace \ $\tilde {\cal C}_j(p)$ of the
    leg when its endpoint is fixed at a point in one of the seven regions
    delineated by the critical circles (one of the four circles or one of
    the three open annular regions between them).  {\bf Right:} The inverse
    image of the curve $\gamma$ - a ``pair of pants."}
 \label{fig:single-leg}
\end{figure}\newnot{2-8}
%
Now consider the case where the endpoint of leg $j$ is fixed to
the point $p$.  In other words, we are interested in the \cspace \
$\tilde {\cal C}_j(p)$ of $\tilde M_j(p)$, which amounts to
calculating the inverse kinematics (IK) $f_j^{-1}(p)$. The
structure of $f_j^{-1}(p)$ has been established in
\cite{Burdick89} for four-link single-loop closed chains, and in
\cite{MT2,MT1} for chains with an arbitrary number of links.
\newnot{2-9}

Next, we compute $f_j^{-1}(p)$ for a three-link serial chain to
explain the basic idea used in \cite{MT2, MT1} for analyzing the
IK map of a closed chain with an arbitrary number of links. As
shown in Fig. \ref{IK}, we begin with the IK of a two-link serial
chain. Each point in the workspace could have 0,1,or 2 IK
solutions, and the set of points with constant number of IK
solutions forms annular regions separated by the critical set of
this chain. The IK of a three-link serial chain is then deduced by
breaking the chain into a two-link serial chain and a one-link
serial chain, and taking the union of the IK solutions of the
two-link chain for all points in the workspace of the one-link
chain. It is easy to check that for points in the outer most
annular region in the workspace of the three-link chain, the
workspace of the one-link chain (a circle) always intersects with
the outer-most critical circle of the two-link chain at two
points, indicating that the IK of this point is the union of two
curve segments with a pair of endpoints identified, respectively,
i.e., a circle. The inverse kinematics for points in other regions
can be derived similarly. The results are shown in
Fig.~\ref{fig:single-leg}. In the 12 o'clock position, points,
circles, and figure eights are drawn to represent the global
structures of $\tilde {\cal C}_j(p)$ in the seven regions of
$W_j$. Specifically, when $A$ is fixed to a point $p$ on the
outer-most critical circle, $\tilde {\cal C}_j(p)$ is a single
point. For $p$ fixed to any point in the largest open annular
region, \cspace \ is a single circle. Continuing inward, the
possible \cspace \ types are a figure eight (on the second largest
critical circle), two disconnected circles, a figure eight again,
a single circle, and a single point (on the inner-most critical
circle).

First, the connectivity of $\tilde {\cal C}_j(p)$ is uniquely
determined by the number of ``long links."  Consider the augmented
link set composed of the links of $M_j$ and $\overline {o_jp}$,
which will be called the fixed base link with length denoted by
$l_{j,0}$. Let $L_j$ be the sum of all the link lengths including
the fixed base link ({\em i.e.,} $L_j = \sum_{i=0}^{n_j}
l_{j,i}$). Further, let ${\cal L}_j(p)$ be a subset of
$\{0,1,...,n_j\}$ such that $l_{j,\alpha}+l_{j,\beta} > L_j / 2; \
\alpha,\beta \in {\cal L}_j(p), \ \alpha \neq \beta$. Over all
such sets, let ${\cal L}_j^*(p)$ be a set of maximal cardinality.
Then the number of long links of $\tilde M_j(p)$ is defined as
$|{\cal L}_j^*(p)|$, where $| \cdot |$ denotes set cardinality.
\newnot{2-3}

\medskip
\begin{Lemma} {\bf Kapovich and Milson \cite{KM}, Trinkle and
Milgram \cite{MT2}}\\
\label{lem-02} \rm The \cspace \ $\tilde {\cal C}_j(p) =
f_j^{-1}(p)$ has two components if and only if $|{\cal
L}_j^*(p)|=3$, and is connected if and only if $|{\cal
L}_j^*(p)|=2$ or $0$. No other cardinality is possible.
\end{Lemma}

\medskip

Let us return to the discussion of Fig.~\ref{fig:single-leg}.
Viewing $W_j$ as a base manifold and the \cspace \ corresponding
to each end point location as a fibre, it is apparent that the
critical set $\Sigma_j$ partitions $W_j$ into regions over which
the \cspace s $\tilde {\cal C}_j(p)$ form a trivial fibration. The
implications of this observation are useful in determining the
\cspace \ of more complicated mechanisms.  Consider a modification
to $\tilde M_j(p)$ that allows the endpoint to move along a
one-dimensional curve segment $\gamma$ within $W_j$.  Then as long
as $\gamma$ is entirely contained in one of the regions defined by
the critical circles, $\tilde {\cal C}_j(\gamma) = \tilde {\cal
C}_j(p) \times I$, where $I$ is the interval.  If $\gamma$ crosses
a critical circle transversally, then $\tilde {\cal C}_j(\gamma) =
(\tilde {\cal C}_j(p_1) \times I) \bigcup \tilde {\cal C}_j(p_3)
\bigcup (\tilde {\cal C}_j(p_2) \times I)$, where $p_1$ is a point
in one of the two open annular regions containing $\gamma$, $p_2$
is a point in the other, and $p_3$ is a point on the critical
circle crossed by $\gamma$, and $\bigcup$ denotes the standard
``gluing" operation. In Fig.~\ref{fig:single-leg}, an example
$\gamma$ and the corresponding \cspace \ $\tilde {\cal
C}_j(\gamma)$ are shown.

\sect{Analysis of Star-Shaped Manipulators} \label{section-2} For
star-shaped manipulators with one or two legs, the global
topological properties of the \cspace \ ${\cal C}$ are fully
understood (for one, see \cite{Lat92}; for two, see
\cite{MT2,MT1}). The goals of this section are to study the global
properties of ${\cal C}$ when $M$ has more than two legs and to
derive necessary and sufficient conditions for solution existence
to the motion planning problem.

\subsubsection{Local Analysis}
As a direct generalization of the critical set of a single leg, we
define the critical set of a star-shaped manipulator as a subset
$\Sigma$ of $W_A$ such that for every $p \in \Sigma$, there exists
a configuration $c$ such that at least one of the Jacobians
$\{Df_1(c),\cdots,Df_k(c)\}$ drops rank.  By definition we have:
%
\begin{eqnarray}
\label{eqn-01}
 \Sigma=\left(\bigcup_{i=1}^k \Sigma_i \right)\bigcap W_A.
\end{eqnarray}
%
An advantage of this definition is that $\Sigma$ can be used to
stratify $W_A$ such that each stratum is trivially fibred.
Figure~\ref{fig:double-leg} shows a star-shaped manipulator with
two legs.  The critical set $\Sigma$ is the boundary of the lune
formed by the intersection of the outer critical circles of their
individual workspaces.  For every point interior to the lune, the
fibre is two circles (the direct product of two points with one
circle). The fibres associated to the vertices of the lune are
single points, which correspond to simultaneous full extension of
the two legs.
%
\begin{figure}
  \centering
  \includegraphics[width=3in]{fig_2/nir-paper-2a.eps}
 \caption{The workspace $W_A$ of $A$ for a star-shaped manipulator with $k=2$
    is the intersection of the workspaces of $A$ for each leg considered
    separately.  The critical set $\Sigma$ is composed of the black
    circular arcs where they bound or intersect the gray area.
    }
 \label{fig:double-leg}
\end{figure}
%
Fig.~\ref{fig:chambers} shows a possible workspace for a
star-shaped manipulator with three legs. The critical set defines
65 distinct sets $^dU_i$ of varying dimension $d$, where $i$ is an
arbitrarily assigned index that simply counts components. We will
refer to these sets as {\sl chambers}. There are 12
two-dimensional, 32 one-dimensional, and 21 zero-dimensional
chambers, each of which is trivially fibred. Removing the
$^0\!U_i$ from $\Sigma$ partitions it into open one-dimensional
chambers $^1\!U_i$, $i=1,\cdots,^1\!m$. Removing $^0\!U_i$ and
$^1\!U_i$ from $W_A$ yields open two-dimensional sets $^2\!U_i, \
i=1,\cdots,^2\!m$, for which the following relationships hold:
%
\newnot{2-5}

\begin{eqnarray}
\Sigma & = & \left(\bigcup_{i=1}^{^0\!m} {}^0\!U_i \right) \bigcup
\left(\bigcup_{i=1}^{^1\!m} {}^1\!U_i \right) \\[5pt]
W_A - \Sigma & = & \bigcup_{i=1}^{^2\!m} {}^2\!U_i.
\end{eqnarray}
\begin{figure}
 \vbox{
      \centering {\epsfysize=2.3in \epsffile{fig_2/nir-paper-3a.eps}}
      }
 \caption{Workspace (shaded gray) of a star-shaped manipulator with
 three legs. The critical set partitions $W_A$ into $12$ two-dimensional,
 32 one-dimensional, and 21 zero-dimensional chambers.}
 \label{fig:chambers}
\end{figure}

\smallskip

\begin{Proposition}
\label{prop-01} \rm For all $d=0,1,2$ and $i$, $f^{-1}(^d\!U_i) \
= \ ^d\!U_i \times f^{-1}(p)$, where $p$ is any point in $^d\!U_i$
and the operator $\times$ denotes the direct product. Gluing the
$f^{-1}(^d\!U_i)$ for all $i$ and $d$ gives the total \cspace \
${\cal C}$.
\end{Proposition}

\medskip

{\bf Proof:} When $d=0$, ${}^0\!U_i$ contains a single point, the
result follows. When $d=1$, ${}^1\!U_i$ belongs to one critical
circle of one leg, say $\tilde M_j$. Any two points $p_1,p_2 \in
{}^1\!U_i$ are related by a Euclidean rotation $p_2-o_j =
R(p_1-o_j)$, indicating that $\tilde {\cal C}_j(p_1)$ and $\tilde
{\cal C}_j(p_2)$ are  homotopic. %homeomorphic.
Thus $\tilde {\cal C}_j(p)$ for all $p \in {}^1\!U_i$ have
equivalent topological structure. For the other legs $\tilde M_l$,
$l \neq j$, according to \cite{MT1} (Lemma $6.1$ and
Corollary~$6.5$) $\tilde {\cal C}_l(p)$ for all $p \in {}^1\!U_i$
have equivalent topological structures as ${}^1\!U_i$ is free of
critical points of $\tilde M_l(p)$. Thus $f^{-1}(p) = \tilde {\cal
C}_1(p)\times \cdots \times \tilde {\cal C}_k(p)$ for all $p \in
{}^1\!U_i$ have equivalent topological structures. The case when
$d=2$ can be proved by applying Lemma $6.1$ and Corollary $6.5$ of
\cite{MT1} to all legs. \hfill$\blacksquare$

\medskip

Proposition \ref{prop-01} and the fact that ${}^d\!U_i$ is a
simply connected set, reveal that each component of
$f^{-1}({}^d\!U_i)$ is a direct product of one component of
${\tilde {\cal C}}_j(p)$, $j=1,\cdots,k$, with a $d$-dimensional
disk. Using $|{\cal L}_j^*(p)|$, $j=1,\cdots,k$ and Lemma
\ref{lem-02}, one can show that the number of components of
$f^{-1}({}^d\!U_i)$ is $2^{k_0}$, where $k_0 \leq k$ is the number
of legs for which $|{\cal L}_j^*(p)| = 3$.

\smallskip

\subsubsection{Local Path Existence}
Before considering the global path existence problem, consider
motion planning between two valid configurations $c_{\rmtt[init]}$
and $c_{\rmtt[goal]}$ for which the junction $A$ lies in the same
chamber. Since the fibre over every point in ${}^d\!U_i$ is
equivalent, path existence amounts to checking the component
memberships of the configurations $c_{\rmtt[init]}$ and
$c_{\rmtt[goal]}$.

For a single leg $\tilde{M}_j(p)$, if the number of long links
$|{\cal L}_j^*(p)|$ is not three, then any two configurations of
$\tilde{M}_j(p)$ are in the same component. When $|{\cal
L}_j^*(p)|=3$, choose any two long links and test the sign of the
angle between them (with full extension taken as zero).  There are
two possible signs, one corresponding to {\sl elbow-up} and the
other to {\sl elbow-down}. If for two distinct configurations of
$\tilde{M}_j$, $A$ lies in the same chamber, there is a continuous
motion between them while keeping $A$ in this chamber, if and only
if the elbow sign is the same at both configurations (naturally,
one must perform the sign test with the same two links and in the
same order for both configurations). Considering all the legs
together, a continuous motion of $A$ in ${}^d\!U_i$ exists if and
only if a motion exists for each leg individually. The previous
discussion serves to prove the following result.

\medskip

\begin{Proposition}
\label{prop-2} \rm Restricted to $f^{-1}({}^d\!U_i)$, two
configurations $c_1,c_2 \in f^{-1}({}^d\!U_i)$ are path connected
if and only if for each leg $\tilde M_j$ with $|{\cal L}_j^*|=3$
in ${}^d\!U_i$, the elbow angle of $\tilde M_j$ has the same sign
at $c_1$ and $c_2$.
\end{Proposition}

\medskip

Proposition \ref{prop-2} completely solves the path existence
problem if $W_A$ consists of a single chamber. However, things
become complex when $W_A$ has more than one chamber.

\subsubsection{Singular Set and Global \cspace \ Analysis}

Recall that the \cspace \ ${\cal C}$ is a union of
$f^{-1}({}^d\!U_i)$, $d \in \{0,1,2\}$, $i=1,\cdots,{}^d\!m$ and
that $f^{-1}(p)$, $p\in {}^d\!U_i$ for $d\neq 2$ and all $i$ is a
set containing at least a singularity of $f$. Combining the local
\cspace \ and singular set analysis yields the global structure of
\cspace.

\medskip

\begin{Proposition}
\label{prop-sing} \rm For all $p \in \Sigma_j$, $f_j^{-1}(p)$ is a
singular set containing isolated singularities. If a singularity
separates its neighborhood $V$ in $f_j^{-1}(p)$, then it is these
singularities which glue the two separated components in
$f_j^{-1}(q)$ where $q \in W_A-\Sigma_j$ is a point sufficiently
close to $p$.
\end{Proposition}

\medskip

{\bf Proof:} First it is obvious that $f_j^{-1}(p)$ contains
isolated singularities for there are finite ways to colinearize
all the links of a close chain. Second, let
\[
   \gamma: (-\varepsilon,\varepsilon) \rightarrow W_A,\, \gamma(0)=p
\]
\newnot{2-6}
be a curve that is transverse to $\Sigma_j$. According to
Corollary 6.6 of \cite{MT1}, the distance function
$s(\gamma(t))=\int_{0}^t |{\dot \gamma}|dt$ defines a Morse
function on $f_j^{-1}(\gamma)$
\[
   s \circ f_j: f_j^{-1}(\gamma) \rightarrow \reals.
\]
Note that $0$ is a singular value of $s \circ f_j$ and the
isolated singularities of $f_j^{-1}(p)$ are also singularities of
$s \circ f_j$. The result of Morse theory applying to $s \circ
f_j$ yields that $(s \circ f_j)^{-1}(0)=f_j^{-1}(p)$ is given by
attaching a handle to $(s \circ
f_j)^{-1}(\varepsilon_0)=f_j^{-1}(q)$ for a sufficiently small
$\varepsilon_0$ and $q$ a point sufficiently close to $p$. The
Proposition follows.
  \hfill$\blacksquare$\medskip

%jct 1/16/06 - changed all {\cal J} to J based on reviewer question - 4 places.
Next, we establish necessary and sufficient conditions for the
connectivity of ${\cal C}$.  Let ${J}$ be the index set such that
for all $j \in {J}$, $|{\cal L}_j^*|=3$ for at least one chamber
${}^d\!U_i$. We prove the following theorem.

\medskip

\begin{Theorem}
\label{them-1} \rm Suppose $W_A = \bigcup_{d=0}^2 \left(
\bigcup_{i=1}^{{}^d\!m} {}^d\!U_i \right)$.  Then ${\cal C} =
f^{-1}(W_A)$ is connected if and only if:
\begin{enumerate}
\item
    $W_A$ is connected;
\item
    $\Sigma_j \bigcap W_A \ne \emptyset$ for all $j \in {J}$.
\end{enumerate}
\end{Theorem}

\medskip

{\bf Proof:} (i) ``Necessity:" Since ${\cal C}$ is a fibration of
the base manifold $W_A$, it can have one component only when $W_A$
has one component.  Thus item $1$ of Theorem~\ref{them-1} is
required.
%
Second, in order that ${\cal C}$ be connected, for each leg $M_j$
restricted to $W_A$, the \cspace \ ${\tilde {\cal C}}_j(W_A) =
f_j^{-1}(W_A)$ must be connected. By definition, for all $j \in
{J}$, there exists a chamber ${}^d\!U_i$ such that $|{\cal
L}_j^*|=3$. The result of Proposition \ref{prop-sing} means that
${\tilde {\cal C}}_j(W_A)$ is connected only if $W_A \bigcap
\Sigma_j \ne \emptyset$.

(ii) ``Sufficiency:" Item 1 and 2 imply that ${\tilde {\cal
C}}_j(W_A)$ are path connected for all $j$. Moreover, ${\cal C}$
is a fibration over $W_A$. The result follows.
 \hfill$\blacksquare$\medskip

Fig.~\ref{fibre} illustrates the global connectivity for an
example $W_A$ corresponding to a star-shaped manipulator with two
legs and a workspace for which there are two chambers $^2U_1$ and
$^2U_3$ where leg~1 has three long links and another chamber
$^2U_4$ where both legs have three long links.  Among these
chambers, ${}^1\!U_1$ and ${}^1\!U_2$ belong to $\Sigma_1$, and
${}^1\!U_3$ belongs to $\Sigma_2$. According to
Theorem~\ref{them-1}, the \cspace \ is path connected.  In this
example, the ${\cal C}$ is the product of the two structures
shown.
%
\begin{figure}
 \vbox{
      \centering {\epsfysize=3.2in \epsffile{fig_2/fibre.eps}}
      }
 \caption{\cspace \ of a star-shape manipulator with two legs.
 For simplicity, only the portion of $f^{-1}(\gamma)$
 is shown, where $\gamma$ is a continuous curve in $W_A$
 that visits all chambers. }
 \label{fibre}
\end{figure}

\smallskip

\begin{Corollary}
\label{cor-1} \rm Two configurations $c_1$ and $c_2$ of a
star-shaped manipulator are in the same component if and only if
\begin{enumerate}
\item $f(c_1)$ and $f(c_2)$ are in the same component of $W_A$;
\item For each leg $j$ with $|{\cal L}_j^*|=3$ for all chambers
${}^d\!U_i$ in the component of $W_A$ which contains $f(c_1)$ and
$f(c_2)$, the elbow sign is same at both $c_1$ and $c_2$.
\end{enumerate}
\end{Corollary}

\medskip

\begin{Remark}
\rm As a matter of fact, $\Sigma$ completely determines the
connectivity of \cspace. When computing a path between two given
configurations, often motions of the junction to points on
$\Sigma$ are incorporated to allow adjustment of leg-angle signs.
However, inevitable deviations of the junction from $\Sigma$
caused by numerical errors, make it impossible to adjust the sign
of legs while fixing its end point. For these reasons, points in
2D chambers are preferred for sign adjustment.
\end{Remark}

\sect{A Polynomial-Time, Exact, Complete Algorithm}
\label{section-3} Our algorithm consists of two main routines,
{\tt PathExists} and {\tt ConstructPath}. {\tt PathExists} solves
the path existence problem, i.e., determining if an initial and an
goal configuration are path-connected, and {\tt ConstructPath}
constructs a path between them if there exists a path.

Notice that the \cspace \ of a star-shaped manipulator could be
very complex. Even the simple planar five-link single-loop closed
chain, its \cspace \ could be as complex as the connected sum of
four torii \cite{MT1}. So determining the path existence without
using the \cspace \ information is difficult. Our strategy is to
solve the path existence based on the set of critical circles
$\Sigma_j$ in the workspace, and then construct the path combining
our knowledge of the workspace and the structure of the \cspace \
of single-loop closed chains. We emphasize here that the problem
is not just moving the junction point between an initial and a
goal position, but moving the manipulator along with all its legs
from an initial configuration to a goal configuration. So, the
workspace information will be insufficient for path construction.
In {\tt ConstructPath}, we employ a move that changes the shape of
a leg with its endpoint fixed in the workspace. This move, called
the sign-adjust move, uses the knowledge of the \cspace \ of a
single-loop closed chain. Below we will show that the overall
complexity of {\tt PathExists} and {\tt ConstructPath} is
$O(k^3N^3)$, where $N$ is the maximum number of links in a leg and
$k$ is the number of legs. The polynomial complexity is key to the
applications like folding of macromolecules, which can can be
modeled as a closed chain with large $k$ and $N$.


The logical flow of {\tt PathExists} is illustrated in
Figure~\ref{PathExists}. Its input is the topology and link
lengths of a star-shaped manipulator and two valid configurations,
$c_{\rmtt[init]}$ and $c_{\rmtt[goal]}$.  The output is the answer
to the path existence question.
%
\begin{figure}
  \centering
  \includegraphics[width=2.8in]{fig_2/star-shaped-flow-chart6_1.eps}
  \caption{Logical flow and complexity of the major steps
  of {\tt PathExists}.}
  \label{PathExists}
\end{figure}
%
The approach taken is to compute $W_A$ and then, for each leg with
its end point constrained to lie in $W_A$, to determine if its
initial and goal configurations are path connected. Notice that
$W_j$ is either a disk or an annular region,  $W_A$ can be
constructed by calculating the intersections between no more than
$2n$ circles. So constructing $W_A$ is polynomial-complex.  The
most difficult issue is to check the path existence. Since the
\cspace \ of a leg is guaranteed to be connected if one of its
critical circles $\Sigma_j$ intersects $W_A$, the most straight
forward way to test connectivity is to explicitly perform the
intersections. However, since there are as many as $2^{n_j-1}$
critical circles, any algorithm based on this approach will have
worst-case complexity that is at least exponential in $N$. The key
contribution of {\tt PathExists} is a polynomial-time algorithm
for checking the existence of an intersection between $W_A$ and a
critical circles - even though there is an exponential number of
these circles.  Recall that if a leg has three long links, then it
is impossible to move the leg so that the three long links change
from an elbow-up configuration to an elbow-down configuration. The
following algorithm constructs a novel polynomial-complex method
that can determine if there is point in $W_A$ in which the number
of long links of a leg is not $3$ (see Step 4 in {\tt
PathExists}).

\medskip

\noindent \framebox{1. Construct $W_A$}
We compute $W_A$ in three steps.\\
Step 1: Compute the boundary circles of $W_j$. In general, $W_j$
is an
%jct 1/16/06 - typo
annulus. The radius of its outer boundary circle is
$r_{\rmtt[max]}=\sum_{i=1}^{n_j} l_{j,i}$, while that of its inner
boundary circle, $r_{\rmtt[min]}$, can be determined by comparing
$l_{\rmtt[max]}:=\max_i l_{j,i}$ and
$r_{\rmtt[max]}-l_{\rmtt[max]}$.

If $l_{\rmtt[max]}>r_{\rmtt[max]}-l_{\rmtt[max]}$, then
$r_{\rmtt[min]}=2l_{\rmtt[max]}-r_{\rmtt[max]}$, else,
$r_{\rmtt[min]}=0$;\\
Step 2: Decompose the whole plane into cells using all boundary
circles of all legs (e.g., the line sweeping algorithm can do
this), and construct the cell adjacency graph;\\
Step 3: Pick a point from the interior of each cell, compute its
distance from each base point, and compare the distance with the
radii of the two boundary circles of $W_j$. The set of cells which
can be reached by all legs constitute $W_A$.\\
The complexity of this 2-D cell decomposition algorithm is
$O(k^2+kN)$.
\medskip \\
\framebox{2. Are $p_{\rmtt[init]}$ and $p_{\rmtt[goal]}$ in same
component of $W_A$?} As an immediate consequence of the cell
decomposition, this can be answered directly by searching the cell
graph.
\medskip  \\
\framebox{3. Compute $J$} This step is used to filter out easy
solution existence checks, based on the cardinality and members of
the sets ${\cal L}_j^*(p_{\rmtt[init]})$ and ${\cal
L}_j^*(p_{\rmtt[goal]})$. For each leg $\tilde
M_j(p_{\rmtt[init]})$, compute $L_j$ (see Section~III) and find
the three longest links of the set $\{l_{j,0},...,l_{j_{n_j}} \}$.
Denote these links by $(p_{\rmtt[init]};
\lambda_{j,1},\lambda_{j,2},\lambda_{j,3})$. Do the same for
$(p_{\rmtt[goal]})$ and define $(p_{\rmtt[goal]};
\lambda_{j,1},\lambda_{j,2},\lambda_{j,3})$. This requires $O(N)$
work. Finally, $|{\cal L}_j^*(p_{(\cdot)})| = 3$ if and only if
$\lambda_{j,2} + \lambda_{j,3} > L_j/2$.  If ${\cal
L}_j^*(p_{\rmtt[init]}) = {\cal L}_j^*(p_{\rmtt[goal]})$ and
$|{\cal L}_j^*(p_{\rmtt[init]})| = 3$, and if the signs of the
long links are different at $c_{\rmtt[init]}$ and
$c_{\rmtt[goal]}$, then add $j$ into $J$.
Computing $J$ is $O(kN)$. \medskip  \\

\framebox{4. Does the set of long links vary for all $j \in J$?}
If and only if $q \in W_A$ exists such that ${\cal L}_j^*(q) \neq
{\cal L}_j^*(p_{\rmtt[init]})$, then it is possible to make the
long links colinear and thus change the signs of their relative
angles. This can be done by computing a point $q \in W_A$ on the
boundary of the cell that contains $p_{\rmtt[goal]}$ and keeps the
same set ${\cal L}_j^*(p)$ for all $p$ in this cell. This boundary
is characterized by $\lambda_{j,2}+\lambda_{j,3}=L_j/2$. Since
$l_{j,0}$ is the only link whose length varies along with $p$,
this boundary must be one or two circles (called inner and outer
circles, respectively) whose radii, denoted $d_{\rmtt[max]}$ and
$d_{\rmtt[min]}$, depend on the link lengths of the leg. Let
$L_{j,0}=\sum_{i=1}^{n_j} l_{j,i}$ and suppose the four longest
links at $p_{\rmtt[goal]}$ are
$(\lambda_{j,1}>\lambda_{j,2}>\lambda_{j,3}>\lambda_{j,4})$ with
$\lambda_{j,2}+\lambda_{j,3}>L_j/2$, we deduce the radii of the
boundary circles for
four different cases:\\
Case 1: if $l_{j,0}(p_{\rmtt[goal]})=\lambda_{j,1}$, then
$d_{\rmtt[max]}= 2(\lambda_{j,2}+\lambda_{j,3})-L_{j,0}$, and
$d_{\rmtt[min]}=\max \{L_{j,0}-2\lambda_{j,3},
2(\lambda_{j,3}+\lambda_{j,4})-L_{j,0}\}$.\\
Case 2: if $l_{j,0}(p_{\rmtt[goal]})=\lambda_{j,2}$,
$d_{\rmtt[max]}=2(\lambda_{j,1}+\lambda_{j,3})-L_{j,0}$, and
$d_{\rmtt[min]}=\max\{L_{j,0}-2\lambda_{j,3},
2(\lambda_{j,3}+\lambda_{j,4})-L_{j,0}\}$.\\
Case 3: if $l_{j,0}(p_{\rmtt[goal]})=\lambda_{j,3}$,
$d_{\rmtt[max]}=2(\lambda_{j,1}+\lambda_{j,2})-L_{j,0}$, and
$d_{\rmtt[min]}=\max\{L_{j,0}-2\lambda_{j,2},2(\lambda_{j,2}+\lambda_{j,4})-L_{j,0}\}
$. \\
Case 4: Otherwise,
$d_{\rmtt[max]}=\min\{2(\lambda_{j,2}+\lambda_{j,3})-L_{j,0},L_{j,0}-2\lambda_{j,2}\}$,
and $d_{\rmtt[min]}=0$.\\
If there is no overlap between the two boundary circles and the
component of $W_A$ that contains $p_{\rmtt[init]}$ and
$p_{\rmtt[goal]}$, then no path exists between $c_{\rmtt[init]}$
and $c_{\rmtt[goal]}$. Otherwise, path exists and we obtain way
points $p_j$ for all legs $j\in J$. Computing $d_{\rmtt[max]}$,
$d_{\rmtt[min]}$, and the way points $p_j$ is $O(kN)$.

\medskip

The basic idea of {\tt ConstructPath} is that when moving from
$c_{\rmtt[init]}$ to $c_{\rmtt[goal]}$, those legs $j \in J$ may
require a change in the signs of relative angles between long
links, which is always possible at the way point $p_j$ or other
critical points of the corresponding leg. A natural approach then
is to use two motion generation primitives: {\sl accordion move}
and {\sl sign-adjust move}.  The former moves the thorax endpoint
(at $A$) along a specified path segment with all legs moving
compliantly so that all loop closures are maintained. The latter
keeps the endpoint fixed at a way point $q_j \in \Sigma_j$ (e.g.,
$q_j=p_j$ or other critical points) while moving leg $j$ into a
singular configuration and then to a nearby configuration with the
sign of the relative angle between a pair of long links in this
leg chosen to match those of $c_{\rmtt[goal]}$.

Note that though the guard points $q_j$ are in the critical set
$\Sigma_j$, the configuration at which leg $j$ approaches these
points need not to be aligned. Without considering the control
issue, a leg can be moved to a colinear configuration with the
thorax fixed. Even if control is considered for the sign-adjust
move, the thorax can still be maintained to be fixed by keeping
all other legs not aligned in the vicinity of $q_j$.

The input of {\tt ConstructPath} is $W_A$ and its cell graph,
$c_{\rmtt[init]}$, $c_{\rmtt[goal]}$, and the set of way points
$p_{j}\, \in W_A, \, j\in J$ computed during the execution of {\tt
PathExist}.

\noindent \framebox{1. Construct an initial path} {\tt
ConstructPath} explores the cell graph of $W_A$, and constructs a
path in $W_A$ connecting $p_{\rmtt[init]}$ to $p_{\rmtt[goal]}$
and visiting all of the way points. Since there are at most $k$
way points, this can be done in $O(k^3)$ time (the path has $k+1$
segments each with $O(k^2)$ arcs).

\noindent \framebox{2. Construct {\em guards} and insert the {\rm
guards} into the path} Notice that when one accordion moves a leg
in a cell in which the number of long links is not $3$ (called
one-component cell), neither the signs of concatenating angles,
nor the sign between any pair of links in this leg will be kept
invariant. Thus even if the desired sign between a pair of long
links is adjusted at a way point, it still could change if the leg
keeps moving in a one-component cell. For this reason, we set {\em
guards} for legs which have three long links at $p_{\rmtt[goal]}$.
These are the set of points $q_j$, each of which is the last
intersection point between the above constructed path in $W_A$ and
the boundary of the two-component cell of leg $j$ containing
$p_{\rmtt[goal]}$. Thus the number of guards ($q_j$'s) may be more
than the number of way points since the number of legs that have
three long links at $p_{\rmtt[goal]}$ may be more than the
cardinality of $J$. Next the {\em guards} are inserted into the
path. Later when we construct the path in ${\cal C}$, sign-adjust
moves are only performed at {\em guards} $q_j$ (but not $p_j$) for
after that the thorax endpoint gets into the two-component cell
and the sign between a pair of long links will not change during
accordion moves, i.e., the leg will always remain in the right
component of its \cspace.  Assuming each arc in the path is
approximated by a fixed number of line segments, finding guards is
$O(k^3)$.

\noindent \framebox{3. Accordion moves and sign-adjust moves} The
path in ${\cal C}$ then is produced by using accordion moves along
the path and sign-adjust moves at the {\em guards}. At each {\em
guard}, one checks the sign between a pair of long links of the
corresponding leg. If it does not match the goal one, then the
junction point is fixed while a sign-adjust move is executed,
otherwise, the accordion move continues. Once $A$ is coincident
with $p_{\rmtt[goal]}$, one is assured by the previous steps, that
with $A$ fixed at $p_{\rmtt[goal]}$, the configuration of each leg
is in the same component of its current \cspace \ $\tilde
C_j(p_{\rmtt[goal]})$ as $c_{\rmtt[goal]}$. The final move can be
accomplished using a special accordion move algorithm found in
\cite{MT1}. At this stage, we remark that finding the set of way
points $p_j$ and planning an initial path visiting all $p_j$ is
necessary for otherwise, an arbitrary path between
$p_{\rmtt[init]}$ and $p_{\rmtt[goal]}$ may not intersect the
boundary of the two-component cell of a leg that contains
$p_{\rmtt[goal]}$.

The complexity of the accordion move algorithms reported in
\cite{MT2} are $O(N^3)$. Since the path has $O(k^3)$ line segments
the complexity of {\tt ConstructPath} is $O(k^3N^3)$. Note that
accordion move algorithms with the required behavior can be
designed to be $O(N^2)$, so the complexity of {\tt ConstructPath}
could be reduced.

Overall, our path planning algorithm is  $O(k^3N^3)$.

\sect{Path Optimization and Robustness} \label{section-4} If a
path between two given configurations exists, it is obvious that
in our algorithm the choice of way points, and thus the path
between the two configurations, is not unique. So a natural
problem is path optimization with respect to meaningful metrics
such as path length and singularity avoidance. Basically we say
that many possible optimization objectives are potentially useful,
but we consider the shortest path in this section. Notice that the
way points are necessary for successively constructing a path from
$c_{\rmtt[init]}$ to $c_{\rmtt[goal]}$ since an arbitrary chosen
path of the thorax from $p_{\rmtt[init]}$ to $p_{\rmtt[goal]}$
(like the line connecting them) will either go out of $W_A$, or
have no intersection with the critical set $\Sigma_j$, in which a
sign-adjust move is required for a leg. Thus the optimization
problem arises in the choice of way points, the order of the way
points, and the path between two consecutive way points.

One may also take into consideration parallel singularities (cf.
\cite{SG95}) and try to avoid them as much as possible by
minimizing the number of singularities crossings. More precisely
this may be done for example by first clustering singularity
regions (see \cite{DCSY03}) and then choosing way points with the
corresponding path having minimum number of crossings of these
regions, finally their connecting path or trajectories (i.e. a
path with temporal relations as well as the geometrical ones)
should be chosen in the proper way (cf. \cite{NTU00}).

Since ${\cal C}$ is a fibration over $W_A$, a meaningful metric
for ${\cal C}$ is
\begin{eqnarray}
\label{rieman-metric}
  ds^2=a_1 dp^Tdp+ a_2\sum_{j=1^k} du_j^TV_j^TV_jdu_j,\, a_1, a_2>0,
\end{eqnarray}
where $a_1$ and $a_2$ are two weights assigned to $dp^Tdp$ and
$\sum_{j=1^k} du_j^TV_j^TV_jdu_j$, respectively for they are
quantities with different physical meaning. The column vectors of
$V_j \in \reals^{n_j \times (n_j-2)}$ forms a basis for the null
space of the Jacobian $J_j=\frac{\partial f_j}{\partial \Theta_j}$
of leg $j$, i.e.,
\[
    J_jV_j=0.
\]
$du_j$ denotes the incremental changes of the local coordinates on
$\tilde {\cal C}_j(p)$. $dp$ and $V_j du_j$ stands for the
infinitesimal motion along the base manifold and the fibre,
respectively. The shortest path problem is to find a path
$(p(t),\Theta_1(t),\cdots,\Theta_k(t))$ such that
\begin{eqnarray}
\label{obj}
   \int_{t=0}^1 ds
\end{eqnarray}
is minimal. The optimal solution to (\ref{obj}) satisfies the
geodesic equation
\begin{eqnarray}
\label{geodesic-eqn}
   {\ddot v}_k+\Gamma_{ij}^k {\dot v}_i {\dot v}_j=0
\end{eqnarray}
where $v=[p^T,u_1^T,\cdots,u_k^T]^T$, and $\Gamma_{ij}^k$ denotes
the Christoffel symbol of the metric (\ref{rieman-metric}).
Solving (\ref{geodesic-eqn}) exactly is difficult. However, an
approximation solution can be derived. Since a path from
$c_{\rmtt[init]}$ to $c_{\rmtt[goal]}$ is globally optimal if and
only if this path is also locally optimal, we construct an
approximate shortest path in a way so that (i) $p(t)$ connects
$p_{\rmtt[init]}$ and $p_{\rmtt[goal]}$ and visits all $p_i$.
Moreover, $\int_{t=0}^1 \sqrt{dp^Tdp}$ is minimal; (ii) There is
minimal number of accordion moves, and each accordion move is
minimal; (iii) Except for the accordion moves, there is no other
motions along the fibre.
\begin{Remark}
\rm The path constructed in this way is shortest if no accordion
moves are needed for we always achieve minimal
$ds=\sqrt{a_1dp^Tdp}$ infinitesimally, while it is only an
approximation if there is at least one accordion move.
\end{Remark}
Mathematically, this problem can be described as
%
\[
    \begin{array}{c}
     \min  \int_0^1 \|dp\|   \\
      p(t) \in W_A, \forall t \in [0,1] \\
      p_i \in \{p(t)\}, \forall i \\
      p_i \in {\cal A}_{\delta(i)}
    \end{array}
\]
%
where ${\cal A}_j, j\in J$ is the boundary arcs of the
two-component cell of leg $j$ that contains $p_{\rmtt[goal]}$.
$\delta(J)$ is a permutation of $J$ with $\delta(J) =J$. Solving
this problem exactly is extremely hard, but a random search method
(for example, the Controlled Random Search Method \cite{BS96}) can
be used to quickly find a good approximate solution. Using the
minimal number of accordion moves has been solved in {\tt
ConstructPath}, and the minimal accordion move problem has been
solved in \cite{MT1}. Combining these two, (ii) is solved.

To solve (iii), we notice that for a local motion $dp=[dx,dy]$ of
the thorax endpoint, $d\Theta_j^Td\Theta_j$ is minimized if and
only if.
%
\[
   d\Theta_j=J_j^+ dp
\]
%
where $J_j^+$=$J_j^T(J_jJ_j^T)^{-1}$.

Another important issue about our planning algorithm is
robustness. The sign-adjust move of leg $j$ performed at a guard
$q_j$ is only feasible when ${\cal C}_j(q_j)$ is connected. Since
$q_j \in \Sigma_j$ which is only 1-D, a small perturbation of the
junction point in $W_A$ (e.g., due to numerical errors) will
violate the condition $p \in \Sigma_j$. When $p$ moves into a
two-component cell, then the sign-adjust move may fail. A remedy
to this is to modify the path of the thorax in the neighborhood of
$q_j \in \Sigma_j$ so that a point $q_j'$ in the interior of a
one-component cell is reached. After the sign of leg $j$ is
adjusted to the desired one with $p$ fixing at $q_j'$, we apply a
constrained accordion move algorithm to ensure that the leg $j$
stays in the right component of ${\cal C}_j(p)$ just before its
thorax endpoint enters the two-component cell containing
$p_{\rmtt[goal]}$.  This resulting algorithm will also be robust
to other errors such as the control and sensor errors if they are
taken into account.


\sect{Examples} \label{section-5}
%jct 1/16/06 - efficiency -> complexity? correctness?
In this section, we demonstrate the correctness and complexity of
our algorithm through two examples: a manipulator with three
three-link legs, and a manipulator with three five-link legs.
Movies of the motion plans are very helpful in understanding the
figures.  They can be found at
\verb$http://www.cs.rpi.edu/~trink/ccwo.html$.
%
\begin{figure}
  \centering
  \includegraphics[width=3in]{fig_2/initial_goal_config.eps}
  \caption{Manipulator's initial configuration (junction on the right, drawn red)
  and goal configuration (junction just below the top left, drawn blue.) The boundary
  circles of $W_j$ are drawn as dashed green lines.}
  \label{start-goal}
\end{figure}
%
\begin{figure}
  \centering
  \includegraphics[width=3in]{fig_2/workspace_path_2.eps}
  \caption{A path between $p_{\rmtt[init]}$ and $p_{\rmtt[goal]}$ that is completely
  contained in $W_A$.}
  \label{workspace-path}
\end{figure}
%
\begin{figure}
  \centering
  \includegraphics[width=3in]{fig_2/guards.eps}
  \caption{The two guard points are the last intersection points between the
  path of $A$ and the boundary circles of two two-component cells.}
  \label{guard}
\end{figure}
%
\begin{figure}
  \centering
  \includegraphics[width=3.5in]{fig_2/first_interval_2.eps}
  \caption{All legs use an accordion move to move the junction $A$
  to the first guard $q_1$ of leg $1$.}
  \label{accordion-1}
\end{figure}
%
\begin{figure}
  \centering
  \includegraphics[width=3.5in]{fig_2/second_interval_2.eps}
  \caption{With the junction $A$ at $q_1$, the joint angles of leg $1$
can be adjusted to achieve the signs required at the goal
configuration. All other legs are fixed in place.}
  \label{adjust_sign-1}
\end{figure}
%
\begin{figure}
  \centering
  \includegraphics[width=3.5in]{fig_2/third_interval_2.eps}
  \caption{All legs use an accordion move to move the junction $A$ to
  the second guard $q_2$ of leg $2$.}
  \label{accordion-2}
\end{figure}
%
\begin{figure}
  \centering
  \includegraphics[width=3.5in]{fig_2/fourth_interval_2.eps}
  \caption{With the junction $A$ at $q_2$, the joint angles of leg $2$
can be adjusted to achieve the signs required at the goal
configuration. All other legs are fixed in place.}
  \label{adjust_sign-2}
\end{figure}
%
\begin{figure}
  \centering
  \includegraphics[width=3.5in]{fig_2/five_interval_2.eps}
  \caption{All legs use an accordion move to move the junction $A$ to its
  goal location.  The signs of the joint angles are preserved guaranteeing
  that legs~1 and~2 will be in the correct \cspace \ component once $A$ is fixed at
  the goal position.}
  \label{accordion_goal}
\end{figure}
%
\begin{figure}
  \centering
  \includegraphics[width=3.5in]{fig_2/six_interval_2.eps}
  \caption{All legs use the Trinkle-Milgram algorithm to achieve their
  goal configurations with the junction $A$ fixed.}
  \label{TMalg}
\end{figure}

In the first example, two of the three legs of the manipulator
have three long links when $A$ is fixed at $p_{\rmtt[goal]}$.
Figure~\ref{start-goal} shows the manipulator in its starting and
goal configurations. Our algorithm predicts $J=\emptyset$.

Then the algorithm constructs a path in $W_A$ from
$p_{\rmtt[init]}$ to $p_{\rmtt[goal]}$, drawn as the dark solid
lines in Fig. \ref{workspace-path}.  This path intersects

the boundary of the two-component annular region of leg $j$ that
contains

$p_{\rmtt[goal]}$ several times, among which $q_j$, $j=1,2$ are
the last ones. These two points are the guards (drawn as diamonds)
where sign-adjust moves are performed.

At $q_j$, $j=1,2$, we check the sign of a pair of long links of
leg~$j$ and see if it matches its sign at the goal. If not, we fix
the other two legs and adjust the sign of the chosen long links in
leg $j$. In this particular example, we chose the two longest
links as the pair of long links,and we find that at $q_1$ the sign
of leg $1$ does not match that at the goal (while at $q_2$, leg
$2$ has the same sign as the goal). Before leaving $q_j$ via the
next accordion move, the pair of long links of leg $j$ was moved
to the elbow-opposite configuration (recall that there are two
configurations for these two links, one is ``elbow up", the other
is ``elbow down"), which has exactly the same sign as the goal
configuration. The Trinkle-Milgram algorithm \cite{MT2} is used to
plan such a motion between the two elbow-opposite configurations.
Figures~\ref{accordion-1} \--\ \ref{TMalg} show the progress of
the manipulation plan as the steps of the complete planning
algorithm are carried out.

A bit more complex example in which the star-shaped manipulator
has three five-link legs is shown in
\verb$http://www.cs.rpi.edu/~trink/ccwo.html$. The computation
time for path existence for star-shaped manipulators with less
than 10 legs, and legs of less than 10 links is typically from
less than 1 second to a few seconds when run in a Matlab, P4,
WindowsXP system.



\sect{Discussion} Star-shaped manipulators are closed chain
manipulators subject to multiple loop closure constraints. The
\cspace \ of these manipulators is often a lower-dimensional
submanifold with high genus \footnote{The genus of a surface is
defined as the largest number of nonintersecting simple closed
curves that can be drawn on the surface without separating it.}
embedded in the ambient space. Computing the silhouette of this
manifold requires solving the extreme points of the manifold
either in the ambient space whose dimension is much higher than
that of the manifold itself, or in a set of local neighborhoods
(local coordinate charts) whose number grows exponentially along
with the genus of the submanifold. Although Canny's algorithm is
very efficient in general, there is difficulty in implementation
for star-shaped manipulators. Second, the classical cylindrical
decomposition of \cspace \ (e.g. collin's decomposition) is a
partition into simple connected subsets of \cspace \ called cells.
However, this algorithm requires a description of the \cspace \ in
terms of a set of polynomials over its ambient space. Again
because the dimension of the ambient space could be very high, the
computation time of this algorithm could become formidable.

Our algorithm employs the special structural properties (fibration
over the workspace) of the \cspace \ of star-shaped manipulators.
It avoids using the coordinates of the ambient space as well as
the local coordinate charts that covers the \cspace. In our
algorithm the path existence and path construction are solved in
polynomial time by combining the cell decomposition of the
workspace (which is two dimensional and with simple shape) and the
structure of the \cspace \ of single-loop closed chains. The
critical set $\Sigma_j$, which marks the change of the topology of
the \cspace \ of each leg, plays a key role in this algorithm.

\sect{Conclusion} \label{section-6} In this paper, we studied the
global structural properties of planar star-shaped manipulators.
Via the analysis of the critical set $\Sigma$, we derived the
global connectivity of the \cspace, and necessary and sufficient
conditions for path existence. Based on these results, we devised
a complete polynomial algorithm for motion planning. Simulation
examples were used to illustrate the key ideas behind the motion
planning problem of planar star-shaped manipulators.
