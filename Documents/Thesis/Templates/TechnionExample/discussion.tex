\chapter{Discussion}
\label{Disc}

Recently a growing interest has arise applying toplogy theory to
Robotics in general and to parallel robots configuration spaces in
particular. To this end, main focus had been set on the \cspace s
of a type of robot called \emph{polygonal linkage}, which is
simply a concatenation of links and hinged joints forming a closed
chain, obviously these are of limited practical use. Generally
speaking for a given robot the \cspace \ dimension equals its
mobility, thus for most robots in use the \cspace \ is of
dimension three or more. Furthermore since investigation focuses
usually on global properties of the \cspace,
common mathematical tools are useless. \\

In what follows we discuss our investigation generally without
getting into detailed mathematics and formal defintions.  In order
to keep things simple we have started our investigation studying a
robot we named \emph{arachnoid mechanism} which admits a two
dimensional \cspace \ if as depicted in Figure
\ref{fig:mechanism}. The robot is constructed of $k$ branches
fixed at their one end and a common other end for all
branches,(the figure depicts a planar robot with branches having
only two links while our discussion holds also for the spatial
case where each branch may have an arbitrary number of links).
Note that this type of mechanism resembles some parallel robots
which are in practical use. Our first task was to examine if its
\cspace\ is a manifold, this could be easily done by applying the
regular value theorem (see \cite[I, Thm.~3.2]{Hi}): By identifying
the \cspace\ as the pre-image of a certain map $G :\Re^{d(N-k+1)}
\to R^N$, that sends each two successive joint loci to their
distance and finding the conditions for the jacobian $dG$ to have
full rank amounts to Theorem \ref{thm:main}. Thus \cspace\ of an
arachnoid mechanism is generically a closed orientable manifold of
dimension $d(N - k + 1)-N$, where $N$ is the total number of links
in the given mechanism and $d=2,3$ for the planar and spatial
cases respectively. Furthermore the cases in which this is not
true (i.e. the \cspace \ possess at least one topological
singularity) are when two branches (or more) are co-aligned (see
Definition
\ref{def:aligned},1) or when $d+1$ branches are aligned (Definition \ref{def:aligned},2).\\

We continue our investigation by explicitly characterizing the
topology of a certain kind of arachnoid mechanism: The \cspace \
for any branch and base-point $x$ is an $n$-torus $T^n$, with
$\phi : T^n \to W$ which maps into the associated work space. Note
that the fiber $\phi^{-1}(z)$ over any point $z \in \Int W$ is the
\cspace\ for the closed chain with links lengths
$\ell_{0},\ell_{1},\cdots,\ell_{n}$, where $\ell_{0} := z - x$. If
$z$ is on the boundary of the work space, then $\phi^{-1}(z)$ is
evidently discrete. Thus the \cspace \ is the pullback:
\begin{equation}
\label{eq:cspace}
\C~=~\{(\tau_{1},\dotsc,\tau_{k})\in\prod_{i=1}^{k}~\T{\nj{i}}~ |\
\phi_{1}(\tau_{1})=\dotsc=\phi_{k}(\tau_{k})\in\Wc\}.
\end{equation}

For the special planar case where all branches have exactly two
links each point within the work space $\textrm{Int}(W)$
corresponds (within the \cspace) to a discrete number of copies
(see Figure \ref{fig:annulus intersection}), thus
$$\textrm{Int}(W)\times \{0,1,2,\cdots,k\} \in \C$$ while a point
within the work space boundary $\partial W$ corresponds to certain
suitable gluing of the $\textrm{Int}(W)$ replicas, this way we can
explicitly calculate the Euler characteristics from the work space
as stated in Theorem \ref{thm:sss}. In the non-generic case we can
carefully follow the singularities emerging by "traveling" through
the \emph{moduli space}: the latter is the space of geometric
characterization of the mechanism in hand, i.e. lengths of rods
and platforms geometry. This space is divided by \emph{walls} into
\emph{cells} where all points (mechanisms) within a given cell are
associated to the same \cspace. Hence in order to understand the
topological singularities within a wall (non-generic mechanisms)
we simply travel from cell to cell and track changes. We have
found that the singularities emerging are \emph{pinch points} of a
discrete number of $2$-discs which is given as a theorem in
\ref{prop:2node}. \\

In order to take advantage of the results discussed above we
continued our study with the problem of motion planning for a
\emph{star-shaped mechanisms} (see Figure \ref{X-shape})which are
simply planar arachnoid mechanisms. Since we consider only a point
end-effector, the direct kinematics is straightforward while the
inverse kinematics is more complex. Thus, while for most parallel
mechanisms using \cspace as a mean for path planning should be
carefully considered, here our approach is natural. Actually (a
slightly modified) motion planing strategy taken below can be
applied to any planar robot with a simple graph topology, though
we will not consider such here. In view of equation
\ref{eq:cspace} and the question of motion planning physibility we
would like to inquire if a fiber contains a given configuration
\--\ which later on we think of as the goal configuration:\\
We first follow some topological properties of the \cspace s (the
number of components and the structures of the components) of
single-loop closed chains with spherical joints (see Figure
\ref{fig:single-leg}). These properties drove (see Theorem~2 in
\cite{MT2}) the design of a complete, polynomial-time motion
planning algorithm, which we shall use here. "Adding up" more
branches (see Figure \ref{fig:double-leg}) we can define the work
space \emph{critical} parts $\Sigma=\left(\bigcup_{i=1}^k \Sigma_i
\right)\bigcap W_A$ as the regions where at least one of the
branches is aligned (see Figure \ref{fig:chambers}). \\
We proceed by introducing the conditions (Theorem \ref{prop-2} and
Corollary \ref{cor-1}) in which two given configurations are in
the same connected component of the \cspace \ which simply put is
true iff each branch \textbf{that cannot be aligned in any way}
need not change its \emph{Elbow UP/Down}
signs (of a set of links called \emph{large links}) in order to move from
initial configuration to goal (see Figure \ref{fibre}).\\

Our strategy is to solve the path existence based on the set of
critical regions in the workspace, and then construct the path
combining our knowledge of the workspace and the structure of the
\cspace \ of single-loop closed chains. Note that the problem is
not just moving the junction point between an initial and a goal
position, but moving the manipulator along with all its legs from
an initial configuration to a goal configuration. So, the
workspace information will be insufficient for path construction.
We employ a move that changes the shape of a leg with its endpoint
fixed in the workspace. This move, called the \emph{sign-adjust
move}, uses the knowledge of the \cspace \ of a single-loop closed
chain. We then showed that the overall complexity of the motion
planning algorithm is $O(k^3N^3)$, where $N$ is the maximum number
of links in a leg and $k$ is the number of branches. The
polynomial complexity is key to the applications like folding of
macro- molecules, which can be modeled as a closed chain with
large $k$
and N. \\

Our next goal was to study the \cspace\ of mechanism we call a
\emph{polygonal mechanism} which consists of a moving polygonal
platform, having branches attached to each vertex, with the other
end fixed in $\Re^d$, while each branch is a concatenation of rods
with revolute (i.e., rotational) joints at the consecutive meeting
points.(see Figure \ref{fig:kgon}). As done before, we use the
regular value theorem to find the conditions in which a polygonal
mechanism \cspace\ is a manifold. Proving the above involves a
careful examination of the geometry that uniquely determines the
mechanism (see Remark \ref{rpolygon}). Furthermore by
\emph{generic} we mean that there is a zero-measure of non-generic
cases, formally we define generic in Definition \ref{fsingular}.

The map we consider $F:\RR{dN}\to\RR{N+|\Ic|}$, \ (see proof of
Theorem \ref{tone}) is defined as:
%
\begin{equation}\label{econstr}
%
F(\Vc)~=~F(\Vj{1},\dotsc,\Vj{i})~:=~
(f_{\nj{1}}(\Vj{1}),\dotsc,f_{\nj{k}}(\Vj{k}),
\|\aj{1,2}\|^{2},\dotsc,\|\aj{k-1,k}\|^{2})~,
%
\end{equation}
%
Which admits a full ranked jacobian iff all participant branches
are aligned and  $\sum^{k}_{i=1}\vj{i} $, on the moving platform.
Concluding these results:
\begin{enumerate}
\item Two participants vectors is not a generic case (Figure
\ref{fig:sing1}).
\item Four or more participants vectors is not a generic case and we disallow
it.
\item Three participants vectors is a generic case - (coupler curve
intersection with a circle)
\end{enumerate}
So we need the �finer� \emph{Transversality theorem} handling the
three aligned legs case to find that for a generic polygonal
mechanism, any configuration having at most three aligned branches
is smooth (Proposition \ref{pzero}). Finally by redefining the
"three aligned branches case" as one where the three branches are
aligned, with direction lines in the same plane meeting in a
single point as depicted in Figure \ref{fig:sing3} (which is
referred to in literature as a flat pencil) we complete Theorem
\ref{tone} proof.\\

Thus we gave a necessary condition for an uncertainty singularity
to occur in a polygonal spatial/planar mechanism with arbitrary
number of branches and arbitrary number of links in each branch.
Naturally one can inquired about the corresponding instantaneous
kinematic singularities:\\
First (Section \ref{sarch}) we numerate all edundant/non-redundant
architecture possibilities for statically defining the moving
platform, then we use screw theory to describe (Proposition
\ref{pkinsing}) the kinematic singularities emerging due to the
topological ones. Finally we give a visual presentation (Figure
\ref{fig:box_kiss}) of an
uncertainty singularity in an explicit 3-URU 3-DOF mechanism.\\

We Conclude our study by constructing a morse function for the
simplest type of polygonal mechanism \ -- \ namely, planar
mechanisms \ ($d=2$) \ having triangular platforms \ ($k=3$) \ and
exactly two links per branch \ ($\nj{1}=\nj{2}=\nj{3}=2$). We
propose a morse function \ref{ttwo}, prove it is Morse and find
its critical points (Figures
\ref{fcaseI},\ref{fcaseII},\ref{fcaseIII}), and give a computation
for a mechanism having $S^3$ \cspace (Figure \ref{fmechanism}).

\section{conclusions}

Configuration space is hard thing to follow. Nevertheless for the
two dimensional case (see chapter \ref{chap2} - the
\emph{Arachnoid mechanism}) were actual visualization is possible
one can "find his way". Configuration space had been found
explicitly: we devised an easy way to "build up" this space as a
handle body, which if followed can be used as a guide for further
bigger dimensioned \cspace \ investigations. Singularities results
in this regard can be easily applicable either in design stage or
for motion planning and control in cases were the robot \cspace \
is intentionally designed to contain topology singularities. Thus
a further investigation of topological singularities may naturally
begin with these results. We have found that said singularities in
the two dimensional case, are simply a wedge of discs. Clearly,
generally this is not the case, and apart of presentations of
$3$-dimensional \cspace \ singularities for given mechanisms, we
haven't found any characterization, but it seems like a good
subject to deal with in the future.

The question of the manifold character of \cspace \ which arose at
the beginning of our research had popped again once we had begun
thinking of the \emph{polygonal mechanisms}, except in this case,
solution was not an easy task. We had found that polygonal
mechanisms have generically a manifold \cspace. Moreover we
introduced three necessary conditions for a polygonal mechanism to
have a topological singularity. We underline here that these are
not sufficient, for example consider a triangular mechanism having
$2,1,1$ links in its associated first, second and third branches.
If we (virtually) disconnect the first branch from the rest of the
mechanism we end with an open $2$-chain and a coupler mechanism.
Consequently the work space of the $2$-open chain is an annulus,
while the coupler's work space is a coupler curve. Note that for
the situation depicted below where the coupler curve is tangent to
the annulus, while the annulus intersects the coupler curve into
two segments, all branches are aligned and their lines of
alignment intersect in one point (the tangency point). Still the
configuration where both curves are tangent does not induces a
topological singular configuration. Thus there is a need to
continue the research, and come up with a sufficient condition for
topological singularities.
%
\begin{figure}[h]
  \centering
  \includegraphics[width=3in]{fig-conclusion/non-singular.eps}
  \caption{A counter example for singular criterion presented in last chapter.}
  \label{non-singular}
\end{figure}
%
The manifold character of the generic star-shaped mechanism,
generic arachnoid mechanism, generic polygonal mechanisms is
herein proved, while the manifold character for simple closed
chain mechanism (link loops) had been proven in literature.

On the other hand Kempe \cite{Kem} proved in 1876 that given an
arbitrary real algebraic curve there exist a linkage such that one
of its vertices will trace the curve. Jordan and Steiner \cite{JS}
construct an homeomorphism between any given algebraic variety and
some components of a mechanical linkage \cspace.\\

The above may indicate a more general truth. Thus we conclude this
dissertation with the conjecture:\\

\begin{conj}
All planar (spatial) graph mechanisms comprised of rigid rods and
rotational (spherical) joints have a smooth orientable manifold
\cspace s for a generic set of lengths
\end{conj}

By the term \emph{generic} we mean "almost every set of lengths".


\begin{center}Thats all folks...\end{center}
