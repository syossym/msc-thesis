\pagestyle{empty} % No headers or page numbers
% ---------------------  Title page --------------------------------
\begin{center}
\vspace*{2.0cm} \Huge {\bf \Huge The Configuration Space of
Parallel Mechanisms}


\vspace*{6.0cm} \Large
Nir Shvalb\\
\end{center}
\newpage
% ---------------------  inner title page --------------------------------
\begin{center}
\Huge {\bf \Huge The Configuration Space of Parallel Mechanisms}

\vspace*{1.5cm} \Large
 Research thesis

\vspace*{1.5cm} \Large
Submitted in Partial Fulfilment of the \\
Requirement for the degree of \\
Doctor of Philosophy \\
\vspace*{3.0cm} \Large
Nir Shvalb\\



\vspace*{2.0cm} \Large
Submitted to Senate of\\
the Technion - Israel Institute of Technology\\

\end{center}
\begin{center} \vspace*{.5cm} Haifa, Israel, Nov 2006, Kislev 5776
\end{center}
\newpage

%%%%%%%%%%%%%%%%%%%%%%%%%%%%%%%%%%%%%%%%%%%%%%%%%%%%%%%%%%%%%
% PRELIMINARY PAGES

\pagestyle{plain} % No headers, just page numbers
\pagenumbering{roman} % Roman numerals

%
\prefacesection{Acknowledgments}
The Research Thesis Was Done Under The Supervision of Prof. Moshe
Shoham in the Department of mechanical Engineering - The Technion
and Dr. David Blanc from the Department of mathematics - Haifa
University.

%\setcounter{page}{1}
\tableofcontents
%\listoftables
% --------------- list of notations ---------------
\prefacesection{List of Notations}
\listofnotations
% -------------------list of figures --------------
\listoffigures
\addcontentsline{toc}{chapter}{List of figures}
\newpage


\prefacesection{Acknowledgments}
is the longest word in this dissertation. Well, Under such a
formal title. I would like to express my love and gratitude to
some of the people that convinced me that torturing myself is good , so here goes\\
\textbf{Mom and Dad} without your love (and evidently a lucky day
for me 34 years ago) I wouldn't be at all, which is not a good
start, thanks for that and for your endless love and care. I love you both very much.\\
\textbf{Prof. David Blanc} - My advisor and my math teacher thank
you, without your advices and ruthless dedication I would probably
be still wondering with a dumb smile somewhere between Homotopy and Morse Theory .\\
\textbf{Oded Salomon} thanks for your friendship, thanks for
hearing my mathematic nonsense for three years, and for all the
wonderfull advices you gave me.\\ \textbf{Micha, Rani} thanks for
making me laugh.\\ \textbf{Naama, Gabriela, Meital, Pnina, Noga,
Tzahi, Daniel,Yonatan} for being real friends. Lastly I would like
to thank \textbf{Prof. Moshe Shoham} - My advisor, thanks for your
guidance, support and encouragement throughout the research.
thanks for the freedom you gave me and for your faith in me.\\
$\heartsuit$ you all\\
Nir.


\prefacesection{Abstract}

We study the \cspace \ of some parallel mechanism that are in use.
starting with the \cspace s of arachnoid mechanisms which consist
of $k$\ branches each of which has an arbitrary number of links
and a fixed initial point, while all branches end at one common
end-point. We show that generically, the \cspace s of such
mechanisms are manifolds, and determine the conditions for the
exceptional cases. The \cspace\  of planar arachnoid mechanisms
having $k$ branches, each with two links is fully characterized
for both the non-singular and the singular cases.

Applying these results we study the motion planning problem for
such planar manipulators. A topological analysis is used to
understand the global structure of the configuration space so that
motion planning problem can be solved exactly. The worst-case
complexity of our algorithm is $O(k^3N^3)$, where $N$ is the
maximum number of links in a leg. Examples illustrating our method
are given.

Next we study the  \cspace\ $\C$ of a parallel polygonal
mechanism, having a moving polygonal platform, and a flexible leg
(consisting of concatenated rods) attached to each vertex, with
the other end fixed in  \ $\RR{d}$. \ We give necessary conditions
for the existence of uncertainty singularities; and show that
generically $\C$ is a smooth manifold. In the planar case, we
construct an explicit Morse function on $\C$, and show how
geometric information about the mechanism can be used to identify
the critical points. Finally we describe how topological
singularities give rise to instantaneous kinematic and explicit
example is provided.




%===================================================
\prefacesection{Methods and preliminaries}
We now introduce some definitions, theorems and methods which are
in extensive use in this dissertation.\\
\emph{\textbf{Smooth Manifolds:}} An $n$-dimensional
\emph{manifold} is a \emph{topological space} which "near" each of
its points is locally like $\Re^n$ i.e. there is a neighborhood
which is topologically "the same" as an open unit ball in $\Re^n$.
Formally, two additional properties are needed to form a manifold,
being \emph{Hausdorff}
and \emph{Second Countable} (see for example \cite{C}).\\
A differential manifold $\M$ \ is a manifold for which overlapping
\emph{charts} "relate smoothly" to each other, where a chart
$\alpha:U\to\Re^n$ is a continuous injection from an open subset
$U$\ of $\M$\ to $\Re^n$.\\
A manifold is \emph{connected} if, roughly speaking, it is all in one piece.\\
A \emph{submanifold} is a subset of a manifold which is itself a
manifold, but has smaller dimension. For example, the equator of a sphere is a submanifold.\\
If $\M$\ is an orientable surface, its genus $g(\M)$ is the number
of ``handles'' it possesses. Furthermore, any orientable surface
is a sphere, or the \emph{connected sum} of $n$\ tori. We say the
sphere has genus $0$, and that the connected sum of $n$\ tori has
genus $n$. Also, $g(\M)=1-\chi(\M)/2$\ where $\chi(\M)$\ is the
Euler characteristic of $\M$.

\emph{\textbf{Smooth Mappings and Singularity}}: The following
enabled us to prove smoothness of generic \smech , and used also
in finding necessary conditions for uncertainty singularities in
polygonal mechanisms:\\
Let $f:\N^n\to\M^m$\ be a smooth map on smooth manifolds. A
critical point of $f$\ is a point $x\in\N$\ such that the
differential $df$\ considered as a linear transformation of real
vector spaces has rank $<m$. A critical value of $f$\ is the image
$f(x)$\ of a critical point. A regular value of $f$\ is a point
which is not the image of any critical point.
\begin{thm}(Regular value theorem) Let $f:\N^n\to\M^m$\ be a smooth mapping. if $y$\ is a regular
value of $f$\ then $f^{-1}(y)$\ is a smooth $n-m$ manifold.
\end{thm}


The notion of objects \emph{intersecting transversally} is
fundamental to singularity theory. The simplest situation to look
at is two subspaces of a vector space $V$: We say that they
intersect transversally when their vector sum is $V$, formally:

\begin{defn}
Let $f: \N \to \M$\ be a smooth mapping, and let $Q$ be a smooth
submanifold of $\M$. $f$\ and $Q$\ are said to intersect
transversally (denoted $f\pitchfork Q$) iff for all $x\in\N$\ with
$y=f(x)\in Q$\ we have $d_xf(T_x\N)+T_yQ=T_y\M$.
\end{defn}

where $T_x\M$\ is the space spanned by the set of all vectors
tangent to $\M$\ at point $x\in\M$, and $d_xf$\ the jacobian of
the map $f$\ calculated at point $x$. The next proposition (see
also \cite[pp31]{AGV}) provides us a tool for examining
uncertainty singularities in the special case where the regular
value theorem fail.
\begin{prop}
Let $f:\N\to \M$\ be a smooth mapping, and let $Q$\ be a smooth
submanifold of $\M$ with $f\pitchfork Q$: then $f^{-1}(Q)$\ is a
smooth submanifold of $\N$, having the same codimension as $Q$ or
is empty.
\end{prop}



%===================================================

\prefacesection{Introduction}
Generally speaking a \emph{configuration} is a set of independent
parameters uniquely specifying the position of each of its
mechanical components relative to a fixed frame of reference. A
\emph{\cspace} of a given mechanism is therefore the totality of
all its admissible positions in the Euclidean space. For example,
a simple $n$-linked branch mechanism with revolute joints have the
$n$-torus \ $\TT^n$ \ as its configuration space.

Obviously knowing the configuration space is a powerful tool for
motion planning problems, singularities characterization and
global understanding of a system. Main research have been set on
the \cspace s of a type of mechanism called \emph{polygonal
linkage}, which is simply a concatenation of links and hinged
joints forming a closed chain in \ $\RR{2}$ and \ $\RR{3}$. A
substantial amount of mathematical literature on polygonal
linkage's \cspace\ has has been written using various methods like
computation of Euler characteristics, homology groups and
handle-body surgery;

Worth special note is the work of J. C. Hausmann, and A. Knutson
\cite{HK} who determine the cohomology rings for these spaces and
the co-work of R.J. Milgram and J. Trinkle \cite{MT1} who
presented full surgery analysis for the same kind of mechanism.
The results of \cite{KM}, provide a good review of previous work
and some interesting results on the structure of these \cspace s.

However, to the best of the author's knowledge \cspace s of more
common parallel mechanisms, i.e. mechanisms with multiple branches
having a \emph{tree} topology or a spider like mechanisms where
all branches are attached in one rigid platform (\emph{polygonal
mechanism}) or one point (\emph{Arachnoid mechanism}), have never
been dealt with before. Such mechanisms are of great importance in
the "real life" due to their accuracy and stiffness properties.

In this dissertation we deal with mechanisms which are constructed
by rigid rods and spherical (rotational)  joints, called
\emph{graph mechanisms}, which mathematically speaking are simply
graphs \ $(G(v,e),\mathcal{L})$ \ having s ets of vertices and
edges such that \ $e(G)=v(g)\times v(g)$ \ and a set of fixed
lengths $\mathcal{L}$.

The \cspace \ is sometimes non-accurately defined in literature.
Here we define the \cspace \ as the quotient \
$\mathcal{C}=\mathcal{R}(G)/\Lambda$ \ where
%
$$
\R(G)=\{\textbf{x}_i \in \RR{v(G)}|i\in v(G),
\|\textbf{x}_i-\textbf{x}_j\|=\L(i,j) \}
$$
%
and $\Lambda$ \ is the group of isometries of translations and
rotations in \ $\RR{2}$, or in \ $\RR{3}$. Thus \ $\Lambda$ \
"fixes" the the mechanism to a fixed frame. The difference between
these two definitions for the \cspace \ can be realized in the
following: A subgraph $H$ \ of graph \ $G$ \ is a graph such that
\ $v(H) \ \subset \ v(G)$ \ and \ $e(H) \ \subset e(G)$, which
leads to the definition:
\begin{defn}
A \emph{sub-mechanism} \ $(H,\mathcal{L}|_{H})\subset (G,\L)$ \ is
a subgraph \ $H$ \ of \ $G$\ together with the restricted length
subset \ $\L|_{H}$, and we denote such a sub-mechanism simply by \
$H$. \ A \emph{branch} is a sub-mechanism such that all vertices
in $H$\ have valence $1$\ or $2$.
\end{defn}
In light of these definitions one can define a feasible
configuration of a parallel mechanism on condition that all of its
branches "agree" upon one (or more) branch configuration. One
could claim that for  arachnoid mechanisms the \cspace \ is simply
the fibered product of the \cspace \ of its branches, but then he
would be wrong. Alternatively we know that:
\begin{thm}
Given a mechanism \ $(G,\L)$ \ with a set \ $H_{1},H_{2},H \subset
G$ \ of its sub-mechanisms such that \ $H_{1}\cap H_{2}=H$ \ then
\ $\R(\L)$ \ is the pullback of the the natural projections:
%
$$
\R(H_{1}) \xrightarrow{\pi_{H_{1}}} \R(H) \xleftarrow{\pi_{H_{2}}}
\R(H_{2}).
$$
\end{thm}
Note that this does not extend to the \cspace \ $\C$ \ since the
induced projections \ $\{\R(H_{i})/\Lambda \rightarrow
\R(H)/\Lambda\}$ \ are determined up to isometry \ $\Lambda$ \ and
thus do not determine a unique point (A proof for this theorem and
a detailed discussion can be found in the Appendix \ref{appendix}).\\

We first analyze a simple class of mechanism called
\emph{arachnoid} which consists of multiple branches each having
arbitrary number of links a fixed initial point, and all end at a
common point (this type of mechanism resembles some parallel
robots which are in practical use \cite{LLL}: parallel
manipulators, walking robots, and dexterous manipulation). It is
shown (see Chapter \ref{chap1}) that generically, the \cspace s of
such mechanisms are manifolds, and the conditions for the
exceptional cases are then determined. We analyze the \cspace\ of
planar arachnoid mechanisms having $k$ branches, each with two
links. Such mechanisms have a $2$ dimensional \cspace \ which
enables a full classification. The non-manifold cases where
singularities emerges are calculated.

As mentioned above, knowing the \cspace \ often enables problems
solving, like the existence of continuous motion planners (Farber
cf. \cite{F}) or the motion planning problem (Note that this kind
of approach completely discards the nature of the mechanism in
issue). Due to the computational complexity and difficulty of
implementing general exact motion planning algorithms, such as
Canny's \cite{Can88}, today sample-base algorithms, such as
Kavraki's \cite{KSLO96} dominate motion planning research.
However, there are important classes of problems for which these
algorithms do not perform well. These arise in systems whose
\cspace \ cannot effectively be represented as a set of parameters
with simple bounds, but rather is most naturally represented as a
variety of co-dimension one or greater embedded in a
higher-dimensional ambient space \cite{YLK01}. Examples of these
systems include manipulators with one or multiple closed loops,
whose configuration space is defined by loop closure constraints.
The \emph{Random Loop Generator} method \cite{Cortes02,CS03}
improves the sampling techniques through estimating the regions of
sampling parameters. However, its efficiency relies on the
accuracy of the estimation, which is often case-variant. Moreover,
it ignores the global structure of \cspace \, and may fail to
sample globally important regions. For which specialized exact
methods have not been developed. This provides motivation to try
to develop effective exact algorithms. For most mechanisms \cspace
\ is of dimension greater than two which makes them impossible to
be fully characterized. Yet, global properties like connectivity
and smoothness are attainable. Recent advances (especially
connectivity analysis methods) in the understanding of the global
structure of \cspace \ of single-loop closed chains
\cite{MT2,SSB05} allow us (Chapter \ref{chap2} here) to develop a
polynomial-time exact algorithm for arachnoid manipulators (which
actually can be extended to any planar mechanism with an arbitrary
tree topology).

Next we focus attention on parallel robots (in any dimension)
comprised of arbitrary number of branches each having arbitrary
number of links concatenated by rotational joints and attached to
a moving platform, which we shall refer to as \emph{polygonal
mechanisms}, these are the mathematical parallel of most parallel
mechanisms which are increasingly in use. Still their \cspace \
had never been analyzed. We examine closely the case of parallel
robots singularity called uncertainty singularity which is simply
a topological singularity of the configuration space:

Kinematic singularities of parallel mechanisms have been studied
extensively in the literature (cf.\ \cite{Hu2}, and the example of
open kinematic chains in \cite{Hu1,WW}). On the other hand, the
\cspace\ of such a mechanism may have topological singularities.
Our goal here is to try to relate these two kinds of singularity,
for that certain class of mechanisms.

Choosing appropriate local coordinates for a given mechanism, we
try to endow its \cspace \ with the structure of a differentiable
manifold. The points where this cannot be done constitute the
\emph{topological singularities} of the mechanism. We shall be
concerned only with \emph{uncertainty singularities} (where  the
\emph{instantaneous mobility} is greater than the \emph{full cycle
mobility} \ -- \ cf.\ \cite{Hu2}); In such a position the
mechanism as a whole gains an additional degree of freedom. Such
points may sometimes separate \cspace \ into distinct regions,
allowing transition between different operating modes. (An
extensive examination of such remarkable transitions in a special
case (a \emph{constraint} singularity for $3$-URU parallel
mechanism with three degrees of freedom) is given in \cite{ZBC}).
In Chapter \ref{chap3} we give a necessary geometric condition for
uncertainty singularities in said mechanisms, and explain how such
topological singularities give rise to instantaneous kinematic
singularities in Chapter \ref{chap4}. Finally, the occurrence of
an uncertainty singularity is illustrated visually in an explicit
example.

% Change page numbering back to Arabic numerals
%\pagenumbering{arabic}

%\setcounter{page}{1}

% Pages which are generated automatically
%\setcounter{page}{6} % Set this counter to get correct page numbers



%%%%%%%%%%%%%%%%%%%%%%%%%%%%%%%%%%%%%%%%%%%%%%%%%%%%%%%%%%%%%
