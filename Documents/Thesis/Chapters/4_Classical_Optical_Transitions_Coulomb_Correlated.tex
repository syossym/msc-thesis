\selectlanguage{english}%

\chapter{Coulomb Correlated Optical Transitions}

\label{cha:Coulomb_Correlated_Optical_Transitions}The preceding chapter
only considered free carriers and omitted their Coulomb interaction.
The Coulomb interaction leads to the formation of electron-hole pairs,
also denoted as \emph{excitons}. Due to their bound nature, the transition
energy is lowered, leading to absorption below the fundamental band
gap, while the correlated movement of the electron-hole pair increases
the transition probability and consequently the transitions strengths.
As in the previous chapter, the theory review is based mainly on \citet{Chuang1995,Jackson1998,Chow,Haug2009,Schafer2002}.

The present chapter therefore focuses on the inclusion of such many
body effects in the evaluation of optical properties. A large part
of the theory presented in the preceding chapter can be recycled.
In general, only the omitted two-particle interaction Hamiltonian
$H_{2}$ in \ref{eq:Two_Particle_Hamiltonian} needs to be added to
the equation of motion for the microscopic polarization \ref{eq:SC_Bloch_Eq_1}.
The chapter is therefore structured as follows: first, the Coulomb
Hamiltonian is derived for Bloch functions and transformed into the
electron-hole picture. Then the Coulomb Hamiltonian is included into
the equation of motion, and the so called Hartree-Fock approximation
is performed along with the introduction of screening.


\section{Second Quantization}


\subsection{Introduction}

If $n_{i},\mathbf{k}_{i}$ denote the quantum number of a Bloch state,
then the Coulomb Hamiltonian \ref{eq:Two_Particle_Hamiltonian} can
be written as\begin{equation}
\hat{H}_{2}=\frac{1}{2}\sum_{\begin{array}{c}
n_{1}\mathbf{k}_{1},n_{2}\mathbf{k}_{2},\\
n_{3}\mathbf{k}_{3},n_{4}\mathbf{k}_{4}\end{array}}\left\langle n_{1}\mathbf{k}_{1}n_{2}\mathbf{k}_{2}\mid v\mid n_{3}\mathbf{k}_{3}n_{4}\mathbf{k}_{4}\right\rangle \hat{a}_{n_{1}\mathbf{k}_{1}}^{\dagger}\hat{a}_{n_{2}\mathbf{k}_{2}}^{\dagger}\hat{a}_{n_{3}\mathbf{k}_{3}}\hat{a}_{n_{4}\mathbf{k}_{4}}\label{eq:Coulomb_Hamiltonian_1}\end{equation}
and the matrix element of the Coulomb operator reads\begin{eqnarray}
\left\langle n_{1}\mathbf{k}_{1}n_{2}\mathbf{k}_{2}\mid v\mid n_{3}\mathbf{k}_{3}n_{4}\mathbf{k}_{4}\right\rangle  & = & \frac{e^{2}}{4\pi\epsilon_{0}}\int_{\Omega}\int_{\Omega}d\mathbf{r}dzd\mathbf{r}'dz'\varphi_{n_{1}\mathbf{k}_{1}}^{*}(\mathbf{r},z)\varphi_{n_{2}\mathbf{k}_{2}}^{*}(\mathbf{r}',z')\nonumber \\
 &  & \cdot\frac{1}{\left|(\mathbf{r},z)-(\mathbf{r}',z')\right|}\varphi_{n_{4}\mathbf{k}_{4}}(\mathbf{r},z)\varphi_{n_{3}\mathbf{k}_{3}}(\mathbf{r}',z').\label{eq:Coulomb_Operator_Matrix_Element}\end{eqnarray}
The Coulomb-potential is mainly long-range and therefore more or less
constant within a lattice cell. Consequently, the lattice periodic
part can be averaged over a single crystal cell. The integration is
then performed over lattice averaged quantities, as suggested in \ref{eq:Matrix_Element_Lattice_Cell_Average}.
Therefore, rewriting \ref{eq:Coulomb_Operator_Matrix_Element} into
a more suitable form including lattice cell averages, yields\begin{eqnarray}
\left\langle n_{1}\mathbf{k}_{1}n_{2}\mathbf{k}_{2}\mid v\mid n_{3}\mathbf{k}_{3}n_{4}\mathbf{k}_{4}\right\rangle  & = & \frac{e^{2}}{4\pi\epsilon_{0}}\int_{\Omega}\int_{\Omega}d\mathbf{r}dzd\mathbf{r}'dz'\frac{1}{\mathcal{A}^{2}}e^{i\left(\mathbf{k}_{4}-\mathbf{k}_{1}\right)\cdot\mathbf{r}}e^{i\left(\mathbf{k}_{3}-\mathbf{k}_{2}\right)\cdot\mathbf{r}'}\nonumber \\
 &  & \cdot g_{n_{1}\mathbf{k}_{1},n_{4}\mathbf{k}_{4}}(z)g_{n_{2}\mathbf{k}_{2},n_{3}\mathbf{k}_{3}}(z')\frac{1}{\left|(\mathbf{r},z)-(\mathbf{r}',z')\right|},\end{eqnarray}
where $g_{n_{1}\mathbf{k}_{1},n_{4}\mathbf{k}_{4}}(z)\approx\sum_{i}F_{n_{1}\mathbf{k}_{1},i}^{*}(z)F_{n_{4}\mathbf{k}_{4},i}(z)$
and $g_{n_{2}\mathbf{k}_{2},n_{3}\mathbf{k}_{3}}(z')$ is defined
accordingly. The next step is to get rid of the free directions $\mathbf{r}$
and $\mathbf{r}'$, where the slowly varying part of the Bloch-function
is given by the plane wave. The first step is to rewrite the term
$\frac{1}{\left|(\mathbf{r},z)-(\mathbf{r}',z')\right|}$ as $\frac{1}{\sqrt{\left(\mathbf{r}-\mathbf{r}'\right)^{2}+\left(z-z'\right)^{2}}}$.
Then, the integration over $\mathbf{r}'$ is transformed into the
integration over the distance between $\mathbf{r}$ and $\mathbf{r}'$,
$\mathbf{s}=\mathbf{r}-\mathbf{r}'$. So, instead of performing for
every point r the integration over $\mathbf{r}'$, one integrates
for every point $\mathbf{r}$ over all possible distances $\mathbf{s}$.
This approach leads to several simplifications\begin{eqnarray}
\left\langle n_{1}\mathbf{k}_{1}n_{2}\mathbf{k}_{2}\mid v\mid n_{3}\mathbf{k}_{3}n_{4}\mathbf{k}_{4}\right\rangle  & = & \frac{e^{2}}{4\pi\epsilon_{0}}\int_{\mathcal{L}}\int_{\mathcal{L}}dzdz'\frac{1}{\mathcal{A}^{2}}g_{n_{1}\mathbf{k}_{1},n_{4}\mathbf{k}_{4}}(z)g_{n_{2}\mathbf{k}_{2},n_{3}\mathbf{k}_{3}}(z')\nonumber \\
 &  & \cdot\int_{\mathcal{A}}\int_{\mathcal{A}}d\mathbf{r}d\mathbf{s}\left[\frac{1}{\sqrt{\mathbf{\left|s\right|}^{2}+\left(z-z'\right)^{2}}}\right.\nonumber \\
 &  & \left.\cdot e^{i\left(\mathbf{k}_{4}+\mathbf{k}_{3}-\mathbf{k}_{1}-\mathbf{k}_{2}\right)\cdot\mathbf{r}}e^{i\left(\mathbf{k}_{2}-\mathbf{k}_{3}\right)\cdot\mathbf{r}'}\right].\end{eqnarray}
The integration over $\mathbf{r}$ and $\mathbf{s}$ is decoupled
and therefore the plane-wave leads to a delta function\begin{equation}
\int_{\mathcal{A}}d\mathbf{r}e^{i\left(\mathbf{k}_{4}+\mathbf{k}_{3}-\mathbf{k}_{1}-\mathbf{k}_{2}\right)\cdot\mathbf{r}}=\mathcal{A}\delta_{\mathbf{k}_{4}+\mathbf{k}_{3},\mathbf{k}_{1}+\mathbf{k}_{2}}.\label{eq:Delta_1}\end{equation}
The remaining integration over $z,z'$ and $\mathbf{s}$ depends on
the dimensionality of the system. 

The Fourier transformation along the free direction $\mathcal{A}$
defined in \ref{eq:Fourier_Transform} is used to expand $\frac{1}{\mathbf{r}}$\begin{equation}
\vartheta(\mathbf{s},z,z')=\frac{1}{\sqrt{\mathbf{\left|s\right|}^{2}+\left(z-z'\right)^{2}}}=\sum_{\mathbf{q}}\vartheta_{\mathbf{q}}(z,z')e^{i\mathbf{q}\cdot\mathbf{s}}\label{eq:Inverse_Distance_Fourier_Expansion}\end{equation}
in terms of its Fourier representation. Taking the integral over $\mathbf{s}$\textbf{\begin{equation}
\int_{\mathcal{A}}d\mathbf{s}\sum_{\mathbf{q}}\vartheta_{\mathbf{q}}(z,z')e^{i\mathbf{q}\cdot\mathbf{s}}e^{i\left(\mathbf{k}_{2}-\mathbf{k}_{3}\right)\cdot\mathbf{s}}=\mathcal{A}\sum_{\mathbf{q}}\vartheta_{\mathbf{q}}(z,z')\delta_{\mathbf{k}_{3}-\mathbf{k}_{2},\mathbf{q}}\end{equation}
}leads to another delta function, which can together with \ref{eq:Delta_1}
be used to simplify \ref{eq:Coulomb_Hamiltonian_1} into\begin{equation}
\hat{H}_{2}=\frac{1}{2}\sum_{\begin{array}{c}
n_{1},n_{2}\\
n_{3},n_{4}\end{array}}\sum_{\mathbf{k}\mathbf{k}'\mathbf{q}}\Theta_{\mathbf{q},\mathbf{k},\mathbf{k}'}^{n_{1},n_{2},n_{3},n_{4}}\hat{a}_{n_{1}\mathbf{k}+\mathbf{q}}^{\dagger}\hat{a}_{n_{2}\mathbf{k}'-\mathbf{q}}^{\dagger}\hat{a}_{n_{3}\mathbf{k}'}\hat{a}_{n_{4}\mathbf{k}}.\label{eq:Coulomb_Hamiltonian_2}\end{equation}
Here, $\mathbf{k}_{4}$ has been relabeled to $\mathbf{k}$ and $\mathbf{k}_{3}$
to $\mathbf{k}'$. The term $\Theta_{\mathbf{q},\mathbf{k},\mathbf{k}'}^{n_{1},n_{2},n_{3},n_{4}}$
contains all parts of \ref{eq:Coulomb_Operator_Matrix_Element} depending
on actual wavefunctions and on the dimensionality of the nanostructure.
It is given by\begin{equation}
\Theta_{\mathbf{q},\mathbf{k},\mathbf{k}'}^{n_{1},n_{2},n_{3},n_{4}}=\frac{e^{2}}{4\pi\epsilon_{0}}\int_{\mathcal{L}}\int_{\mathcal{L}}dzdz'g_{n_{1}\mathbf{k}+\mathbf{q},n_{4}\mathbf{k}}(z)g_{n_{2}\mathbf{k}'-\mathbf{q},n_{3}\mathbf{k}'}(z')\vartheta_{\mathbf{q}}(z,z').\label{eq:Coulomb_Interaction}\end{equation}


In the case of a bulk crystal, there is no integration over $z,z'$.
Then, $\vartheta_{\mathbf{q}}^{3D}$ is given by the 3D Fourier transform
of the Coulomb potential \begin{equation}
\vartheta_{\mathbf{q}}^{3D}=\frac{4\pi}{\Omega q^{2}}.\label{eq:3D_Form_Factor}\end{equation}


The quantum well case has a 2D plane as the translational invariant
directions and a 1D axis where symmetry is broken. The 2D Fourier
transformation of \ref{eq:Inverse_Distance_Fourier_Expansion} gives\begin{equation}
\vartheta_{\mathbf{q}}^{2D}(z,z')=\frac{2\pi}{\mathcal{A}q}e^{-q\left|z-z'\right|}.\label{eq:2D_Form_Factor}\end{equation}
Hereby, the factor $\frac{2\pi}{\mathcal{A}q}$ denotes the Fourier
transform of an ideal quantum well with no extension in the symmetry
broken direction, while the exponential factor comprises of the finite
extension of the quantum well wavefunctions. 

For the sake of convinience, we define the form factor term to be
\begin{equation}
G_{\mathbf{q},\mathbf{k},\mathbf{k}'}^{n_{1},n_{2},n_{3},n_{4}}=\int_{\mathcal{L}}\int_{\mathcal{L}}dzdz'g_{n_{1}\mathbf{k}+\mathbf{q},n_{4}\mathbf{k}}(z)g_{n_{2}\mathbf{k}'-\mathbf{q},n_{3}\mathbf{k}'}(z')e^{-q\left|z-z'\right|},.\end{equation}
in terms of which $\Theta_{\mathbf{q},\mathbf{k},\mathbf{k}'}^{n_{1},n_{2},n_{3},n_{4}}$
can be written as\begin{equation}
\Theta_{\mathbf{q},\mathbf{k},\mathbf{k}'}^{n_{1},n_{2},n_{3},n_{4}}=G_{\mathbf{q},\mathbf{k},\mathbf{k}'}^{n_{1},n_{2},n_{3},n_{4}}V_{\mathbf{q}}.\label{eq:Coulomb_Interaction_Alternative}\end{equation}
For the 2D case we can write \begin{equation}
\Theta_{\mathbf{q},\mathbf{k},\mathbf{k}'}^{n_{1},n_{2},n_{3},n_{4}}=G_{\mathbf{q},\mathbf{k},\mathbf{k}'}^{n_{1},n_{2},n_{3},n_{4}}\frac{e^{2}}{2\epsilon_{r}q\mathcal{A}}.\label{eq:Coulomb_Interaction_Alternative_2D}\end{equation}


Inspecting the Coulomb terms, it is clear that these terms diverge
in the limit of $\mathbf{q}=0$. The divergence can be resolved within
the \emph{Jellium model} \citet{Haug2009}, where the semiconductor
is assumed to be intrinsically charge free. As a result, the diverging
term of the electron-electron interaction at $\mathbf{q}=0$ is canceled
by the diverging term of the interaction of electrons with the homogeneous
positive charge background. Hence, terms $\mathbf{q}=0$ within the
summation over $\mathbf{q}$ are discarded.


\subsection{Diagonal Approximation}

The next approximation performed aims to remove two of the four subband
indices in \ref{eq:Coulomb_Hamiltonian_2} by assuming that Coulomb
interactions will not cause interband transitions. Then, $n_{1}=n_{4}$
and $n_{2}=n_{3}$, which is called here diagonal Coulomb approximation.
This assumption is for example justified at an approximate level in
symmetrical quantum wells, as the diagonal Coulomb matrix elements
are due to the symmetry of the wavefunctions, two orders of magnitude
larger than the off-diagonal \citet{Hader2003}. Also, for single-band
effective mass wavefunctions, only the diagonal terms remain, while
others vanish due to the orthogonality of the zone-center functions.
Using this approximations, the Coulomb Hamiltonian reads\begin{equation}
\hat{H}_{2}=\frac{1}{2}\sum_{m,n}\sum_{\mathbf{k}\mathbf{k}'\mathbf{q}}\Theta_{\mathbf{q},\mathbf{k},\mathbf{k}'}^{m,n,n,m}\hat{a}_{m\mathbf{k}+\mathbf{q}}^{\dagger}\hat{a}_{n\mathbf{k}'-\mathbf{q}}^{\dagger}\hat{a}_{n\mathbf{k}'}\hat{a}_{m\mathbf{k}}.\label{eq:Coulomb_Hamiltonian_3}\end{equation}
Consequently, the notation $\Theta_{\mathbf{q},\mathbf{k},\mathbf{k}'}^{m,n,n,m}$
can be reduced to $\Theta_{\mathbf{q},\mathbf{k},\mathbf{k}'}^{m,n}$.


\subsection{Holes}

The next step is to transform the Coulomb Hamiltonian into the electron-hole
picture. In the diagonal approximation \ref{eq:Coulomb_Hamiltonian_3},
one distinguishes between electron states in the conduction bands
$c,d$ and hole states in the valence bands $v,w$. Therefore, the
following four combinations are possible\begin{equation}
n=c,\, m=d\,\,\,\, n=c,\, m=v\,\,\,\, n=v,\, m=c\,\,\,\, n=v,\, m=w.\end{equation}
As next, hole operators are inserted for electrons in the valence
band. Further, a natural ordering is reestablished by propagating
the creation operators to the left and the annihilation operators
to the right and setting the hole creation operator before, the hole
annihilation operator after the one of the electron. The terms involving
the sum $\sum_{n=c,m=d}$ are already in that ordering. Permuting
the hole terms in the sum $\sum_{n=v,m=w}$ introduces a new two-operator
term while the sums $\sum_{n=c,m=v}$ and $\sum_{n=v,m=c}$ can be
summed together. Hence, the final Hamiltonian which will be used to
calculate the microscopic equation of motion is obtained as\begin{eqnarray}
\hat{H} & = & \sum_{c,\mathbf{k}}E_{c}(\mathbf{k})\hat{a}_{c\mathbf{k}}^{\dagger}\hat{a}_{c\mathbf{k}}-\sum_{v,\mathbf{k}}E_{v}(\mathbf{k})\hat{b}_{v\mathbf{k}}^{\dagger}\hat{b}_{v\mathbf{k}}\label{eq:Full_H_1-1}\\
 &  & -\mathbf{E}(\mathbf{r},t)\cdot\sum_{c,v,\mathbf{k}}\boldsymbol{\mu}_{cv,\mathbf{k}}\hat{a}_{c\mathbf{k}}^{\dagger}\hat{b}_{v\mathbf{-k}}^{\dagger}+\boldsymbol{\mu}_{cv,\mathbf{k}}^{*}\hat{b}_{v-\mathbf{k}}\hat{a}_{c\mathbf{k}}\label{eq:Full_H_2}\\
 &  & +\frac{1}{2}\sum_{c,d}\sum_{\mathbf{k}\mathbf{k}'\mathbf{q}}\Theta_{\mathbf{q},\mathbf{k},\mathbf{k}'}^{c,d}\hat{a}_{c\mathbf{k}+\mathbf{q}}^{\dagger}\hat{a}_{d\mathbf{k}'-\mathbf{q}}^{\dagger}\hat{a}_{d\mathbf{k}'}\hat{a}_{c\mathbf{k}}\label{eq:Full_H_3}\\
 &  & +\frac{1}{2}\sum_{v,w}\sum_{\mathbf{k}\mathbf{k}'\mathbf{q}}\Theta_{-\mathbf{q},-\mathbf{k}'+\mathbf{q},-\mathbf{k}-\mathbf{q}}^{w,v}\hat{b}_{v\mathbf{k}+\mathbf{q}}^{\dagger}\hat{b}_{vk'-\mathbf{q}}^{\dagger}\hat{b}_{w\mathbf{k}'}\hat{b}_{v\mathbf{k}}\label{eq:Full_H_4}\\
 &  & \sum_{v,\mathbf{k}',\mathbf{q}}\Theta_{-\mathbf{q},-\mathbf{k}'+\mathbf{q},-\mathbf{k}}^{v,v}\hat{b}_{v\mathbf{k}'}^{\dagger}\hat{b}_{v\mathbf{k}'}\label{eq:Full_H_5}\\
 &  & -\frac{1}{2}\sum_{\mathbf{k}\mathbf{k}'\mathbf{q}}\left\{ \left(\Theta_{-\mathbf{q},-\mathbf{k}'+\mathbf{q},\mathbf{k}}^{v,c}+\Theta_{\mathbf{q},\mathbf{k},\mathbf{q}-\mathbf{k}'}^{c,v}\right)\hat{a}_{c\mathbf{k}+\mathbf{q}}^{\dagger}\hat{b}_{v\mathbf{k}'-\mathbf{q}}^{\dagger}\hat{b}_{v\mathbf{k}'}\hat{a}_{c\mathbf{k}}\right\} .\label{eq:Full_H_6}\end{eqnarray}
The indices $\mathbf{k}$ and $\mathbf{k}'$ have in some cases been
swapped, shifted by $\mathbf{q}$ and occasionally changed sign.


\section{Transitions Calculation}


\subsection{Equations of Motion}

The next task is to evaluate the equation of motion \ref{eq:Heisenberg_Eq_Motion}
for $\hat{p}_{nm,\mathbf{k}}$, as it has already been done in previous
chapter, excluding the Coulomb interaction. Therefore, the remaining
task is to evaluate the commutator of the microscopic polarization
and the Coulomb Hamiltonian,\begin{equation}
\frac{d}{dt}\hat{p}_{nm,\mathbf{k}}=\frac{i}{\hbar}\left[\hat{H}_{2}\hat{p}_{nm,\mathbf{k}}\right]+rest\end{equation}
where $\hat{H}_{2}$ are the Coulomb terms in \ref{eq:Full_H_1-1}.
The evaluation is straight forward but lengthy: For each four-operator
term, one writes $\hat{H}_{2}\hat{p}_{nm,\mathbf{k}}-\hat{p}_{nm,\mathbf{k}}\hat{H}_{2}$
and then permutes $\hat{p}_{nm,\mathbf{k}}$ in the first term to
the left to obtain the remaining terms which do not cancel finally.
The remaining terms are resorted at the end into the natural ordering.
The resulting equation of motion is given by

\begin{eqnarray}
\frac{d}{dt}\hat{p}_{nm,\mathbf{k}} & = & -\frac{i}{\hbar}\left\{ E_{m}(\mathbf{k})-\left(E_{n}(\mathbf{k})-\sum_{\mathbf{q}}\Theta_{-\mathbf{q},-\mathbf{k}+\mathbf{q},\mathbf{k}}^{n,n}\right)\right\} \hat{p}_{nm,\mathbf{k}}\label{eq:Coulomb_Inter_Polarization_Eq_1}\\
 &  & -\frac{i}{\hbar}\mathbf{E}(z,t)\cdot\boldsymbol{\mu}_{mn,\mathbf{k}}\left(\sum_{c}\hat{a}_{c\mathbf{k}}^{\dagger}\hat{a}_{m\mathbf{k}}+\sum_{v}\hat{b}_{v-\mathbf{k}}^{\dagger}\hat{b}_{n-\mathbf{k}}-1\right)\label{eq:Coulomb_Inter_Polarization_Eq_2}\\
 &  & +\frac{i}{2\hbar}\sum_{c,\mathbf{k}',\mathbf{q}}\left(\Theta_{-\mathbf{q},\mathbf{k}+\mathbf{q},\mathbf{k}'}^{m,c}+\Theta_{\mathbf{q},\mathbf{k}',\mathbf{k}+\mathbf{q}}^{c,m}\right)\hat{a}_{c\mathbf{k}'+\mathbf{q}}^{\dagger}\hat{b}_{n-\mathbf{k}}\hat{a}_{c\mathbf{k}'}\hat{a}_{m\mathbf{k}+\mathbf{q}}\label{eq:Coulomb_Inter_Polarization_Eq_3}\\
 &  & +\frac{i}{2\hbar}\sum_{v,\mathbf{k}',\mathbf{q}}\left(\Theta_{-\mathbf{q},-\mathbf{k}'+\mathbf{q},\mathbf{k}}^{v,n}+\Theta_{\mathbf{q},\mathbf{k},-\mathbf{k}'+\mathbf{q}}^{n,v}\right)\hat{b}_{v\mathbf{k}'-\mathbf{q}}^{\dagger}\hat{b}_{v\mathbf{k}'}\hat{a}_{m\mathbf{k}}\hat{b}_{n-\mathbf{k}-\mathbf{q}}\label{eq:Coulomb_Inter_Polarization_Eq_4}\\
 &  & -\frac{i}{2\hbar}\sum_{c,\mathbf{k}',\mathbf{q}}\left(\Theta_{-\mathbf{q},\mathbf{k},\mathbf{k}'}^{n,c}+\Theta_{\mathbf{q},\mathbf{k}',\mathbf{k}}^{c,n}\right)\hat{a}_{c\mathbf{k}'+\mathbf{q}}^{\dagger}\hat{b}_{n-\mathbf{k}+\mathbf{q}}\hat{a}_{c\mathbf{k}'}\hat{a}_{m\mathbf{k}}\label{eq:Coulomb_Inter_Polarization_Eq_5}\\
 &  & -\frac{i}{2\hbar}\sum_{v,\mathbf{k}',\mathbf{q}}\left(\Theta_{-\mathbf{q},-\mathbf{k}'+\mathbf{q},\mathbf{k}-\mathbf{q}}^{v,m}+\Theta_{\mathbf{q},\mathbf{k}-\mathbf{q},-\mathbf{k}'+\mathbf{q}}^{m,v}\right)\hat{b}_{v\mathbf{k}'-\mathbf{q}}^{\dagger}\hat{b}_{n-\mathbf{k}}\hat{b}_{v\mathbf{k}'}\hat{a}_{m\mathbf{k}-\mathbf{q}}\label{eq:Coulomb_Inter_Polarization_Eq_6}\\
 &  & +\frac{i}{2\hbar}\sum_{\mathbf{q}}\left(\Theta_{-\mathbf{q},\mathbf{k},\mathbf{k}-\mathbf{q}}^{n,m}+\Theta_{\mathbf{q},\mathbf{k}-\mathbf{q},\mathbf{k}}^{m,n}\right)\hat{b}_{n-\mathbf{k}+\mathbf{q}}\hat{a}_{m\mathbf{k}-\mathbf{q}}.\label{eq:Coulomb_Inter_Polarization_Eq_7}\end{eqnarray}
Taking the two-band approximation of this particular equation and
ignoring the $\mathbf{k}$ dependence of the Coulomb matrix element
results in Eq. (3.9) of ref. \citet{Chow}. Comparing this equation
with the free-carrier result \ref{eq:SC_Bloch_Eq_1} reveals that
in \ref{eq:Coulomb_Inter_Polarization_Eq_1}, the valence band energy
is shifted to\begin{equation}
E_{n}(\mathbf{k})\rightarrow E_{n}(\mathbf{k})-\sum_{\mathbf{q}}\Theta_{-\mathbf{q},-\mathbf{k}+\mathbf{q},\mathbf{k}}^{n,n}.\end{equation}
The values usually inserted into $E_{v}(\mathbf{k})$ are results
of a preceding band structure calculation, where parameters are obtained
from low excitation experiments, where the valence band is full \citet{Chow}.
Therefore, the $E_{v}(\mathbf{k})$ actually already holds the Coulomb
energy of the full valence band. The free-carrier calculation implicitly
included this energy within the band structure result, while it appears
here explicitly.

Similar equations of motion can be obtained for the carrier population
operators 

The terms \ref{eq:Coulomb_Inter_Polarization_Eq_3} and \ref{eq:Coulomb_Inter_Polarization_Eq_4}
are the result of the equation of motion of the conduction-conduction
and valence-valence terms \ref{eq:Full_H_3}, \ref{eq:Full_H_4} and
\ref{eq:Full_H_5}. The terms \ref{eq:Coulomb_Inter_Polarization_Eq_5}-\ref{eq:Coulomb_Inter_Polarization_Eq_7}
are the results of the commutator with the conduction-valence terms
\ref{eq:Full_H_6}. The term \ref{eq:Coulomb_Inter_Polarization_Eq_7}
is obtained at the end when \ref{eq:Coulomb_Inter_Polarization_Eq_6}
is brought into proper order.%
\begin{figure}
\begin{centering}
\includegraphics[scale=0.25]{\string"C:/Users/Yossi Michaeli/Documents/Thesis/Documents/Thesis/Figures/4_Infinite_Operator_Hirarchy\string".eps}
\par\end{centering}

\caption{\label{fig:Infinite_Hierarchy_Operators}Infinite hierarchy of operator
products for the equtions of motion.}

\end{figure}


The obtained equation of motion for the two-particle operator term
$\hat{p}_{nm,\mathbf{k}}$ obviously depends on the four-particle
operator terms. Therefore, it would be necessary to calculate the
equation of motion of the four-particle operator term which would
couple to a six-particle operator term: one obtains an infinite hierarchy
of equations, as demonstatred in Fig. \ref{fig:Infinite_Hierarchy_Operators}.
For this reason, one usually truncates the hierarchy at a certain
order \citet{Chow}. Factorizing the four-particle operator into two
two-particle operators yields the\emph{ Hartree-Fock approximation},
which is considered in the next section.


\subsection{Hartree-Fock Approximation}

In the next step, all four-particle operators are factorized into
all possible products of two-particle operators, of which the expectation
values are taken. While permuting these operators, the correct sign
due to the anti-commutation relations has to be tracked. Due to the
random-phase approximation \ref{eq:Random_Phase_Apprx}, all expectation
values of two-particle operators reduce either to microscopic polarizations
$p_{nm,\mathbf{k}}$, particle densities $n_{n,\mathbf{k}}$ or to
zero. We summarize this approach for a two-operator expectation value%
\begin{figure}
\begin{centering}
\includegraphics[scale=0.25]{\string"C:/Users/Yossi Michaeli/Documents/Thesis/Documents/Thesis/Figures/4_HF_Operator_Approximation\string".eps}
\par\end{centering}

\caption{\label{fig:Schematic_HF_Operator_Approximation}Schematic representation
of the Hartree-Fock approximation operator product in the equation
of motion.}



\end{figure}
 \begin{eqnarray}
\frac{d}{dt}\left\langle AB\right\rangle  & = & \frac{d}{dt}\left\langle AB\right\rangle _{HF}+\left(\frac{d}{dt}\left\langle AB\right\rangle -\frac{d}{dt}\left\langle AB\right\rangle _{HF}\right)\nonumber \\
 & = & \frac{d}{dt}\left\langle AB\right\rangle _{HF}+\frac{d}{dt}\left\langle AB\right\rangle _{col.}\end{eqnarray}
where $\left\langle AB\right\rangle _{HF}$ is the Hartree-Fock contribution
to the equation if motion which will be treated in this chapter (see
Fig. \ref{fig:Schematic_HF_Operator_Approximation}). The term $\left\langle AB\right\rangle _{col.}$
us a higher-order contribution containing four-operator expectation
values and they will be discussed briefly later in this chapter. Here,
we only mention that with the full many-body Hamiltonian \ref{eq:Separated_Hamiltonian-1},
the Heisenberg equation of motion gives the equation of motion for
$\left\langle ABCD\right\rangle $ as \begin{equation}
\frac{d}{dt}\left\langle ABCD\right\rangle =\frac{d}{dt}\left\langle ABCD\right\rangle _{F}+\left(\frac{d}{dt}\left\langle ABCD\right\rangle -\frac{d}{dt}\left\langle ABCD\right\rangle _{F}\right),\end{equation}
where $\frac{d}{dt}\left\langle ABCD\right\rangle $ contains expectation
values of products of up to six operators. The label $F$ is used
to indicate the result from a Hartree-Fock factorization of the four
and six operator expectation values. We can continue by deriving the
equation of motion for \begin{equation}
\left\langle ABCDEF\right\rangle \equiv\left(\frac{d}{dt}\left\langle ABCD\right\rangle -\frac{d}{dt}\left\langle ABCD\right\rangle _{F}\right)\end{equation}
and so on. The result is a hierarchy of equations, where each succeeding
equation describes a correlation contribution that is of higher order
than the one before. In practice, we truncate the hierarchy at some
point. 

As an example, the factorization of \ref{eq:Coulomb_Inter_Polarization_Eq_3}
leads to\begin{eqnarray*}
\left\langle \hat{a}_{c\mathbf{k}'+\mathbf{q}}^{\dagger}\hat{b}_{n-\mathbf{k}}\hat{a}_{c\mathbf{k}'}\hat{a}_{m\mathbf{k}+\mathbf{q}}\right\rangle _{HF} & = & \left\langle \hat{a}_{c\mathbf{k}'+\mathbf{q}}^{\dagger}\hat{b}_{n-\mathbf{k}}\right\rangle \left\langle \hat{a}_{c\mathbf{k}'}\hat{a}_{m\mathbf{k}+\mathbf{q}}\right\rangle \\
 &  & -\left\langle \hat{a}_{c\mathbf{k}'+\mathbf{q}}^{\dagger}\hat{a}_{c\mathbf{k}'}\right\rangle \left\langle \hat{b}_{n-\mathbf{k}}\hat{a}_{m\mathbf{k}+\mathbf{q}}\right\rangle \\
 &  & +\left\langle \hat{a}_{c\mathbf{k}'+\mathbf{q}}^{\dagger}\hat{a}_{m\mathbf{k}+\mathbf{q}}\right\rangle \left\langle \hat{b}_{n-\mathbf{k}}\hat{a}_{c\mathbf{k}'}\right\rangle \\
 & = & 0-\delta_{\mathbf{q}0}+\delta_{\mathbf{k}'\mathbf{k}}\delta_{cm}n_{m,\mathbf{k}+\mathbf{q}}p_{nm,\mathbf{k}}.\end{eqnarray*}
Other terms are factorized accordingly. The factorized terms of \ref{eq:Coulomb_Inter_Polarization_Eq_3}
and \ref{eq:Coulomb_Inter_Polarization_Eq_4} can be added to the
transition energy, while the remaining terms are proportional to the
inversion factor. Finally, the equation for the microscopic polarization
in the Hartree-Fock approximation is obtained as\begin{eqnarray}
\frac{d}{dt}p_{nm,\mathbf{k}} & = & -\frac{i}{\hbar}\hbar\tilde{\omega}_{mn}(\mathbf{k})p_{nm,\mathbf{k}}\nonumber \\
 &  & -\frac{i}{\hbar}\left\{ \mathbf{E}(z,t)\cdot\boldsymbol{\mu}_{mn,\mathbf{k}}+\sum_{\mathbf{q}}\Theta_{-\mathbf{q},\mathbf{k},\mathbf{k}-\mathbf{q}}^{n,m}p_{nm,\mathbf{k}-\mathbf{q}}\right\} \left(n_{m\mathbf{k}}+n_{n\mathbf{k}}-1\right)\nonumber \\
 &  & +\frac{\partial}{\partial t}\left.p_{nm,\mathbf{k}}\right|_{col.},\label{eq:HF_Polarization_Eq_Motion}\end{eqnarray}
and the renormalized transition energy is given as\begin{eqnarray}
\hbar\tilde{\omega}_{mn}(\mathbf{k}) & = & E_{m}(\mathbf{k})-\left(E_{n}(\mathbf{k})-\sum_{\mathbf{q}}\Theta_{-\mathbf{q},-\mathbf{k}+\mathbf{q},\mathbf{k}}^{n,n}\right)\nonumber \\
 &  & -\sum_{\mathbf{q}}\left(\Theta_{\mathbf{q},\mathbf{k},\mathbf{k}+\mathbf{q}}^{m,m}n_{m,\mathbf{k}+\mathbf{q}}+\Theta_{\mathbf{q},\mathbf{k},\mathbf{k}+\mathbf{q}}^{n,n}n_{n,\mathbf{k}+\mathbf{q}}\right).\label{eq:Transition_Energy}\end{eqnarray}
The sum term in \ref{eq:Transition_Energy} is called the \emph{exchange
shift} term. Inspecting \ref{eq:HF_Polarization_Eq_Motion} reveals
that the microscopic polarization $p_{nm,\mathbf{k}}$ at $\mathbf{k}$
is now coupled to the one at $\mathbf{k}'$ by the Coulomb interaction,
while in the free-carrier expression, the microscopic polarization
was entirely determined by the momentum matrix element and the inversion.
This coupling introduces the excitonic effects such as enhancing gain
or absorption. The higher-order correlation (collisions) contributions
are denoted similarly to previous chapter with the subscript col.
The expressions for the carrier distribution operators $\hat{n}_{n\mathbf{k}}$
and $\hat{n}_{m\mathbf{k}}$ can be obtained similarly. By introducing
\emph{renormalized Rabi frequency} \begin{equation}
\Omega_{\mathbf{k}}(z,t)=\mathbf{E}(z,t)\cdot\boldsymbol{\mu}_{mn,\mathbf{k}}+\sum_{\mathbf{q}}\Theta_{-\mathbf{q},\mathbf{k},\mathbf{k}-\mathbf{q}}^{n,m}p_{nm,\mathbf{k}-\mathbf{q}},\label{eq:Renormalized_Rabi_Freq}\end{equation}
we can rewrite the semicondictor Bloch equations as\begin{eqnarray}
\frac{d}{dt}p_{nm,\mathbf{k}} & = & -i\tilde{\omega}_{mn}(\mathbf{k})p_{nm,\mathbf{k}}-i\Omega_{\mathbf{k}}(z,t)\left(n_{m\mathbf{k}}+n_{n\mathbf{k}}-1\right)+\frac{\partial}{\partial t}\left.p_{nm,\mathbf{k}}\right|_{col.},\label{eq:HF_Semicondictor_Bloch_Eq_1}\\
\frac{d}{dt}n_{n\mathbf{k}} & = & i\left[\Omega_{\mathbf{k}}(z,t)p_{nm,\mathbf{k}}^{*}-\Omega_{\mathbf{k}}^{*}(z,t)p_{nm,\mathbf{k}}\right]+\frac{\partial}{\partial t}\left.n_{n\mathbf{k}}\right|_{col.},\label{eq::HF_Semicondictor_Bloch_Eq_2}\\
\frac{d}{dt}n_{m\mathbf{k}} & = & i\left[\Omega_{\mathbf{k}}(z,t)p_{nm,\mathbf{k}}^{*}-\Omega_{\mathbf{k}}^{*}(z,t)p_{nm,\mathbf{k}}\right]+\frac{\partial}{\partial t}\left.n_{m\mathbf{k}}\right|_{col.}.\label{eq::HF_Semicondictor_Bloch_Eq_3}\end{eqnarray}
The second term in \ref{eq:Renormalized_Rabi_Freq} is called the
\emph{Coulomb field renormalization}.

The semiconductor Bloch equations look like the two-level Bloch equations,
with the exceptions that the transition energy and the electric-dipole
interaction are renormalized, and the carrier probabilities $n_{n\mathbf{k}}$
enter instead of the probability difference between upper and lower
levels. The renormalizations are due to the many-body Coulomb interactions,
and they couple equations for different $\mathbf{k}$ states. This
coupling leads to significant complications in the evaluation of \ref{eq:HF_Semicondictor_Bloch_Eq_1}-\ref{eq::HF_Semicondictor_Bloch_Eq_3}
in comparison to the corresponding free-carrier equations, \ref{eq:SC_Bloch_Eq_1}-\ref{eq:SC_Bloch_Eq_3}.
If all Coulomb-potential contributions are dropped, the semiconductor
Bloch equations reduce to the undamped inhomogeneously broadened two-level
Bloch equations \citet{Chow}. Of course, this limit is unacceptable
for semiconductors. 


\subsection{Many-Body Effects}

There are two important many-body effects that do not appear in the
Hartree-Fock limit of the equation of motion, the collisions between
particles and the plasma screening. They occur at high carrier densities.
The main consequences of the collisions between particles are: 
\begin{itemize}
\item A decay of the microscopic polarization $p_{nm,\mathbf{k}}$ that
we can model with a relaxation time $\gamma$. This implies $\frac{\partial}{\partial t}\left.p_{nm,\mathbf{k}}\right|_{col.}=-\gamma p_{nm,\mathbf{k}}$. 
\item A rapid equilibration of electrons and holes into Fermi-Dirac distributions.
We can model this, assuming that the conduction band and the valence
band are two carrier reservoirs, which are Fermi-distributed, i.e.
$n_{e\mathbf{k}}=f_{e\mathbf{k}}$ and $n_{h\mathbf{k}}=f_{h\mathbf{k}}$.
\end{itemize}
It is more difficult to include the plasma screening effect into \ref{eq:HF_Semicondictor_Bloch_Eq_1}-\ref{eq::HF_Semicondictor_Bloch_Eq_3}.
The procedure is outlined in this subsection. 


\subsubsection{Background Screening}

In the crystal electron Hamiltonian \ref{eq:crystal_H_final}, the
interaction between ion cores and the interaction between electrons
and ion cores is merged into the potential $U(\mathbf{r})$, while
the Coulomb energy between electrons has been retained and later transformed
into the Bloch states. With the introduction of $\mathbf{k}\cdot\mathbf{p}$
states, only a restricted number of valence electrons are considered,
while core electrons are included implicitly. These implicitly included
core electrons and ion cores screen the Coulomb interaction between
explicitly included electrons. The effect can be considered by replacing
the dielectric permittivity of the vacuum $\epsilon_{0}$ with the
static semiconductor crystal background permittivity $\epsilon_{b}$
in the Coulomb matrix element \ref{eq:Coulomb_Operator_Matrix_Element}.


\subsubsection{Plasma Screening}

A profound weakness of the Hartree-Fock approximation is the lack
of screening of Coulomb interactions at elevated carrier densities.
The presence of unbound carriers of the electron-hole plasma leads
to an adjustment of the carriers to a charge, effectively screening
it and thereby reducing the Coulomb interaction energy. The lack of
screening in the Hartree-Fock equations is caused by the early truncation
of the infinite hierarchy of equations of motion. One therefore introduces
the screening effect phenomenologically using a dynamic dielectric
function $\epsilon(\mathbf{q},\omega)$\begin{equation}
V_{s\mathbf{q}}=\frac{V_{\mathbf{q}}}{\epsilon(\mathbf{q},\omega)},\end{equation}
where $V_{s\mathbf{q}}$ denotes the screened potential and $V_{\mathbf{q}}$
the unscreened one%
\footnote{Note that the Coulomb interaction Hamiltonian with the bare Coulomb
potential already contains the mechanism for plasma screening. Therefore,
one should be concerned that an \emph{ad hoc} replacement of $V_{\mathbf{q}}$
with $V_{s\mathbf{q}}$ might count some screening effects twice.
Such problems can only be avoided within a systematic many-body approach. %
}. There are several model dielectric functions, where all have their
strengths, limitations and weaknesses. One standard approach is the
\emph{Lindhard's formula} \citet{Chow,Haug2009},\begin{equation}
\epsilon(\mathbf{q},\omega)=1-V_{\mathbf{q}}\sum_{n\mathbf{k}}\frac{n_{n,\mathbf{k}-\mathbf{q}}-n_{n,\mathbf{k}}}{E_{n}(\mathbf{k}-\mathbf{q})-E_{n}(\mathbf{k})+\hbar\omega+\hbar i\delta},\label{eq:Full_Lindhards_Eq}\end{equation}
also denoted as randomphase approximation. The Lindhard formula can
be derived using the self-consistent field (SCF) approach \citet{Ehrenreich1959},
or more profound, by neglecting vertex corrections in the equation
of motion for the Green\textquoteright{}s function \citet{Binder1995}
of the designated system and using the Kadanoff-Baym formalism \citet{Binder1995}.
The derivation using the SCF approach involves the reaction of a homogeneous
electron gas to a test charge (see detailed derivation in Appendix
\ref{cha:Appendix_Screening}). To use Lindhard's formula in optical
calculations, the test charge potential $V_{\mathbf{q}}$ is replaced
with the Coulomb potential $\Theta_{\mathbf{q},\mathbf{k},\mathbf{k}'}^{n_{1},n_{2},n_{3},n_{4}}$
defined in \ref{eq:Coulomb_Interaction_Alternative}, including the
factor due to the finite extend of the wavefunctions.

There are several far better models \citet{mahan_many-particle_1997}
for the dielectric screening but the Lindhard formula is still popular
and commonly used in optoelectronic modeling \citet{Chow} due to
its simple structure. For many practical applications, the damped
response of the screening is ignored, therefore setting $(\lyxmathsym{\textgreek{w}}+i\lyxmathsym{\textgreek{d}})$
to zero. Once $\epsilon(\mathbf{q},0)$ is obtained, one replaces
the Coulomb potential within all equations with \begin{equation}
\Theta_{s,\mathbf{q},\mathbf{k},\mathbf{k}'}^{n_{1},n_{2},n_{3},n_{4}}=\frac{1}{\epsilon(\mathbf{q},0)}\Theta_{\mathbf{q},\mathbf{k},\mathbf{k}'}^{n_{1},n_{2},n_{3},n_{4}}.\end{equation}
Beside its physical importance, the screening introduces the desired
effect of removing the divergence of the Coulomb potential with $\mathbf{q}\rightarrow0$,
thereby facilitating the numerical evaluation.%
\begin{figure}
\begin{centering}
\includegraphics{\string"C:/Users/Yossi Michaeli/Documents/Thesis/Documents/Thesis/Figures/4_Screened_HF_Procedurel\string".eps}
\par\end{centering}

\caption{\label{fig:Screened_HF_Outline}Outline of an approach that takes
advantage of a phenomenological derivation of plasma screening, which
is incorporated directly into the many-body Hamiltonian to give the
screened Hartree-Fock equations (after \citet{Chow}). }



\end{figure}


The outline of the screened Hatree-Fock approach is presented in Fig.
\ref{fig:Screened_HF_Outline}.


\subsubsection{Coulomb Hole Self Energy}

By replacing the electron operators with hole operators in \ref{eq:Coulomb_Inter_Polarization_Eq_1},
the valence band energy was shifted by a constant value,\begin{equation}
E_{n}(\mathbf{k})\rightarrow E_{n}(\mathbf{k})-\sum_{\mathbf{q}}\Theta_{-\mathbf{q},-\mathbf{k}+\mathbf{q},\mathbf{k}}^{n,n}.\end{equation}
This shift is already implicitly included in the band structure. In
the high density limit, the Coulomb interaction is screened\begin{equation}
\Theta_{s,-\mathbf{q},-\mathbf{k}+\mathbf{q},\mathbf{k}}^{n,n}=\frac{\Theta_{-\mathbf{q},-\mathbf{k}+\mathbf{q},\mathbf{k}}^{n,n}}{\epsilon(\mathbf{q},0)}\end{equation}
and the difference between the unscreened and the screened Coulomb
interaction leads to a density dependent shift\begin{equation}
\Delta E_{CH,n}=\sum_{\mathbf{q}}\Theta_{-\mathbf{q},-\mathbf{k}+\mathbf{q},\mathbf{k}}^{n,n}\left(\frac{1}{\epsilon(\mathbf{q})}-1\right).\label{eq:Coulomb_Hole_Self_Energy}\end{equation}
The shift is denoted as \emph{Debye shift} or \emph{Coulomb-hole self
energy}.


\subsubsection{Screened Exchange Energy}

Beside the Coulomb-hole self energy, the transition energy \ref{eq:Transition_Energy}
is, compared to the free carrier result, reduced by the density dependent
\emph{screened exchange shift},\begin{equation}
\Delta E_{SX,mn}(\mathbf{k})=\sum_{\mathbf{q}}\left(\Theta_{\mathbf{q},\mathbf{k},\mathbf{k}+\mathbf{q}}^{m,m}n_{m,\mathbf{k}+\mathbf{q}}+\Theta_{\mathbf{q},\mathbf{k},\mathbf{k}+\mathbf{q}}^{n,n}n_{n,\mathbf{k}+\mathbf{q}}\right).\label{eq:Screened_Exchange_Shift}\end{equation}
where the screened Coulomb potentials are used. Within the screened
Hartree-Fock limit, the transition energy is therefore renormalized\begin{equation}
\hbar\tilde{\omega}_{mn}(\mathbf{k})=\Delta E_{mn}(\mathbf{k})+\Delta E_{CH,n}-\Delta E_{SX,mn}(\mathbf{k}).\label{eq:Band_Gap_Renormalization}\end{equation}
Note that both contributions \ref{eq:Coulomb_Hole_Self_Energy} and
\ref{eq:Screened_Exchange_Shift} reduce the transition energy and
can lead to a significant red shift.


\subsection{Solving the Correlated Equation}

In order to solve the equation of motion \ref{eq:HF_Polarization_Eq_Motion},
the collision term is approximated using a decay rate $-\gamma p_{mn,\mathbf{k}}$.
As equation \ref{eq:HF_Polarization_Eq_Motion} couples microscopic
polarizations $p_{mn,\mathbf{k}}$ of different $\mathbf{k}$ values,
the equation system has to be solved self-consistently. Therefore,
the equation is transformed using the same ansatz as in Chapter \ref{cha:Free_Carriers_Optical_Transitions}.
The fast oscillating $p_{mn,\mathbf{k}}$ is replaced with its slowly-varying
envelope\begin{equation}
s_{nm,\mathbf{k}}=p_{mn,\mathbf{k}}e^{-i\left(k_{0}z-\nu t-\phi(z)\right)}.\label{eq:Polarization_Slow_Envelope_Def}\end{equation}
Inserting \ref{eq:Polarization_Slow_Envelope_Def} into \ref{eq:HF_Polarization_Eq_Motion},
taking the time derivative, using the plane wave expression for the
electrical field \ref{eq:Electrical_Field_Ansatz}, skipping the fast
oscillating parts (as they should average out to zero) and solving
for the steady state leads to \begin{eqnarray}
s_{nm,\mathbf{k}} & = & -\frac{1}{i\left(\tilde{\omega}_{mn}(\mathbf{k})-\nu\right)+\gamma}\frac{i}{\hbar}\nonumber \\
 &  & \cdot\left\{ \mu_{mn,\mathbf{k}}\frac{E(z)}{2}+\sum\Theta_{\mathbf{q},\mathbf{k},\mathbf{k}+\mathbf{q}}^{n,m}s_{nm,\mathbf{k}+\mathbf{q}}\right\} \label{eq:Polarization_Slow_Envelope_Eq}\\
 &  & \cdot\left(n_{m\mathbf{k}}+n_{n\mathbf{k}}-1\right).\nonumber \end{eqnarray}
The polarization envelope still depends on the intensity of the considered
field. Therefore, the field independent solution variable is introduced,
given by\begin{equation}
\lambda_{nm,\mathbf{k}}=\frac{2s_{nm,\mathbf{k}}}{E(z)}.\label{eq:Field_Independent_Sol_Var}\end{equation}
In order to obtain a practicable expression, the lineshape and the
inversion factor are combined into\begin{equation}
\Lambda_{nm,\mathbf{k}}=\frac{i}{\hbar}\frac{\left(n_{m\mathbf{k}}+n_{n\mathbf{k}}-1\right)}{i\left(\tilde{\omega}_{mn,\mathbf{k}}-\nu\right)+\gamma}.\label{eq:Lineshape_Inversion_Factor}\end{equation}
Inserting \ref{eq:Field_Independent_Sol_Var} and \ref{eq:Lineshape_Inversion_Factor}
into \ref{eq:Polarization_Slow_Envelope_Eq}, yields\begin{equation}
\Lambda_{nm,\mathbf{k}}\sum\Theta_{\mathbf{q},\mathbf{k},\mathbf{k}+\mathbf{q}}^{n,m}\lambda_{nm,\mathbf{k}+\mathbf{q}}+\lambda_{nm,\mathbf{k}}=-\mu_{mn,\mathbf{k}}\Lambda_{nm,\mathbf{k}}.\end{equation}
This equation is the self-consistency equation for $\lambda_{nm,\mathbf{k}}$.
Once solved, the optical susceptibility is calculated using\begin{equation}
\chi(\nu)=\frac{1}{n_{b}^{2}\epsilon_{0}}\frac{P(z)}{E(z)}=\frac{1}{n_{b}^{2}\epsilon_{0}}\frac{1}{V}\sum_{n,m,\mathbf{k}}\mu_{mn,\mathbf{k}}^{*}\lambda_{nm,\mathbf{k}}.\label{eq:HF_Susceptability}\end{equation}
Absorption, refractive index change and spontaneous emission can therefrom
be calculated as already presented in the preceding Chapter \ref{cha:Free_Carriers_Optical_Transitions}.
In the next section we present the numerical implementation details
of the solution. 


\subsection{Numerical Implementation}

As a consequence of the collisions between particles in the assumed
equalibrium state, we consider the conduction band and the valence
band as two carriers reservoirs with Fermi-Dirac distributions for
the electrons and holes. Thus, we only need to solve the equation
of motion for the polarization $\hat{p}_{nm,\mathbf{k}}$. Using the
decay model for the microscopic polarization we obtain\begin{equation}
\frac{d}{dt}p_{nm,\mathbf{k}}=-i\tilde{\omega}_{mn}(\mathbf{k})p_{nm,\mathbf{k}}-i\Omega_{\mathbf{k}}(z,t)\left(f_{e\mathbf{k}}^{n}+f_{h\mathbf{k}}^{m}-1\right)-\gamma p_{nm,\mathbf{k}},\label{eq:HF_Polarization_Eq_Motion_Numerics}\end{equation}
where $n$ and $m$ denote the conduction and valence subbands, respectvely,
involved in the transition. As for the free carrier model, the electron
and hole densities are given as input parameters, and the chemical
potentials, $E_{Fc}$ and $E_{Fv}$, are calculated as described in
section \ref{sub:Carrier_Statistics}. 

As the first step for the calculation, the dielectric function $\epsilon(\mathbf{q})$,
the Coulomb-hole self energy $\Delta E_{CH,n}$ and the screened exchange
shift energy $\Delta E_{SX,mn}(\mathbf{k})$ must be evaluated. 
\begin{itemize}
\item It is more convinient to evaluate $q\epsilon(\mathbf{q})$ than only
$\epsilon(\mathbf{q})$. With Fermi-Dirac electron and hole distributions,
we have from \ref{eq:Full_Lindhards_Eq} \begin{equation}
q\epsilon(\mathbf{q})=q-\frac{e^{2}}{2\epsilon_{r}A}\sum_{n\mathbf{k}}\frac{f_{n,\mathbf{k}-\mathbf{q}}-f_{n,\mathbf{k}}}{E_{n}(\mathbf{k}-\mathbf{q})-E_{n}(\mathbf{k})}.\end{equation}
Converting the sum over the $\mathbf{k}$-states into an integral,
adding the sum over the spin, choosing $\theta$ as the angle between
$\mathbf{k}$ and $\mathbf{q}$ and assuming no angular dependence
for the subband structure, we obtain\begin{equation}
q\epsilon(q)=q-\frac{e^{2}}{4\pi^{2}\epsilon_{r}}\sum_{n}\int_{0}^{\infty}dk\, k\int_{0}^{2\pi}d\theta\frac{f_{n,\left|\mathbf{k}-\mathbf{q}\right|}-f_{n,\mathbf{\left|k\right|}}}{E_{n}\left(\left|\mathbf{k}-\mathbf{q}\right|\right)-E_{n}(\mathbf{\left|k\right|})}.\end{equation}
with \begin{equation}
\left|\mathbf{k}'\right|=\left|\mathbf{k}-\mathbf{q}\right|=\left(k^{2}+q^{2}-2kq\cos\theta\right)^{\frac{1}{2}}.\end{equation}
The integrals can be eveluated here, as before, using the trapezoidal
method for integration. 
\item For the Coulomb-hole self energy with Fermi-Dirac distributions we
can write =\begin{eqnarray}
\Delta E_{CH,n} & = & \sum_{\mathbf{q}\neq0}\Theta_{-\mathbf{q},-\mathbf{k}+\mathbf{q},\mathbf{k}}^{n,n}\left(\frac{1}{\epsilon(\mathbf{q})}-1\right)\nonumber \\
 & = & \frac{e^{2}}{4\pi\epsilon_{r}}\int_{0^{+}}^{\infty}dq\, G_{\mathbf{q}}^{n,n}\left(\frac{1}{\epsilon(q)}-1\right).\end{eqnarray}

\item The screened-exchange shift energy becomes\begin{equation}
\Delta E_{SX,nm}(\mathbf{k})=\sum_{\mathbf{k}'\neq\mathbf{k}}\left(\Theta_{s,\mathbf{q}}^{n,n}f_{e,\mathbf{k}'}^{n}+\Theta_{s,\mathbf{q}}^{m,m}f_{h,\mathbf{k}'}^{m}\right).\end{equation}
The band indices $m$ and $n$ are defined as before for the interacting
subbands. If we convert the sum over $\mathbf{k}'$ into an integral
with $\theta'$ defining the angle between $\mathbf{k}$ and $\mathbf{k}'$
and we assume a rotationally symmetric band structure, we obtain\begin{equation}
E_{SX,nm}(\mathbf{k})=\frac{e^{2}}{8\pi^{2}\epsilon_{r}}\int_{0}^{\infty}dk'\int_{0}^{2\pi}d\theta'\,\frac{k'}{q\epsilon(q)}\left(G_{\mathbf{q}}^{n,n}f_{e,\mathbf{k}'}^{n}+G_{\mathbf{q}}^{m,m}f_{h,\mathbf{k}'}^{m}\right)\end{equation}
where \begin{equation}
\left|\mathbf{q}\right|=\left|\mathbf{k}-\mathbf{k}'\right|=\left(k^{2}+k'^{2}-2kk'\cos\theta'\right)^{\frac{1}{2}}.\end{equation}

\end{itemize}
In all the expressions above we use \ref{eq:Coulomb_Interaction_Alternative_2D}
to describe the coulomb interaction for the 2D case. The form factor
is explicetly calculated using \begin{equation}
G_{\mathbf{q}}^{n,m}=\int_{-\infty}^{\infty}dz\int_{-\infty}^{\infty}dz'\left|F_{n,\mathbf{k}}(z)\right|^{2}\left|F_{m,\mathbf{k}}(z')\right|^{2}e^{-q\left|z-z'\right|},\end{equation}
where $F_{n,\mathbf{k}}(z)$ is the envelope function of the specific
subband considered.

With the knowledge of these terms, we can now concentrate on \ref{eq:HF_Polarization_Eq_Motion_Numerics}.
The last difficulty resides in $\Omega_{\mathbf{k}}(z,t)$. The first
step is to eliminate the rapidly varying phase factor from $p_{nm,\mathbf{k}}$
and to work with $s_{nm,\mathbf{k}}$ as in \ref{eq:Polarization_Slow_Envelope_Eq}\begin{eqnarray}
\frac{d}{dt}s_{nm,\mathbf{k}} & = & -\left(i\left(\tilde{\omega}_{n,m}(\mathbf{k})-\nu\right)+\gamma\right)s_{nm,\mathbf{k}}-\frac{i}{\hbar}\mu_{\mathbf{k}}\frac{E(z)}{2}w_{\mathbf{k}}\nonumber \\
 &  & -\frac{i}{\hbar}w_{\mathbf{k}}\sum_{\mathbf{k}\neq\mathbf{k}'}\Theta_{s,\mathbf{q}}^{n,m}s_{nm,\mathbf{k}'},\label{eq:HF_Slow_Polarization_Eq_Motion_Numerics}\end{eqnarray}
where we defined $w_{\mathbf{k}}=f_{e\mathbf{k}}^{n}+f_{h\mathbf{k}}^{m}-1$.
To obtain suscptability spectrum of a medium, \ref{eq:HF_Slow_Polarization_Eq_Motion_Numerics}
is solved for the steady state (as in the previous section), which
leads to \begin{equation}
s_{nm,\mathbf{k}}=-\frac{i}{\hbar}\frac{w_{\mathbf{k}}\mu_{\mathbf{k}}}{i\left(\tilde{\omega}_{n,m}(\mathbf{k})-\nu\right)+\gamma}\left(\frac{E(z)}{2}+\frac{1}{\mu_{\mathbf{k}}}\sum_{\mathbf{k}\neq\mathbf{k}'}\Theta_{s,\mathbf{q}}^{n,m}s_{nm,\mathbf{k}'}\right).\label{eq:Steady_State_Sol_Polarization}\end{equation}
We define two new variables, the $\mathbf{k}$-dependent complex susceptability
function \begin{equation}
\chi_{\mathbf{k}}^{0}=-\Lambda_{nm,\mathbf{k}}\mu_{\mathbf{k}},\end{equation}
and the Coulomb enhancement factor\begin{equation}
\Gamma_{\mathbf{k}}^{n,m}=\frac{\lambda_{nm,\mathbf{k}}}{\chi_{\mathbf{k}}^{0}},\end{equation}
where $\Lambda_{nm,\mathbf{k}}$ and $\lambda_{nm,\mathbf{k}}$ where
defined in the previous section. Substituting these two expressions
into \ref{eq:Steady_State_Sol_Polarization} we obtian\begin{equation}
\Gamma_{\mathbf{k}}^{n,m}(\nu)=1+\frac{1}{\mu_{\mathbf{k}}}\sum_{\mathbf{k}\neq\mathbf{k}'}\Theta_{s,\mathbf{q}}^{n,m}\chi_{\mathbf{k}'}^{0}\Gamma_{\mathbf{k}'}^{n,m}(\nu).\label{eq:Coulomb_Enhancement_Factor_1}\end{equation}
We consider the last part of this equation and transfor the sum over
$\mathbf{k}'$ into an integral as before. Furthermore, we assume
that the microscopic polarization $p_{nm,\mathbf{k}}$ has no angular
dependence\begin{equation}
\frac{1}{\mu_{\mathbf{k}}}\sum_{\mathbf{k}\neq\mathbf{k}'}\Theta_{s,\mathbf{q}}^{n,m}\chi_{\mathbf{k}'}^{0}\Gamma_{\mathbf{k}'}^{n,m}(\nu)=\frac{1}{\mu_{\mathbf{k}}}\int_{0}^{\infty}dk'\,\varepsilon_{k,k'}\chi_{k'}^{0}\Gamma_{k'}^{n,m}(\nu),\label{eq:Coulomb_Enhancement_Part}\end{equation}
where \begin{equation}
\varepsilon_{k,k'}=\frac{e^{2}}{2\pi^{2}\epsilon_{r}}k'\int_{0}^{2\pi}d\theta'\frac{G_{\mathbf{q}}^{n,m}}{q\epsilon(q)}.\end{equation}
Inserting \ref{eq:Coulomb_Enhancement_Part} into \ref{eq:Coulomb_Enhancement_Factor_1},
we obtain \begin{equation}
\Gamma_{k}^{n,m}(\nu)=1+\frac{1}{\mu_{k}}\int_{0}^{\infty}dk'\,\varepsilon_{k,k'}\chi_{k'}^{0}\Gamma_{k'}^{n,m}(\nu).\label{eq:Coulomb_Enhancement_Factor_2}\end{equation}
Various approaches exist in the literature to solve this self-consistent
equation \citet{Haug2009,Chow1994}. We choose to use the most general
matrix inversion method, in-spite of its tendency to be rater resource
consuming, as apposed to various approximate methods (such as Pade
and the dominant momentum methods). To this end we approximate the
integral over the $k$ space by a sum of trapezoidal areas and discretize
$k$ and $k'$ with $n$ values and constant step size $\Delta k$%
\footnote{An elternative choise for $\Delta k$ is to take the elements of the
$k$ vector as the support of a Gaussian qudrature \citet{haug_quantum_1994}. %
}. Eq. \ref{eq:Coulomb_Enhancement_Factor_2} can thus be written in
a matrix form\begin{equation}
\mathbf{M}\cdot\mathbf{\boldsymbol{\Gamma}}_{\mathbf{k}}^{n,m}=\mathbf{1},\label{eq:Coulomb_Enhancement_Factor_3}\end{equation}
with \begin{equation}
\mathbf{\boldsymbol{\Gamma}}_{\mathbf{k}}^{n,m}=\left(\Gamma_{k_{1}}^{n,m},\Gamma_{k_{2}}^{n,m},\cdots,\Gamma_{k_{n}}^{n,m}\right)^{T}\end{equation}
and \begin{equation}
\mathbf{M}=\left(\begin{array}{cccc}
1-\frac{\Delta_{k}}{\mu_{k_{1}}}\varepsilon_{k_{1},k_{1}'}\chi_{k_{1}'}^{0} & -\frac{\Delta_{k}}{\mu_{k_{1}}}\varepsilon_{k_{1},k_{2}'}\chi_{k_{2}'}^{0} & \cdots & -\frac{\Delta_{k}}{\mu_{k_{1}}}\varepsilon_{k_{1},k_{n}'}\chi_{k_{n}'}^{0}\\
-\frac{\Delta_{k}}{\mu_{k_{1}}}\varepsilon_{k_{1},k_{1}'}\chi_{k_{1}'}^{0} & 1-\frac{\Delta_{k}}{\mu_{k_{2}}}\varepsilon_{k_{2},k_{2}'}\chi_{k_{2}'}^{0} & \cdots & -\frac{\Delta_{k}}{\mu_{k_{2}}}\varepsilon_{k_{2},k_{n}'}\chi_{k_{n}'}^{0}\\
\vdots & \vdots & \ddots & \vdots\\
-\frac{\Delta_{k}}{\mu_{k_{n}}}\varepsilon_{k_{n},k_{1}'}\chi_{k_{1}'}^{0} & -\frac{\Delta_{k}}{\mu_{k_{n}}}\varepsilon_{k_{n},k_{2}'}\chi_{k_{2}'}^{0} & \cdots & 1-\frac{\Delta_{k}}{\mu_{k_{n}}}\varepsilon_{k_{n},k_{n}'}\chi_{k_{n}'}^{0}\end{array}\right).\end{equation}
Eq. \ref{eq:Coulomb_Enhancement_Factor_3} can now be solved using
standard mathematical methods to obtain $\mathbf{\boldsymbol{\Gamma}}_{\mathbf{k}}^{n,m}$.
The microscopic polarization can be obtain using \begin{equation}
p_{nm,\mathbf{k}}=\frac{1}{2}e^{-i\left(k_{0}z-\nu t-\phi(z)\right)}E(z)\chi_{k}^{0}\Gamma_{k}^{n,m}.\label{eq:HF_Final_Polarization}\end{equation}


Insering \ref{eq:HF_Final_Polarization} into \ref{eq:HF_Susceptability}
and using the definitions in chapter \ref{cha:Free_Carriers_Optical_Transitions},
\ref{eq:Intensity_Gain} and \ref{eq:Sp_Emission_Definition}, we
can write the expressions for the absorption (or the gain amplitude)
and spontanious emission spectra\begin{eqnarray}
\alpha_{HF}(\nu) & = & -\Im\left\{ \frac{i\nu}{\hbar c\epsilon_{0}n_{b}\pi\mathcal{L}}\sum_{n,m,\mathbf{k}}\left|\mu_{nm,\mathbf{k}}\right|^{2}\frac{w_{\mathbf{k}}}{i\left(\tilde{\omega}_{n,m}(\mathbf{k})-\nu\right)+\gamma}\Gamma_{\mathbf{k}}^{n,m}(\nu)\right\} \nonumber \\
 & = & -\Im\left\{ \frac{i\nu}{\hbar c\epsilon_{0}n_{b}\pi\mathcal{L}}\sum_{n,m}\int_{0}^{\infty}dk\, k\left|\mu_{nm,\mathbf{k}}\right|^{2}\frac{w_{k}}{i\left(\tilde{\omega}_{n,m}(\mathbf{k})-\nu\right)+\gamma}\Gamma_{\mathbf{k}}^{n,m}(\nu)\right\} \label{eq:HF_Absorption}\end{eqnarray}
and \begin{equation}
r_{sp,HF}(\nu)=-\frac{n_{b}^{2}(\hbar\nu)^{2}}{\pi^{2}\hbar^{3}c^{2}}\frac{1}{1-\exp\left(\left(\hbar\nu-\left(E_{Fc}-E_{Fv}\right)\right)/k_{B}T\right)}\alpha_{HF}(\nu),\label{eq:HF_Sp_Emission}\end{equation}
where $\Gamma_{\mathbf{k}}^{n,m}(\nu)$ are the solution of \ref{eq:Coulomb_Enhancement_Factor_3},
$\tilde{\omega}_{n,m}(\mathbf{k})$ is the renormalized bandgap defined
in \ref{eq:Band_Gap_Renormalization}, and $\mathcal{L}$ is the length
if the 2D quantum structure. 


\section{Beyond the Hartree-Fock Approximation}

%
\begin{lyxgreyedout}
Put here a short overview of the higher order effects...
\end{lyxgreyedout}
\selectlanguage{english}

