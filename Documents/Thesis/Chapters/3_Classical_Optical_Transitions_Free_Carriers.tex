\selectlanguage{english}%

\chapter{Free Carriers Optical Transitions}

\label{cha:Free_Carriers_Optical_Transitions}Absorption and emission
of photons in semiconductors and semiconductor nanostructures is the
result of the complex interaction between light and condensed matter,
i.e. photons, electrons, and ions. In order to capture the physics
behind these processes, electromagnetism and quantum mechanics have
to be combined, leading to a multitude of physical phenomena. This
thesis focuses on the static properties of nanostructures, such as
the constant emmision or absorption of light. Therefore, the theory
will be treated within the time-independent limit. The theory here
is formulated within the classical limit of optical transitions, where
the quantum-mechanically treated carriers couple to a classical electromagnetic
field. The resulting equations are famous and denoted as \emph{semiconductor-Bloch
equations} \citet{haug_quantum_1994}. These equations have been used
successfully over decades to describe optical properties within semiconductors.
A fully consistent treatment of the light-emission would in principle
require to work in a fully quantized picture (see the review \citet{Kira1999}),
leading to the \emph{semiconductor-luminescence equations}. Such a
treatment is beyond the scope of this thesis. The derivation presented
here serves to illustrate the imposed approximations and to clearly
document the implemented equations for nanostructures of any dimensionality
using the electronic structure obtained via the $\mathbf{k}\cdot\mathbf{p}$
envelope equations. This theory review is based mainly on \citet{Chuang1995,Jackson1998,Chow,Haug2009,Schafer2002}.

The chapter is organized as follows. The first section covers the
transformation of the crystal Hamiltonian \ref{eq:crystal_H_final}
into the second quantization. Then, the representation will be rewritten
in terms of the Bloch states, with a particular focus on the details
arising when electronic states obtained using the $\mathbf{k}\cdot\mathbf{p}$
envelope function method are used. In the present chapter, the Coulomb
interaction will be excluded and thereby, only transitions between
free carriers will be considered. Coulomb-correlated transitions will
be treated in Chapter \ref{cha:Coulomb_Correlated_Optical_Transitions}. 

The second section introduces the classical (monochromatic) light
field and the self-consistent coupling to the electrons within the
semiconductor nanostructure. The resulting equations allow to calculate
the optical susceptibility and therefrom absorption, stimulated emission
and refractive index change. The third section briefly explains how
spontaneous emission can be obtained from optical susceptibility.
The fourth section covers the procedure to calculate required matrix
elements from wavefunctions obtained using the $\mathbf{k}\cdot\mathbf{p}$
envelope equations.


\section{Second Quantization}


\subsection{Introduction}

The electrons are fermions and consequently their wavefunction must
be antisymmetric to obey the Pauli exclusion principle. But instead
of using a complicated expression for the antisymmetric wavefunction,
restricted to a constant number of particles, the Hamiltonian can
be formulated in the second quantization that allows to include a
varying number of particles and maintain the antisymmetry of the wavefunction
naturally. Starting with the Hamiltonian \ref{eq:crystal_H_final}
and separating the interaction with the external electromagnetic field
leads to\ \begin{eqnarray}
H & = & \sum_{i}\underbrace{\left(\frac{\mathbf{p}_{i}^{2}}{2m_{0}}+U(\mathbf{r}_{i})\right)}_{H_{1}}+\underbrace{\sum_{i}\left(\frac{e}{m_{0}}\mathbf{A}_{i}\cdot\mathbf{p}_{i}+\frac{e^{2}}{2m_{0}}\mathbf{A}_{i}^{2}\right)}_{H_{e-EM}}\nonumber \\
 &  & +\underbrace{\frac{1}{2}\sum_{i,j}\frac{e^{2}}{4\pi\epsilon_{0}\left|\mathbf{r}_{i}-\mathbf{r}_{j}\right|}}_{H_{2}}.\label{eq:Separated_Hamiltonian-1}\end{eqnarray}
Here the pure electro-magnetic Hamiltonian $H_{EM}$ has been dropped
as it constitutes within the present theory only an additional energy.
Let $\varphi_{n}$ be the eigenstates of the single particle Hamiltonian
$H_{1}$\begin{equation}
H_{1}\varphi_{n}=E\varphi_{n}\end{equation}
with resulting energies and wavefinctions. Next, the vacuum ground
state $\left|0\right\rangle $ and creation and annihilation operators
$\hat{a}_{n}^{\dagger}$ and $\hat{a}_{n}$ are introduced. These
operators create and destroy a particle with $\varphi_{n}$ and therefore
act on the state\begin{equation}
\hat{a}_{n}^{\dagger}\left|0\right\rangle =\left|1_{n}\right\rangle ,\,\,\hat{a}_{n}\left|1_{n}\right\rangle =\left|0\right\rangle ,\,\,\hat{a}_{n}\left|0\right\rangle =0,\,\,\hat{a}_{n}^{\dagger}\left|1_{n}\right\rangle =0.\end{equation}
The state span an Hilbert space of varying number of particles termed
as the \emph{Fock space. }The Fock space $\mathcal{F}$ can be defined
in terms of the direct sum of $N$-particle Hilbert spaces $\mathcal{H}_{N}$\[
\mathcal{F}=\mathcal{H}_{0}+\mathcal{H}_{1}+\mathcal{H}_{2}+\ldots\]
The antisymmetry of the state is preserved by the fermion commutation
rule\begin{equation}
\left[\hat{a}_{n}^{\dagger},\hat{a}_{n'}\right]_{+}=\hat{a}_{n}^{\dagger}\hat{a}_{n'}+\hat{a}_{n'}\hat{a}_{n}^{\dagger}=\delta_{nn'}\label{eq:commutation_rel_1}\end{equation}
and\begin{equation}
\left[\hat{a}_{n},\hat{a}_{n'}\right]_{+}=\left[\hat{a}_{n}^{\dagger},\hat{a}_{n'}^{\dagger}\right]_{+}=0.\label{eq:commutation_rel_2}\end{equation}
A valuable tool to transfer the Hamiltonian \ref{eq:Separated_Hamiltonian-1}
(or any other operator) into second quantization representation is
given by the electron field operator\begin{equation}
\hat{\Phi}(\mathbf{r},t)=\sum_{n}\varphi_{n}(\mathbf{r})\hat{a}_{n}(t).\end{equation}
Here, we switch from the Schr\"{o}dinger to the Heisenberg picture.
For an operator $A$ in the Schr\"{o}dinger picture, the corresponding
operator in the Heisenberg picture is given by \begin{equation}
A_{H}=e^{iHt/\hbar}Ae^{-iHt/\hbar},\end{equation}
where $H$ is the Hamilton operator. As the Heisenberg picture will
be generally used, the time-dependence of the operators is dropped
from now on.


\subsubsection{Single Particle Hamiltonian}

Using the field operator $\hat{\Phi}(\mathbf{r})$, the representation
of the single-particle Hamiltonian $H_{1}$ in second quantization
is given by\begin{eqnarray}
\hat{H}_{1} & = & \int d\mathbf{r}\hat{\Phi}^{\dagger}(\mathbf{r})H_{1}\hat{\Phi}(\mathbf{r})\nonumber \\
 & = & \sum_{j,k}\hat{a}_{j}^{\dagger}\hat{a}_{k}\int d\mathbf{r}\varphi_{j}^{*}H_{1}\varphi_{k}\label{eq:Single_Particle_H_Second_Q}\\
 & = & \sum_{j}\hat{a}_{j}^{\dagger}\hat{a}_{j}E_{j},\nonumber \end{eqnarray}
which is diagonal as $\varphi_{n}$ is assumed to be an eigenfunction
of $H_{1}$. The representation in terms of second quantization could
also be performed using another basis of the single-particle Hilbert
space, but then \ref{eq:Single_Particle_H_Second_Q} would not be
diagonal in $jk$. Another detail is that here, the first quasi-particle,
a particle occupying the eigenstates of the single particle Hamiltonian
$H_{1}$, has been introduced.


\subsubsection{Two Particle Hamiltonian: Coulomb Interaction}

The transformation of the two-particle interaction, the Coulomb term
$H_{2}$, is similar, but more complicated. The result of the expansion
is\begin{equation}
\hat{H}_{2}=\frac{1}{2}\sum_{j,k,l,m}\left\langle jk\mid v\mid lm\right\rangle \hat{a}_{j}^{\dagger}\hat{a}_{k}^{\dagger}\hat{a}_{l}\hat{a}_{m}.\label{eq:Two_Particle_Hamiltonian}\end{equation}
The matrix element $\left\langle jk\mid v\mid lm\right\rangle $ is
given by \begin{equation}
\left\langle jk\mid v\mid lm\right\rangle =\int d\mathbf{r}d\mathbf{r}'\varphi_{j}^{*}(\mathbf{r})\varphi_{k}^{*}(\mathbf{r}')\frac{e^{2}}{4\pi\epsilon_{0}\left|\mathbf{r}-\mathbf{r}'\right|}\varphi_{m}(\mathbf{r})\varphi_{l}(\mathbf{r}').\end{equation}
Note that the $m$ and the $l$ are swapped in the integral compared
to the annihilation and creation operators and that in $\left\langle jk\mid v\mid lm\right\rangle $,
the spin variable must be included, so\[
\left\langle jk\mid v\mid lm\right\rangle \rightarrow\delta_{s_{j}s_{m}}\delta_{s_{k}s_{l}}\left\langle jk\mid v\mid lm\right\rangle .\]



\subsubsection{Particle \textendash{} Electromagnetic Field Interaction Hamiltonian}

The next Hamiltonian to quantize is the electron\textendash{}EM field
interaction Hamiltonian $H_{e\lyxmathsym{\textminus}EM}$ defined
in \ref{eq:Separated_Hamiltonian-1}. It is clear that $H_{e\lyxmathsym{\textminus}EM}$
is a one-particle Hamiltonian and therefore its representation in
the second quantization is obtained by applying the field operators
as in \ref{eq:Single_Particle_H_Second_Q}, with the difference that
$\varphi_{n}$ is diagonal in $H_{1}$but not in $H_{e\lyxmathsym{\textminus}EM}$.
The resulting expression reads\begin{equation}
\hat{H}_{e-EM}=\sum_{j,k}\left(\frac{e}{m_{0}}\left\langle j\mid\mathbf{A}\cdot\mathbf{p}\mid k\right\rangle +\frac{e^{2}}{2m_{0}}\left\langle j\mid\mathbf{A}^{2}\mid k\right\rangle \right)\hat{a}_{j}^{\dagger}\hat{a}_{k}.\label{eq:Particle_EM_Interaction_H}\end{equation}
To simplify the Hamiltonian and in order to get rid of the vector
potential $\mathbf{A}$ in favor of the electric field $\mathbf{E}=-\frac{\partial}{\partial t}\mathbf{A}$
(Coulomb gauge assumed as before), the \emph{dipole approximation}
can be applied. As the crystal is a quasi-periodic structure, where
the periodicity is much smaller than the photon wavelength of usual
electromagnetic fields, the electric field can be assumed to be constant
within the range of a lattice cell. This approximation removes the
spatial dependence of $\mathbf{A}$ and reduces the second term $\left\langle j\mid\mathbf{A}^{2}\mid k\right\rangle $
to a contribution diagonal in $j,k$, and is therefore proportional
to the number of electrons and does not contribute to any interband
transitions. Therefore, it can be treated as an additional energy
constant and thus be neglected. The remaining term, $\frac{e}{m_{0}}\left\langle j\mid\mathbf{A}\cdot\mathbf{p}\mid k\right\rangle $
is more difficult to simplify and thus a simplified argument is brought
here, while the full justification can be found in \citet{Schafer2002}.
The first step is to relate the momentum matrix element $\mathbf{p}$
and the dipole transition matrix element $\mathbf{r}$. The relation
is given by\begin{equation}
\left[\mathbf{r},H_{1}\right]=\frac{i\hbar}{m_{0}}\mathbf{p},\end{equation}
where $H_{1}$ is the single particle Hamiltonian. Thus, the second
term in \ref{eq:Particle_EM_Interaction_H} can transformed into\begin{eqnarray}
\frac{e}{m_{0}}\left\langle j\mid\mathbf{A}\cdot\mathbf{p}\mid k\right\rangle  & = & \mathbf{A}\cdot\frac{e}{i\hbar}\left\langle j\mid\left[\mathbf{r},H_{1}(\mathbf{r})\right]\mid k\right\rangle \nonumber \\
 & = & \mathbf{A}\cdot\frac{e}{i\hbar}\left\langle j\mid\mathbf{r}H_{1}(\mathbf{r})-H_{1}(\mathbf{r})\mathbf{r}\mid k\right\rangle \\
 & = & \mathbf{A}\cdot\frac{e}{i\hbar}\left\langle j\mid\mathbf{r}\mid k\right\rangle \left(E_{k}-E_{j}\right)\nonumber \\
 & = & -\frac{i}{\hbar}\left(E_{k}-E_{j}\right)\mathbf{A}\cdot\left\langle j\mid\mathbf{d}\mid k\right\rangle ,\nonumber \end{eqnarray}
where $\mathbf{d}=-e\mathbf{r}$ is the dipole moment. The last step
involved adding the term $\mathbf{d}_{j,k}\cdot\mathbf{E}-\mathbf{d}_{j,k}\cdot\mathbf{E}$,
which with the defintion of $\mathbf{E}$ gives\begin{equation}
\frac{e}{m_{0}}\left\langle j\mid\mathbf{A}\cdot\mathbf{p}\mid k\right\rangle =-\mathbf{d}_{j,k}\cdot\mathbf{E}-\left(\frac{\hbar}{i}\frac{\partial}{\partial t}+E_{j}-E_{k}\right)\mathbf{A}\cdot\mathbf{d}_{j,k}.\end{equation}
If $\mathbf{A}$ is given by a plane wave $\mathbf{A}_{0}e^{i\left(\omega_{0}t-k_{0}r\right)}$
with center fraquency $\omega_{0}$ and the transition energies $E_{k}-E_{j}$
are ranged around $\hbar\omega_{0}$, the expression in the brackets
vanished and the sipole approxiamtion for the Hamiltonian is obtained,
\begin{equation}
\hat{H}_{e-EM}=-\sum_{j,k}\left\langle j\mid\mathbf{d}\cdot\mathbf{E}\mid k\right\rangle \hat{a}_{j}^{\dagger}\hat{a}_{k}.\end{equation}


The final Hamiltonian for the crystal electrons \ref{eq:Separated_Hamiltonian-1},
cand therefore be written as \begin{equation}
\hat{H}=\sum_{j,k}\left\langle j\mid H_{1}\mid k\right\rangle \hat{a}_{j}^{\dagger}\hat{a}_{k}+\frac{1}{2}\sum_{j,k,l,m}\left\langle jk\mid v\mid lm\right\rangle \hat{a}_{j}^{\dagger}\hat{a}_{k}^{\dagger}\hat{a}_{l}\hat{a}_{m}-\sum_{j,k}\left\langle j\mid\mathbf{d}\cdot\mathbf{E}\mid k\right\rangle \hat{a}_{j}^{\dagger}\hat{a}_{k}.\label{eq:Crystal_Electron_H_General}\end{equation}



\subsection{Bloch States Formulation}

Up to now, we have ignored the specific form of the wavefunction $\varphi_{n}$.
Here we consider the wavefunction to be a Bloch function, as presented
in Chapter \ref{cha:Electronic_Properties_Semiconductors}. Within
a bulk crystal, the Bloch function is given by \begin{equation}
\varphi_{n\mathbf{k}}(\mathbf{r})=u_{n\mathbf{k}}(\mathbf{r})e^{i\mathbf{k}\cdot\mathbf{r}},\end{equation}
where $u_{n\mathbf{k}}(\mathbf{r})$ is lattice periodic and the exponential
term is slowly varying. The current section aims now to transform
\ref{eq:Crystal_Electron_H_General} into Bloch states, which needs
some clarifications and definitions. 

The Bloch-functions are required to be orthonormalized over the crystal
domain $\Omega$\begin{equation}
\int_{\Omega}d\mathbf{r}\varphi_{n'\mathbf{k}'}^{*}\varphi_{n\mathbf{k}}=\delta_{n',n}\delta_{\mathbf{k}',\mathbf{k}}.\label{eq:Bloch_Orthonormality}\end{equation}
The goal now is to represent the Hamiltonian \ref{eq:Crystal_Electron_H_General}
using the field operator \begin{equation}
\hat{\Phi}(\mathbf{r},t)=\sum_{n,\mathbf{k}}\varphi_{n\mathbf{k}}\hat{a}_{n\mathbf{k}}(t)\end{equation}
in terms of the Bloch-functions. As the Bloch-states are not localized,
the Fourier transform of certain quantities (such as the electromagnetic
field or the Coulomb interaction) in the translational invariant directions
will be used to simplify the Hamiltonian. For the spatial Fourier
transform, the conventions are given by\begin{eqnarray}
w(\mathbf{r}) & = & \sum_{\mathbf{q}}w(\mathbf{q})e^{i\mathbf{q}\cdot\mathbf{r}},\label{eq:Fourier_Transform}\\
w(\mathbf{q}) & = & \frac{1}{\mathcal{A}}\int_{\mathcal{A}}d\mathbf{r}w(\mathbf{r})e^{-i\mathbf{q}\cdot\mathbf{r}}.\label{eq:Inverse_Fourier_Transform}\end{eqnarray}



\subsubsection{Normalization}

In the case of a quantum nanostructure, the Bloch function loses it\textquoteright{}s
plane wave dependence in the quantized direction and is therefore
expressed as\begin{equation}
\varphi_{n\mathbf{k}}(\mathbf{r},z)=\left|n\mathbf{k}\right\rangle =u_{n\mathbf{k}}(\mathbf{r},z)e^{i\mathbf{k}\cdot\mathbf{r}}F_{n\mathbf{k}}(z).\label{eq:Bloch_WF_Nanostructure}\end{equation}
The free direction is represented by the plane-wave and the symmetry
broken part by $F(z)$. The indices $n$ now include the subbands.
In the following, the coordinate $z$ will be used for the symmetry
broken direction while $\mathbf{r}$ will denote the translational
invariant direction. 

Regarding the normalization, it is assumed that $\mathcal{A}$ is
the volume of the translational invariant direction and $\mathcal{L}$
is the volume of the quantized direction. Obviously, $\Omega=\mathcal{LA}$.
\ref{eq:Bloch_Orthonormality} is still required to hold, but the
normalization over the translational invariant direction is distributed
into the lattice periodic part $u_{n\mathbf{k}}(\mathbf{r},z)$. The
envelope function $F(z)$ is then normalized over the quantized direction.
Therefore, for a system quantized in $d$ dimensions, the units of
the wavefunction parts are given by\begin{equation}
\underbrace{\varphi_{n\mathbf{k}}(\mathbf{r},z)}_{\frac{1}{\sqrt{m^{3}}}=\frac{1}{\sqrt{\Omega}}}=\underbrace{u_{n\mathbf{k}}(\mathbf{r},z)}_{\frac{1}{\sqrt{m^{3-d}}}=\frac{1}{\sqrt{\mathcal{A}}}}e^{i\mathbf{k}\cdot\mathbf{r}}\underbrace{F_{n\mathbf{k}}(z)}_{\frac{1}{\sqrt{m^{d}}}=\frac{1}{\sqrt{\mathcal{L}}}}.\end{equation}



\subsubsection{Lattice-Cell Average}

A general problem of the approach using wavefunctions of the $\mathbf{k}\cdot\mathbf{p}$
envelope equation is that the lattice-periodic functions $u_{n\mathbf{k}}(\mathbf{r},z)$
are not given in an explicit form. Only their symmetry properties,
the fact that they are orthogonal to each other and some measurable
quantities are available. Nevertheless, these properties are sufficient
for building the Hamiltonian and calculating desired matrix elements,
within the approximation that lattice-cell averaged quantities are
used. In the $\mathbf{k}\cdot\mathbf{p}$ method, the envelopes and
plane waves are assumed to vary little over a crystal cell. If an
operator $A$ mainly acts on the lattice-periodic part, a matrix element
between two wavefunctions \ref{eq:Bloch_WF_Nanostructure} can be
simplified as\begin{equation}
\left\langle n'\mathbf{k}'\mid A\mid n\mathbf{k}\right\rangle \approx\frac{\mathcal{A}}{V_{c}}\left\langle u_{n'\mathbf{k}'}\mid A\mid u_{n\mathbf{k}}\right\rangle _{V_{c}}\delta_{k',k}\int dzF_{n'\mathbf{k}'}^{*}(z)F_{n\mathbf{k}}(z).\end{equation}
Here the operator $A$ is replaced with it's lattice averaged quantity\[
V_{c}^{-1}\left\langle u_{n'\mathbf{k}'}\mid A\mid u_{n\mathbf{k}}\right\rangle _{V_{c}}.\]
$V_{c}$ denotes the crystal cell and $\left\langle \right\rangle _{V_{c}}$
the integration over it. The matrix elements found in the literature
are related by \begin{equation}
\frac{\mathcal{A}}{V_{c}}\left\langle u_{n'\mathbf{k}'}\mid A\mid u_{n\mathbf{k}}\right\rangle _{V_{c}}=\left\langle \tilde{u}_{n'\mathbf{k}'}\mid A\mid\tilde{u}_{n\mathbf{k}}\right\rangle \label{eq:Matrix_Elements_Relation}\end{equation}
where $\tilde{u}_{i}$ is the Bloch function normalized with respect
to a single crystal cell. Therefore to determine values for the short-range
operator $\left\langle n'\mathbf{k}'\mid A\mid n\mathbf{k}\right\rangle $,
only the envelope function $F_{n\mathbf{k}}(z)$ and the experimental
value for $\left\langle \tilde{u}_{n'\mathbf{k}'}\mid A\mid\tilde{u}_{n\mathbf{k}}\right\rangle $
for the from bulk measurements are required.

The other case is given by the long-range operator $B$, such as the
Coulomb potential. In that case, the operator is constant over a lattice
cell and the integral of the matrix element can be reduced using the
orthogonality of the Bloch functions. The lattice-cell average reads\begin{eqnarray}
\left\langle n'\mathbf{k}'\mid B\mid n\mathbf{k}\right\rangle  & = & \int_{\Omega}d\mathbf{r}dzu_{n'\mathbf{k}'}^{*}(\mathbf{r},z)e^{-i\mathbf{k}'\cdot\mathbf{r}}F_{n'\mathbf{k}'}^{*}(z)Bu_{n\mathbf{k}}(\mathbf{r},z)e^{i\mathbf{k}\cdot\mathbf{r}}F_{n\mathbf{k}}(z)\nonumber \\
 & \approx & \int_{\Omega}d\mathbf{r}dz\frac{1}{V_{c}}\left\langle u_{n'\mathbf{k}'}\mid u_{n\mathbf{k}}\right\rangle _{V_{c}}e^{-i\mathbf{k}'\cdot\mathbf{r}}F_{n'\mathbf{k}'}^{*}Bu_{n\mathbf{k}}(\mathbf{r},z)e^{i\mathbf{k}\cdot\mathbf{r}}F_{n\mathbf{k}}(z).\label{eq:Matrix_Element_Lattice_Cell_Average}\end{eqnarray}
A further simplification depends on the actual form of the considered
operator.


\subsubsection{The Kinetic Term}

To transform the kinetic part into the Bloch states, the eigenfunctions
$\varphi_{n\mathbf{k}}(\mathbf{r},z)$ of the single particle Hamiltonian
are inserted into the field operator. Using that particular field
operator, the kinetic term \ref{eq:Crystal_Electron_H_General} is
given by\begin{equation}
\hat{H}_{1}=\sum_{n,\mathbf{k}}\hat{a}_{n\mathbf{k}}^{\dagger}\hat{a}_{n\mathbf{k}}E_{n,\mathbf{k}}.\end{equation}



\subsubsection{The Interaction Term}

The remaining step is to quantize the electron - electromagnetic field
interaction Hamiltonian $H_{e-EM}$ within the dipole approximation.
The Fourier representation of the electric field $\mathbf{E}(\mathbf{r},t)$
is given by\begin{equation}
\mathbf{E}(\mathbf{r},t)=\sum_{\mathbf{q}}\mathbf{E}(\mathbf{q},t)e^{i\mathbf{q}\cdot\mathbf{r}}.\end{equation}
As it is assumed that $\mathbf{E}(\mathbf{r},t)$ is slowly-varying,
the relevant contributions to the sum will be around very small values
of $\mathbf{q}$. This allows to pull the $e^{i\mathbf{q}\cdot\mathbf{r}}$
factor out of the dipole integral in the calculation below. The resulting
Hamiltonian then reads\begin{eqnarray*}
 &  & \int_{\Omega}d\mathbf{r}dz\hat{\Phi}^{\dagger}H_{e-EM}\hat{\Phi}\\
 & = & -\sum_{n',\mathbf{k}',n,\mathbf{k}}\hat{a}_{n'\mathbf{k}'}^{\dagger}\hat{a}_{n\mathbf{k}}\int_{\Omega}d\mathbf{r}dz\varphi_{n'\mathbf{k}'}^{*}\mathbf{d}\cdot\left(\sum_{\mathbf{q}}\mathbf{E}(\mathbf{q},t)e^{i\mathbf{q}\cdot\mathbf{r}}\right)\varphi_{n\mathbf{k}}\\
 & = & -\sum_{n',\mathbf{k}',n,\mathbf{k}}\sum_{\mathbf{q}}\delta_{\mathbf{k}'\mathbf{k}}\boldsymbol{\mu}_{n'n,\mathbf{k}}\cdot\mathbf{E}(\mathbf{r},t)e^{i\mathbf{q}\cdot\mathbf{r}},\end{eqnarray*}
where the dipole matrix element between two states $n$ and $n\lyxmathsym{\textasciiacute}$
with same crystal momentum k is given by\begin{equation}
\boldsymbol{\mu}_{n'n,\mathbf{k}}=\mathcal{A}\int_{\mathcal{L}}dzu_{n'\mathbf{k}}^{*}F_{n'\mathbf{k}}^{*}\mathbf{d}u_{n\mathbf{k}}F_{n\mathbf{k}}.\end{equation}
The delta function $\delta_{\mathbf{k}'\mathbf{k}}$ results from
the plane wave and ensures momentum conservation. In other words,
only direct transitions are allowed. Using the dipole matrix element,
the Hamiltonian can finally be written as\begin{equation}
\hat{H}_{e-EM}=\mathbf{-E}(\mathbf{r},t)\sum_{n',n,\mathbf{k}}\boldsymbol{\mu}_{n',n,\mathbf{k}}\hat{a}_{n'\mathbf{k}}^{\dagger}\hat{a}_{n\mathbf{k}}.\end{equation}


The transformation of the two particle Hamiltonian $H_{2}$ is presented
in the next chapter. For the moment, it is assumed that the carriers
are \emph{uncorrelated} and their coulomb interaction is included
in the single particle Hamiltonians. Collecting all single particle
interactions, the Hamiltonian in the Bloch basis is given as\begin{equation}
\hat{H}=\sum_{n}E_{n}(\mathbf{k})\hat{a}_{n\mathbf{k}}^{\dagger}\hat{a}_{n\mathbf{k}}\mathbf{-E}(\mathbf{r},t)\sum_{n',n,\mathbf{k}}\boldsymbol{\mu}_{n',n,\mathbf{k}}\hat{a}_{n'\mathbf{k}}^{\dagger}\hat{a}_{n\mathbf{k}}.\end{equation}



\subsection{Holes}

In a semiconductor at $T=0K$, the valence band is fully occupied
while the conduction band is empty. With increasing temperature, electrons
in the valence band get excited into the conduction band and leave
behind holes. Consequently, it is common to use for the valence band
the absence of an electron, the hole, as a quasi particle. Therefore,
if an electron with momentum $\lyxmathsym{\textminus}\mathbf{k}$
is created ($=\hat{a}_{-\mathbf{k}}^{\dagger}$), a hole with momentum
$\mathbf{k}$ is annihilated ($=\hat{b}_{\mathbf{k}}$) and vice versa.
The sign is switched by convention. In order to introduce the concept
of holes into the Hamiltonian, the commutator rules \ref{eq:commutation_rel_1}
and \ref{eq:commutation_rel_2} are used to reestablish the normal
ordering of creation and annihilation operators. One obtains for the
kinetic and dipole Hamiltonian\begin{eqnarray}
\hat{H} & = & \sum_{c,\mathbf{k}}E_{c}(\mathbf{k})\hat{a}_{c\mathbf{k}}^{\dagger}\hat{a}_{c\mathbf{k}}-\sum_{v,\mathbf{k}}E_{v}(\mathbf{k})\hat{b}_{b\mathbf{k}}^{\dagger}\hat{b}_{b\mathbf{k}}\nonumber \\
 &  & -\mathbf{E}(\mathbf{r},t)\left(\sum_{c,v,\mathbf{k}}\boldsymbol{\mu}_{cv,\mathbf{k}}\underbrace{\hat{a}_{c\mathbf{k}}^{\dagger}\hat{b}_{v-\mathbf{k}}^{\dagger}}_{\hat{p}_{vc,\mathbf{k}}^{\dagger}}+\boldsymbol{\mu}_{cv,\mathbf{k}}^{*}\underbrace{\hat{b}_{v-\mathbf{k}}\hat{a}_{c\mathbf{k}}}_{\hat{p}_{vc,\mathbf{k}}}\right).\label{eq:Kinetic_Dipole_H}\end{eqnarray}
Note that from now on, the summation is distinguished between summation
over conduction bands indicated with $c$ and valence bands indicated
using the index $v$. In \ref{eq:Kinetic_Dipole_H}, transitions between
conduction subbands and transitions between valence subbands have
been neglected. 

One important point is that in the Hamiltonian \ref{eq:Kinetic_Dipole_H},
some terms can already be identified: $\hat{n}_{c\mathbf{k}}=\hat{a}_{c\mathbf{k}}^{\dagger}\hat{a}_{c\mathbf{k}}$
is the number operator counting the number of electrons in the band
$c$ with crystal momentum $\mathbf{k}$, while $\hat{n}_{v\mathbf{k}}=\hat{b}_{b-\mathbf{k}}^{\dagger}\hat{b}_{b-\mathbf{k}}$
is the number operator counting the holes in the valence band $v$.
The term $\hat{p}_{vc,\mathbf{k}}=\hat{b}_{v-\mathbf{k}}\hat{a}_{c\mathbf{k}}$
is named \emph{microscopic polarization} and is given by the off-diagonal
density matrix element, giving the correlation between a particle
in one band and an empty state in the other. In a simplistic view,
the dipole matrix element $\boldsymbol{\mu}_{cv,\mathbf{k}}$ gives
the coupling strength of such a correlation to an electric field.


\subsection{\label{sub:Carrier_Statistics}Carrier Statistics}

The main interest of the current work is the continuous emission of
light of a semiconductor, which is given in the steady state of the
system. In the limit of small light intensity and therefore absence
of spectral-hole burning, it can be assumed that the carriers relax
into their quasi-equilibrium distributions, given by the Fermi distributions
for the electrons in the conduction band\begin{equation}
n_{c\mathbf{k}}=\left\langle \hat{n}_{c\mathbf{k}}\right\rangle =f_{c\mathbf{k}}=\frac{1}{1+e^{\left(E_{c}(\mathbf{k})-E_{F,c}\right)/k_{B}T}}\label{eq:Conduction_Band_Electrons_Fermi}\end{equation}
and the holes in the valence band\begin{equation}
n_{v\mathbf{k}}=\left\langle \hat{n}_{v\mathbf{k}}\right\rangle =f_{v\mathbf{k}}=\frac{1}{1+e^{\left(E_{F,v}-E_{v}(\mathbf{k})\right)/k_{B}T}}.\label{eq:Valence_Band_Hole_Fermi}\end{equation}
Here, $E_{F,c}$ and $E_{F,v}$ denote the quasi-Fermi levels of the
electrons and holes, $k_{B}$ is Boltzmann\textquoteright{}s constant
and $T$ denotes the temperature. The Fermi levels can be calculated
from the 3D carrier density $N$\begin{equation}
N=\frac{1}{\Omega}\sum_{c\mathbf{k}}f_{c\mathbf{k}}\label{eq:Electron_3D_Density}\end{equation}
for electrons and $P$\begin{equation}
P=\frac{1}{\Omega}\sum_{v\mathbf{k}}f_{v\mathbf{k}}\label{eq:Holes_3D_Density}\end{equation}
for holes. Here $\Omega$ denotes the volume of the system, which
is usually unknown. When the sum over $\mathbf{k}$ is transformed
into an integral\begin{equation}
\sum_{\mathbf{k}}\rightarrow\frac{2\mathcal{A}}{(2\pi)^{d}}\int d\mathbf{k}\end{equation}
the unknown $\Omega$ volume is removed and one ends up with known
quantities ($\mathcal{L}=\Omega/\mathcal{A}$). $d$ denotes the dimensionality
of the $\mathbf{k}$-space. Note the factor 2 which stems from the
implicit summation over the spin states. For an ideal 2D quantum well,
the electrons are free to move only in the plane of the active layer,
so that the summation over $\mathbf{k}$ is restricted to two dimensions
according to \begin{equation}
\sum_{\mathbf{k}}\rightarrow\frac{2\mathcal{A}}{(2\pi)^{2}}\int dk2\pi k,\label{eq:2D_Symmetric_k_Sum_Integral}\end{equation}
where we assume that the function summed over is cylindrically symmetric
(which is a reasonable assumption for the subband structure near the
zone center $\mathbf{k}=0$). As the $\mathbf{k}$ dependence of the
distributions for general band structures is quite evolved, an analytical
inversion of \ref{eq:Electron_3D_Density} and \ref{eq:Holes_3D_Density}
is not feasible. Therefore, the simplest way to calculate the quasi-Fermi
levels is to perform a numerical Newton procedure to find the roots
of $N_{0}\lyxmathsym{\textminus}N(E_{F})=0$.


\section{Transitions Calculation}


\subsection{Introduction}

Classically, an electric field $\mathbf{E}$ in a semiconductor induces
dipoles. These dipoles create a macroscopic polarization $\mathbf{P}$.
The induced polarization $\mathbf{P}(t)$ at time $t$ depends on
the electric field $\mathbf{E}(t')$ at time $t'<t$. Their relation
is defined in terms of a time-dependent susceptibility $\chi(t-t')$
(in this case a scalar)\begin{equation}
\mathbf{P}(t)=\epsilon_{b}\int_{\infty}^{t}\chi(t-t')\mathbf{E}(t')dt'.\label{eq:Polarization_Time_Domain}\end{equation}
Taking the Fourier transform of \ref{eq:Polarization_Time_Domain}
leads to the more convenient form\begin{equation}
\mathbf{P}(\omega)=\epsilon_{b}\chi(\omega)\mathbf{E}(\omega).\label{eq:Polarization_Freq_Domain}\end{equation}
The polarization influences the electric field via the electric displacement\begin{equation}
\mathbf{D}=\epsilon\mathbf{E}=\epsilon_{b}\mathbf{E}+\mathbf{P}.\label{eq:Electric_Displacement}\end{equation}
Therefore, the creation of dipoles and the amplification of an electric
field needs to be treated self-consistently. The aim is therefore
to calculate the polarization quantum-mechanically in the presence
of an electric field and obtain a self-consistent formula for the
steady state. 

As a first step, Maxwell\textquoteright{}s equations are rewritten
to give a single relation between the electric field E and the macroscopic
polarization $\mathbf{P}$. Taking the curl of\begin{equation}
\nabla\times\mathbf{H}-\frac{\partial\mathbf{D}}{\partial t}=\mathbf{j}\label{eq:Maxwells_1}\end{equation}
and using \begin{equation}
\nabla\times\mathbf{E}=-\frac{\partial\mathbf{B}}{\partial t},\label{eq:Maxwells_2}\end{equation}
ones obtains\begin{eqnarray}
\nabla\times\nabla\times\mathbf{E} & = & -\nabla\times\frac{\partial\mu_{0}\mathbf{H}}{\partial t},\\
\nabla\left(\nabla\cdot\mathbf{E}\right)-\nabla^{2}\mathbf{E} & = & -\mu_{0}\frac{\partial}{\partial t}\left(\nabla\times\mathbf{H}\right).\end{eqnarray}
On the left hand side, the term $\nabla\cdot\mathbf{E}=0$ vanishes,
assuming charge neutrality and homogeneous media. On the right hand
side, using \ref{eq:Maxwells_1}, $\nabla\times\mathbf{H}$ is replaced
by the electric displacement $\mathbf{D}$\begin{equation}
\nabla\left(\nabla\cdot\mathbf{E}\right)-\nabla^{2}\mathbf{E}\approx-\nabla^{2}\mathbf{E}=-\mu_{0}\frac{\partial^{2}\mathbf{D}}{\partial t^{2}},\end{equation}
which now allows to introduce the macroscopic polarization $\mathbf{P}$
using \ref{eq:Electric_Displacement}\begin{equation}
-\nabla^{2}\mathbf{E}+\mu_{0}\epsilon_{b}\frac{\partial^{2}\mathbf{E}}{\partial t^{2}}=-\mu_{0}\frac{\partial^{2}\mathbf{P}}{\partial t^{2}}.\label{eq:Helmholtz_Eq}\end{equation}
This equation is the\emph{ inhomogeneous Helmholtz equation}. The
next step is to assume a monochromatic electrical field $\mathbf{E}$
(traveling into $z$-direction) given by\begin{equation}
\mathbf{E}(z,t)=\frac{1}{2}\hat{e}_{i}E(z)e^{i\left(k_{0}z-\nu t-\phi(z)\right)}+\mbox{c.c},\label{eq:Electrical_Field_Ansatz}\end{equation}
where $k_{0}=\nu n/c$ is the photo wavenumber and $\hat{e}_{i}$is
a unit vector orthogonal to $\hat{e}_{z}$. $\phi(z)$ us the real
phase shift and $E(z)$ is the real field amplitude, varying little
within an optical wavelength. $\nu$ denotes the field frequency.
It is assumed that the spacial strong oscillatory part is properly
described by the plane wave. The electric field induces a polarization
\begin{equation}
\mathbf{P}(z,t)=\frac{1}{2}\hat{e}_{i}P(z)e^{i\left(k_{0}z-\nu t-\phi(z)\right)}+\mbox{c.c}.\label{eq:Polarization_Ansatz}\end{equation}
The next step is to insert \ref{eq:Electrical_Field_Ansatz} and \ref{eq:Polarization_Ansatz}
into \ref{eq:Helmholtz_Eq} and neglect all terms containing $\partial_{z}^{2}E(z)$,
$\partial_{z}^{2}\phi(z)$ and $\partial_{z}E(z)\partial_{z}\phi(z)$.
This approach is denoted as the \emph{slowly varying envelope approximation}
and one ends up with an equation for the amplitudes $E(z)$ and $P(z)$,\begin{equation}
\partial_{z}E(z)-iE(z)\partial_{z}\phi(z)=i\frac{\mu_{0}\nu^{2}}{2k_{0}}\chi(z)E(z).\end{equation}
Splitting into real and imaginary parts and using \ref{eq:Polarization_Freq_Domain},
the \emph{self consistent} equations are obtained as\begin{eqnarray}
\partial_{z}E(z) & = & -\frac{\nu}{2\epsilon_{0}nc}\Im\left\{ P(z)\right\} =-\frac{k_{0}}{2}\chi^{\prime\prime}(z)E(z)\\
\partial_{z}\phi(z) & = & -\frac{1}{E(z)}\frac{\nu}{2\epsilon_{0}nc}\Re\left\{ P(z)\right\} =-\frac{k_{0}}{2}\chi^{\prime}(z)\end{eqnarray}
where $\chi=\chi^{\prime}+i\chi^{\prime\prime}$. Consequently, the
intensity gain (amplitude is half of it) is defined by\begin{equation}
G=-k_{0}\chi^{\prime\prime},\label{eq:Intensity_Gain}\end{equation}
and the change of refractive index (via a continuous phase change)
is given by the real part of the susceptibility\begin{equation}
\frac{\delta n}{n}=\frac{\chi^{\prime}}{2}.\label{eq:Refractive_Index_Change}\end{equation}
The absorption can be obtained from the intensity gain through $\alpha=-G$.


\subsection{Quantum Microscopic Polarization}

The second step is to couple quantum mechanical observables to the
solids properties. The dipole interaction between electrons and the
electric field is given by \begin{equation}
H_{e-EM}=-V\mathbf{P}\cdot\mathbf{E},\end{equation}
where $V$ is the volume of the system and $\mathbf{P}$ is defined
as the macroscopic polarization density\begin{equation}
\mathbf{P}=\frac{1}{V}\sum_{n,n',\mathbf{k}}\boldsymbol{\mu}_{n'n,\mathbf{k}}\left\langle \hat{a}_{n'\mathbf{k}}^{\dagger}\hat{a}_{n\mathbf{k}}\right\rangle =\frac{1}{V}\sum_{c.v.\mathbf{k}}\boldsymbol{\mu}_{cv,\mathbf{k}}p_{vc,\mathbf{k}}^{\dagger}+\boldsymbol{\mu}_{cv,\mathbf{k}}^{*}p_{vc,\mathbf{k}}\end{equation}
from which the optical properties \ref{eq:Intensity_Gain} and \ref{eq:Refractive_Index_Change}
can be obtained. Here, the expectation value $p_{vc,\mathbf{k}}=\left\langle \hat{p}_{vc,\mathbf{k}}\right\rangle $
has been used. The amplitude $P(z)$ of the macroscopic polarization
$\mathbf{P}(z,t)$ used for the optical properties is given by \begin{equation}
P(z)=2e^{-i\left(k_{0}z-\nu t-\phi(z)\right)}\frac{1}{V}\sum_{c.v.\mathbf{k}}\boldsymbol{\mu}_{cv,\mathbf{k}}^{*}p_{vc,\mathbf{k}}\end{equation}
The other term of the polarization containing $p_{vc,\mathbf{k}}^{\dagger}$
is related to the complex conjugate part of \ref{eq:Polarization_Ansatz}
and is not required to determine the macroscopic amplitude $P(z)$.


\subsection{Heisenberg's Equation of Motion}

In order to calculate the expectation value $p_{nm,\mathbf{k}}$,
the equation of motion of the microscopic polarization operator $\hat{p}_{nm,\mathbf{k}}$
 has to be solved. Here the index $m$ is used for the conduction
band and $n$ for the valence band of interest. The indices $c$ and
$v$ will be used for the summation over remaining conduction and
valence bands. The Heisenberg equation of motion for a time dependent
operator $\hat{O}$ is given by \citet{Chow}\begin{equation}
\frac{d}{dt}\hat{O}=\frac{i}{\hbar}\left[\hat{H},\hat{O}\right].\label{eq:Heisenberg_Eq_Motion}\end{equation}
Applying to the microscopic polarization operator, one obtains\begin{equation}
\frac{d}{dt}\hat{p}_{nm,\mathbf{k}}=\frac{i}{\hbar}\left[\hat{H}_{1},\hat{p}_{nm,\mathbf{k}}\right]+\frac{i}{\hbar}\left[\hat{H}_{e-EM},\hat{p}_{nm,\mathbf{k}}\right].\end{equation}
The evaluation of the first commutator gives\begin{equation}
-\frac{i}{\hbar}\left(E_{c}(\mathbf{k})-E_{v}(\mathbf{k})\right)\hat{p}_{nm,\mathbf{k}}.\end{equation}
For the second operator\begin{equation}
\frac{1}{V}\sum_{c,v,\mathbf{k}'}\boldsymbol{\mu}_{cv,\mathbf{k}'}\hat{p}_{vc,\mathbf{k}'}^{\dagger}+\boldsymbol{\mu}_{cv,\mathbf{k}'}^{*}\hat{p}_{vc,\mathbf{k}'},\end{equation}
the second term containing $\hat{p}_{vc,\mathbf{k}}$ vanishes because
four anticommuting exchanges are required (leading to no sign change)
to obtain $\hat{p}_{nm,\mathbf{k}}\hat{H}-\hat{p}_{nm,\mathbf{k}}\hat{H}=0$.
Therefore, one is left with the first term, where operators are exchanged
to transform $\hat{H}\hat{p}$ into $\hat{p}\hat{H}$\begin{eqnarray}
\hat{a}_{c\mathbf{k}'}^{\dagger}\hat{b}_{c\mathbf{-k}'}^{\dagger}\hat{b}_{n\mathbf{k}}\hat{a}_{m\mathbf{k}} & \rightarrow & \hat{a}_{c\mathbf{k}'}^{\dagger}\hat{a}_{m\mathbf{k}}\delta_{v,n}\delta_{\mathbf{k}',\mathbf{k}}\nonumber \\
 &  & +\hat{b}_{v-\mathbf{k}'}^{\dagger}\hat{b}_{n\mathbf{-k}}\delta_{c,m}\delta_{\mathbf{k}',\mathbf{k}}\\
 &  & -\delta_{\mathbf{k}',\mathbf{k}}\delta_{n,v}\delta_{m,c}+\hat{p}\hat{H}.\nonumber \end{eqnarray}
The delta functions of the remaining terms lead in the sum over $c,v$
and $\mathbf{k}'$ to\begin{equation}
\sum_{c}\hat{a}_{c\mathbf{k}}^{\dagger}\hat{a}_{m\mathbf{k}}+\sum_{v}\hat{b}_{v-\mathbf{k}}\hat{b}_{n-\mathbf{k}}-1.\end{equation}
Here, the sums over $c$ and $v$ except for$c=m$ and $v=n$ will
vanish when later the expectation value $\left\langle \hat{a}_{c\mathbf{k}}^{\dagger}\hat{a}_{m\mathbf{k}}\right\rangle $
is taken. Therefore, the sums are neglected from now on. This approximation
scheme is called the \emph{random phase approximation}. The hand waving
argument is that the expectation value $\left\langle \hat{a}_{n'\mathbf{k}'}^{\dagger}\hat{a}_{n\mathbf{k}}\right\rangle $
has a dominant time-dependence\begin{equation}
\left\langle \hat{a}_{n'\mathbf{k}'}^{\dagger}\hat{a}_{n\mathbf{k}}\right\rangle \propto e^{i\left(\omega_{n'\mathbf{k}'}-\omega_{n\mathbf{k}}\right)t}\label{eq:Random_Phase_Apprx}\end{equation}
and therefore rapidly oscillates for $\omega_{n'\mathbf{k}'}\neq\omega_{n\mathbf{k}}$
and then averages out over time.

The terms with $c=m$ and $v=n$ are the density operators $\hat{n}_{m\mathbf{k}}$
and $\hat{n}_{n\mathbf{k}}$. Collecting everything and taking the
expectation values, the equation of motion of the free carrier microscopic
polarization is\begin{eqnarray}
\frac{d}{dt}\hat{p}_{nm,\mathbf{k}} & = & -\frac{i}{\hbar}\left(E_{m}(\mathbf{k})-E_{n}(\mathbf{k})\right)\hat{p}_{nm,\mathbf{k}}\nonumber \\
 &  & -\frac{i}{\hbar}\mathbf{E}(z,t)\cdot\boldsymbol{\mu}_{mn,\mathbf{k}}\left(\hat{n}_{m\mathbf{k}}+\hat{n}_{n\mathbf{k}}-1\right)\nonumber \\
 &  & +\left.\frac{\partial}{\partial t}\hat{p}_{nm,\mathbf{k}}\right|_{col.}.\label{eq:SC_Bloch_Eq_1}\end{eqnarray}
Similar equations can be formulated for the carriers density operators
$\hat{n}_{m\mathbf{k}}$ and $\hat{n}_{n\mathbf{k}}$\begin{eqnarray}
\frac{d\hat{n}_{n\mathbf{k}}}{dt} & = & -\frac{2}{\hbar}\Im\left\{ \mathbf{E}(z,t)\cdot\boldsymbol{\mu}_{\mathbf{k}}\hat{p}_{nm,\mathbf{k}}^{*}\right\} \nonumber \\
 &  & +\left.\frac{\partial}{\partial t}\hat{n}_{n,\mathbf{k}}\right|_{col.},\label{eq:SC_Bloch_Eq_2}\\
\frac{d\hat{n}_{m\mathbf{k}}}{dt} & = & -\frac{2}{\hbar}\Im\left\{ \mathbf{E}(z,t)\cdot\boldsymbol{\mu}_{\mathbf{k}}\hat{p}_{nm,\mathbf{k}}^{*}\right\} \nonumber \\
 &  & +\left.\frac{\partial}{\partial t}\hat{n}_{m,\mathbf{k}}\right|_{col.}.\label{eq:SC_Bloch_Eq_3}\end{eqnarray}
These equations constitute the\emph{ semiconductor Bloch equations
\citet{Haug2009},} for the free-carrier case. These semiconductor
Bloch equations constitute the basis for most of our understanding
of the optical properties of semiconductors and semiconductor microstructures.
Depending on the strength and time dynamics of the applied laser light
field, one can distinguish several relevant regimes: 
\begin{itemize}
\item the low excitation regime in which the exciton resonances - sometimes
accompanied by exciton molecule (biexciton) resonances - dominate
the optical properties. The interaction with phonons provides the
most important relaxation and dephasing mechanism. As the density
increases gradually, scattering between electron-hole excitations
also becomes important. 
\item the high excitation regime in which an electron-hole plasma is excited.
Here, the screening of the Coulomb interaction by the optically excited
carriers and the collective plasma oscillations are the relevant physical
phenomena. The main dissipative mechanism is the carrier-carrier Coulomb
scattering. 
\item the quasi-equilibrium regime which can be realized on relatively long
time scales. Here, the excitations have relaxed into a quasi-equilibrium
and can be described by thermal distributions. The relatively slow
approach to equilibrium can be described by a semi-classical relaxation
and dephasing kinetics.
\item the ultrafast regime in which quantum coherence and the beginning
dissipation determine the optical response. The process of decoherence,
of the beginning relaxation and the build-up of correlations, e.g.,
by a time-dependent screening are governed by the quantum kinetics
with memory structure of the carrier-carrier and carrier-phonon scattering.
\end{itemize}
Thoughout this thesis, the high-excitation and quasi-equalibrium regimes
dominate the theoretical analysis of the discussed quantum structures.
Generally, the scattering terms in the semiconductor Bloch equations
describe all the couplings of the polarizations and populations, i.e.,
of the single-particle density matrix elements to higher-order correlations,
such as two-particle and phonon- or photon-assisted density matrices.
However, in many cases one can identify specific physical mechanisms
that dominate the scattering terms in some of the excitation regimes
listed above. For example, in the low excitation regime often the
coupling of the excited carriers to phonons determines relaxation
and dephasing, whereas at high carrier densities carrier-carrier scattering
dominates. For relatively long pulses, Markov approximations for the
scattering processes are often justified, and the scattering terms
can be described by Boltzmann-like scattering rates due to carrier-phonon
or carrier-carrier scattering, respectively \citet{Haug2009}.

In all situations, the semiconductor Bloch equations are a very suitable
theoretical framework which, however, has to be supplemented with
an appropriate treatment of the scattering terms in order to describe
the various aspects of the rich physics which one encounters in pulse
excited semiconductors. In general, the semiconductor Bloch equations
have to be treated together with the Maxwell equations for the light
field in order to determine the optical response. This self consistent
coupling of Maxwell and semiconductor Bloch equations (for shortness
also called Maxwell-semiconductor-Bloch equations) is needed as soon
as spatially extended structures are analyzed where light propagation
effects become important. Relevant examples are the polariton effects,
as well as semiconductor lasers or the phenomenon of optical bistability
\citet{Haug2009}. In optically thin samples, however, where propagation
effects are unimportant, the transmitted light field is proportional
to the calculated polarization field. Under these conditions one can
treat the semiconductor Bloch equations separately from Maxwell's
equations to calculate the optical response.

The first term in \ref{eq:SC_Bloch_Eq_2} and \ref{eq:SC_Bloch_Eq_3}
describes the generation of electrons and holes pairs by the absorption
of light. As long as the scattering terms are ignored, the rate of
change of the hole population \ref{eq:SC_Bloch_Eq_3} is identical
to the rate of change of the electron population \ref{eq:SC_Bloch_Eq_2}.
The scattering terms require a more sophisticated treatment and some
of these will be addressed in chapter \ref{cha:Coulomb_Correlated_Optical_Transitions}.
For the moment, scattering is approximated using a a simple phenomenological
decay rate model given by\begin{eqnarray}
\left.\frac{\partial}{\partial t}\hat{p}_{nm,\mathbf{k}}\right|_{col.} & \approx & -\gamma\hat{p}_{nm,\mathbf{k}},\end{eqnarray}



\subsection{Solving the Free Carrier Equation}

To solve \ref{eq:SC_Bloch_Eq_1} one could use the approach given
in \citealp{Chow} and formally integrate the differential equation.
A simpler way is to replace the oscillating microscopic polarization
$p_{nm,\mathbf{k}}$ by its slowly varying envelope\begin{equation}
s_{nm,\mathbf{k}}=p_{nm,\mathbf{k}}e^{-i\left(k_{0}z-\nu t-\phi(z)\right)},\label{eq:Slowly_Varying_Envelope_Polarization}\end{equation}
where $\nu$ is the frequency of the optical field, and solve for
the steady state of \begin{equation}
\frac{d}{dt}s_{nm,\mathbf{k}}=0.\end{equation}
Inserting \ref{eq:Slowly_Varying_Envelope_Polarization} and \ref{eq:Electrical_Field_Ansatz}
into \ref{eq:SC_Bloch_Eq_1}, and skipping all fast oscillating parts
(as they should average out over time), leads to\begin{equation}
p_{nm,\mathbf{k}}=-\frac{i}{\hbar}\frac{E(z)}{2}\mu_{mn,\mathbf{k}}\frac{n_{m\mathbf{k}}+n_{n\mathbf{k}}-1}{i\left(\omega_{n,m}(\mathbf{k})-\nu\right)+\gamma}e^{-i\left(k_{0}z-\nu t-\phi(z)\right)},\end{equation}
where $\omega_{n,m}(\mathbf{k})$ is defined by \begin{equation}
\hbar\omega_{n,m}(\mathbf{k})=E_{m}(\mathbf{k})-E_{n}(\mathbf{k}).\end{equation}
 Inserting this equation into \ref{eq:Intensity_Gain} we obtain the
expression for the absorption spectrrum \begin{equation}
\alpha_{FCT}(\nu)=k_{0}\chi^{\prime\prime}=-\frac{\nu}{\epsilon_{0}n_{b}c}\frac{1}{\hbar}\frac{1}{\Omega}\sum_{n,m,\mathbf{k}}\left|\mu_{nm,\mathbf{k}}\right|^{2}\left(n_{nk}+n_{m\mathbf{k}}-1\right)\frac{\gamma}{\left(\omega_{n,m}(\mathbf{k})-\nu\right)^{2}+\gamma^{2}}.\label{eq:Absorption_FCT}\end{equation}
Here, the$\mu_{nm,\mathbf{k}}$ denotes the dipole matrix element,
between the subbands $n$ and $m$, along the polarization of the
light field, resulting from the scalar product between the monochromatic
light field $\mathbf{E}$ and the dipole $\boldsymbol{\mu}_{nm,\mathbf{k}}$.

In order to avaluate numericaly the absorption spectrum we make the
following assumptions:
\begin{itemize}
\item The system under discussion is 2 dimensional and all $\mathbf{k}$-vector
dependencies are cylindrically symmetric. Thus, we can use \ref{eq:2D_Symmetric_k_Sum_Integral}
transformation from summation to integration for the $\mathbf{k}$-vector.
\item The system undergoes a rapid equilibration of electrons and holes
into Fermi-Dirac distributions, which, assuming that the conduction
band and the valence band are two carrier reservoirs, leads to $n_{e\mathbf{k}}=f_{e\mathbf{k}}$
and $n_{h\mathbf{k}}=f_{h\mathbf{k}}$.
\end{itemize}
Under these assumptions, \ref{eq:Absorption_FCT} becomes \begin{equation}
\alpha_{FCT}(\nu)=-\frac{\nu}{\hbar c\epsilon_{0}n_{b}\pi\mathcal{L}}\sum_{n,m}\int_{0}^{\infty}dk\, k\left|\mu_{nm,\mathbf{k}}\right|^{2}\left(f_{ek}^{n}+f_{hk}^{m}-1\right)\frac{\gamma}{\left(\omega_{n,m}(\mathbf{k})-\nu\right)^{2}+\gamma^{2}},\end{equation}
where $\mathcal{L}$ is the length of the 2D quantum system.


\section{Spontaneous Emission}

The spontaneous emission within a semiconductor nanostructure can
be obtained from the amplification $G$ of the photon field, i.e.
the complex part of the optical susceptibility $\chi^{\prime\prime}$.
Using a phenomenological approach, \ref{eq:Absorption_FCT} can be
divided into an emitting and an absorbing part, $G=G_{e}-G_{a}$.
The term $n_{c\mathbf{k}}+n_{v\mathbf{k}}-1$ is the inversion of
the electron-hole population in the semiconductor, which can be rewritten
to\begin{equation}
n_{c\mathbf{k}}+n_{v\mathbf{k}}-1=n_{c\mathbf{k}}n_{v\mathbf{k}}-\left(1-n_{c\mathbf{k}}\right)\left(1-n_{v\mathbf{k}}\right).\end{equation}
The first term on rhs denotes the probability of a photon emission,
i.e. electron and hole is occupied, while the second term denotes
the probability of a photon absorption. Therefore by evaluating \ref{eq:Absorption_FCT}
including only the emission probability term $G_{e}\sim n_{c\mathbf{k}}n_{v\mathbf{k}}$,
the spontaneous emission probability per unit length is obtained.
As the velocity of a photon in the semiconductor is given by $c/n_{b}$,
the spontaneous emission probability per second is given by $G_{e}c/n_{b}$.
Neglecting the existence of a cavity, the photon density of states
(of photons with energy $\hbar\omega$) is given by\begin{equation}
N(\hbar\omega)=\frac{n_{b}^{3}(\hbar\omega)^{2}}{\pi^{2}\hbar^{3}c^{3}}\end{equation}
and therefore the spontaneous emission ($s^{\lyxmathsym{\textminus}1}m^{\lyxmathsym{\textminus}3}eV^{\lyxmathsym{\textminus}1}$)
per second per unit volume per unit energy is given by\begin{equation}
r_{sp}(\hbar\omega)=\frac{n_{b}^{3}(\hbar\omega)^{2}}{\pi^{2}\hbar^{3}c^{2}}G_{e}\end{equation}
where as the spontaneous emission intensity ($s^{\lyxmathsym{\textminus}1}m^{\lyxmathsym{\textminus}3}$)
is given by\begin{equation}
I_{sp}\left(\hbar\omega\right)=\hbar\omega r_{sp}\left(\hbar\omega\right).\end{equation}
Note that the result can be related to the Kubo-Martin-Schwinger (KMS)
\citet{Martin1959a} relation. The relation between emission and inversion
gives\begin{equation}
\frac{f_{c\mathbf{k}}f_{v\mathbf{k}}}{f_{c\mathbf{k}}+f_{v\mathbf{k}}-1}=\frac{1}{1-\exp\left(\left(E_{c}(\mathbf{k})-E_{v}(\mathbf{k})-\left(E_{Fc}-E_{Fv}\right)\right)/k_{B}T\right)}.\label{eq:Inversion_Emmision_Relation}\end{equation}
Due to the Lorentzian in \ref{eq:Absorption_FCT}, the transition
energy $E_{c}(\mathbf{k})-E_{v}(\mathbf{k})$ is close to the photon
energy. Consequently, it may be replaced with $\hbar\omega$ on the
r.h.s of \ref{eq:Inversion_Emmision_Relation}, which allows to pull
the factor \ref{eq:Inversion_Emmision_Relation} out of the $k$-sum
in \ref{eq:Absorption_FCT}, leading to the KMS relation\begin{equation}
r_{sp}(\hbar\omega)=\frac{n_{b}^{2}(\hbar\omega)^{2}}{\pi^{2}\hbar^{3}c^{2}}\frac{1}{1-\exp\left(\left(\hbar\omega-\left(E_{Fc}-E_{Fv}\right)\right)/k_{B}T\right)}G(\hbar\omega)\label{eq:Sp_Emission_Definition}\end{equation}
between gain and spontaneous emission.

Another aspect to consider is the dependence of the dipole matrix
element $\mu_{cv,\mathbf{k}}$ on the polarization of the light field.
Therefore, the average of all possible transitions is usually taken
for the spontaneous emission\begin{equation}
\mu_{cv,\mathbf{k}}^{sp}=\frac{1}{3}\sum_{i=x,y,z}\mu_{cv,\mathbf{k}}^{i}.\end{equation}
The spontaneous emission rate per unit volume ($s^{\lyxmathsym{\textminus}1}m^{\lyxmathsym{\textminus}3}$)
is given by integrating $r_{sp}(\hbar\omega)$ over the energy\begin{equation}
R_{sp}=\int_{0}^{\infty}r_{sp}(\hbar\omega)d\hbar\omega=Bnp\end{equation}
which defines another figure of merit, the spontaneous emission $B$
coefficient ($m^{3}s^{\lyxmathsym{\textminus}1}$).

There is an obvious pathological feature when such a relation is used
to obtain the spontaneous emission: if the electric field is zero,
then no spontaneous emission would exist. The root of this misbehavior
is due to the fact that the quantization of the electro-magnetic field
is not included. Including this, spontaneous emission would occur
- spontaneously. Nevertheless, the spontaneous emission of a two level
system treated using quantized electro-magnetic fields does not differ
from dipole radiation, therefore the error of the classical treatment
may be small enough for the present purpose.


\section{Dipole Matrix Element}

The evaluation of the interband dipole matrix element $\boldsymbol{\mu}_{mn,\mathbf{k}}$
between the solutions of the $\mathbf{k}\cdot\mathbf{p}$ band structure
calculation is not straight-forward. In the general case, the envelope
functions are given by\begin{equation}
\varphi_{m\mathbf{k}}=\sum_{i}F_{m,i\mathbf{k}}u_{i0},\end{equation}
where $u_{i0}$ are again zone-center Bloch functions. Using this
expansion, the dipole matrix element between two states is given by
\begin{equation}
\boldsymbol{\mu}_{n'n,\mathbf{k}}=\mathcal{A}\int_{\mathcal{L}}dz\sum_{i,j}u_{i0}^{*}F_{m,i\mathbf{k}}^{*}\mathbf{d}u_{j0}F_{n,j\mathbf{k}}.\end{equation}
The question to answer is to whether the dipole operator is acting
on the lattice periodic part $u_{j0}$ or on the envelope part $F_{i}$.
The issue has been addressed by Burt and others \citealp{Burt1995,Burt1993},
concluding that the expression is dominated by\begin{equation}
\boldsymbol{\mu}_{n'n,\mathbf{k}}\approx\frac{\mathcal{A}}{V_{c}}\sum_{i,j}\left\langle u_{i0}\mid u_{j0}\right\rangle _{V_{c}}\int_{\mathcal{L}}dzF_{m,i\mathbf{k}}^{*}\mathbf{d}F_{n,j\mathbf{k}}.\end{equation}
But, within a $\mathbf{k}\cdot\mathbf{p}$ calculation involving only
single-band models for both, conduction and valence band, the last
expression evaluates due to the orthogonality of $u_{c0}$ and $u_{v0}$
to zero%
\footnote{This fact also applies to $4\times4$ and $6\times6$ $\mathbf{k}\cdot\mathbf{p}$
valence band models.%
}. Therefore, no transition between conduction and valence subbands
in such models would exist. The reason lies in the single-band approximation.
While the wavefunction of e.g. conduction subbands is dominated by
the envelope modulating $u_{c0}$, the interband dipole moment is
dominated by the admixture of $u_{v0}$ into the conduction subbands
and $u_{c0}$ into the valence subbands. Working within the $8\times8$
model would therefore resolve the problem.

A simpler solution, allowing to use non-$8\times8$ models without
the pathological feature of vanishing transition probabilities, is
to use the momentum- instead of the dipole matrix element. The relation
can be derived using the Heisenberg equation of motion for the position
operator $r$, given by\begin{equation}
\frac{\partial}{\partial t}\mathbf{r}=\frac{i}{\hbar}\left[H,\mathbf{r}\right].\end{equation}
Assuming that $n$ and $m$ are eigenfunctions of the Hamiltonian
$H$ with corresponding eigenenergies $E_{n}$ and $E_{m}$, replacing
the derivative of the position operator by the momentum operator and
$H\left|m\right\rangle $ by $E_{m}\left|m\right\rangle $ , one obtains\begin{equation}
\left\langle n\mid\mathbf{p}\mid m\right\rangle =\frac{im}{\hbar}\left(E_{n}-E_{m}\right)\left\langle n\mid\mathbf{r}\mid m\right\rangle .\end{equation}
The electron dipole operator (electrons are negatively charged) is
related to the position operator by $\mathbf{d}=e\mathbf{r}$ leading
to \begin{equation}
\left\langle n\mid\mathbf{d}\mid m\right\rangle =e\frac{\hbar}{im}\frac{\left\langle n\mid\mathbf{p}\mid m\right\rangle }{\left(E_{n}-E_{m}\right)}.\end{equation}
Within direct transitions, the energy difference $E_{n}-E_{m}$ can
be replaced by the photon energy $\hbar\omega$, leading to the simple
expression for $\boldsymbol{\mu}_{mn,\mathbf{k}}$ in terms of the
momentum matrix element\begin{equation}
\boldsymbol{\mu}_{mn,\mathbf{k}}=\frac{e}{im\omega}\mathbf{p}_{mn,\mathbf{k}}.\end{equation}
In the approximation of slowly varying envelopes, the derivatives
of the envelopes are small and the momentum matrix element is dominated
by\begin{equation}
\mathbf{p}_{mn,\mathbf{k}}\approx\frac{\mathcal{A}}{V_{c}}\sum_{i,j}\left\langle u_{i0}\mid\mathbf{p}\mid u_{j0}\right\rangle _{V_{c}}\int_{\mathcal{L}}dzF_{m,i\mathbf{k}}^{*}F_{n,j\mathbf{k}}.\label{eq:Momentum_Matrix_Element_Approx}\end{equation}
The momentum matrix elements are related via \ref{eq:Matrix_Elements_Relation}
to band structure parameters $P$, defined e.g. in \ref{eq:Valence_Band_Edge_Energy}.
This approximation is commonly applied and is also used in the remainder
of the thesis.

Beside the neglected derivatives of the envelopes, the approximation
\ref{eq:Momentum_Matrix_Element_Approx} neglects the effect of remote
bands. \ref{eq:Momentum_Matrix_Element_Approx} considers only transitions
between zone-center functions $u_{i0}$ included explicitly in the
Hamiltonian. In the $\mathbf{k}\cdot\mathbf{p}$ envelope equations,
the effect of remote bands is included using L\"{o}wdins perturbation
theory. Therefore, beside the mixing of the explicitly considered
states in the Hamiltonian, the zone-center functions away from $k$
include remote contributions. Consequently, transitions between remote
contributions have to be included in a consistent definition of the
momentum matrix elements \citealp{Enders1995a}.

%
\begin{lyxgreyedout}
Put here the momentum matrix calculation in the 2 band model.
\end{lyxgreyedout}
\selectlanguage{english}

