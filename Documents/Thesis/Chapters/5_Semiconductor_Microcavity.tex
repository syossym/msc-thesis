\selectlanguage{english}%

\chapter{Semiconductor Microcavity}

\label{cha:Microcavity}%
\begin{lyxgreyedout}
 Put intro here... 
\end{lyxgreyedout}



\section{Optical Resonator}

An optical resonator is a device used for confining light at certain
frequencies. The classical resonator (such a the Fabri-Perot etalon
seen in Fig. \ref{fig:etalon-multiple_reflections}) consists of two
planar mirrors, separated by a distance $l$. The region between the
two mirror is called the spacer layer and its refractove index is
denoted by $n$. For now we assume that the medium outside the structure
is plain air. The transmitions spectrum of this structure exhibits
a pattern of repetative peaks of large transmission corresponding
to resonances of the etalon. This varying transmission function is
caused by an interference between the multiple reflections of light
between the two reflecting surfaces (see Fig. \ref{fig:etalon-multiple_reflections}).
Contructive interference iccurs if the transmitted beams are in phase,
which corresponds to a high-transmission peak. for transmitted beams
that are out-of-phase a destructive interference occurs, which corresponds
to a minimum in the transmission spectrum. The multiple-reflected
beams' phase matching depends in the wavelength $\lambda$ of the
light, the beam incidence angle $\theta$, the etalon thickness $l$
and the refractive index of the spacer material $n$.

%
\begin{figure}
\begin{centering}
\includegraphics[scale=0.6]{\string"C:/Users/Yossi Michaeli/Documents/Thesis/Documents/Thesis/Figures/5_Fabri_Perot_Etalon\string".eps}
\par\end{centering}

\caption{\label{fig:etalon-multiple_reflections}Fabri-Perot etalon multiple
reflections.}



\end{figure}


The phase between each successive reflection is \citet{Born2002}
\begin{equation}
\delta=\left(\frac{2\pi}{\lambda}\right)2nl\cos\left(\theta\right)\end{equation}
If both surfaces have reflectance $R$, the transmittance function
of the entire structure is given by \begin{equation}
T=\frac{(1-R)^{2}}{1+R^{2}-2R\cos\left(\delta\right)}=\frac{1}{1+F\sin^{2}\left(\frac{\delta}{2}\right)},\end{equation}
where $F=\frac{4R}{(1-R)^{2}}$ is the finesse coefficient. Fig. \ref{fig:etalon-transmittance-function}
shows the dependence of the trasmittence on the phase parameter $\delta$,
for various values of $R$ (or equavalently of $F$). The maximum
tranmission occurs when the optical path length difference $2nl\cos(\theta)$
between each transmitted beam is an interger multiple of the wavelength.
In the absence of absorption, the reflectance of the structure $R$
is the complement of the transmittence, i.e. $R+T=1$. The maximum
reflectivity is given by \begin{equation}
R_{max}=1-\frac{1}{1+F},\end{equation}
which occurs when the path-length difference is equal to half an odd
multiple of the wavelength.

%
\begin{figure}
\begin{centering}
\includegraphics[scale=0.5]{\string"C:/Users/Yossi Michaeli/Documents/Thesis/Documents/Thesis/Figures/5_Fabri_Perot_Transmission\string".eps}
\par\end{centering}

\caption{\label{fig:etalon-transmittance-function}Resonator transmittance
function for various values of mirror reflectrance $R$. $\delta\lambda$
is the full-width half-maximum of the transmission band.}



\end{figure}


The wavelength separation between adjacent transmission peaks or free
spectral range (FSR) of the etalon, $\lyxmathsym{\textgreek{D}\textgreek{l}}$,
is given by \begin{equation}
\Delta\lambda=\frac{\lambda_{0}^{2}}{2nl\cos(\theta)+\lambda_{0}},\end{equation}
where $\lyxmathsym{\textgreek{l}}_{0}$ is the central wavelength
of the nearest transmission peak. The FSR is related to the full-width
half-maximum (FWHM) $\lyxmathsym{\textgreek{d}\textgreek{l}}$ of
any one transmission band by a quantity known as the \emph{finesse\begin{equation}
\mathcal{F}=\frac{\Delta\lambda}{\delta\lambda}=\frac{\pi}{2\arcsin(1/\sqrt{F})}\end{equation}
 }


\section{Distributed Bragg Reflector}

The reflecting surfaces (or mirrors) modeled in previous section by
a reflectivity parameter $R$ are realized in practive by a stack
of multiple layers of alternating semicondictor materials with varying
refractive indices called Distributed Bragg Reflector (DBR) (see Fig.
\ref{fig:Schematic-DBR}). The alternating high and low indices, denoted
respectively by $n_{H}$ and $n_{L}$, and the semiconductor layer
thicknesses satisfy the following condiction\begin{equation}
n_{L}l_{L}=n_{H}l_{H}=\frac{\lambda_{c}}{4},\end{equation}
where $\lambda_{c}$ is the center wavelength of the high reflectevity
region of the structure. The two indices $n_{l}$ and $n_{r}$ are,
respectively, the left- and right-hand cladding material refractive
indices.

The principle of operation can be understood as follows. Each interface
between the two materials contributes a Fresnel reflection. For the
design wavelength $\lambda_{c}$, the optical path length difference
between reflections from subsequent interfaces is half the wavelength;
in addition, the reflection coefficients for the interfaces have alternating
signs. Therefore, all reflected components from the interfaces interfere
constructively, which results in a strong reflection. The reflectivity
achieved is determined by the number of layer pairs and by the refractive
index contrast between the layer materials. The reflection bandwidth
is determined mainly by the index contrast, $n_{H}-n_{L}$.

%
\begin{figure}
\begin{centering}
\includegraphics[clip,scale=0.5]{\string"C:/Users/Yossi Michaeli/Documents/Thesis/Documents/Thesis/Figures/5_DBR_Schematic\string".eps}
\par\end{centering}

\caption{\label{fig:Schematic-DBR}Schematic illustration of a Distributed
Bragg Reflector (DBR). $z$ represents the growth direction of the
layered structure.}

\end{figure}


In order to analyze the reflection and transmittance of the DBR, we
use the Transfer Matrix formalism \citet{Macleod2001}, as is described
in the next section.


\subsection{Transfer Matrix Method (TMM)}

A monochromatic electric field of frequency $\omega$ propagating
in a dielecric medium, assuming a frequency independent dielectric
function, can be shown to satisfy the following wave eqaution \begin{equation}
\nabla^{2}\mathbf{E}(\mathbf{r})+\frac{\omega^{2}}{c^{2}(r)}\mathbf{E}(\mathbf{r})=0,\label{eq:E_wave_eqaution_r}\end{equation}
where \[
\mathbf{E}(\mathbf{r},t)=\Re\left\{ \mathbf{E}(\mathbf{r})e^{i\omega t}\right\} ,\]
the speed of light in the dielectric $c(r)=c_{0}/n(r)$, $c_{0}$
being the speed on light in the vacuum and $n(r)$ the location dependent
refractive index.

A medium consisting of pairs of two different dielectric materials
with different refraction indices has a growth axis, $\hat{z}$, dependent
refraction index, i.e. $n(r)=n(z)$. Therefore equation \ref{eq:E_wave_eqaution_r}
can be written as\begin{equation}
\nabla^{2}\mathbf{E}(\rho,z)+\frac{\omega^{2}}{c^{2}(z)}\mathbf{E}(\rho,z)=0.\label{eq:E_wave_equation_z}\end{equation}
Because of the translational invariance along the in-plane, the solution
of Eq. \ref{eq:E_wave_equation_z} consists of plane waves along the
in-plane direction. For a given in-plane wave vector $k_{||}$ (parallel
to the dielectric layer plane) we can make the following ansatz for
the solution \begin{equation}
\mathbf{E}_{\mathbf{k}_{||}}=E_{\mathbf{k}_{||},\omega}(z)e^{i\mathbf{k}_{||}\cdot\mathbf{\rho}}.\end{equation}
Substituting into Eq. \ref{eq:E_wave_equation_z} yields the following
one-dimensional equation\begin{equation}
\frac{d^{2}E_{\mathbf{k}_{||},\omega}(z)}{dz^{2}}+\left(\frac{\omega^{2}}{c^{2}(z)}-k_{||}^{2}\right)E_{\mathbf{k}_{||},\omega}(z)=0.\end{equation}
This equation can be solved separately for each layer of the stack.
The solution for a homogeneouos layer with a constant refractive index
is of the form\begin{eqnarray}
E_{\mathbf{k}_{||},\omega}(z) & = & E^{+}e^{ik_{z}z}+E^{-}e^{-ik_{z}z},\nonumber \\
k_{z} & = & \sqrt{\frac{\omega^{2}}{c^{2}}-\mathbf{k}_{||}^{2}}.\end{eqnarray}


The solution is a linear cobnination of two traveling waves in opposite
direction along the $\hat{z}$ axis. A non-evanescent wave solution
exist only if $\frac{\omega}{c}>\mathbf{k}_{||}$. The complex amplitudes
of the forward ($E^{+})$ and backward ($E^{-}$) propagating waves
are determined from the boundary cnditions at each interface between
two adjacent layers of the stack.

The relation between the two complex amplitudes between two point
along the propagation direction, $z_{1}$ and $z_{2}$, can be expressed
as a $2\times2$ matrix transfer matix. From here on we assume the
non-unique basis for the transfer matrix is $(E^{+},E^{-})$. The
transfer matrix takes into account the propagation through the deielectric
media and the boundary conditions at each interface between two adjacent
layers. The boundary conditions resulting from Maxwell's equations
%
\begin{lyxgreyedout}
Put here the number of Maxwell's equations 
\end{lyxgreyedout}
 state that tangential components of the electric and magnetic fields
are continuous across the interface. %
\begin{lyxgreyedout}
Furhter discuss the boundary conditions here
\end{lyxgreyedout}
 The boundary conditions for the field components can be wtritten
as\begin{align}
D_{\perp1} & =D_{\perp2}\nonumber \\
B_{\perp1} & =B_{\perp2}\nonumber \\
E_{\parallel1} & =E_{\parallel2}\label{eq:field_boundary_conditions}\\
H_{\parallel1} & =H_{\parallel2}\nonumber \end{align}
For siplicity we assume propagation only in the $\hat{z}$ direction.
The boundary conditions for the eletric field become (see Fig. %
\begin{lyxgreyedout}
Schematic description of (a) light propagation through an inteface
between two andjacent dielectric layers and (b) light propagation
in a homogenous medium.
\end{lyxgreyedout}
 \begin{eqnarray}
E_{1}^{+}+E_{1}^{-} & = & E_{2}^{+}+E_{2}^{-}\\
n_{2}\left(E_{2}^{+}-E_{2}^{-}\right) & = & n_{1}\left(E_{1}^{+}-E_{1}^{-}\right)\end{eqnarray}


The interface transfer matrix $M_{i}$, for both linear polarizations
of the field, TE and TM, can be written as\begin{equation}
\left(\begin{array}{c}
E_{2}^{+}\\
E_{2}^{-}\end{array}\right)=\frac{1}{2n_{2}}\left(\begin{array}{cc}
n_{2}+n_{1} & n_{2}-n_{1}\\
n_{2}-n_{1} & n_{2}+n_{1}\end{array}\right)\left(\begin{array}{c}
E_{1}^{+}\\
E_{1}^{-}\end{array}\right)\equiv M_{i}\left(\begin{array}{c}
E_{1}^{+}\\
E_{1}^{-}\end{array}\right)\end{equation}
for a wave propagating from layer 1 to layer 2. The in-layer propagation
trasfer matrix $M_{p}$, which relates different vectors at $z_{1}$
and $z_{2}$ in the same layer can be written as (Fig. %
\begin{lyxgreyedout}
same figure as above, b
\end{lyxgreyedout}
)\begin{equation}
\left(\begin{array}{c}
E_{2}^{+}\\
E_{2}^{-}\end{array}\right)=\left(\begin{array}{cc}
e^{ikl} & 0\\
0 & e^{-ikl}\end{array}\right)\left(\begin{array}{c}
E_{1}^{+}\\
E_{1}^{-}\end{array}\right)\equiv M_{p}\left(\begin{array}{c}
E_{1}^{+}\\
E_{1}^{-}\end{array}\right),\end{equation}
where $k=\left(\nicefrac{\omega}{c_{0}}\right)n$.

This amplitude propagation approach can be applyed to the entire multilayer
structure, for which the transfer matrix is simply the multipliction
of $M_{i}$ and $M_{p}$ matrices for each boundary and layer\begin{equation}
M_{DBR}=\prod_{m}M_{i,m}M_{p,m},\end{equation}
where $m$ is the dielectric layer index. The explicit relation between
the amplitudes of the incident electric field in the DBR structure
and the transmitted field on the other side can be written as \begin{equation}
\left(\begin{array}{c}
E_{out}^{+}\\
0\end{array}\right)=\left(\begin{array}{cc}
M_{11} & M_{12}\\
M_{21} & M_{22}\end{array}\right)\left(\begin{array}{c}
E_{in}^{+}\\
E_{in}^{-}\end{array}\right).\end{equation}


The reflection and tranmission coefficients of the entire structure
can thus be written as\begin{eqnarray}
r_{DBR} & = & -\frac{M_{21}}{M_{22}}\\
t_{DBR} & = & \frac{\det(M_{DBR})}{M_{22}}\end{eqnarray}


%
\begin{figure}
\begin{centering}
\includegraphics[scale=0.6]{\string"C:/Users/Yossi Michaeli/Documents/Thesis/Documents/Thesis/Figures/5_DBR_Profile_Reflectivity\string".eps}
\par\end{centering}

\caption{\label{fig:Sample_DBR_reflection}Simulation of a DBR structure with
35 alternating high and low layers, where (a) is the refreactive index
profile as a function of the growth axis, (b) the amplitude and phase
of the normal incidence reflection function this structure as a fucntion
of normalized wavelength, $\lambda/\lambda_{c}$, and (c) the same
amplitude as a function of the phase acquired by the EM field at each
layer, $\phi=\frac{2\pi l_{i}n_{i}}{\lambda}$.}

\end{figure}


The reflection coefficient can be expressed as $r_{DBR}(\lambda)=|r|e^{i\alpha_{r}}$.
It can be shown that for a structure containing $2N+1$ layers of
high and low refractive indices (as in Fig. \ref{fig:Schematic-DBR}),
so that the outermost layers are of high refractive index, $n_{H}$,
and for the case of $k_{i}l_{i}=\frac{\pi}{2}$ \begin{equation}
|r|=\left(\frac{1-\left(\frac{n_{H}}{n_{L}}\right)^{2N}\frac{n_{H}^{2}}{n_{r}n_{l}}}{1+\left(\frac{n_{H}}{n_{L}}\right)^{2N}\frac{n_{H}^{2}}{n_{r}n_{l}}}\right)^{2}.\end{equation}
%
\begin{lyxgreyedout}
The phase within the region of high reflectivity (the \emph{stop band})
of the reflector can be expressed as \begin{equation}
\alpha_{r}=\frac{n_{c}L_{DBR}}{c}\left(\omega-\omega_{c}\right),\end{equation}
where $L_{DBR}$ is the penetration depth into the DBR and is given
by \citet{Savona1995}\begin{equation}
L_{DBR}=\frac{\lambda_{c}}{2}\frac{n_{L}n_{H}}{n_{c}\left(n_{H}-n_{L}\right)}.\end{equation}

\end{lyxgreyedout}


As an illustration, we present in Fig. \ref{fig:Sample_DBR_reflection}
an exact calculation of the reflection coefficient of a DBR containing
35 pairs of alternating high and low reflractive index layers. The
refractive indices for the simulation are $n_{l}=1$ (air), $n_{H}=3.45$
($Al_{0.2}Ga_{0.8}As$), $n_{L}=2.98$ ($AlAs$) and $n_{r}=3.59$
($GaAs$) (the structure refractive index profile is presented in
(a)). In subplots (b) and (c) we plot the reflection coefficient as
a function of the normalized wavelength, $\nicefrac{\lambda}{\lambda_{c}}$,
and the phase acquired by the electromagnetic field at each layer,
$\phi=\frac{2\pi l_{i}n_{i}}{\lambda}$. The stop-band, as can be
clearly seen from these plots, is centered at $\phi=\frac{\pi}{2}$,
which corresponds to $\lambda_{c}$. In subplot (b) we also present
the reflectivity phase $\alpha_{r}$, where the zero-phase crossing
is observed at $\lambda_{c}$ as expected for the left-face reflection. 


\section{Microvity Optical Characteristics }


\subsection{Microcavity Reflection Spectrum }

A microcavity (MC) is a Fabry-Perot resonator, whose mirrors are two
DBRs and the spacer material is a semicondictor layer with refractive
index $n_{c}$ and of length $l_{c}=\frac{\lambda_{c}}{2n_{c}}m$
($m$ is a positive integer). The microcavity can be analyzed similarly
to the DBR using the transfer matrix formalism. The MC's transfer
matrix for a wave traveling from left to right can be written as \begin{equation}
M_{MC}=M_{DBR}^{r}M_{c}M_{DBR}^{l},\label{eq:MC_transfer_matrix_1}\end{equation}
where $M_{c}$ is the transfer matrix of the spacer (cavity) layer.
We can write the these three transfer matrices in their most general
form as\begin{eqnarray}
M_{DBR}^{r} & = & \frac{n_{c}}{n_{r}}\left(\begin{array}{cc}
\frac{1}{t_{r}^{\star}} & -\frac{r_{r}^{\star}}{t_{r}^{\star}}\\
-\frac{r_{r}}{t_{r}} & \frac{1}{t_{r}}\end{array}\right),\\
M_{DBR}^{l} & = & \frac{n_{l}}{n_{c}}\left(\begin{array}{cc}
\frac{1}{t_{l}^{\star}} & \frac{r_{l}}{t_{l}}\\
\frac{r_{l}^{\star}}{t_{l}^{\star}} & \frac{1}{t_{l}}\end{array}\right),\\
M_{c} & = & \left(\begin{array}{cc}
e^{ikl_{c}} & 0\\
0 & e^{-ikl_{c}}\end{array}\right),\end{eqnarray}
where $\left(r_{l},\, t_{l}\right)$ and $\left(r_{r},\, t_{r}\right)$
are, respectively, the reflection and transmission coefficients of
the left- and right-hand DBR mirrors (see Fig. \ref{fig:Microcavity_schematic_structure})
and $n_{l}$ and $n_{r}$ are the left- and right-hand cladding material
refractive indices, respecively.

%
\begin{figure}
\begin{centering}
\includegraphics[scale=0.4]{\string"C:/Users/Yossi Michaeli/Documents/Thesis/Documents/Thesis/Figures/5_MC_Schematic\string".eps}
\par\end{centering}

\caption{\label{fig:Microcavity_schematic_structure}Microcavity schematic
structure.}



\end{figure}


Insering these matrices into Eq. \ref{eq:MC_transfer_matrix_1} yields
\begin{equation}
M_{MC}=\left(\frac{n_{L}}{n_{H}}\right)^{2}\left(\begin{array}{cc}
\frac{e^{ikl_{eff}}-R^{\star}e^{-ikl_{eff}}}{T^{\star}} & \frac{r_{l}e^{ikl_{eff}}-r_{r}^{\star}e^{-ikl_{eff}}}{t_{l}t_{r}^{\star}}\\
\frac{r_{l}^{\star}e^{-ikl_{eff}}-r_{r}e^{ikl_{eff}}}{t_{l}^{\star}t_{r}} & \frac{e^{-ikl_{eff}}-Re^{ikl_{eff}}}{T}\end{array}\right),\end{equation}
where $T=t_{l}t_{r}$ and $R=r_{l}r_{r}$. $l_{eff}$ is not equal
to $l_{c}$ but rather to $l_{eff}=l_{c}+l_{p}$, where $l_{p}$ is
the penetration depth of the cavity field into the DBR \citet{Savona1995}
and is given by \begin{equation}
l_{p}=\frac{\lambda_{c}}{2}\frac{n_{L}n_{H}}{n_{c}\left(n_{H}-n_{L}\right)}.\end{equation}
The reflectance and transmittance of the MC can thus be wtritten as
\begin{eqnarray}
T_{MC} & = & \frac{1}{\det\left(M_{MC}\right)|M_{MC}^{22}|^{2}}=\frac{T}{\left(1-R\right)^{2}+4R\sin^{2}\left(k_{z}l_{eff}\right)}\\
R_{MC} & = & \frac{|M_{MC}^{21}|^{2}}{|M_{MC}^{22}|^{2}}=\frac{\left(|r_{r}|-|r_{l}|\right)^{2}+4R\sin^{2}\left(k_{z}l_{eff}\right)}{\left(1-R\right)^{2}+4R\sin^{2}\left(k_{z}l_{eff}\right)}\end{eqnarray}
It is clear that there are several modes which stisfy the condition
$k_{z}l_{eff}=m\pi$. As we increase the number of layers in the DBR
mirrors the stop-band reflectivity approaches unity and the cavity
field reflection line becomes narrower. Because of the finite transmittion
probability of the DBR, the MC mode has a FWHM which can be shown
to be \citet{Savona1995} \begin{equation}
2\gamma_{c}=\frac{1-R_{MC}}{\sqrt{R_{MC}}}\frac{c}{n_{c}l_{eff}}.\end{equation}
$\gamma_{c}$ is a homogenous lifetime broadening of the confined
cavity mode, caused by the decay through the mirrors.

%
\begin{figure}
\begin{centering}
\includegraphics[scale=0.6]{\string"C:/Users/Yossi Michaeli/Documents/Thesis/Documents/Thesis/Figures/5_MC_Profile_Reflectivity\string".eps}
\par\end{centering}

\caption{\label{fig:Sample_MC_reflection}Simulation of a microcavity with
two DBR's with 35 alternating high and low refractive index layers
each, where (a) is the refreactive index profile as a function of
the growth axis, (b) the amplitude and phase of the normal incidence
reflection function this structure as a fucntion of normalized wavelength,
$\lambda/\lambda_{c}$, and (c) the same amplitude as a function of
the phase acquired by the EM field at each layer, $\phi=\frac{2\pi l_{i}n_{i}}{\lambda}$.
The inset in (c) shows the reflection profile in the vicinity of $\phi=\frac{\pi}{2}$.}



\end{figure}


As an illustration, we present in Fig. \ref{fig:Sample_MC_reflection}
the simulation results of a sample microcavity with two DBR mirrors
with 35 alternating layers of high and low refractive index each and
cavity length of $l_{c}=\frac{\lambda_{c}}{2n_{c}}$. This calculation
was performed using the direct TMM calculation of the entire the structure,
similarly to the DBR simulation presented above. The DBR parameters
are the same as in Fig. \ref{fig:Sample_DBR_reflection}, while for
the cavity we choose $n_{c}=n_{H}=3.45$ ($Al_{0.2}Ga_{0.8}As$).

%
\begin{lyxgreyedout}

\subsection{Microcavity Confined Photon}

The photon confinement along the $\hat{z}$ axis leads to the following
photon energy dispersion is \citet{Skolnick1998} \begin{equation}
E_{ph}(k_{\parallel})=\frac{\hbar ck}{n_{c}}=\frac{\hbar c}{n_{c}}\sqrt{\left(\frac{2\pi m}{l_{c}}\right)^{2}+k_{\parallel}^{2}},\end{equation}
where $m$ is a positive integer. For small $k_{\parallel}$ we approximate
the dispersion relation to a parabola \begin{equation}
E_{ph}(k_{\parallel})\cong\frac{2\pi\hbar c}{n_{c}l_{c}}\left(1+\frac{\hbar^{2}k_{\parallel}^{2}}{2m_{ph}}\right).\end{equation}
Here we define the photon in-plane effective mass $m_{ph}=\frac{\hbar n_{c}}{cl_{c}}\approx10^{-5}m_{0}$,
where $m_{0}$ is the free electron mass.
\end{lyxgreyedout}
\selectlanguage{english}

