\selectlanguage{english}%

\chapter{Electronic Properties of Semiconductors}

\label{cha:Electronic_Properties_Semiconductors}The optical properties
of nanostructures are intimately connected to the electronic states,
the response of electrons within the atomic lattice to external perturbations.
As such, a precise knowledge of the electronic properties is required
for a proper device analysis. In bulk semiconductors, the lattice
periodicity and its symmetry leads to the formation of electronic
bands, continuously (within the band) relating the crystal momentum$\mathbf{k}$
of an electron to its energy $E_{n}\left(\mathbf{k}\right)$. The
combination of different materials within a nanostructure breaks the
symmetry of the semiconductor crystal, leading to \emph{quantization
effects}, i.e. modifications of the electronic properties from their
bulk properties. These modifications have a pronounced impact on the
electronic and optical properties of nanostructures. The present chapter
covers the theory of the calculation of the electronic states in nanostructures
using the \emph{envelope function method} (EFM). 

The first section contains an extensive introduction to the Schr\"{o}dinger 
equation within a semiconductor crystal, which will later also serve
as the basis for the analysis of optical effects covered in Chapter
\ref{cha:Free_Carriers_Optical_Transitions} and the many body effects
in optical spectra in Chapter \ref{cha:Coulomb_Correlated_Optical_Transitions}. 

After introducing certain approximations to the crystal Schr\"{o}dinger 
equation, the concept of solving the resulting single-particle Schr\"{o}dinger 
equation using the bulk $\mathbf{k\cdot p}$ method is presented.
The subsequent section derives the equations of the $\mathbf{k\cdot p}$
envelope function method used in nanostructures and simultaneously
introduces the $\mathbf{k\cdot p}$ model for zinc-blende crystals,
with an emphasis on the two-band model . Note that the presented discusion
doesn't include the influance of strain effects on the electronic
properties. 

Finally, we present the Schr\"{o}dinger-Poisson model of the electronic
bands structure where the electrostatics of the system of relevance.


\section{Crystal Schr\"{o}dinger Equation}


\subsection{Introduction}

The total Hamiltonian of all electrons, atoms and the electro-magnetic
field of a solid state crystal is given by\begin{equation}
H=\sum_{i}\frac{1}{2m_{i}}\left(\mathbf{p}_{i}-ez_{i}\mathbf{A}\left(\mathbf{r}_{i}\right)\right)^{2}+\frac{\epsilon_{0}}{2}\int\left(\mathbf{E}^{2}+c_{0}^{2}\mathbf{B}^{2}\right)d\mathbf{r}\label{eq:full_crystal_H}\end{equation}
In the second term $\mathbf{E}$ denotes the electric- and $\mathbf{B}$
the magnetic field. These fields are related to the vector potential
$\mathbf{A}$ and the scalar potential $\phi$ via \citet{Jackson1998}\begin{align}
\mathbf{E} & =-\frac{\partial\mathbf{A}}{\mathbf{\partial}t}-\nabla\phi\label{eq:E_potentials}\\
\mathbf{B} & =\nabla\times\mathbf{A}.\label{eq:B_vector_potential}\end{align}
In the first term of \ref{eq:full_crystal_H}, $e$ is the elementrary
charge, $z_{i}$ is the charge of particle $i$ in units of $e$ (-1
for electrons) and $m_{i}$ denotes the particle mass. The term $\mathbf{p}_{i}-ez_{i}\mathbf{A}\left(\mathbf{r}_{i}\right)$
is the mechanical momentum of particle $i$, which is invariant to
our gauge choise (see below) for $\mathbf{A}$ and $\phi$. The electrical
field term $\mathbf{A}^{2}$ holds both transverse and longitudinal
contributions. In the following discussion we assume a Coulomb gauge
for the vector potential, i.e. $\nabla\cdot\mathbf{A}=0$ \citet{Jackson1998}.
In this gauge the term $\frac{\partial\mathbf{A}}{\partial t}\nabla\phi$
in $\mathbf{E}^{2}$ vanishes, and so we can split this term to the
transverse and longitudinal contributions $\mathbf{E}^{2}=\mathbf{E}_{l}^{2}+\mathbf{E}_{t}^{2}$.
Using Poisson's equation, the longitudinal contribution can be written
in terms of the charge density \begin{equation}
\int\mathbf{E}_{l}^{2}d\mathbf{r}=\int\left(\nabla\phi\right)^{2}d\mathbf{r}=\langle\phi\nabla\phi\rangle-\int\phi\nabla^{2}\phi d\mathbf{r}=\int\phi\frac{\rho}{\epsilon_{0}}d\mathbf{r}.\end{equation}
Next, the contributions to $\phi$ are distinguished between internal
$\phi_{int}$ and external $\phi_{ext}$, where the internal contributions
are created by the crystal electrons and atomic cores. The internal
contributions are given by the longitudinal Coulomb interaction between
the charged particles $i$ and $j$\begin{equation}
V_{i,j}=\frac{e^{2}z_{i}z_{j}}{4\pi\epsilon_{0}|\mathbf{r}_{i}-\mathbf{r}_{j}|},\end{equation}
with unit charges $z_{i}$ and $z_{j}$. The external contribution
are \begin{equation}
V_{ext}=\sum_{i}z_{i}e\phi_{ext}(\mathbf{r}_{i}).\end{equation}


The crystal Hamiltonian for electrons $i,j$ and atoms $I,J$ then
reads\begin{eqnarray}
H & = & \underbrace{\sum_{i}\frac{\left(\mathbf{p}_{i}+e\mathbf{A}\right)^{2}}{2m_{0}}}_{T_{e}}+\underbrace{\sum_{I}\frac{\left(\mathbf{p}_{I}+ez_{I}\mathbf{A}\right)^{2}}{2m_{I}}}_{T_{a}}+\underbrace{\frac{1}{2}\sum_{i,j}\frac{e^{2}}{4\pi\epsilon_{0}|\mathbf{r}_{i}-\mathbf{r}_{j}|}}_{V_{ee}}\nonumber \\
 &  & +\underbrace{\frac{1}{2}\sum_{I,J}\frac{e^{2}z_{I}z_{J}}{4\pi\epsilon_{0}|\mathbf{r}_{I}-\mathbf{r}_{J}|}}_{V_{aa}}-\underbrace{\frac{1}{2}\sum_{i,J}\frac{e^{2}z_{J}}{4\pi\epsilon_{0}|\mathbf{r}_{i}-\mathbf{r}_{J}|}}_{V_{ea}}+V_{ext}\nonumber \\
 &  & +\underbrace{\frac{\epsilon_{0}}{2}\int\left(\mathbf{E}_{t}^{2}+c_{0}\mathbf{B}^{2}\right)d\mathbf{r}}_{H_{EM}},\end{eqnarray}
where $T$ is the kinetic energy part of the Hamiltonian, and $V$
the potential energy one. As the atoms have a much higher mass than
the electrons, the Born-Oppenheimer approximation \citet{Yu2005}
can be used to separate the motion of electrons and atoms. The electrons
will react on movements of the atoms instantaneously. Therefore, the
interaction between ion cores and between electrons and ion cores
can be concentrated into a potential $U(\mathbf{x})$, leading to
the final crystal electron Hamiltonian\begin{eqnarray}
H & = & \sum_{i}\left(\frac{\left(\mathbf{p}_{i}+e\mathbf{A}(\mathbf{r}_{i})\right)^{2}}{2m_{0}}+U(\mathbf{r}_{i})\right)\nonumber \\
 &  & +\frac{1}{2}\sum_{i,j}\frac{e^{2}}{4\pi\epsilon_{0}|\mathbf{r}_{i}-\mathbf{r}_{j}|}+V_{ext}+H_{EM}.\label{eq:crystal_H_final}\end{eqnarray}


The next step is to perform the \emph{single electron approximation}
or \emph{mean field approximation} by assuming that the electron-electron
interaction $V_{ee}$ with the core- and valence electrons can also
be concentrated into the potential $U(\mathbf{r})$. This step is
necessary as solving the equation for $10^{23}$ explicitly considered
electrons is an impossible task. The equation reduces to a single
electron equation, where an electron experiences the mean-field potential
$U(\mathbf{r})$. Within this step, the vector potential $\mathbf{A}$
is restricted to external excitations, where the local oscillations
within the crystal lattice are averaged out. Overall, the effect of
the external electro-magnetic field on the atoms has been excluded.
This approximation is usually justified in the case of weak fields
and heavy atomic masses. The interaction between electrons and atoms
is restricted to a frozen lattice. The interaction between the movement
of the atoms and the electrons can be reintroduced using \emph{phonons}
\citet{Yu2005}, but this will be omitted throughout the thesis.


\subsection{Bloch's Theorem}

Assuming no external field, the time independent single electron Schr\"{o}dinger 
equation to solve is given by\begin{equation}
\left(-\frac{\hbar^{2}}{2m_{0}}\nabla^{2}+U(\mathbf{r})\right)\psi=E\psi.\end{equation}
%
\begin{figure}
\begin{centering}
\includegraphics[scale=0.4]{\string"C:/Users/Yossi Michaeli/Documents/Thesis/Documents/Thesis/Figures/2_Bloch_Theorem_Sch\string".eps}
\par\end{centering}

\caption{\label{fig:Bloch_Theorem_Sch}Schematic representation of electronic
functions in a crystal (a) potential plotted along a row of atoms,
(b) free electron wave function, (c) amplitude factor of Bloch function
having the periodicity of the lattice, and (d) Bloch function $\psi$
(after \citet{Piprek2003})}



\end{figure}


For the moment, the crystal is assumed to be homogeneous and perfect.
The potential $U(\mathbf{r})$ is periodic within the lattice (see
Figure \ref{fig:Bloch_Theorem_Sch}). For any translation vector $\mathbf{R}$
mapping the infinite crystal lattice to itself, the potential obeys\begin{equation}
U(\mathbf{r})=U(\mathbf{r}+\mathbf{R})\end{equation}
and the wavefunction $\psi_{n\mathbf{k}}$ can be expressed in the
form known as \emph{Bloch function\begin{equation}
\psi_{n\mathbf{k}}=u_{n\mathbf{k}}(\mathbf{r})e^{i\mathbf{k}\cdot\mathbf{r}},\end{equation}
}where $u_{n\mathbf{k}}(\mathbf{r})$ denotes the lattice periodic
path with \begin{equation}
u_{n\mathbf{k}}(\mathbf{r})=u_{n\mathbf{k}}(\mathbf{r}+\mathbf{R})\end{equation}
and the plane wave is the slowly modulating envelope. The $n\mathbf{k}$
are the quantum numbers indexing the solutions. Applying the differential
operators to the plane wave and multiplying the equation on both sides
from the left with $e^{\lyxmathsym{\textminus}i\mathbf{k}\cdot\mathbf{r}}$
gives the equation for the lattice periodic functions as\begin{eqnarray}
H_{\mathbf{k}\cdot\mathbf{p}}(\mathbf{k})u_{n\mathbf{k}}(\mathbf{r}) & = & \left(-\frac{\hbar^{2}}{2m_{0}}\nabla^{2}+\frac{\hbar}{m_{0}}\mathbf{k}\cdot\mathbf{p}+\frac{\hbar^{2}k^{2}}{2m_{0}}+U(\mathbf{r})\right)u_{n\mathbf{k}}(\mathbf{r})\nonumber \\
 & = & E_{n}(\mathbf{k})u_{n\mathbf{k}}(\mathbf{r}).\label{eq:Bloch_Hamiltonian}\end{eqnarray}
An important consequence of the Bloch theorem is the fact that wavefunctions
with different $\mathbf{k}$ values are not coupled together (due
to the slowly varying plane wave) and therefore \ref{eq:Bloch_Hamiltonian}
has a parametric dependence on the crystal momentum $\mathbf{k}$.


\subsection{The $\mathbf{k}\cdot\mathbf{p}$ Method}


\subsubsection{Introduction}

There is a vast number of methods to obtain solutions to Eq. \ref{eq:Bloch_Hamiltonian}.
One of the most frequently used is the semi-empirical $\mathbf{k}\cdot\mathbf{p}$
method that serves to derive analytical expressions for the band structure,
using symmetry arguments and experimental observations. The method
was initially introduced by Bardeen and Seitz (see references in \citet{Chuang1995})
and applied by many researchers. It is particularly useful to describe
the band structure for direct semiconductors used in optoelectronic
devices at the $\Gamma$ point of the Brillouin zone with a high precision. 

Hereafter a short introduction of the general concepts will be given.
These concepts will later reappear within the Envelope Fucntion Methods
(EFM). This introduction is mainly based on \citet{Yu2005}.

The basic idea within the $\mathbf{k}\cdot\mathbf{p}$ theory \citet{kane_handbooksemiconductors_1982}
is to solve \ref{eq:Bloch_Hamiltonian} for an extremal point with
high symmetry of the band structure, usually the $\Gamma$ point at
$\mathbf{k}=0$. There, \ref{eq:Bloch_Hamiltonian} reduces to\begin{equation}
\left(-\frac{\hbar^{2}}{2m_{0}}\nabla^{2}+U(\mathbf{r})\right)u_{n0}(\mathbf{r})=E_{n}(0)u_{n0}(\mathbf{r}).\label{eq:zero_center_Bloch_Equation}\end{equation}
The solutions $u_{n0}(\mathbf{r})$ are denoted as \emph{zone-center}
functions and span the complete Hilbert space of all solutions to
\ref{eq:Bloch_Hamiltonian}. Therefore, one may express the lattice
periodic functions $u_{n\mathbf{k}}(\mathbf{r})$ away from the $\Gamma$
point in terms of the zone-center functions\begin{equation}
u_{m\mathbf{k}}(\mathbf{r})=\sum_{n'}a_{m\mathbf{k},n'}u_{n'0}(\mathbf{r}).\label{eq:Bloch_Function_k_Expansion}\end{equation}
To obtain the coefficients $a_{m\mathbf{k},n'}$, the expansion \ref{eq:Bloch_Function_k_Expansion}
is inserted into \ref{eq:Bloch_Hamiltonian}, multiplied from the
left by $u_{n0}(\mathbf{r})$ and integrated over the crystal unit-cell
to obtain\begin{equation}
\sum_{n'}\left(\left(E_{n'}(0)+\frac{\hbar^{2}k^{2}}{2m_{0}}\right)\delta_{nn'}+\frac{\hbar}{m_{0}}\mathbf{k}\cdot\mathbf{p}_{nn'}\right)a_{m\mathbf{k},n'}=E_{m}(\mathbf{k})a_{m\mathbf{k},n'},\end{equation}
where \begin{equation}
\mathbf{p}_{nn'}=\int u_{n0}^{*}(\mathbf{r})\mathbf{p}u_{n'0}(\mathbf{r})d\mathbf{r}\label{eq:momentum_matrix_element}\end{equation}
is used to express the momentum matrix element between two zone-center
Bloch functions. The above equation can obviously be written in matrix
form, where a matrix entry is given by\begin{equation}
h_{ij}=\left(E_{j}(0)+\frac{\hbar^{2}k^{2}}{2m_{0}}\right)\delta_{ij}+\frac{\hbar}{m_{0}}\mathbf{k}\cdot\mathbf{p}_{ij}\label{eq:k_dot_p_H_matrix}\end{equation}
and a diagonalization of the\textbf{$\mathbf{k}$} dependent, infinite
matrix would lead to the exact coefficients and energies $E_{m}(\mathbf{k})$.
As the matrix is continuous in \textbf{$\mathbf{k}$}, is is clear
that the dispersion $E_{m}(\mathbf{k})$ will also be continuous.

As a remark, we can point out that \ref{eq:zero_center_Bloch_Equation}
is not solved explicitly and no closed expression for $u_{n0}(\mathbf{r})$
is needed. Instead, the matrix \ref{eq:k_dot_p_H_matrix} is constructed
by using group theory to derive symmetry properties of the zone-center
functions $u_{n0}(\mathbf{r})$. Using these symmetry properties,
similarities and equivalences for the momentum matrix elements $\mathbf{p}_{ij}$
can be deduced. A profound introduction into the group theory for
semiconductors is beyond the current scope. A rough guideline illustrating
the essence required by the $\mathbf{k}\cdot\mathbf{p}$ theory is
presented in Appendix \ref{cha:Appendix_Schrodinger_Poisson}. An
extensive presentation can be found in \citet{Yu2005} and \citet{Bir1974a}.


\subsubsection{L\"{o}wdin-Renormalization}

While the matrix $h_{ij}$ \ref{eq:k_dot_p_H_matrix} is infinite,
leading to an infinite number of bands, the assessment of light emission
in optoelectronic devices usually requires only the knowledge of the
lowest conduction and highest valence bands. For the relevant III-V
semiconductors, the conduction and the valence bands lie at the $\Gamma$
point energetically close together. Other bands can be regarded as
being remote, so the interaction in-between these bands can be considered
to dominate for the band structure of interest. In the model developed
by Kane \citet{Kane1957}, therefore a submatrix $h_{ij}$ of the
conduction- and valence band at the $\Gamma$ point is extracted out
of \ref{eq:k_dot_p_H_matrix} and diagonalized. Although the primary
interest of Kane\textquoteright{}s model was to include spinorbit
interaction, the missing interaction with remote bands resulted in
a heavy-hole band structure, given by the free-electron dispersion,
bending for a hole into the wrong direction. Luttinger and Kohn \citet{Kohn1955a}
included the interaction of remote states onto the selected set of
bands, using L\"{o}wdin\textquoteright{}s perturbation theory \citet{Lowdin1951a}.
The idea of L\"{o}wdin perturbation theory is to divide the bands
in two classes $S$ and $R$ (see Fig. \ref{fig:Schematic-band-structure}).
Bands in class $S$ are considered explicitly, i.e. the respective
rows and columns of the matrix $h_{ij}$ are kept. The bands $R$
are considered being remote and their effect on the bands in class
$S$ are included in the submatrix of $h_{ij}$ using perturbation
theory. Hereby, the renormalized matrix elements are given by\begin{equation}
h_{ij}^{\prime}=h_{ij}+\sum_{\nu}^{R}\frac{h_{i\nu}h_{\nu j}}{E_{i}-h_{\nu\nu}},\end{equation}
where $i$ and $j$ are in class $S$ and $\nu$ is in $R$. %
\begin{figure}
\begin{centering}
\includegraphics[scale=0.4]{\string"C:/Users/Yossi Michaeli/Documents/Thesis/Documents/Thesis/Figures/2_Luttinger_Kohn_Sch\string".eps}
\par\end{centering}

\caption{\label{fig:Schematic-band-structure}Schematic band structure classification
in the L\"{o}wdin perturbation theory.}



\end{figure}
 Group theory and symmetry arguments are then used to derive similarities
and vanishing terms and reduce the perturbation expressions to a few
constants. By analytically diagonalizing the remaining matrix, the
few remaining constants can be determined by comparing analytical
dispersion expressions to experimentally determined effective masses.
An important note here is that $E_{i}$ denotes the energy of band
$i$ such that the renormalization of the matrix element depends on
the result of the eigenvalue calculation, actually requiring self-consistency.
In practice, the energies $E_{i}$ are approximated by the zone-center
energy $E_{i}(0)$, given by the matrix element $h_{ii}$, which is
fine close the zone-center. By keeping only one single band in class
$S$, one obtains the single band effective mass dispersion for band
$i$ as\begin{equation}
E_{i}(0)+\left(\frac{\hbar^{2}k^{2}}{2m_{0}}+\frac{\hbar}{m_{0}}\sum_{\nu}^{R}\frac{\mathbf{k}\cdot\mathbf{p}_{s\nu}\mathbf{p}_{\nu s}\cdot\mathbf{k}}{E_{i}-h_{\nu\nu}}\right)=E_{i}(0)+\frac{\hbar^{2}}{2}\mathbf{k}^{T}\frac{1}{\mathbf{m}^{*}}\mathbf{k}.\end{equation}
In the case of several bands within the class S, the resulting matrix
can be written, ordered by the dependence on the wavenumber $\mathbf{k}$,
as\begin{equation}
\sum_{i,j=x,y,z}\mathbf{H}_{i,j}^{(2)}k_{i}k_{j}+\sum_{i=x,y,z}\mathbf{H}_{i,j}^{(1)}k_{i}+\mathbf{H}^{(0)}.\end{equation}
To summarize, the second order terms $\mathbf{H}_{i,j}^{(2)}$ are
a result of the combination of the free carrier dispersion and the
perturbation treatment of remote states in class $R$, while first
order terms $\mathbf{H}_{i,j}^{(1)}$ stem from the direct treatment
of the $\mathbf{k}\cdot\mathbf{p}$ interaction (and later from linear
spin-orbit terms), while zero order terms $\mathbf{H}^{(0)}$, contain
zone-center energies (and possible terms if the zone-center basis
is not orthogonal).


\section{The $\mathbf{k}\cdot\mathbf{p}$ Envelope Function Method}


\subsection{Introduction}

In previous sections, the crystal was assumed to be homogeneous and
infinitely extended. In nanostructures, this assumption is no longer
valid and the translational symmetry is broken in certain directions.
As a consequence, a carrier might be energetically confined within
a lower-bandgap material, but still free to propagate within the translational
invariant direction. In the case of a quantum well, the symmetry is
broken by the atoms of the other species in one direction, for a quantum
wire, it is broken in two directions and for a quantum dot in all
three directions. Consequently, the number of translational invariant
directions is reduced from two for a quantum well to zero for a quantum
dot. As a result, the Bloch function employing the plane wave ansatz
has to be modified\begin{equation}
\varphi_{m\mathbf{k}_{t}}(\mathbf{r},z)=\sum_{n}u_{n}(\mathbf{r},z)e^{i\mathbf{k}_{t}\cdot\mathbf{r}}F_{n\mathbf{k},m}(z).\label{eq:Envelope_Function_Bloch_Function}\end{equation}
Here, $\mathbf{r}$ denotes the coordinate of translational invariant
direction(s), $z$ is the coordinate of the direction(s) where the
crystal symmetry is broken and $u_{n}(\mathbf{r},z)$ is a lattice-periodic
function. The crystal momentum \textbf{$\mathbf{k}$} is only defined
within the translational invariant direction. The expression $F_{n\mathbf{k},m}(z)$
is referred to as slowly-varying envelope and denotes at every position
in the symmetry broken direction $z$, how the lattice-periodic functions
are mixed together. In the bulk crystal, the plane wave term decouples
the wavefunctions with different crystal momenta $\mathbf{k}$. In
a nanostructure, this decoupling is only true for the translational
invariant direction, while in the symmetry broken direction, the states
are now mixed together. 

As a result of the symmetry breaking, the bands are split into \emph{subbands},
depending on the transversal crystal momentum $\mathbf{k}_{t}$. The
task of the envelope function method is now to select a suitable set
of lattice periodic functions $u_{n}(\mathbf{r},z)$ for \ref{eq:Envelope_Function_Bloch_Function}
and derive a proper equation to determine the envelopes $F_{n\mathbf{k},m}(z)$.


\subsection{$\mathbf{k}\cdot\mathbf{p}$ Envelope Function Ansatz}

The traditional ansatz is to use the zone-center $\mathbf{k}\cdot\mathbf{p}$
lattice periodic functions $u_{n0}(\mathbf{r})$ for the expansion
\ref{eq:Envelope_Function_Bloch_Function}, together with a set of
matching conditions. The equation for the envelopes is obtained by
replacing the wave number $k_{i}$ by the corresponding operator $\lyxmathsym{\textminus}i\partial_{i}$.
As in the nanostructure, different materials are involved, the wavefunction
is in each material expanded into the materials zone-center functions.
The result is that the effective mass like parameters from the perturbation
interaction with remote states and the zone-center energies are position
dependent. For the single-band effective mass approximation for e.g.
the conduction band in a quantum well for $\mathbf{k}_{t}=0$, the
envelope equation is given by\begin{equation}
\left(-\frac{\hbar^{2}}{2m^{*}(z)}\frac{\partial^{2}}{\partial z^{2}}+E_{c}(z)\right)F_{s}(z)=E(0)F_{s}(z).\label{eq:QW_Conduction_Band_Envelope_Schr_Eq}\end{equation}
$E_{c}(z)$ denotes the position dependent bulk band edge, corresponding
in the direct band gap semiconductors to the conduction band energy
at $\Gamma$. Once the eigenvalues $E_{m}(0)$ and normalized eigenfunctions
$F_{s,m}(z)$ (indexed by the subband quantum number $m$) of \ref{eq:QW_Conduction_Band_Envelope_Schr_Eq}
are obtained, the in-plane dispersion is usually approximated using
the dominant effective mass of the quantum well material\begin{equation}
E_{m}(\mathbf{k})=E_{m}(0)+\frac{\hbar^{2}k^{2}}{2m_{c}^{*}}.\end{equation}
The way the equation is written, it is in the presence of a material
interface not Hermitian and therefore unexpected imaginary eigenvalues
for the energy would result. The usual ad-hoc fix, justified by the
requirement of a continuous probability flow, is to rewrite the second
order differential operator \citet{Chuang1995}\begin{equation}
-\frac{\hbar^{2}}{2m^{*}(z)}\nabla^{2}\rightarrow-\nabla\frac{\hbar^{2}}{2m^{*}(z)}\nabla.\label{eq:Second_Order_Diff_Operator}\end{equation}
The particular order of the coefficient and the differential operators
is commonly referred to as operator ordering. The form of \ref{eq:Second_Order_Diff_Operator}
is denoted as Ben-Daniel and Duke ordering \citet{BenDaniel1966},
but within the single-band effective mass theory, other orderings
are suggested too (see \citet{Morrow1984} and references therein).
The ordering is irrelevant within bulk materials, but plays a substantial
role at a material interface: it is equivalent to the matching conditions
for the bulk Bloch functions. In the case of a multi-band equation,
where not only one, but several lattice periodic functions are included,
one obtains in analogy to \ref{eq:Second_Order_Diff_Operator} for
all envelopes involved $\mathbf{F}(z)=(F_{1}(z),F_{2}(z),...,F_{M}(z))^{T}$
a system of coupled partial differential equations\begin{equation}
\mathbf{H}_{\mathbf{k}\cdot\mathbf{p}}(z)\mathbf{F}(z)=E\mathbf{F}(z),\label{eq:General_Envelope_Function_Schr_Eq}\end{equation}
where the $\mathbf{k}\cdot\mathbf{p}$ differential operator is given
by\begin{eqnarray}
\mathbf{H}_{\mathbf{k}\cdot\mathbf{p}}(z) & = & -\sum_{i,j}\partial_{i}\mathbf{H}_{i,j}^{(2)}(\mathbf{r};\mathbf{k}_{t})\partial_{j}\nonumber \\
 &  & +\sum_{i}\left(\mathbf{H}_{i;L}^{(1)}(\mathbf{r};\mathbf{k}_{t})\partial_{i}+\partial_{i}\mathbf{H}_{i;R}^{(1)}(\mathbf{r};\mathbf{k}_{t})\right)\nonumber \\
 &  & +\mathbf{H}^{(0)}(\mathbf{r};\mathbf{k}_{t}).\end{eqnarray}
Here the Hermitian operator ordering has already been introduced.
The problem is that within the bulk $\mathbf{k}\cdot\mathbf{p}$ Hamiltonian,
terms of the type $Nk_{i}k_{j}$ with $i\neq j$ can appear, for which
the operator ordering is not clear. The usual fix is to split the
contribution symmetrically\begin{equation}
Nk_{i}k_{j}\rightarrow\partial_{i}\frac{N}{2}\partial_{j}-\partial_{j}\frac{N}{2}\partial_{i}.\label{eq:Symmetrized_Operator_Ordering}\end{equation}
It can be shown that this particular choice, which is called \emph{symmetrized
operator ordering}, may lead to erroneous results. The traditional
envelope equations for one and several bands are widely applied and
used in a variety of numerical calculations of quantum wells (and
superlattices) \citet{Altarelli1983,Bastard1981,Bastard1982,Eppenga1987,Chuang1991a}.

Beside its successful application to some material systems, the traditional
way of deriving the envelope equations contains several ad hoc fixes,
which are required within the presence of material interfaces. The
operator ordering is crucial at a material interface, where the material
parameters change and therefore the ad-hoc operator ordering involves
unknown approximations made to the effect of the interface. In order
to analyze the involved approximations, Burt \citet{Burt1994,Burt1999,Burt1992,Burt1988}
derived the operator ordering \ref{eq:Second_Order_Diff_Operator}
from the nanostructure\textquoteright{}s Schr\"{o}dinger  equation
and demonstrated that the symmetrization approach for the terms \ref{eq:Symmetrized_Operator_Ordering}
in the multi-band envelope equation is incorrect. The derived multi-band
Hamiltonian is Hermitian, but not necessarily symmetric. The derivation
focusing on operator ordering is not summarized here, but the reader
is reffered to reviews \citet{Burt1992,Burt1999} and for the numerical
estimates of the introduced errors to \citet{Burt1994}. In essence,
Burt\textquoteright{}s theory clearly demonstrated the existence and
omission of interface terms and showed that exact envelope equations
could be derived from the original Schr\"{o}dinger  equation. But
Burt\textquoteright{}s theory is not widely applied. The issue of
operator ordering attracts growing interest and it is crucial for
the mathematical stability of the envelope equations \citet{Veprek2008}.
Foreman \citet{Foreman1993} derived the correct operator ordering
for zinc-blende crystals and therefore the derived operator ordering
is commonly referred to as Burt-Foreman operator ordering. Beside
the operator ordering, the theory regarding the omitted interface
terms in the traditional model is only used rarely \citet{Foreman1998a,Foreman1995b}.
The effect of the interface terms is to perturb the system and create
an additional mixing of the bands. In the traditional envelope equations,
the effect of the interface was introduced using phenomenological
models (see e.g. \citet{Krebs1999b} and references therein) or by
a variational least action principle \citet{Rodina2006}. The reason
of the fundamental popularity of the traditional envelope function
method over Burt\textquoteright{}s method is given by its simple form,
which allows to write quantum well solvers within days, and the fact
that mostly known, bulk input parameters can be used, which is not
the case for the interface terms within the exact envelope function
theory. 


\subsection{The Zinc-blende Models}


\subsubsection{Direct Interaction}

To apply the envelope equations \ref{eq:General_Envelope_Function_Schr_Eq}
to direct-bandgap semiconductor nanostructures, the exact form of
the $\mathbf{k}\cdot\mathbf{p}$ Hamiltonian at the $\Gamma$ point
is required. Ignoring spin, the top of the valence band at $\Gamma$
is triply degenerate, corresponding to the $\Gamma_{15}$ representation
with $p$-type basis functions $x,y$ and $z$, while the lowest-lying
conduction band is of $s$-type symmetry, corresponding to $\Gamma_{1}$.
The direct interaction terms \ref{eq:k_dot_p_H_matrix} of the $\mathbf{k}\cdot\mathbf{p}$
matrix for $s,x,y$ and $z$ are given by \citet{kane_handbooksemiconductors_1982}\begin{equation}
\mathbf{H}_{d}^{4\times4}=\left(\begin{array}{r|lccc}
 & |s\rangle & |x\rangle & |y\rangle & |z\rangle\\
\hline |s\rangle & E_{c}+\frac{\hbar^{2}}{2m_{0}} & iPk_{x} & iPk_{y} & iPk_{z}\\
|x\rangle & -ik_{x}P & E_{v}+\frac{\hbar^{2}}{2m_{0}} & 0 & 0\\
|y\rangle & -ik_{y}P & 0 & E_{v}+\frac{\hbar^{2}}{2m_{0}} & 0\\
|z\rangle & -ik_{z}P & 0 & 0 & E_{v}+\frac{\hbar^{2}}{2m_{0}}\end{array}\right).\label{eq:Direct_Int_Matrix}\end{equation}
Here, $E_{c}$ and $E_{v}$ correspond to the zone-center energies\begin{eqnarray}
E_{c} & = & \langle s|H|s\rangle\\
E_{v} & = & \langle x|H|x\rangle\label{eq:Valence_Band_Edge_Energy}\\
P & = & -\frac{i\hbar}{m_{0}}\langle s|p_{x}|s\rangle\end{eqnarray}
and $P$ the nonzero interband momentum matrix element from the direct
$\mathbf{k}\cdot\mathbf{p}$ interaction between the conduction- and
the valence band. The momentum matrix element is often given as an
energy parameter $E_{p}$ , the \emph{optical matrix parameter}, related
to $P$ by\begin{equation}
E_{p}=\frac{2m_{0}}{\hbar^{2}}P^{2}.\end{equation}
For the $\Gamma_{15}$ states, the nonzero momentum matrix element
is of the type $\langle x|p_{y}|z\rangle$, where no coordinate appears
twice \citet{Yu2005} (the crystal is invariant under a rotation of
180\textdegree{} around one of the axes). Therefore, one would expect
the direct interaction within the valence band $\mathbf{k}\cdot\langle x|\mathbf{p}|y\rangle$
resulting into a linear term in $\mathbf{k}$ given explicitly by
$k_{z}\langle x|p_{z}|y\rangle$. In fact, the first order matrix
elements within $\Gamma_{15}$ vanish due to time reversal symmetry
of the Hamiltonian \citet{Dresselhaus1955a}. For example, a reflection
in the (110) plane for $\langle x^{a}|p_{z}|y^{b}\rangle$ results
in $\langle y^{a}|p_{z}|x^{b}\rangle$, while integration by parts
gives $\langle x^{a}|p_{z}|y^{b}\rangle=-\langle y^{b}|p_{z}|x^{a}\rangle$.
Here $a$ and $b$ index the degenerate level. If the states are from
the same degenerate level, then $a=b$ and the matrix elements vanish.
If the states are not from the same degenerate level, the interaction
is nonzero.

The operator ordering in the first order terms, e.g. $H_{sx}=iPk_{x}$
and $H_{xs}=-ik_{x}P$ is an ad hoc fix. If $P$ is allowed to vary,
the here chosen ordering is required to ensure the Hamiltonians Hermiticity.


\subsubsection{Remote Contributions}

The next step is to include the interaction between the remote states
and keep track of the correct operator ordering. It is clear from
table \ref{tab:Symmetry_operations_Td} that the remote contributions
to the conduction band stem from remote $\Gamma_{15}$ states, while
the valence band states have contributions from all except the $\Gamma_{2}$
states. The corresponding terms are given by the matrix $\mathbf{H}_{r}^{4\times4}$
\ref{eq:Remote_Contributions_Hamiltonian} and the coefficients \ref{eq:Hr4_Coefficients}.
In deriving these terms, the energy for the energy-dependent renormalization
has been replaced by the according band edge energies $E_{v}$ and
$E_{c}$. Although the approximation might be reasonable for very
remote bands $\nu$, it remains crude. The derivation of the terms
tese terms involves only symmetry arguments similar to the already
used arguments within this chapter. The bar in $\bar{A_{c}}$, $\bar{L}$
and $\bar{M}$ indicates the close relationship to the parameters
$\tilde{A_{c},}\tilde{L}'$ and $\tilde{M}$ , which include the free-electron
dispersion\begin{equation}
\tilde{A_{c}}=\bar{A_{c}}+\frac{\hbar^{2}}{2m_{0}},\,\,\tilde{L}'=\bar{L}+\frac{\hbar^{2}}{2m_{0}},\,\,\tilde{M}=\bar{M}+\frac{\hbar^{2}}{2m_{0}},\end{equation}
whereas in this case the free electron dispersion is still contained
in the direct interaction matrix \ref{eq:Direct_Int_Matrix}. The
operator ordering in the offdiagonal terms $k_{y}\tilde{N}_{+}k_{x}+k_{x}\tilde{N}_{-}k_{y}$
has been derived by Foreman \citealp{Foreman1995b}. The term $\tilde{N}_{+}$
contains the contribution from $\Gamma_{1}\,(F)$ and $\Gamma_{12}\,(G)$
bands, while $\tilde{N}_{-}$ contains the contributions from $\Gamma_{15}\,(H_{1})$
and $\Gamma_{25}\,(H_{2})$ bands. In the bulk $\mathbf{k}\cdot\mathbf{p}$
theory, the ordering is irrelevant and the terms are summed together
into the Kane's parameter $\tilde{N}'$\begin{equation}
\tilde{N}'=\tilde{N}_{+}+\tilde{N}_{-}.\label{eq:Kane_Parameter}\end{equation}
Within the traditional $\mathbf{k}\cdot\mathbf{p}$ envelope function
method applied to heterostructure, the symmetrized operator ordering
\ref{eq:Symmetrized_Operator_Ordering} simply uses\begin{equation}
\tilde{N}_{+}=\tilde{N}_{-}=\frac{\tilde{N'}}{2}.\end{equation}
The parameters $\tilde{L}'$, $\tilde{M}$ and $\tilde{N}$ are usually
determined by calculating analytical expressions for the dispersion
and relating them to measured effective masses. The detailed splitting
of $\tilde{N}'$ into $\tilde{N}_{+}$ and $\tilde{N}_{-}$ is not
directly accessible from experiment, but can be estimated \citet{Foreman1993}
using the following arguments: The term $\tilde{N}_{-}$ contains
contributions of $\Gamma_{15}$ and $\Gamma_{25}$ bands. The $\Gamma_{25}$
bands can only be formed by $f$-type and higher atomic orbitals,
while $\Gamma_{15}$ bands can be formed by $p$-,$d$-,$f$- and
higher type orbitals \citet{Foreman1995b}. Within the usual semiconductors,
the influence of $f$ orbitals is insignificant. Therefore, neglecting
the influence of the $\Gamma_{25}$ bands ($H_{2}=0$) and using \ref{eq:Hr4_coefficients},
the value for $\tilde{N}_{-}$ is given by \begin{equation}
\tilde{N}_{-}=\bar{M}=\tilde{M}-\frac{\hbar^{2}}{2m_{o}}.\end{equation}
$\tilde{N}_{+}$ can be deduced using \ref{eq:Kane_Parameter} and
the experimentally determined value for $\tilde{N}'$. In a usual
zinc-blende semiconductor such as $GaAs$, $\tilde{M}$ = \textminus{}2.65
and $\tilde{N}'$ = \textminus{}17.4 (in units of $\frac{\hbar^{2}}{2m_{0}}$).
The symmetric operator ordering would therefore use $\tilde{N}_{+/-}$
= \textminus{}8.7, while the Burt-Foreman ordering gives strongly
asymmetric values $\tilde{N}_{+}$= \textminus{}13.75 and $\tilde{N}_{-}$
= \textminus{}3.65. In contrast to the operator ordering within the
valence band, the ordering in the off-diagonal terms between the conduction
and valence bands, given by the terms $k_{y}Ck_{z}+k_{z}Ck_{y}$,
is required to be symmetric. Only remote $\Gamma_{15}$ type bands
mix into the $s$-type conduction band, and therefore, the non-vanishing
elements from the perturbation are\begin{equation}
k_{y}\left\langle s\mid p_{y}\mid y^{r}\right\rangle \left\langle y^{r}\mid p_{z}\mid x\right\rangle k_{z}+k_{z}\left\langle s\mid p_{z}\mid y^{r}\right\rangle \left\langle z^{r}\mid p_{y}\mid x\right\rangle k_{y}.\end{equation}
By crystal symmetry, these terms are equal. The term $C$ is related
to the common Kane parameter $B=2C$. As $B$ is usually small, it
is commonly neglected, i.e. $C=0$.

\newpage{}

\begin{landscape}

\begin{equation}
\mathbf{H}_{r}^{4\times4}=\left(\begin{array}{r|lccc}
 & |s\rangle & |x\rangle & |y\rangle & |z\rangle\\
\hline |s\rangle & k\bar{A}_{c}k & k_{y}Ck_{z}+k_{z}Ck_{y} & k_{x}Ck_{z}+k_{z}Ck_{x} & k_{y}Ck_{x}+k_{x}Ck_{y}\\
|x\rangle & k_{z}Ck_{y}+k_{y}Ck_{z} & k_{x}\bar{L}k_{x}+k_{y}\bar{M}k_{y}+k_{z}\bar{M}k_{z} & k_{x}\tilde{N}_{+}k_{y}+k_{y}\tilde{N}_{-}k_{x} & k_{x}\tilde{N}_{+}k_{z}+k_{z}\tilde{N}_{-}k_{x}\\
|y\rangle & k_{z}Ck_{x}+k_{x}Ck_{z} & k_{y}\tilde{N}_{+}k_{x}+k_{x}\tilde{N}_{-}k_{y} & k_{y}\bar{L}k_{y}+k_{x}\bar{M}k_{x}+k_{z}\bar{M}k_{z} & k_{y}\tilde{N}_{+}k_{z}+k_{z}\tilde{N}_{-}k_{y}\\
|z\rangle & k_{x}Ck_{y}+k_{y}Ck_{x} & k_{z}\tilde{N}_{+}k_{x}+k_{x}\tilde{N}_{-}k_{z} & k_{z}\tilde{N}_{+}k_{y}+k_{y}\tilde{N}_{-}k_{z} & k_{z}\bar{L}k_{z}+k_{x}\bar{M}k_{x}+k_{y}\bar{M}k_{y}\end{array}\right),\label{eq:Remote_Contributions_Hamiltonian}\end{equation}


\begin{gather}
\bar{L}=F+2G,\,\,\,\,\,\,\,\,\,\,\bar{M}=H_{1}+H_{2},\nonumber \\
\tilde{N}_{+}=F-G,\,\,\,\,\,\,\,\,\,\,\tilde{N}_{-}=H_{1}-H_{2},\nonumber \\
F=\frac{\hbar^{2}}{m_{0}^{2}}\sum_{\nu}^{\Gamma_{1}}\frac{\left|\left\langle x\mid p_{x}\mid u_{\nu}\right\rangle \right|^{2}}{\left(E_{v}-E_{\nu}\right)},\,\,\,\,\, G=\frac{\hbar^{2}}{m_{0}^{2}}\sum_{\nu}^{\Gamma_{12}}\frac{\left|\left\langle x\mid p_{x}\mid u_{\nu}\right\rangle \right|^{2}}{\left(E_{v}-E_{\nu}\right)},\label{eq:Hr4_Coefficients}\\
H_{1}=\frac{\hbar^{2}}{m_{0}^{2}}\sum_{\nu}^{\Gamma_{15}}\frac{\left|\left\langle s\mid p_{x}\mid u_{\nu}\right\rangle \right|^{2}}{\left(E_{c}-E_{\nu}\right)},\,\,\,\,\, H_{2}=\frac{\hbar^{2}}{m_{0}^{2}}\sum_{\nu}^{\Gamma_{25}}\frac{\left|\left\langle x\mid p_{x}\mid u_{\nu}\right\rangle \right|^{2}}{\left(E_{v}-E_{\nu}\right)},\nonumber \\
\bar{A}_{c}=\frac{\hbar^{2}}{m_{0}^{2}}\sum_{\nu}^{\Gamma_{15}}\frac{\left|\left\langle s\mid p_{x}\mid u_{\nu}\right\rangle \right|^{2}}{\left(E_{c}-E_{\nu}\right)},\,\,\,\,\, C=\frac{\hbar^{2}}{m_{0}^{2}}\sum_{\nu}^{\Gamma_{15}}\frac{\left\langle s\mid p_{x}\mid u_{\nu}\right\rangle \left\langle u_{\nu}\mid p_{x}\mid z\right\rangle }{\left(\frac{1}{2}\left(E_{c}+E_{v}\right)-E_{\nu}\right)}.\nonumber \end{gather}


\end{landscape}

\newpage{}


\subsubsection{Spin-Orbit Interaction}

Up to this point, the electron spin has been omitted. Within the semiconductors
involving heavier atoms, the spin-orbit interaction has a large impact
on the electron dispersions. The spin-orbit energy leads to additional
terms in the equation \ref{eq:QW_Conduction_Band_Envelope_Schr_Eq}
for the zone-center functions, namely\begin{eqnarray}
H_{SO,\mathbf{p}} & = & \frac{\hbar}{4m_{0}^{2}c^{2}}\left(\nabla V\times\mathbf{p}\right)\cdot\boldsymbol{\sigma},\label{eq:spin_orbit_int_p}\\
H_{SO,\mathbf{k}} & = & \frac{\hbar}{4m_{0}^{2}c^{2}}\left(\nabla V\times\mathbf{k}\right)\cdot\mathbf{\boldsymbol{\sigma}},\label{eq:spin_orbit_int_k}\end{eqnarray}
where $c$ is the vacuum speed of light and $\boldsymbol{\sigma}$
are the \emph{Pauli operators} acting on the electoron-spin variable\[
\sigma_{x}=\left(\begin{array}{cc}
0 & 1\\
1 & 0\end{array}\right),\,\,\sigma_{y}=\left(\begin{array}{cc}
0 & -i\\
i & 0\end{array}\right),\,\,\sigma_{z}=\left(\begin{array}{cc}
1 & 0\\
0 & -1\end{array}\right).\]
The Bloch basis functions have to be extended to include the spin
degree of freedom, which is done by giving the spin $z$-component,
which can be either up $\left|\uparrow\right\rangle $ or down $\left|\downarrow\right\rangle $.
The spin variable is diagonal, meaning that $\left\langle \uparrow\mid\uparrow\right\rangle =\left\langle \downarrow\mid\downarrow\right\rangle =1$
and $\left\langle \uparrow\mid\downarrow\right\rangle =0$. Using
the spin, the initial basis of four states describing the lowest conduction
band and the top valence bands is doubled to eight states\begin{equation}
\left|s\uparrow\right\rangle ,\,\left|x\uparrow\right\rangle ,\,\left|y\uparrow\right\rangle ,\,\left|z\uparrow\right\rangle ,\,\left|s\downarrow\right\rangle ,\,\left|x\downarrow\right\rangle ,\,\left|y\downarrow\right\rangle ,\,\left|z\downarrow\right\rangle .\label{eq:levels_basis_with_spin}\end{equation}
All operators in the zone-center Hamiltonian \ref{eq:zero_center_Bloch_Equation}
do not act on the spin variable. Therefore, the Hamiltonian is diagonal
in the spin variable. Within the basis \ref{eq:levels_basis_with_spin},
the Hamiltonian is given by\begin{equation}
\mathbf{H}_{rd}^{8\times8}=\left(\begin{array}{cc}
\mathbf{H}_{d}^{4\times4}+\mathbf{H}_{r}^{4\times4} & 0\\
0 & \mathbf{H}_{d}^{4\times4}+\mathbf{H}_{r}^{4\times4}\end{array}\right).\end{equation}
In contrast to this Hamiltonian, the spin-orbit interaction \ref{eq:spin_orbit_int_p}
and \ref{eq:spin_orbit_int_k} is not diagonal in the basis \ref{eq:levels_basis_with_spin}.
The reason to this lies in the symmetry of the spin. The crystal potential
$V(\mathbf{r})$ without spin terms is invariant under any rotation
around an angle of $2\lyxmathsym{\textgreek{p}}$, denoted here as
$\hat{E}$. Under such a rotation, the wavefunction including the
spin switches sign and is only invariant under the rotation of $4\lyxmathsym{\textgreek{p}}$.
Therefore, if the point group of the crystal neglecting spin is given
by $\mathcal{G}$, then the point group including spin will be given
by the elements of $\mathcal{G}$ and $\hat{E}\mathcal{G}$, leading
to the definition of the \emph{double-group} $\mathcal{\tilde{G}}$
of $\mathcal{G}$ defined as\begin{equation}
\mathcal{\tilde{G}}=\left\{ g,\,\tilde{g}=-g\right\} \,\,\,\forall g\in\mathcal{G}.\end{equation}
The spin-orbit interaction for $H_{SO,\mathbf{p}}$of \ref{eq:spin_orbit_int_p}
leads, for the basis in \ref{eq:levels_basis_with_spin}, to \citet{Enders1995a}\begin{equation}
\mathbf{H}_{SO,\mathbf{p}}=\frac{\Delta}{3}\left(\begin{array}{cccccccc}
0 & 0 & 0 & 0 & 0 & 0 & 0 & 0\\
0 & 0 & -i & 0 & 0 & 0 & 0 & 1\\
0 & i & 0 & 0 & 0 & 0 & 0 & -i\\
0 & 0 & 0 & 0 & 0 & -1 & i & 0\\
0 & 0 & 0 & 0 & 0 & 0 & 0 & 0\\
0 & 0 & 0 & -1 & 0 & 0 & i & 0\\
0 & 0 & 0 & -i & 0 & -i & 0 & 0\\
0 & 1 & i & 0 & 0 & 0 & 0 & 0\end{array}\right).\end{equation}
Here, the spin-orbit energy $\Delta$ is defiend as \begin{equation}
\Delta=-3i\frac{\hbar}{4m_{0}^{2}c^{2}}\left\langle x\mid\left(\nabla V\times\mathbf{p}\right)_{y}\mid z\right\rangle .\end{equation}
The $k$-dependent spin-orbit interaction $H_{SO,\mathbf{k}}$ of
\ref{eq:spin_orbit_int_k} is small and therefore commonly neglected
\citet{kane_handbooksemiconductors_1982}. The result of $H_{SO,\mathbf{k}}$
are terms linear in $k$, leading to an off-diagonal coupling between
the conduction and the valence band. Further, the spin-orbit interaction
$H_{SO,\mathbf{p}}$ is usually only included in the direct interaction,
while perturbative contributions from remote bands are in general
neglected. The effect of the spin-orbit interaction is to split the
six-fold (threefold without spin) degeneracy of the valence band at
$\Gamma$ into a fourfold degeneracy with eigenvalue $\frac{\Delta}{3}$
and a two-fold degeneracy with an eigenvalue of $-\frac{2\Delta}{3}$
. In other words, the states are separated by $\Delta$. The four-fold
degenerate states corresponds to the irreducible representation $\Gamma_{8}$
of the double group of $T_{d}$. Away from $\Gamma$, these states
create the \emph{heavy-hole} (HH) and the\emph{ light-hole} (LH) bands.
The two-fold degenerate states correspond to the $\Gamma_{7}$ representation
and lie below the $\Gamma_{8}$ states by $-\Delta$. The $\Gamma_{7}$
band is referred to as \emph{spin-orbit split off band} (SO). The
lowest conduction band (CB) states finally correspond to $\Gamma_{6}$
states. The situation is illustrated in Fig. \ref{fig:GaAs_Bulk_Bands},
where the band structure of $GaAs$ calculated using the here presented
$\mathbf{k}\cdot\mathbf{p}$ model is plotted around $\mathbf{k}=0$.
The experimental accessible bandgap is given by the difference between
the $\Gamma_{6}$ and the $\Gamma_{8}$ states.%
\begin{figure}
\begin{centering}
\includegraphics[scale=0.5]{\string"C:/Users/Yossi Michaeli/Documents/Thesis/Documents/Thesis/Figures/2_GaAs_Bulk_k_p\string".eps}
\par\end{centering}

\caption{\label{fig:GaAs_Bulk_Bands}Band structure of bulk $GaAs$ around
the $\Gamma$ point at room temperature, calculated using the $8\times8$
$\mathbf{k}\cdot\mathbf{p}$ for two crystal directions.}



\end{figure}
Therefore, the valence band edge $E_{v}$ in \ref{eq:Valence_Band_Edge_Energy},
needs to be shifted by $-\frac{\Delta}{3}$. 

The here elaborated model, including the CB, HH, LH and SO band, is
commonly referred to as the $8\times8$ $\mathbf{k}\cdot\mathbf{p}$
model. For large band gap materials, the coupling between the conduction
and valence bands is weak and therefore the $8\times8$ $\mathbf{k}\cdot\mathbf{p}$
model can be divided into to a $6\times6$ model, including only the
valence bands and a single band model for the spin degenerate conduction
band. This division is achieved by perturbatively folding the conduction
band onto the valence bands and vice versa. The $6\times6$ $\mathbf{k}\cdot\mathbf{p}$
is therefore obtained by using only the valence band part of $\mathbf{H}_{rd}^{8\times8}+\mathbf{H}_{SO,\mathbf{p}}$
with modified parameters $L\lyxmathsym{\textasciiacute}$ and $N\lyxmathsym{\textasciiacute}$
given by\begin{equation}
L'=\tilde{L}'+\frac{P^{2}}{E_{g}},\,\,\,\, N_{+}=\tilde{N}_{+}+\frac{P^{2}}{E_{g}}\label{eq:Kanes_params_renormalization}\end{equation}
instead of $\tilde{L}'$ and $\tilde{N}_{+}$. The parameter $\tilde{M}$,
and consequently $\tilde{N}_{-}$, remain unchnged. Thereby, the correct
operator ordering is conserved. The conduction band parameter is obtained
using\begin{equation}
A_{c}=\tilde{A}_{c}+\frac{P^{2}}{E_{g}},\end{equation}
which is then equal to the common conduction band effective mass.
For some semiconductors, the spin-orbit splitting is large and the
SO band has only little influence on the top of the valence band.
Then, the $6\times6$ $\mathbf{k}\cdot\mathbf{p}$ model can be further
reduced to only include the HH and LH bands. The reduction is performed
by choosing a combination of basis functions \ref{eq:levels_basis_with_spin}
effectively diagonalizing the spin-orbit interaction $\mathbf{H}_{SO,\mathbf{p}}$.
The new basis is labelled according to its total angular momentum
$J$ and the angular momentum around the $z$-axis, $J_{z}$. A possible
choice, using $\left|J,J_{z}\right\rangle $ as notation, is given
by\begin{eqnarray}
u_{hh}=\left|\frac{3}{2},\frac{3}{2}\right\rangle  & = & -\frac{1}{\sqrt{2}}\left(\left|x\uparrow\right\rangle +i\left|y\uparrow\right\rangle \right),\nonumber \\
u_{lh}=\left|\frac{3}{2},\frac{1}{2}\right\rangle  & = & -\frac{1}{\sqrt{6}}\left(\left|x\downarrow\right\rangle +i\left|y\uparrow\right\rangle \right)+\sqrt{\frac{2}{3}}\left|z\uparrow\right\rangle ,\nonumber \\
\bar{u}_{lh}=\left|\frac{3}{2},-\frac{1}{2}\right\rangle  & = & \frac{1}{\sqrt{6}}\left(\left|x\uparrow\right\rangle -i\left|y\uparrow\right\rangle \right)+\sqrt{\frac{2}{3}}\left|z\downarrow\right\rangle ,\label{eq:Angular_Momentum_Basis}\\
\bar{u}_{hh}=\left|\frac{3}{2},-\frac{3}{2}\right\rangle  & = & \frac{1}{\sqrt{2}}\left(\left|x\downarrow\right\rangle -i\left|y\downarrow\right\rangle \right),\nonumber \\
u_{SO}=\left|\frac{1}{2},\frac{1}{2}\right\rangle  & = & \frac{1}{\sqrt{3}}\left(\left|x\downarrow\right\rangle +i\left|y\downarrow\right\rangle +\left|z\uparrow\right\rangle \right),\nonumber \\
\bar{u}_{SO}=\left|\frac{1}{2},-\frac{1}{2}\right\rangle  & = & \frac{1}{\sqrt{3}}\left(\left|x\uparrow\right\rangle -i\left|y\uparrow\right\rangle -\left|z\downarrow\right\rangle \right).\nonumber \end{eqnarray}
The four states with total angular momentum of $\frac{3}{2}$ belong
to $\Gamma_{8}$ and the two states with an angular momentum of $\frac{1}{2}$
form the $\Gamma_{7}$ (or SO) bands. Transforming the Hamiltonian
into the basis \ref{eq:Angular_Momentum_Basis} and neglecting the
$\Gamma_{7}$ rows and columns finally results in the $4\times4$
$\mathbf{k}\cdot\mathbf{p}$ model. Within the $4\times4$ $\mathbf{k}\cdot\mathbf{p}$
model, it is common to use the Luttinger parameters $\gamma_{1},\gamma_{2}$
and $\gamma_{3}$%
\footnote{The full listing of these parameters is given is Appendix %
\begin{lyxgreyedout}
XXX
\end{lyxgreyedout}
.%
}\citet{Luttinger1956} instead of the Kane's parameters of the $6\times6$
model $L\lyxmathsym{\textasciiacute},\, M$ and $N\lyxmathsym{\textasciiacute}$.
The Luttinger parameters are usually those given in literature. The
parameters are related to each other via\begin{eqnarray}
L' & = & -\frac{\hbar^{2}}{2m_{0}}\left(\gamma_{1}+4\gamma_{2}\right)\nonumber \\
M & = & -\frac{\hbar^{2}}{2m_{0}}\left(\gamma_{1}-2\gamma_{2}\right)\label{eq:Luttinger_Kane_parameters_relation}\\
N' & = & -\frac{\hbar^{2}}{2m_{0}}6\gamma_{3}.\nonumber \end{eqnarray}
From \ref{eq:Luttinger_Kane_parameters_relation}, the parameters
for $8\times8$ model can be calculated using the renormalization
\ref{eq:Kanes_params_renormalization}.

In terms of the Luttinger parameters the full $8\times8$ Hamiltonian
can be written explicitly as \citet{Chuang1995,Luttinger1956,Kohn1955a}\begin{equation}
\mathbf{H}^{8\times8}=\left(\begin{array}{cccccccc}
E_{c} & P_{z} & \sqrt{2}P_{z} & -\sqrt{3}P_{+} & 0 & \sqrt{2}P_{-} & P_{-} & 0\\
P_{z}^{\dagger} & P+\Delta & \sqrt{2}Q^{\dagger} & -S^{\dagger}/\sqrt{2} & -\sqrt{2}P_{+}^{\dagger} & 0 & -\sqrt{3/2}S & -\sqrt{2}R\\
\sqrt{2}P_{z}^{\dagger} & \sqrt{2}Q & P-Q & -S^{\dagger} & -P_{+}^{\dagger} & \sqrt{3/2}S & 0 & R\\
-\sqrt{3}P_{+}^{\dagger} & -S/\sqrt{2} & -S & P+Q & 0 & \sqrt{2}R & R & 0\\
0 & -\sqrt{2}P_{+} & -P_{+} & 0 & E_{c} & P_{z} & -\sqrt{2}P_{z} & -\sqrt{3}P_{-}\\
\sqrt{2}P_{-}^{\dagger} & 0 & \sqrt{3/2}S^{\dagger} & \sqrt{2}R^{\dagger} & P_{z}^{\dagger} & P+\Delta & \sqrt{2}Q^{\dagger} & -S/\sqrt{2}\\
P_{-}^{\dagger} & -\sqrt{3/2}S^{\dagger} & 0 & R^{\dagger} & -\sqrt{2}P_{z}^{\dagger} & \sqrt{2}Q & P-Q & S\\
0 & -\sqrt{2}R^{\dagger} & R^{\dagger} & 0 & -\sqrt{3}P_{-}^{\dagger} & -S^{\dagger}/\sqrt{2} & S^{\dagger} & P+Q\end{array}\right)\end{equation}
where \begin{eqnarray}
E_{c} & = & E_{g}+\frac{\hbar^{2}}{2m_{0}}\left(k_{x}^{2}+k_{y}^{2}+k_{z}^{2}\right),\\
P & = & \frac{\hbar^{2}}{2m_{0}}\gamma_{1}\left(k_{x}^{2}+k_{y}^{2}+k_{z}^{2}\right),\\
P_{\pm} & = & \sqrt{\frac{1}{6}}\left[i\mathcal{P}\left(k_{x}\pm ik_{y}\right)+\mathcal{B}k_{z}\left(k_{y}\pm ik_{x}\right)\right]\\
P_{z} & = & \sqrt{\frac{1}{3}}\left(i\mathcal{P}k_{z}+\mathcal{B}k_{x}k_{y}\right),\\
Q & = & \frac{\hbar^{2}}{2m_{0}}\gamma_{2}\left(k_{x}^{2}+k_{y}^{2}-2k_{z}^{2}\right),\\
R & = & \frac{\hbar^{2}}{2m_{0}}\left[-\sqrt{3}\gamma_{2}\left(k_{x}^{2}-k_{y}^{2}\right)+i2\sqrt{3}\gamma_{3}k_{x}k_{y}\right],\\
S & = & \frac{\hbar^{2}}{2m_{0}}\gamma_{3}\left(k_{x}-ik_{y}\right)k_{z}.\end{eqnarray}
The parameter $\Delta$ is, as before, the spin-orbit splitting energy.
The coupling between the $\Gamma$ conduction band edge state $\left|s\right\rangle $
and the $\Gamma$ valence band edge state $\left|z\right\rangle $
is given by\begin{equation}
\mathcal{P}=-\frac{\hbar^{2}}{m_{0}}\int_{V_{c}}\varphi_{s}\frac{\partial}{\partial z}\varphi_{z}.\end{equation}
The Kane parameter $\mathcal{B}$ describes the inversion asymmetry.
In most practical calculations, this parameter is neglected. The parameters
$\gamma_{1},\gamma_{2},\gamma_{3}$ and $\mathcal{P}$ can ne determined
from effective masses at the $\Gamma$ point of the bulk semiconductor
\citet{Chuang1995}\begin{eqnarray}
\frac{m_{0}}{m_{hh}(001)} & = & \gamma_{1}-2\gamma_{2},\\
\frac{m_{0}}{m_{lh}(001)} & = & \gamma_{1}+2\gamma_{2}+\lambda,\\
\frac{m_{0}}{m_{SO}(001)} & = & \gamma_{1}+\frac{1}{2}\lambda r,\\
\frac{m_{0}}{m_{hh}(111)} & = & \gamma_{1}-2\gamma_{3},\end{eqnarray}
where the dimensionless parameters $\lambda$ and $r$ are given by\begin{eqnarray}
\lambda & = & \frac{4m_{0}\mathcal{P}^{2}}{3\hbar^{2}E_{g}},\\
r & = & \frac{E_{g}}{E_{g}+\Delta}.\end{eqnarray}


%
\begin{lyxgreyedout}
Put here the 6X6 and 4X4 Hamiltonians
\end{lyxgreyedout}



\subsection{Two Band Model}

The conduction band can be modeled quite easily if we assume that
the interaction with the other bands is weak enough for it to be treated
perturbatively, i.e. use a simple effective mass model. In the case
of the valence band, however, the strong interaction between the degenerate
light and heavy hole bands (near the band edge) requires that these
bands are taken into account explicitly. Only when we consider energy
levels deep into the valence bands (close to the SO splitting energy,
about $300\, meV$ in $GaAs)$ do the coupling terms to the SO and
conduction bands ($1.5\, eV$ splitting) can be introduced through
the effective mass. 

The degeneracy of the light and heavy hole bands near the band edge
generates a coupling term (as in the Luttinger Hamiltonian). Including
spin degeneracy, this yeilds a set of four coupled effective mass
equations \citet{Broido1985,Andreani1987,Chuang1995}. 

Fortunately, this set of coupled equations can be greatly simplified
by a method described in \citet{Broido1985}. Here a unitary transformation
of the four basis Bloch functions in \ref{eq:Angular_Momentum_Basis}
into a new set $u_{A},u_{B},u_{C,}u_{D}$ to decouple the set of four
coupled equations into two coupled ones. The Bloch functions $u_{i}$
are given by\begin{eqnarray}
u_{A} & = & \frac{1}{\sqrt{2}}\left(u_{hh}-\bar{u}_{hh}\right),\\
u_{B} & = & \frac{1}{\sqrt{2}}\left(-u_{lh}-\bar{u}_{lh}\right),\\
u_{C} & = & \frac{1}{\sqrt{2}}\left(u_{lh}+\bar{u}_{lh}\right),\\
u_{D} & = & \frac{1}{\sqrt{2}}\left(u_{hh}+\bar{u}_{hh}\right).\end{eqnarray}
Consequently, the $4\times4$ $\mathbf{k}\cdot\mathbf{p}$ Hamiltonian
\begin{equation}
\mathbf{H}^{4\times4}=\left(\begin{array}{cccc}
P-Q & -S^{\dagger} & 0 & R\\
-S & P+Q & R & 0\\
0 & R^{\dagger} & P-Q & S\\
R^{\dagger} & 0 & S^{\dagger} & P+Q\end{array}\right).\end{equation}
In terms of the the new base propesed above, it can be diagonalized
into two $2\times2$ block matrices, upper $H^{U}$ and lower $H^{L}$,
given by\begin{equation}
H^{\sigma}=\left(\begin{array}{cc}
P\pm Q & W\\
W^{\dagger} & P\mp Q\end{array}\right),\label{eq:Block_2_2_H}\end{equation}
where $W=\left|R\right|-i\left|S\right|$. The index $\sigma=U(L)$
refers to the upper (lower) $\pm$ signs. The upper and lower blocks
are quivalent, showing the double degeneracy of the heavy and light
hole bands. It is therefore sufficient to solve the upper block and
obtain its solutions. The solutions of the lower block can easily
be determined from the latter.

We can identify $P-Q$ and $P+Q$ with the light hole energy (operator)
$\hat{H}_{lh}$ and the heavy hole energy $\hat{H}_{hh}$, respectively.
Similarly to the conduction band case, the Schr\"{o}dinger  equation
with the Hamiltonian \ref{eq:Block_2_2_H} can be simplified into
an effective-mass formalism with \begin{eqnarray}
\hat{H}_{lh} & = & -\left(\gamma_{1}+2\gamma_{2}\right)\frac{\partial^{2}}{\partial z^{2}}+\left(\gamma_{1}-\gamma_{2}\right)k_{t}^{2},\\
\hat{H}_{hh} & = & -\left(\gamma_{1}-2\gamma_{2}\right)\frac{\partial^{2}}{\partial z^{2}}+\left(\gamma_{1}+\gamma_{2}\right)k_{t}^{2},\\
\hat{W} & = & \left\{ \begin{array}{c}
\begin{array}{c}
\sqrt{3}k_{t}\left(\gamma_{2}k_{t}-2\gamma_{3}\frac{\partial}{\partial z}\right)\,\,\,\textrm{for}\,[100]\end{array}\\
\begin{array}{c}
\sqrt{3}k_{t}\left(\gamma_{3}k_{t}-2\gamma_{3}\frac{\partial}{\partial z}\right)\,\,\,\textrm{for}\,[110]\end{array}\end{array}\right.\end{eqnarray}
Finally, we take into account the potential $V(z)$, which represents
the (bulk) valence-band-edge offset with respect to an arbitrary reference
energy. This allows us write the effective mass equation as \begin{equation}
\left(\begin{array}{cc}
\hat{H}_{hh}+V & \hat{W}\\
\hat{W}^{\dagger} & \hat{H}_{lh}+V\end{array}\right)\left(\begin{array}{c}
F_{hh}\\
F_{lh}\end{array}\right)=E(\mathbf{k})\left(\begin{array}{c}
F_{hh}\\
F_{lh}\end{array}\right),\label{eq:Effective_Mass_Equation}\end{equation}
where $F_{hh}$ and $F_{lh}$ are the envelope fucntions corresponding
to $u_{A}$ and $u_{B}$ respectively. Note that in this formalism,
hole energies are taken to be positive.

The first step in solving the quantum well problem, is finding the
solution in bulk material, where we take $V$ to be a constant $V_{0}$.
The value of $V_{0}$ will be different in well material and barriersm
reflecting the different valence band edge offsets. We can now easily
solve for the eigenenergies $E(\mathbf{k})$, yeilding the bulk energy
dispersion relations for the heavy and light hole subbands. We consider
the case of a $\left[100\right]$ plane, writing the in-plane $\mathbf{k}$
component as $k_{t}$\begin{equation}
E(\mathbf{k})-V_{0}=\gamma_{1}\left(k_{z}^{2}+k_{t}^{2}\right)\pm\sqrt{4\gamma_{2}^{2}\left(k_{z}^{2}+k_{t}^{2}\right)+12\left(\gamma_{3}^{2}-\gamma_{2}^{2}\right)k_{z}^{2}k_{t}^{2}},\label{eq:Bulk_Dispersion_Relation_HH_LH}\end{equation}
where the plus sign referes to the light hole solution, and the minus
to the heavy hole one. This expression can be rewritten to \begin{equation}
E(\mathbf{k})=V_{0}+\left[\gamma_{1}\pm\gamma_{2}\sqrt{1+3\frac{\gamma_{3}^{2}-\gamma_{2}^{2}}{\gamma_{2}^{2}}\frac{k_{z}^{2}k_{t}^{2}}{\left(k_{z}^{2}+k_{t}^{2}\right)^{2}}}\right]\left(k_{z}^{2}+k_{t}^{2}\right).\end{equation}
A similar derivation can be formulated for the {[}110{]} crystal planes.

Constant energy contours are shown in Fig. \ref{fig:HH_LH_Contours_Of_Constant_Energy},
illustrating that $\gamma_{3}$can be related to the mass anisotropy
along the $\left[100\right]$ and $\left[110\right]$ directions.
If $k_{t}$ small compared to $k_{z}$, we can expand the square root
in \ref{eq:Bulk_Dispersion_Relation_HH_LH}\begin{equation}
E(\mathbf{k})=V_{0}+\left(\gamma_{1}\pm\gamma_{2}\right)\left(k_{z}^{2}+k_{t}^{2}\right)\pm3\frac{\gamma_{3}^{2}-\gamma_{2}^{2}}{\gamma_{2}}k_{t}^{2}.\end{equation}


%
\begin{figure}
\begin{centering}
\includegraphics[scale=0.1]{\string"C:/Users/Yossi Michaeli/Documents/Thesis/Documents/Thesis/Figures/2_HH_LH\string".eps}
\par\end{centering}

\caption{\label{fig:HH_LH_Contours_Of_Constant_Energy}Contours of constant
energy within any $[100]$ plane of $k$-space for the heavy (right)
and light (left) hole subbands in bulk $GaAs$. The energy spacing
between each contour level is $0.5\, meV$ for the HH band and $3\, meV$
for the LH band. The effective HH mass is much larger along the$\left[110\right]$
direction than along the $\left[100\right]$ direction, as indicated
by the larger contour spacing. The effective LH mass is seen to be
much more isotropic (after \citet{Zory1993}). }



\end{figure}


The Energy term accounting for anisotropy for a given $k_{t}$ and
$k_{z}$ is equal fot the HH and LH subbands. However, due to the
lower energy of the HH bands the anisotropy term is relatively more
important for HH than for LH, resulting in a clearly anisotropic HH
band and a quasi isotropic LH band.

Still, we see that in bulk material, the effective masses along the
$z$-axis $\left[001\right]$ and $x$- and $y$- axes $\left[100\right]$
and $\left[010\right]$ are identical (as expected), as the dispersion
realtion is given by $E(\mathbf{k})=V_{0}+\left(\gamma_{1}\pm2\gamma_{2}\right)k^{2}.$We
can easily find this from \ref{eq:Bulk_Dispersion_Relation_HH_LH}
with $k_{t}=0$ for $\left[001\right]$, and $k_{z}=0$ for the $x$-
and $y$- directions.

The eigenvectors of \ref{eq:Effective_Mass_Equation} are found to
be, apart from a normalization constant \begin{eqnarray}
\varphi_{1}(\mathbf{k},\mathbf{r}) & = & \left(\begin{array}{c}
F_{hh,1}\\
F_{lh,1}\end{array}\right)=e^{i\mathbf{k}\cdot\mathbf{r}}\left(\begin{array}{c}
H_{lh}+V_{0}-E_{hh}\\
-W^{\dagger}\end{array}\right),\label{eq:Phi_1}\\
\varphi_{2}(\mathbf{k},\mathbf{r}) & = & \left(\begin{array}{c}
F_{hh,2}\\
F_{lh,2}\end{array}\right)=e^{i\mathbf{k}\cdot\mathbf{r}}\left(\begin{array}{c}
H_{lh}+V_{0}-E_{lh}\\
-W^{\dagger}\end{array}\right),\label{eq:Phi_2}\end{eqnarray}
where the matrix notation implies\begin{equation}
\varphi=F_{hh}u_{A}+F_{lh}u_{B}.\end{equation}


To solve the quantum well problem, we choose the well growth direction
(direction of confinment) along the $z$-axis. %
\begin{figure}
\begin{centering}
\includegraphics[scale=0.3]{\string"C:/Users/Yossi Michaeli/Documents/Thesis/Documents/Thesis/Figures/2_QW_Schematic_Transverse_Dispersion\string".eps}
\par\end{centering}

\caption{\label{fig:QW_In_Plane_Dispersion}The \textquotedbl{}in-plane\textquotedbl{}
subband structure of the quantum well in the conduction band. Within
the plane of the well, the electron still behaves like a \textquotedbl{}free\textquotedbl{}
electron. Thus, for each quantized level, a parabolic energy subband
exist (after \citet{Zory1993}).}



\end{figure}
 The $xy$-plane is the plane of the well, as in Fig. \ref{fig:QW_In_Plane_Dispersion}.
We can construct a confined solution from the bulk plane wave solutions
by imposing boundary conditions along the confinement axis. In the
plane of the well, there is no confinment and hence we retain the
bulk plane wave solution. By taking a linear combination of the bulk
solutions in each material, a general solution can be cinstructed.
As illustrated in %
\begin{figure}
\begin{centering}
\includegraphics[scale=0.4]{\string"C:/Users/Yossi Michaeli/Documents/Thesis/Documents/Thesis/Figures/2_Four_Plane_Wave_States_kt_0\string".eps}
\par\end{centering}

\caption{\label{fig:Bulk_Dispersion_Coefficients}At any one enrgy on a bulk
material, we can find four wavevectors corresponding to the heavy
and light hole bands. An eigenstate of the Hamiltonian in a quantum
well is then madde of a linear combination of the bulk plane waves
corresponding to those wave vectors. The amplitudes $A_{\pm}$ and
$B_{\pm}$ in (a) correspond, respectively, to $\pm k_{z1}$(light
hole) and $\pm k_{z2}$ in (b) for the well layer of the quantum well
($GaAs$ throughout this thesis). In the barriers ($AlGaAs)$ a similar
mechanism is employed. The boundary conditions at the interfaces then
determine the energy eigenvalues and the coefficients.}

\end{figure}
 Fig. \ref{fig:Bulk_Dispersion_Coefficients}, four plane wave solutions
exist at a given energy, yielding a general solution $\Phi$ of the
form\begin{equation}
\Phi=\sum A_{\pm}\varphi_{1}(\pm k_{z1},k_{t},\mathbf{r})+\sum B_{\pm}\varphi_{2}(\pm k_{z2},k_{t},\mathbf{r}).\end{equation}
The four coefficients $A_{\pm}$ and $B_{\pm}$ are unknown constants.
Both $\varphi_{1}$ and $\varphi_{2}$ are two-component vectors as
seen in \ref{eq:Phi_1} and \ref{eq:Phi_2}. We can write the components
of $\Phi$, $F_{hh}$ and $F_{lh}$, as \begin{eqnarray}
F_{hh} & = & e^{i\mathbf{k}_{t}\cdot\mathbf{r}_{t}}\left[\sum A_{\pm}F_{hh,1}(\pm k_{z1},k_{t})e^{\pm ik_{hh}z}+\sum B_{\pm}F_{hh,2}(\pm k_{z2},k_{t})e^{\pm ik_{hh}z}\right],\\
F_{lh} & = & e^{i\mathbf{k}_{t}\cdot\mathbf{r}_{t}}\left[\sum A_{\pm}F_{lh,1}(\pm k_{z1},k_{t})e^{\pm ik_{lh}z}+\sum B_{\pm}F_{lh,2}(\pm k_{z2},k_{t})e^{\pm ik_{lh}z}\right].\end{eqnarray}
Thus we have four unknown constants in each region, making a total
of 12 unknowns over the three regions. The boundary conditions at
the interfaces between the regions and the demand that the solutions
be confined in the quantum well provide the necessary relations to
solve the problem. In order to symmetr The following quantities have
to be matched across the interfaces\begin{eqnarray}
F_{hh}\,\,\textrm{and}\,\,\left(\gamma_{1}-2\gamma_{2}\right)\frac{dF_{hh}}{dz}+\sqrt{3}\gamma_{3}k_{t}F_{lh},\\
F_{lh}\,\,\textrm{and}\,\,\left(\gamma_{1}-2\gamma_{2}\right)\frac{dF_{lh}}{dz}-\sqrt{3}\gamma_{3}k_{t}F_{hh}.\end{eqnarray}
These boundary conditions were obtained by symmettrizing the Hamiltonian
\ref{eq:Block_2_2_H}. Caution should be issued however that the above
boundary conditions only apply when the Bloch functions in both well
materials and are similar, as is the case for the $GaAs-AlGaAs$ system
used throughout this thesis. The boundary conditions boil down to
the continuity of the wave function and {}``generalized'' continuity
of its derivative, corresponding to current across the interface.
The numerical implementation details of the two bands model is given
in Appendix \ref{cha:Appendix_Two_Band_Numerics}. %
\begin{figure}
\begin{centering}
\includegraphics[scale=0.7]{\string"C:/Users/Yossi Michaeli/Documents/Thesis/Documents/Thesis/Figures/2_100A_GaAs_AlGaAs_QW_Valence_Dispersion_DOS\string".eps}
\par\end{centering}

\caption{\label{fig:QW_Valence_Subbands_DOS}(a) Valence subbands dispersion
relations calculated for a $100\AA$ wide $GaAs/Al_{0.3}Ga_{0.7}As$
quantum well, for {[}100{]} crystal plane. The subbands are named
after their dominant character at the zone center ($k_{t}=0$). (b)
The ratio between the density of states of the valence subbands and
the first conduction subband, calculated for the same structure. }

\end{figure}


As an illustration, we present in Fig. \ref{fig:QW_Valence_Subbands_DOS}(a)
the valence subband structure of a $100\AA$ $GaAs/Al_{0.3}Ga_{0.7}As$
quantum well. The light and heavy holes are very heavily coupled,
giving rise to highly non-parabolic subbands. 

Particularly important is the \emph{density of states} (DOS), which
can be found from \begin{equation}
\rho(E)=\frac{1}{\pi}\frac{dk}{dE},\end{equation}
assuming the dispersion relationship is isotropic (using the axial
approximation, see Appendix \ref{cha:Appendix_Two_Band_Numerics}).
As an illustration, we plot in Fig. \ref{fig:QW_Valence_Subbands_DOS}(b)
the calculated ratio between the DOS of the valence subbands from
\ref{fig:QW_Valence_Subbands_DOS}(a) and the DOS of the first conduction
subband. The spikes in the DOS are sue to the band extrema away from
the zone center.


\section{Schr\"{o}dinger-Poisson Model}

All of the theoretical methods and examples described so far have
concentrated solely on solving systems for a single charge carrier.
In many devices such models would be inadequate as large numbers of
charge carriers, e.g. electrons, can be present in the conduction
band. In order to decide whether or not typical carrier densities
would give rise to a \emph{significant} additional potential on top
of the usual band-edge potential terms (which will be labelled specifically
as $V_{CB}$ or $V_{VB}$), it then becomes necessary to solve the
electrostatics describing the system. 

When considering the case of an n-type material, then (although obvious)
it is worth stating that the number of {}``free'' electrons in the
conduction band is equal to the number of positively charged ionised
donors in the heterostructure. As an example, we can point out the
modulation doped system, where the doping is located in a position
wher the free carriers it produces will become spatially separated
from the ion. The additional potential term $V_{\rho}(z)$ arising
from this, or any other charge distribution $\rho$, can be expressed
by using Poisson's equation\begin{equation}
\nabla^{2}V_{\rho}=-\frac{\rho}{\epsilon}\label{eq:Poissons_Eq}\end{equation}
where $\epsilon$ is the permittivity of the material, i.e. $\epsilon=\epsilon_{r}\epsilon_{0}$.
The solution is generally obtianed via the electric field strength
$\mathbf{E}$. Recalling that \begin{equation}
\mathbf{E}=-\nabla V\end{equation}
the potential then follow in the usual way \citet{Jackson1998} \begin{equation}
V_{\rho}(z)=-\int_{-\infty}^{\mathbf{r}}\mathbf{E}\cdot d\mathbf{r}.\label{eq:Poisson_Potential}\end{equation}
Given that the potential profiles, $V_{CB}(Z)$ for example, are one-dimensional,
then they will also produce a one-dimensional charge distribution.
In addition, remembering that the quantum wells are assumed infinite
in the $x-y$ plane then any charge density $\rho(z)$ can be thought
of as an infinite plane, i.e. a sheet, with areal charge density $\sigma(z)$
and thickness $\delta z$, as shown in Fig. \ref{fig:Electric_Field_Enfinite_Plane}(a).
%
\begin{figure}
\begin{centering}
\includegraphics[scale=0.3]{\string"C:/Users/Yossi Michaeli/Documents/Thesis/Documents/Thesis/Figures/2_Infinite_Plane_E_Field\string".eps}
\par\end{centering}

\caption{\label{fig:Electric_Field_Enfinite_Plane}Electric field strength
from an infinite plane of charge of volume density $d(z)$ and thickness
$\delta z$ (after \citet{harrison_quantum_2000}).}



\end{figure}
Such an infinite plane of charge produces an electric field perpendicular
to it, and with a strength\begin{equation}
\mathbf{E}=\frac{\sigma}{2\epsilon}.\end{equation}
Note that as the sheet is infinite in the plane, then the field strength
is constant for all distances from the plane. The total electric field
strength due to many of these planes of charge, as shown in Fig. \ref{fig:Electric_Field_Enfinite_Plane}(b),
is then the sum of the individual contributions as follows\begin{equation}
\mathbf{E}(z)=\sum_{z'=-\infty}^{\infty}\frac{\sigma(z')}{2\epsilon}\textrm{sign}\left(z-z'\right)\label{eq:Electrical_Field_Poisson}\end{equation}
where the function sign is defined as \begin{equation}
\textrm{sign}\left(z\right)=\left\{ \begin{array}{c}
1,\,\,\,\, z\geq0\\
-1,\,\,\,\, z\leq0\end{array}\right.\end{equation}
and has been introduced to account for the vector nature of $\mathbf{E}$,
i.e. if a single sheet of charge is at a position $z'$, then for
$z>z'$, $\mathbf{E}(z)=+\nicefrac{\sigma}{2\epsilon}$, whereas for
$z<z'$, $\mathbf{E}(z)=-\nicefrac{\sigma}{2\epsilon}$. Note further
that it is only the charge neutrality, there are as many ionised donors
(or acceptors) in the system as there are electrons (or holes), or
expressed mathematically\begin{equation}
\sum_{z=-\infty}^{\infty}\sigma(z)=0,\end{equation}
which ensures that the electric field, and hence the potential, go
to zero at large distances from the charge distribution. For the case
of a doped semiconductor, there would be two contributions to the
charge density $\sigma(z)$, where the first would be the ionised
impurities and the second the free charge carriers themselves. While
the former would be known from the doping density in each semiconductor
layer, as defined at growth time, the latter would be calculated from
the probability distributions of the carriers in the heterostructure.
Thus if $d(z)$ defines the volume density of the dopants at position
$z$, where the planes are separated by the usual step length $\delta z$,
then the total number of carriers, per unit cross-sectional area,
introduced into the heterostructure is given by\begin{equation}
N=\int_{-\infty}^{\infty}d(z)dz.\end{equation}
The net charge density in any of the planes follows as\begin{equation}
\sigma(z)=q\left[N\psi^{*}(z)\psi(z)-d(z)\right]\delta z\end{equation}
where $q$ us the charge on the extrinsic carriers. The step length
$\delta z$ selects the proportion of the carriers that are within
that slab and converts the volume density of dopant, $d(z)$, into
an areal density. %
\begin{figure}
\begin{centering}
\includegraphics[scale=0.5]{\string"C:/Users/Yossi Michaeli/Documents/Thesis/Documents/Thesis/Figures/2_Poisson_Ex_Sigma\string".eps}
\par\end{centering}

\caption{\label{fig:Areal_charge_density_Ex}Areal charge density a for a $100\,\AA$
$GaAs$ well, $n$-type doped to $2\times10^{18}cm^{-3}$, surrounded
by undoped $Ga_{0.8}Al_{0.2}As$ 4 meVbarriers.}

\end{figure}


If the charge carriers are distributed over more than one subband,
then the contribution to the charge density $\sigma(z)$ would have
to be summed over the relevant subbands\begin{equation}
\sigma(z)=q\left[\sum_{i=1}^{n}N_{i}\psi_{i}^{*}(z)\psi_{i}(z)-d(z)\right]\delta z,\end{equation}
where $\sum_{i=1}^{n}N_{i}=N$.

Fig. \ref{fig:Areal_charge_density_Ex} shows the areal charge density
along the growth axis for a $100\,\AA$ $GaAs$ well, $n$-type doped
to $2\times10^{18}cm^{-3}$, surrounded by undoped $Ga_{0.8}Al_{0.2}As$
barriers. The ionized donors yield a constant contributions to $\sigma$
within the well of $d(z)\delta z=2\times10^{24}m^{-3}\times1\AA=2\times10^{14}m^{-2}$,
in each of the $1\,\AA$ thick slabs. Hence, the total number $N$
of electrons in the quantum well is $100\times2\times10^{14}m^{-2}=2\times10^{12}cm^{-2}.$
By assuming that the electrons introduced by such doping all occupy
the ground state of the quantum well, then the curve on top of the
ionised impurity background clearly resembles $-\psi^{*}\psi$, as
expected from the mathematics. The discontinuities in a occur at the
edges of the doping profiles and are of magnitude $2\times10^{14}m^{-2}$,
again as expected.%
\begin{figure}
\begin{centering}
\includegraphics[scale=0.5]{\string"C:/Users/Yossi Michaeli/Documents/Thesis/Documents/Thesis/Figures/2_Poisson_Ex_E\string".eps}
\par\end{centering}

\caption{\label{fig:Poisson_Ex_E}The electric field strength $\mathbf{E}$
due to the charge distribution shown in Fig. \ref{fig:Areal_charge_density_Ex}.}



\end{figure}


There are a number of points to note about Fig. \ref{fig:Poisson_Ex_E},
which plots the electric field strength $\mathbf{E}$ due to the charge
distribution (as defined in equation \ref{eq:Electrical_Field_Poisson})
along the growth axis of the heterostructure. First, the field does
reach zero at either end of the structure, which implies charge neutrality.
In addition, the zero field point at the centre of the structure reflects
the symmetry of the charge distribution. The electric field strength
itself is not an observable, merely an intermediate quantity which
can be useful to plot from time to time; the quantity which is significant
is, of course, the potential due to this charge distribution.%
\begin{figure}
\begin{centering}
\includegraphics[scale=0.5]{\string"C:/Users/Yossi Michaeli/Documents/Thesis/Documents/Thesis/Figures/2_Poisson_Ex_V\string".eps}
\par\end{centering}

\caption{\label{fig:Poisson_Ex_V}The potential due to the ionised donor/electron
charge distribution.}

\end{figure}
 Fig. \ref{fig:Poisson_Ex_V} plots the potential as calculated from
equation \ref{eq:Poisson_Potential}, as usual defining the origin,
in this case for the potential, at the effective infinity u at the
left-hand edge of the barrier-well-barrier structure. Again, the symmetry
of the original heterostructure and doping profiles are reflected
in the symmetric potential. The potential is positive at the centre
of the well since the system under consideration consists of electrons
in the conduction band, so any test charge used to probe the potential
is also an electron which would be repelled by the existing charge.
The carrier density in this single quantum well is reasonably high
at $2\times10^{12}cm^{-2}$, and this produces a potential of up to
$4\, meV$; while this is small compared to the conduction band offset,
which is usually of the order of one or two hundred meV or more, it
could still have a measurable effect on the energy eigenvalues of
the quantum well.

The energy eigenvalues are calculated by considering the introduction
of a further test electron into the system and incorporating the potential
due to the carrier density already present into the standard Schr\"{o}dinger
equation, i.e. the potential term $V(z)$ in equations \ref{eq:Effective_Mass_Equation}
becomes\begin{equation}
V(z)\rightarrow V(z)+V_{\rho}(z),\end{equation}
where $V$ represents the band edge potential at zero doping and the
potential due to the non-zero number of carriers, the charge density
$\rho$, represented by the function $V_{\rho}$.

The numerical shooting method, described in detail in appendix \ref{cha:Appendix_Two_Band_Numerics},
can be used without alteration to solve for this new potential, which
will thus yield new energies and wave functions. The latter is an
important point since the potential due to the charge distribution
is itself dependent on the wave functions. Therefore, it is necessary
to form a closed loop solving Schrodinger's equation, calculating
the potential due to the resulting charge distribution, adding it
to the original bandedge potential, solving Schr\"{o}dinger's equation
again, and so on - a process illustrated schematically in Fig. \ref{fig:Schrodinger_Poisson_Schematic}.%
\begin{figure}
\begin{centering}
\includegraphics[scale=0.5]{\string"C:/Users/Yossi Michaeli/Documents/Thesis/Documents/Thesis/Figures/2_Schrodinger_Poisson_Schematic\string".eps}
\par\end{centering}

\caption{\label{fig:Schrodinger_Poisson_Schematic}Block diagram illustrating
the process of self-consistent iteration.}



\end{figure}
 The process is repeated until the energy eigenvalues converge; at
this point the wave functions are simultaneously solutions to both
Schrodinger's and Poisson's equations\textemdash{}the solutions are
described as \emph{self-consistent. }The numerical implementation
details of the self-consistent Schr\"{o}dinger-Poisson model is presented
in Appendix \ref{cha:Appendix_Schrodinger_Poisson}.

%
\begin{figure}
\begin{centering}
\includegraphics[scale=0.5]{\string"C:/Users/Yossi Michaeli/Documents/Thesis/Documents/Thesis/Figures/2_Poisson_Ex_V_int\string".eps}
\par\end{centering}

\caption{\label{fig:Poisson_Ex_V_int}The sum of the band-edge potential $V_{CB}$
and Poisson's potential $V_{\rho}$ for single quantum well.}



\end{figure}
Figure \ref{fig:Poisson_Ex_V_int} shows the result of adding the
potential due to the charge distribution $V_{\rho}$, as displayed
in Fig. \ref{fig:Poisson_Ex_V}, to the original band-edge potential
 for the single quantum well. The perturbation, even at this relatively
high carrier density of $2\times10^{12}cm^{-2}$, is rather small
compared to the barrier height, for instance. Nonetheless it is important
to calculate the effect of this perturbation on the electron energy
levels by continuing with the iterative process and looking for convergence
of the resulting energy solutions.

Although mention has been made of quantum well systems in which doping
in the barriers leads to a spatial separations of the ions and charge
carriers, which collect in a quantum well, quantitative calculations
presented thus far have not considered these modulation-doped systems.
Fig. \ref{fig:Poisson_Modulation_Doped_V} shows the band-edge potential,
$V_{CB}$, and the self-consistent potential, $V_{CB}+V_{\rho}$,
for a system with. an undoped single quantum well surrounded by doped
barriers; with the full layer definition thus being: $100\,\AA$ $Ga_{0.8}Al_{0.2}As$
doped $n$-type to $2\times10^{17}cm^{-3}$; $100\,\AA$ $GaAs$ undoped;
$100\,\AA$ $Ga_{0.8}Al_{0.2}As$ doped $n$-type to $2\times10^{17}cm^{-3}$.The
electrons introduced into the system are physically separated from
the ionised donors, so therefore instead of an ion/charge carrier
plasma, the mobile charge in this case is often referred to as a\emph{
two-dimensional (2D) electron gas}. The physical separation leads
to a reduction in the ionised impurity scattering and hence increased
electron mobilities for in-plane ($x-y$) transport, a feature which
is exploited in High-Electron- Mobility Transistors (HEMTs).%
\begin{figure}
\begin{centering}
\includegraphics[scale=0.5]{\string"C:/Users/Yossi Michaeli/Documents/Thesis/Documents/Thesis/Figures/2_Poisson_Modulation_Doping_V_int\string".eps}
\par\end{centering}

\caption{\label{fig:Poisson_Modulation_Doped_V}The band-edge potential (solid
blue) and the self-consistent potential (dotted red) of a modulation-doped
single quantum well.}

\end{figure}


%
\begin{lyxgreyedout}
Add here the calculation for a delta doped structure
\end{lyxgreyedout}
\selectlanguage{english}

