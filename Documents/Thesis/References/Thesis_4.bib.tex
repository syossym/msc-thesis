Automatically generated by Mendeley 0.9.6.2
Any changes to this file will be lost if it is regenerated by Mendeley.

@book{Singh2003,
author = {Singh, J.},
pages = {532},
publisher = {Cambridge University Press},
title = {{Electronic and optoelectronic properties of semiconductor structures}},
url = {http://books.google.com/books?hl=en\&amp;lr=\&amp;id=7n2CTG0m0\_EC\&amp;oi=fnd\&amp;pg=PR1\&amp;dq=Electronic+and+Optoelectronic+Properties+of+Semiconductor+Structures\&amp;ots=lo0R48mC-W\&amp;sig=82r\_tA-OvQhr5U4UIp9BVqB8Sgw},
year = {2003}
}
@article{Koch1989,
author = {Koch, S. W.},
journal = {Physical Review A},
pages = {1887--1898},
title = {{Semiconductor Laser Theory with Many-body Effects}},
volume = {39},
year = {1989}
}
@article{Lindberg1996a,
author = {Lindberg, M.},
journal = {Physical Review A},
pages = {3347--3359},
title = {{Maxwell-Bloch Equations for Spatially Inhomogeneous Semiconductor Lasers, Theoretical formulation}},
volume = {54},
year = {1996}
}
@article{Krebs1999b,
abstract = {A Reply to the Comment by Bradley A. Foreman.},
author = {Krebs, O. and Voisin, P.},
doi = {10.1103/PhysRevLett.82.1340},
file = {:C$\backslash$:/Users/Yossi Michaeli/AppData/Local/Mendeley Ltd./Mendeley Desktop/Downloaded/R655U5WU/p1340\_1.html:html},
journal = {Physical Review Letters},
month = {פברואר},
number = {6},
pages = {1340},
shorttitle = {Krebs and Voisin Reply},
title = {{Krebs and Voisin Reply:}},
url = {http://link.aps.org/doi/10.1103/PhysRevLett.82.1340 http://prl.aps.org/abstract/PRL/v82/i6/p1340\_1},
volume = {82},
year = {1999}
}
@book{Datta2005a,
author = {Datta, S.},
pages = {404},
publisher = {Cambridge University Press},
title = {{Quantum transport}},
year = {2005}
}
@article{Rodina2006,
abstract = {We apply the envelope function approach to abrupt heterostructures starting with the least-action principle for the microscopic wave function. The interface is treated nonperturbatively, and our approach is applicable to mismatched heterostructure. We obtain the interface connection rules for the multiband envelope function and the multiband heterostructure Hamiltonian from the k∙p version of the variational principle. The k∙p heterostructure Hamiltonian contains the short-range interface terms which consist of two physically distinct contributions. The first one depends only on the structure of the interface, and the second one is completely determined by the bulk parameters. We discover new structure inversion asymmetry terms and new magnetic energy terms important in spintronic applications.},
author = {Rodina, AV and Alekseev, A.Y.},
doi = {10.1103/PhysRevB.73.115312},
file = {:C$\backslash$:/Users/Yossi Michaeli/AppData/Local/Mendeley Ltd./Mendeley Desktop/Downloaded/MJCHM7MI/e115312.html:html},
journal = {Physical Review B},
month = {מרץ},
number = {11},
pages = {115312},
publisher = {APS},
title = {{Least-action principle for envelope functions in abrupt heterostructures}},
url = {http://link.aps.org/doi/10.1103/PhysRevB.73.115312},
volume = {73},
year = {2006}
}
@article{Chow2002,
author = {Chow, W. W. and Schneider, H. C.},
journal = {MRS bulletin},
number = {7},
pages = {525--530},
title = {{The application of gain theory to vertical-cavity surface-emitting lasers}},
url = {http://www.csa.com/partners/viewrecord.php?requester=gs\&amp;collection=TRD\&amp;recid=A0242178AH},
volume = {27},
year = {2002}
}
@article{Chuang1991c,
author = {Chuang, S. L.},
journal = {Physical Review B},
pages = {9649--9661},
title = {{Efficient band-structure calculations of strained quantum wells}},
volume = {43},
year = {1991}
}
@article{Bastard1988,
author = {Bastard, G.},
journal = {New York},
publisher = {Les ´editions de Physique},
title = {{Wave mechanics applied to semiconductor heterostructures}},
url = {http://www.osti.gov/energycitations/product.biblio.jsp?osti\_id=5095386},
year = {1988}
}
@article{Burt1988,
author = {Burt, M. G.},
journal = {Semiconductor Science and Technology},
number = {12},
pages = {1224--1226},
publisher = {Institute of Physics Publishing},
title = {{A new effective-mass equation for microstructures}},
url = {http://www.iop.org/EJ/abstract/0268-1242/3/12/013},
volume = {3},
year = {1988}
}
@article{Chao1992,
author = {Chao, C. Y. P. and Chuang, S. L.},
journal = {Physical Review B},
number = {7},
pages = {4110--4122},
publisher = {APS},
title = {{Spin-orbit-coupling effects on the valence-band structure of strained semiconductor quantum wells}},
url = {http://link.aps.org/doi/10.1103/PhysRevB.46.4110},
volume = {46},
year = {1992}
}
@article{Chuang1991,
author = {Chuang, S. L.},
journal = {Physical Review B},
pages = {1500--1509},
title = {{Exciton Green’s-Function Approach to Optical Absorbtion in a Quantum Well with an Applied Electric Field}},
volume = {43},
year = {1991}
}
@article{Hess1999a,
author = {Hess, O.},
journal = {Physical Review A},
pages = {2342--2358},
title = {{Maxwell-Bloch Equations for Spatially Inhomogeneous Semiconductor Lasers}},
volume = {59},
year = {1999}
}
@article{Schmitt-Rink1986,
author = {Schmitt-Rink, S.},
journal = {Physical Review B},
pages = {1183--1189},
title = {{Many-Body Effects in the Absorbtion, Gain and Luminescence Spectra of Semiconductor Quantum-Well Structures}},
volume = {33},
year = {1986}
}
@article{Burt1992,
abstract = {The assumptions of conventional effective-mass theory, especially the one of continuity of the envelope function at an abrupt interface, are reviewed critically so that the need for a fresh approach becomes apparent. A new envelope-function method, developed by the author over the past few years, is reviewed. This new method is based on both a generalization and a novel application to microstructures of the Luttinger-Kohn envelope-function expansion. The differences between this new method and the conventional envelope-function method are emphasized. An alternative derivation of the new envelope-function equations, which are exact, to that already published is provided. A new and improved derivation of the author's effective-mass equation is given, in which the differences in the zone-centre eigenstates of the constituent crystals are taken into account. The cause of the kinks in the conventional effective-mass envelope function, at abrupt effective-mass changes, is identified.},
author = {Burt, M. G.},
journal = {Journal of Physics: Condensed Matter},
number = {32},
pages = {6651--6690},
publisher = {Institute of Physics Publishing},
title = {{The justification for applying the effective-mass approximation to microstructures}},
url = {http://www.iop.org/EJ/abstract/0953-8984/4/32/003},
volume = {4},
year = {1992}
}
@article{Chuang1990b,
author = {Chuang, S. L.},
journal = {IEEE Journal of Quantum Electronics},
pages = {13--24},
title = {{Optical Gain and Gain Suppression of Quantum-Well Lasers with Valence Band Mixing}},
volume = {26},
year = {1990}
}
@article{Ning1997,
author = {Ning, C. Z.},
journal = {IEEE Journal of Quantum Electronics},
pages = {1543--1550},
title = {{Effective Bloch Equations for Semiconductor Lasers and Amplifiers}},
volume = {33},
year = {1997}
}
@book{Chuang1995,
author = {Chuang, S. L.},
publisher = {Wiley-Interscience},
title = {{Physics of Optoelectronic Devices}},
year = {1995}
}
@article{Binder1992,
author = {Binder, R. and Scott, D. and Paul, AE and Lindberg, M. and Henneberger, K. and Koch, SW},
journal = {Physical Review B},
number = {3},
pages = {1107--1115},
publisher = {APS},
title = {{Carrier-carrier scattering and optical dephasing in highly excited semiconductors}},
url = {http://link.aps.org/doi/10.1103/PhysRevB.45.1107},
volume = {45},
year = {1992}
}
@article{Burt1994,
abstract = {A direct general method for deriving effective-mass equations for microstructures with atomically abrupt boundaries is presented. The principal assumption is that the envelope functions are slowly varying on the scale of the lattice period. The band-edge Bloch functions are not assumed to be the same on both sides of an interface and it is shown how the differences can be taken into account perturbatively. The particle in a box method is known to work well in many situations. To demonstrate why, a derivation of the effective-mass equation is carried out explicitly for the case of conduction-band states of a type-I microstructure composed of zinc-blende crystals without spin-orbit interaction. The derivation provides much insight into why the effective-mass method works so well. The method is illustrated by applying it to a one-dimensional superlattice problem. For this model problem, the effective-mass approximation to the wave function is seen to be good even for a quantum well one lattice period wide.},
author = {Burt, M. G.},
doi = {10.1103/PhysRevB.50.7518},
file = {:C$\backslash$:/Users/Yossi Michaeli/AppData/Local/Mendeley Ltd./Mendeley Desktop/Downloaded/4AKQEASA/p7518\_1.html:html},
journal = {Physical Review B},
number = {11},
pages = {7518},
title = {{Direct derivation of effective-mass equations for microstructures with atomically abrupt boundaries}},
url = {http://link.aps.org/doi/10.1103/PhysRevB.50.7518 http://prb.aps.org/abstract/PRB/v50/i11/p7518\_1},
volume = {50},
year = {1994}
}
@book{Glutsch2004,
author = {Glutsch, S.},
publisher = {Springer},
title = {{Excitons in low-dimensional semiconductors}},
year = {2004}
}
@book{Basu2002,
author = {Basu, P. K.},
isbn = {0198526202, 9780198526209},
month = {אוקטובר},
pages = {470},
publisher = {Oxford University Press},
title = {{Theory of Optical Processes in Semiconductors}},
url = {http://books.google.com/books?id=8pybYUPeq4QC},
year = {2002}
}
@article{Burt1999,
abstract = {The increasing sophistication used in the fabrication of semiconductor nanostructures and in the experiments performed on them requires more sophisticated theoretical techniques than previously employed. The philosophy behind the author's exact envelope function representation method is clarified and contrasted with that of the conventional method. The significance of globally slowly varying envelope functions is explained. The difference between the envelope functions that appear in the author's envelope function representation and conventional envelope functions is highlighted and some erroneous statements made in the literature on the scope of envelope function methods are corrected. A perceived conflict between the standard effective mass Hamiltonian and the uncertainty principle is resolved demonstrating the limited usefulness of this principle in determining effective Hamiltonians. A simple example showing how to obtain correct operator ordering in electronic valence band Hamiltonians is worked out in detailed tutorial style. It is shown how the use of out of zone solutions to the author's approximate envelope function equations plays an essential role in their mathematically rigorous solution. In particular, a demonstration is given of the calculation of an approximate wavefunction for an electronic state in a one dimensional nanostructure with abrupt interfaces and disparate crystals using out of zone solutions alone. The author's work on the interband dipole matrix element for slowly varying envelope functions is extended to envelope functions without restriction. Exact envelope function equations are derived for multicomponent fields to emphasize that the author's method is a general one for converting a microscopic description to a mesoscopic one, applicable to linear partial differential equations with piecewise or approximately piecewise periodic coefficients. As an example, the method is applied to the derivation of approximate envelope function equations from the Maxwell equations for photonic nanostructures.},
author = {Burt, M. G.},
journal = {Journal of Physics: Condensed Matter},
number = {9},
pages = {53–83},
publisher = {Institute of Physics Publishing},
title = {{Fundamentals of envelope function theory for electronic states and photonic modes in nanostructures}},
url = {http://www.iop.org/EJ/abstract/0953-8984/11/9/002},
volume = {11},
year = {1999}
}
@article{Lindberg1994,
author = {Lindberg, M.},
journal = {Physical Review B},
pages = {18060--18072},
title = {$\chi$(3) formalism in optically excited semiconductors and its applications in four-wave-mixing spectroscopy},
volume = {50},
year = {1994}
}
@article{Foreman1995b,
abstract = {An exact effective-mass differential equation is derived for electrons in heterostructures. This equation is exactly equivalent to the Schr\"{o}dinger equation, and is obtained by applying a k-space transformation of variables to the Burt envelope-function theory in which the Brillouin zone is mapped onto the infinite real axis. The mapping eliminates all nonlocal effects and long-range Gibbs oscillations in the Burt theory, producing an infinite-order differential equation in which interface effects are strongly localized to the immediate vicinity of the interface. A general procedure is given for obtaining finite-order boundary conditions from the infinite-order equation; the second-order theory reduces to the BenDaniel-Duke model with a $\delta$-function potential at the interface. The derivation is presented for a simple one-dimensional crystal but can easily be generalized for more complex situations.},
author = {Foreman, B. A.},
doi = {10.1103/PhysRevB.52.12241},
file = {:C$\backslash$:/Users/Yossi Michaeli/AppData/Local/Mendeley Ltd./Mendeley Desktop/Downloaded/TK8BZXDK/p12241\_1.html:html},
journal = {Physical Review B},
month = {אוקטובר},
number = {16},
pages = {12241},
title = {{Exact effective-mass theory for heterostructures}},
url = {http://link.aps.org/doi/10.1103/PhysRevB.52.12241 http://prb.aps.org/abstract/PRB/v52/i16/p12241\_1},
volume = {52},
year = {1995}
}
@article{Quattropani1995,
abstract = {The system of one or several quantum wells embedded in a planar semiconductor Fabry-P\'{e}rot microcavity with distributed Bragg reflectors is studied in the framework of both a semiclassical and a full quantum theory. The results obtained with the two treatments are proved to be equivalent. Simple analytical expressions for the exciton-radiation mixed mode energies are obtained. In particular the model of two coupled harmonic oscillators for the dispersion is shown to hold in the whole range of coupling constant and mirror reflectivity values. The theory describes at the same time the weak coupling regime, where an enhanced spontaneous emission takes place, and the strong coupling regime where instead a Rabi splitting in the dispersion can be observed. Existing experimental results are described with great accuracy. Analytical expressions for the splittings in reflectivity, transmission, absorption, and photoluminescence are given. The splittings are all different from each other, and the differences could be observed in structures with suitably chosen parameters.},
author = {Quattropani, A. and Schwendimann, P. and Savona, V. and Andreani, L. C.},
doi = {10.1016/0038-1098(94)00865-5},
file = {:C$\backslash$:/Users/Yossi Michaeli/AppData/Local/Mendeley Ltd./Mendeley Desktop/Downloaded/PIF4VGF6/science.html:html},
issn = {0038-1098},
journal = {Solid State Communications},
keywords = {Excitons,Polaritons,Quantum Wells,Semiconductor Microcavities},
mendeley-tags = {Excitons,Polaritons,Quantum Wells,Semiconductor Microcavities},
month = {מרץ},
number = {9},
pages = {733--739},
shorttitle = {Quantum well excitons in semiconductor microcaviti},
title = {{Quantum well excitons in semiconductor microcavities: Unified treatment of weak and strong coupling regimes}},
url = {http://www.sciencedirect.com/science/article/B6TVW-4037RCB-2F/2/f6a5105de2d1698d714d49f58ee57f78 http://www.sciencedirect.com/science?\_ob=ArticleURL\&\_udi=B6TVW-4037RCB-2F\&\_user=32321\&\_coverDate=03\%2F31\%2F1995\&\_alid=1181145237\&\_rdoc=1\&\_fmt=high\&\_orig=search\&\_cdi=5545\&\_sort=r\&\_docanchor=\&view=c\&\_ct=1\&\_acct=C000004038\&\_version=1\&\_urlVersion=0\&\_userid=32321\&md5=1578fa25e0eff3d2ec010fcbb15baa33},
volume = {93},
year = {1995}
}
@article{Ellmers1998,
author = {Ellmers, C.},
journal = {Phys. Stat. Sol. B},
pages = {407--412},
title = {{Gain Spectra of an (InGa)As Single Qunatum-Well Laser Diode}},
volume = {206},
year = {1998}
}
@article{Skolnick1998,
abstract = {The physics of strong coupling phenomena in semiconductor quantum microcavities is reviewed. This is a relatively new field with most important developments having occurred in the last 5 years. We describe how such microcavities enable both electronic and photonic properties of semiconductors, and the interaction between them, to be controlled in the same structure. The resulting coupled exciton-photon eigenstates, cavity polaritons, have many interesting properties including very low mass for small in-plane wavevectors, and can be studied easily and directly in optical experiments, unlike exciton-polaritons in bulk semiconductors. A wealth of new optical phenomena has been reported in this field in the last few years. This review describes the most important of these phenomena. We discuss the reasons why polaritons have fundamentally different properties in microcavities as compared with those in bulk materials or quantum wells. We explain the factors which control the strength of the exciton-photon coupling and how the resulting optical spectra can be modelled. We then emphasize, in the main body of the review, the particularly important results of reflectivity experiments at both normal and oblique angles of incidence, the effects of external electric and magnetic fields, the results of coherent Raman scattering experiments, the effects of disorder on microcavity spectra, including the observation of motional narrowing over the exciton disorder potential, studies of coupled microcavities, and photoluminescence and time-resolved phenomena.},
author = {Skolnick, M. S. and Fisher, T. A. and Whittaker, D. M.},
journal = {Semiconductor Science and Technology},
pages = {645--669},
publisher = {Institute of Physics Publishing},
title = {{Strong coupling phenomena in quantum microcavity structures}},
url = {http://www.iop.org/EJ/abstract/0268-1242/13/7/003},
volume = {13},
year = {1998}
}
@article{Kohn1955a,
abstract = {A new method of developing an "effective-mass" equation for electrons moving in a perturbed periodic structure is discussed. This method is particularly adapted to such problems as arise in connection with impurity states and cyclotron resonance in semiconductors such as Si and Ge. The resulting theory generalizes the usual effective-mass treatment to the case where a band minimum is not at the center of the Brillouin zone, and also to the case where the band is degenerate. The latter is particularly striking, the usual Wannier equation being replaced by a set of coupled differential equations.},
author = {Kohn, W. and Luttinger, J. M.},
doi = {10.1103/PhysRev.97.869},
file = {:C$\backslash$:/Users/Yossi Michaeli/AppData/Local/Mendeley Ltd./Mendeley Desktop/Downloaded/W7GA9665/p869\_1.html:html},
journal = {Physical Review},
month = {פברואר},
number = {4},
pages = {869},
title = {{Motion of Electrons and Holes in Perturbed Periodic Fields}},
url = {http://link.aps.org/doi/10.1103/PhysRev.97.869 http://prola.aps.org/abstract/PR/v97/i4/p869\_1},
volume = {97},
year = {1955}
}
@article{Lowdin1951a,
author = {Lowdin, P.},
doi = {10.1063/1.1748067},
journal = {The Journal of Chemical Physics},
month = {נובמבר},
number = {11},
pages = {1396--1401},
title = {{A Note on the Quantum-Mechanical Perturbation Theory}},
url = {http://link.aip.org/link/?JCP/19/1396/1},
volume = {19},
year = {1951}
}
@article{Dresselhaus1955a,
abstract = {Character tables for the "group of the wave vector" at certain points of symmetry in the Brillouin zone are given. The additional degeneracies due to time reversal symmetry are indicated. The form of energy vs wave vector at these points of symmetry is derived. A possible reason for the complications which may make a simple effective mass concept invalid for some crystals of this type structure will be presented.},
author = {Dresselhaus, G.},
doi = {10.1103/PhysRev.100.580},
file = {:C$\backslash$:/Users/Yossi Michaeli/AppData/Local/Mendeley Ltd./Mendeley Desktop/Downloaded/CUMRJ4T2/p580\_1.html:html},
journal = {Physical Review},
month = {אוקטובר},
number = {2},
pages = {580},
title = {{Spin-Orbit Coupling Effects in Zinc Blende Structures}},
url = {http://link.aps.org/doi/10.1103/PhysRev.100.580 http://prola.aps.org/abstract/PR/v100/i2/p580\_1},
volume = {100},
year = {1955}
}
@book{Chow1994,
author = {Chow, W. W. and Koch, S. W. and {Sargent III}, M.},
publisher = {Springer-Verlag New York, Inc. New York, NY, USA},
title = {{Semiconductor-laser physics}},
url = {http://portal.acm.org/citation.cfm?id=179096},
year = {1994}
}
@article{Burt1993,
abstract = {The expression for the matrix element of position for an interband transition in a wide deep quantum well is derived using the envelope function expansion. It is found that the matrix element is not determined by the intra-atomic-like matrix element evaluated using band-edge Bloch functions, the result one might be led to expect either on intuitive grounds or as the result of a supposedly approximate evaluation in which only the dominant term in the envelope function expansion of each wave function is used. To obtain the correct expression, terms in the envelope function expansion that become vanishingly small in the limit of wide wells must be retained. Usually the interband matrix element is comparable with that for allowed intersubband transitions. Some implications for the DC Stark effect for quantum wells and non-linear refraction in quantum dot structures are mentioned.},
author = {Burt, M. G.},
journal = {Journal of Physics: Condensed Matter},
number = {24},
pages = {4091},
title = {{The evaluation of the matrix element for interband optical transitions in quantum wells using envelope functions}},
volume = {5},
year = {1993}
}
@book{Zory1993,
author = {Zory, P. S.},
isbn = {0127818901},
month = {אפריל},
publisher = {Academic Press},
title = {{Quantum well lasers}},
url = {http://books.google.com/books?hl=en\&amp;lr=\&amp;id=lb-VfmQF8TMC\&amp;oi=fnd\&amp;pg=PR11\&amp;dq=Quantum+Well+Lasers\&amp;ots=CLYbMcyZH0\&amp;sig=EtJ\_gmx2EW3wE6l4SP714UFZ5aw},
year = {1993}
}
@article{Koch1997,
author = {Koch, S. W.},
journal = {Phys. Stat. Sol. B},
pages = {725--738},
title = {{Multi-Band Bloch Equations and Gain Spectra of Highly Excited II-VI Semiconductor Quantum Wells}},
volume = {202},
year = {1997}
}
@book{Piprek2003,
author = {Piprek, J.},
isbn = {0125571909},
month = {ינואר},
publisher = {Academic Press},
shorttitle = {Semiconductor Optoelectronic Devices},
title = {{Semiconductor optoelectronic devices: introduction to physics and simulation}},
url = {http://scholar.google.com/scholar?hl=en\&btnG=Search\&q=intitle:Semiconductor+Optoelectronic+Devices:+Introduction+to+Physics+and+Simulation\#0},
year = {2003}
}
@article{Burt1995,
abstract = {The origin of large dipole matrix elements for interband optical transitions in low-bandgap quantum wells is illustrated using the exact numerical solution of a one-dimensional model. The usual effective mass approximation of the wavefunction as a single product of a slowly varying envelope function modulating a band edge Bloch function is not adequate in this case. Inclusion of at least one other term in the envelope function expansion of the wavefunction involving another band is necessary to get the correct result. It is shown that the omission of these extra terms leads to qualitatively incorrect charge oscillation under optical excitation.},
author = {Burt, M. G.},
journal = {Semiconductor Science and Technology},
number = {4},
pages = {412--415},
publisher = {Institute of Physics Publishing},
title = {{Breakdown of the atomic dipole approximation for the quantum well interband dipole matrix element}},
url = {http://www.iop.org/EJ/abstract/0268-1242/10/4/005},
volume = {10},
year = {1995}
}
@book{Yamamoto2000,
author = {Yamamoto, Y. and Tassone, F. and Cao, H.},
edition = {1},
publisher = {Springer Verlag},
title = {{Semiconductor cavity quantum electrodynamics}},
url = {http://books.google.com/books?hl=en\&amp;lr=\&amp;id=-jMEZ4uxQS0C\&amp;oi=fnd\&amp;pg=PA1\&amp;dq=Semiconductor+Cavity+Quantum+Electrodynamics\&amp;ots=aeAJWaSZ0l\&amp;sig=T7z3wIA73g1hj\_3eKaDQsVI09Zc},
year = {2000}
}
@article{Chuang1997,
abstract = {We present a theoretical model for calculating the band structures of strained quantum-well wurtzite semiconductors. The theory is based on the Hamiltonian for wurtzite semiconductors and includes the strain effects on the shifts of the band edges. We show new results that include the matrix elements of the Hamiltonian using the finite-difference method for the calculations of the valence electronic band structures of GaN=AlxGa1−xN quantum-well wurtzite semiconductors based on the effective-mass theory.},
author = {Chuang, S. L. and Chang, C. S.},
journal = {Semiconductor Science and Technology},
pages = {252},
title = {{A band-structure model of strained quantum-well wurtzite semiconductors}},
volume = {12},
year = {1997}
}
@book{Macleod2001,
author = {Macleod, H. A.},
edition = {3rd},
publisher = {Institute of Physics Publishing},
title = {{Thin-Film Optical Filters}},
year = {2001}
}
@article{Chow1999,
author = {Chow, W. W.},
journal = {Optics Express},
pages = {119--124},
title = {{Calculation of Quantum Well Laser Gain Spectra}},
volume = {2},
year = {1999}
}
@article{Andreani1987,
abstract = {Valence subbands of uniaxially stressed GaAs-Ga1-xAlxAs quantum wells are found by solving exactly the multiband effective-mass equation for the envelope function; as in the particle in a box problem, we first solve the effective-mass equation in each bulk material, and then we impose boundary conditions on the linear combinations of bulk solutions. Discrete symmetries of the effective-mass Hamiltonian are used to decouple the spin-degenerate subbands; the energy levels are obtained as the zeros of an 8×8 determinant. The functional form of the wave functions is given analytically, and is used in order to discuss the heavy-hole–light-hole mixing at finite values of the in-plane vector k?; the mixing greatly increases when the applied stress reduces the energy separation at k?=0. Resonances are shown to arise and are due to the degeneracy of discrete levels with states of the continuum at different values of k?.},
author = {Andreani, L. C. and Pasquarello, A. and Bassani, F.},
doi = {10.1103/PhysRevB.36.5887},
file = {:C$\backslash$:/Users/Yossi Michaeli/AppData/Local/Mendeley Ltd./Mendeley Desktop/Downloaded/XDPPE352/p5887\_1.html:html},
journal = {Physical Review B},
month = {אוקטובר},
number = {11},
pages = {5887},
shorttitle = {Hole subbands in strained GaAs-Ga1-xAlxAs quantum },
title = {{Hole subbands in strained GaAs-Ga1-xAlxAs quantum wells: Exact solution of the effective-mass equation}},
url = {http://link.aps.org/doi/10.1103/PhysRevB.36.5887 http://prb.aps.org/abstract/PRB/v36/i11/p5887\_1},
volume = {36},
year = {1987}
}
@book{Seeger2004,
author = {Seeger, K.},
publisher = {Springer},
title = {{Semiconductor physics}},
year = {2004}
}
@article{Binder1995,
author = {Binder, R.},
doi = {10.1016/0079-6727(95)00001-S},
file = {:C$\backslash$:/Users/Yossi Michaeli/AppData/Local/Mendeley Ltd./Mendeley Desktop/Downloaded/BNC76J9C/science.html:html},
issn = {0079-6727},
journal = {Progress in Quantum Electronics},
number = {4-5},
pages = {307--462},
title = {{Nonequilibrium semiconductor dynamics}},
url = {http://adsabs.harvard.edu/abs/1995PQE....19..307B},
volume = {19},
year = {1995}
}
@article{Luo1993,
author = {Luo, M. S. C. and Chuang, S. L. and Schmitt-Rink, S. and Pinczuk, A.},
journal = {Physical Review B},
number = {15},
pages = {11086--11094},
publisher = {APS},
title = {{Many-body effects on intersubband spin-density and charge-density excitations}},
url = {http://link.aps.org/doi/10.1103/PhysRevB.48.11086},
volume = {48},
year = {1993}
}
@article{Lent1990,
author = {Lent, C. S. and Kirkner, D. J.},
doi = {10.1063/1.345156},
issn = {00218979},
journal = {Journal of Applied Physics},
number = {10},
pages = {6353},
title = {{The quantum transmitting boundary method}},
url = {http://link.aip.org/link/JAPIAU/v67/i10/p6353/s1\&Agg=doi},
volume = {67},
year = {1990}
}
@article{Bastard1981,
abstract = {The band structure of GaAs-GaAlAs and InAs-GaSb superlattices is calculated by matching propagating or evanescent envelope functions at the boundary of consecutive layers. For GaAs-GaAlAs materials, the envelope functions are the solutions of an effective Hamiltonian in which both band edges and effective masses are position dependent. The effective-mass jumps modify the boundary conditions which are imposed to the eigenstates of the effective-mass Hamiltonian. In InAs-GaSb superlattices, the dispersion relations, although quite similar to those obtained in GaAs-GaAlAs materials, reflect the genuine symmetry mismatch of InAs (electrons) and GaSb (light-holes) levels. The evolution of the InAs-GaSb band structure with increasing periodicity is calculated and found to be in excellent agreement with previous LCAO results. The dispersion relations of heavy-hole bands are obtained.},
author = {Bastard, G.},
doi = {10.1103/PhysRevB.24.5693},
file = {:C$\backslash$:/Users/Yossi Michaeli/AppData/Local/Mendeley Ltd./Mendeley Desktop/Downloaded/PBC5RJF9/p5693\_1.html:html},
journal = {Physical Review B},
month = {נובמבר},
number = {10},
pages = {5693},
title = {{Superlattice band structure in the envelope-function approximation}},
url = {http://link.aps.org/doi/10.1103/PhysRevB.24.5693 http://prb.aps.org/abstract/PRB/v24/i10/p5693\_1},
volume = {24},
year = {1981}
}
@article{Koch1998,
author = {Koch, S. W.},
journal = {Applied Physics A},
pages = {1--12},
title = {{Microscopic Theory of Laser Gain in Semiconductor Quantum-Wells}},
volume = {66},
year = {1998}
}
@book{Born2002,
author = {Born, M. and Wolf, E.},
edition = {7th},
publisher = {Cambridge University Press},
title = {{Principles of Optics: Electromagnetic Theory of Propagation, Interference and Diffraction of Light}},
year = {2002}
}
@article{Bastard1982,
abstract = {We extend our previous investigations on the band structure of superlattices by applying the envelope-function approximation to four distinct problems. We calculate the band structure of HgTe-CdTe superlattices and show that these materials can be either semiconducting or zero-gap semiconductors, i.e., behave exactly like the ternary Hg1-xCdx Te random alloys. We analyze the superlattice dispersion relations in the layer planes (Landau superlattice subbands) and we compare the longitudinal and transverse effective masses of semiconducting InAs-GaSb superlattices. We calculate the general equation for the bound states due to aperiodic layers, taking account of the band structure of the host materials. We finally derive the dispersion relations of polytype (ABC or ABCD) superlattices.},
author = {Bastard, G.},
doi = {10.1103/PhysRevB.25.7584},
file = {:C$\backslash$:/Users/Yossi Michaeli/AppData/Local/Mendeley Ltd./Mendeley Desktop/Downloaded/7XIJ26N9/p7584\_1.html:html},
journal = {Physical Review B},
month = {יוני},
number = {12},
pages = {7584--7597},
publisher = {APS},
title = {{Theoretical investigations of superlattice band structure in the envelope-function approximation}},
url = {http://link.aps.org/doi/10.1103/PhysRevB.25.7584},
volume = {25},
year = {1982}
}
@book{Koch1999,
author = {Koch, S. W. and Chow, W. W.},
publisher = {Springer},
title = {{Semiconductor-Laser Fundamentals}},
year = {1999}
}
@article{Koch1999a,
author = {Koch, S. W. and Hader, J. and Moloney, J. V.},
journal = {IEEE Journal of Quantum Electronics},
pages = {1878--1886},
title = {{Microscopic Theory of Gain, Absorbtion, and Refractive Index in Semiconductor Laser Materials-Influence of Conduction-Band Nonparabolicity and Coulomb-Induced Intersubband Coupling}},
volume = {35},
year = {1999}
}
@article{Altarelli1983,
abstract = {Self-consistent electronic structure calculations in the envelope-function approximation are performed for InAs-GaSb superlattices, with a three-band k⃗·p⃗ formalism and suitable boundary conditions. The subband dispersion for k⃗ not parallel to the growth axis, realistically computed for the first time, obeys a no-crossing rule which opens small (≲10 meV) gaps between conduction-band-like and valence-band-like subbands. It is therefore argued that the semimetallic behavior observed for periods d≥180 \AA is dominated by extrinsic effects.},
author = {Altarelli, M.},
doi = {10.1103/PhysRevB.28.842},
file = {:C$\backslash$:/Users/Yossi Michaeli/AppData/Local/Mendeley Ltd./Mendeley Desktop/Downloaded/8VTHAHN7/p842\_1.html:html},
journal = {Physical Review B},
month = {יולי},
number = {2},
pages = {842--845},
publisher = {APS},
title = {{Electronic structure and semiconductor-semimetal transition in InAs-GaSb superlattices}},
url = {http://link.aps.org/doi/10.1103/PhysRevB.28.842},
volume = {28},
year = {1983}
}
@article{Eppenga1987a,
author = {Eppenga, R.},
journal = {Physical Review B},
pages = {1554--1564},
title = {{New k · p theory for GaAs/Ga1-xAlxAs-type quantum wells,}},
volume = {36},
year = {1987}
}
@book{Jackson1998,
author = {Jackson, J. D.},
edition = {3rd},
publisher = {John Wiley and Sons, Inc.},
title = {{Classical Electrodynamics}},
year = {1998}
}
@book{haug_quantum_1994,
author = {Haug, H and Koch, S W},
edition = {4},
publisher = {World Scientific},
title = {{Quantum Theory of the Optical and Electronic Properties of Semiconductors}},
year = {1994}
}
@article{khitrova_nonlinear_1999,
abstract = {The authors review the nonlinear optical properties of semiconductor quantum wells that are grown inside \{high-Q\} Bragg-mirror microcavities. Light-matter coupling in this system is particularly pronounced, leading in the linear regime to a polaritonic mixing of the excitonic quantum well resonance and the single longitudinal cavity mode. The resulting normal-mode splitting of the optical resonance is observed in reflection, transmission, and luminescence experiments. In the nonlinear regime the strong light-matter coupling influences the excitation-dependent bleaching of the normal-mode resonances for nonresonant excitation, leads to transient saturation and normal-mode oscillations for resonant pulsed excitation and is responsible for the density-dependent signatures in the luminescence characteristics. These and many more experimental observations are summarized and explained in this review using a microscopic theory for the Coulomb interacting electron-hole system in the quantum well that is nonperturbatively coupled to the cavity light field.},
annote = {Copyright \{(C)\} 2010 The American Physical Society; Please report any problems to prola@aps.org},
author = {Khitrova, G and Gibbs, H M and Jahnke, F and Kira, M and Koch, S W},
doi = {10.1103/RevModPhys.71.1591},
journal = {Reviews of Modern Physics},
number = {5},
pages = {1591},
title = {{Nonlinear optics of normal-mode-coupling semiconductor microcavities}},
url = {http://link.aps.org/abstract/RMP/v71/p1591},
volume = {71},
year = {1999}
}
@book{bir_symmetry_1976,
author = {Bir, G L and Pikus, G E and Shelnitz, P and Louvish, D},
isbn = {0706513673, 9780706513677},
pages = {484},
publisher = {Israel Program for Scientific Translations},
title = {{Symmetry and strain-induced effects in semiconductors}},
year = {1976}
}
@book{yu_fundamentals_1999,
author = {Yu, P Y and Cardona, M},
edition = {2nd},
isbn = {354065352X},
publisher = {\{Springer-Verlag\} Telos},
shorttitle = {Fundamentals of Semiconductors},
title = {{Fundamentals of Semiconductors: Physics and Materials Properties}},
year = {1999}
}
@article{kane_band_1957,
abstract = {The band structure of \{InSb\} is calculated using the k ·. p perturbation approach and assuming that the conduction and valence band extrema are at k = 0. The small band gap requires an accurate treatment of conduction and valence band interactions while higher bands are treated by perturbation theory. A highly nonparabolic conduction band is found. The valence band is quite similar to germanium. Energy terms linear in k which cannot exist in germanium are estimated and found to be small, though possibly of importance at liquid-helium temperature. An absolute calculation of the fundamental optical absorption is made using the cyclotron resonance mass for n-type \{InSb.\} The agreement with experimental data for the fundamental absorption and its dependence on n-type impurity concentration is quite good. This evidence supports the assumptions made concerning the band structure.},
author = {Kane, E O},
doi = {10.1016/0022-3697(57)90013-6},
issn = {0022-3697},
journal = {Journal of Physics and Chemistry of Solids},
number = {4},
pages = {249--261},
title = {{Band structure of indium antimonide}},
url = {http://www.sciencedirect.com/science/article/B6TXR-46MF59X-8N/2/bef159baa98d883244251c3b38c88fbd},
volume = {1},
year = {1957}
}
@article{foreman_effective-mass_1993,
abstract = {Using the recently developed exact envelope-function theory, an explicit form for the effective-mass Hamiltonian is derived for the valence bands (including the spin-orbit split-off band) of a semiconductor quantum well or superlattice. It is shown that the correct form of the Hamiltonian gives physically reasonable results, while the commonly used ‘‘symmetrized’’ form can produce nonphysical solutions for the heavy-hole subbands in which the quantum-well effective mass is very sensitive to the difference in Luttinger parameters between the well and the barrier. This problem arises because the correct boundary conditions for the heavy-hole states are determined exclusively through interaction with other p states, while the symmetrized boundary conditions implicitly incorporate the much larger s-state interaction, hence they substantially overestimate the magnitude of the interband coupling.},
annote = {Copyright \{(C)\} 2010 The American Physical Society; Please report any problems to prola@aps.org},
author = {Foreman, B A},
doi = {10.1103/PhysRevB.48.4964},
journal = {Physical Review B},
number = {7},
pages = {4964},
title = {{Effective-mass Hamiltonian and boundary conditions for the valence bands of semiconductor microstructures}},
url = {http://link.aps.org/doi/10.1103/PhysRevB.48.4964},
volume = {48},
year = {1993}
}
@article{savona_exact_1995,
abstract = {We develop a general quantum mechanical method to calculate the dispersion relations of polariton modes for a quantum well embedded in a semiconductor planar microcavity. The method is based on the properties of the Green's function of the electromagnetic field in a one dimensional dielectric structure. By means of this approach it is possible to express the dispersion relations in a compact form once the eigenmodes of the electromagnetic field in the region where the quantum well is placed are known. We discuss tile specific example of a microcavity with equal distributed Bragg reflectors at the two ends. The dispersion relation in this case is calculated analytically. The method can in principle be applied to dielectric one dimensional structures of arbitrary dielectric profile.},
author = {Savona, V and Tassone, F},
doi = {10.1016/0038-1098(95)00355-X},
journal = {Solid State Communications},
keywords = { Excitons, Polaritons, Quantum Wells, Semiconductor Microcavities,1},
pages = {673--678},
title = {{Exact quantum calculation of polariton Dispersion in semiconductor microcavities}},
url = {http://www.ingentaconnect.com/content/els/00381098/1995/00000095/00000010/art00355},
volume = {95},
year = {1995}
}
@article{morrow_model_1984,
abstract = {We consider a class of Hermitian effective-mass Hamiltonians whose kinetic energy term is (m$\alpha$p\{$\backslash$textasciicircum\}m$\beta$p\{$\backslash$textasciicircum\}m$\gamma$+m$\gamma$p\{$\backslash$textasciicircum\}m$\beta$p\{$\backslash$textasciicircum\}m$\alpha$)/4 with $\alpha$+$\beta$+$\gamma$=-1. We apply these Hamiltonians to an abrupt heterojunction between two crystals and seek the matching conditions across the junction on the effective-mass wave function ($\psi$) and its spatial derivative ($\psi$̇). For $\alpha$≠$\gamma$ we find that the wave function must vanish at the junction thus implying that the junction acts as an impenetrable barrier. Consequently, the only viable cases are for $\alpha$=$\gamma$ where we show that m$\alpha$$\psi$ and m$\alpha$+$\beta$$\psi$̇ must be continuous across the junction.},
annote = {Copyright \{(C)\} 2010 The American Physical Society; Please report any problems to prola@aps.org},
author = {Morrow, R A and Brownstein, K R},
doi = {10.1103/PhysRevB.30.678},
journal = {Physical Review B},
number = {2},
pages = {678},
title = {{Model effective-mass Hamiltonians for abrupt heterojunctions and the associated wave-function-matching conditions}},
url = {http://link.aps.org/doi/10.1103/PhysRevB.30.678},
volume = {30},
year = {1984}
}
@book{mahan_many-particle_1997,
author = {Mahan, G D},
publisher = {New York, Plenum},
title = {{\{Many-Particle\} Physics}},
year = {1997}
}
@article{veprek_ellipticity_2007,
abstract = {We present an explanation to the spurious solution problem affecting the k∙p envelope function method, indicating that the problem is mathematically caused by loss of ellipticity of the differential operator. Focusing on direct band gap zinc-blende heterostructures, we derive criteria that must be fulfilled by the input parameters in order to establish ellipticity. Using these criteria, we compare symmetrized operators with \{Burt-Foreman\} \{[B.\} A. Foreman, Phys. Rev. B 56, R12748 (1997)] operator ordering. We substantiate our arguments with numerical results obtained using linear finite elements. We find that the space of stable input parameters is very narrow and demonstrate that \{Burt-Foreman\} operator ordering together with experimental k∙p input parameters leads to near-elliptic envelope equations in the 4×4 and 6×6 models, whereas symmetrization yields strong nonellipticity. In the k∙p 8×8 model, the procedure of renormalizing parameters of the 4×4 model generally yields parameters producing spurious solutions, even for \{Burt-Foreman\} operator ordering. We find that this problem can be resolved by using a smaller optical matrix parameter Ep. This suggests that the parametrization of k∙p models for heterostructures of any dimensionality must be reviewed, checking against the mathematical ellipticity of the equation system.},
annote = {Copyright \{(C)\} 2010 The American Physical Society; Please report any problems to prola@aps.org},
author = {Veprek, R G and Steiger, S and Witzigmann, B},
doi = {10.1103/PhysRevB.76.165320},
journal = {Physical Review B},
number = {16},
pages = {165320},
title = {{Ellipticity and the spurious solution problem of kbullp envelope equations}},
url = {http://link.aps.org/doi/10.1103/PhysRevB.76.165320},
volume = {76},
year = {2007}
}
@book{harrison_quantum_2000,
author = {Harrison, P},
publisher = {John Wiley \& Sons Canada, Ltd.},
title = {{Quantum Wells, Wires and Dots Theoretical and Computational Physics}},
year = {2000}
}
@article{ren_valence-band_1999,
author = {Ren, G B and Liu, Y M and Blood, P},
doi = {10.1063/1.123461},
issn = {00036951},
journal = {Applied Physics Letters},
number = {8},
pages = {1117},
title = {{Valence-band structure of wurtzite \{GaN\} including the spin-orbit interaction}},
url = {http://adsabs.harvard.edu/abs/1999ApPhL..74.1117R},
volume = {74},
year = {1999}
}
@inproceedings{koch_microscopic_2002,
author = {Koch, S W},
booktitle = {NUSOD-02},
title = {{Microscopic Modelling of Gain and Luminescence in Semicinductors}},
year = {2002}
}
@article{eppenga_new_1987-1,
abstract = {A new k⋅p theory for the description of \{GaAs/Ga1-xAlxAs-type\} quantum wells is presented. The theory combines a unified description of electron and hole states with only five adjustable parameters for each material constituting the quantum well. Unlike earlier k⋅p work it fully accounts for the coupling between the lowest electron, the light-hole, the heavy-hole, and the spin-orbit split-off hole bands and the coupling to all other bands is taken into account perturbatively. The theory thereby also applies to quantum wells where the spin-orbit splitting is comparable to the hole-confinement potential energies. The full inclusion of the k? dependence of confinement energies and electron-hole transition strengths allows for accurate predictions of excitation spectra of quantum wells. In this respect the results of our simple k⋅p theory stand comparison to the results of the more complicated tight-binding theory of Chang and Schulman. Our theory can thus explain the recently observed $\Delta$n≠0 transitions. As a final application we have calculated gain spectra of quantum-well lasers.},
annote = {Copyright \{(C)\} 2010 The American Physical Society; Please report any problems to prola@aps.org},
author = {Eppenga, R and Schuurmans, M F H and Colak, S},
doi = {10.1103/PhysRevB.36.1554},
journal = {Physical Review B},
number = {3},
pages = {1554},
title = {{New k·p theory for \{GaAs/Ga1-xA1xAs-type\} quantum wells}},
url = {http://link.aps.org/doi/10.1103/PhysRevB.36.1554},
volume = {36},
year = {1987}
}
@article{veprek_operator_2009,
author = {Veprek, R G and Steiger, S and Witzigmann, B},
journal = {Optical and Quantum Electronics},
number = {14-15},
pages = {1169--1174},
title = {{Operator ordering, ellipticity and spurious solutions in k · p calculations of \{III-nitride\} nanostructures}},
volume = {40},
year = {2009}
}
@article{foreman_analytical_1998,
abstract = {An analytical theory of intervalley mixing at semiconductor heterojunctions is presented. Burt's envelope-function representation is used to analyze a pseudopotential Hamiltonian, yielding a simple $\delta$-function mixing between $\Gamma$ and X electrons and light and heavy holes. This coupling exists even for media differing only by a constant band offset (i.e., with no difference in Bloch functions).},
annote = {Copyright \{(C)\} 2010 The American Physical Society; Please report any problems to prola@aps.org},
author = {Foreman, B A},
doi = {10.1103/PhysRevLett.81.425},
journal = {Physical Review Letters},
number = {2},
pages = {425},
title = {{Analytical \{Envelope-Function\} Theory of Interface Band Mixing}},
url = {http://link.aps.org/doi/10.1103/PhysRevLett.81.425},
volume = {81},
year = {1998}
}
@book{kane_handbooksemiconductors_1982,
author = {Kane, E O},
publisher = {W.Paul},
title = {{Handbook on semiconductors}},
volume = {1},
year = {1982}
}
@article{bendaniel_space-charge_1966,
abstract = {The one-electron \{(Bethe-Sommerfeld)\} model of electron tunneling is formulated to describe tunneling when the curvature (electron mass) and centroid of the one-electron constant-energy surfaces vary across the junction. The conductance for an abrupt \{GaAs\} p-n tunnel diode is calculated and shown to exhibit minima near zero bias for highly asymmetrical doping ratios. The conductance of metal-oxide-semimetal \{(M-O-SM)\} tunnel junctions is evaluated both with and without the inclusion of space-charge effects and of surface states. All calculations are performed using solvable models for which the \{WKBJ\} approximation is not imposed. Neither the removal of the \{WKBJ\} approximation nor the space-charge effects give rise to maxima in the conductance of the \{M-O-SM\} junctions near a band edge.},
annote = {Copyright \{(C)\} 2010 The American Physical Society; Please report any problems to prola@aps.org},
author = {BenDaniel, D J and Duke, C B},
doi = {10.1103/PhysRev.152.683},
journal = {Physical Review},
number = {2},
pages = {683},
title = {{\{Space-Charge\} Effects on Electron Tunneling}},
url = {http://link.aps.org/doi/10.1103/PhysRev.152.683},
volume = {152},
year = {1966}
}
@book{klingshirn_semiconductor_2006,
author = {Klingshirn, C},
edition = {3rd},
isbn = {354038345X},
publisher = {Springer},
title = {{Semiconductor Optics}},
year = {2006}
}
@article{enders_kp_1995,
abstract = {This paper examines an eight-band k⋅p theory of strained semiconductors yielding energy bands, wave functions, and momentum matrices. Only if the symmetry of the strained crystal is accounted for in all terms of the Hamiltonian, a consistent definition and calculation of the momentum matrix becomes possible. The band structure and wave functions are nonanalytical functions of strain and crystal momentum. For strained crystals, the extrapolation from the $\Gamma$ point into the Brillouin zone, such as the effective-mass approximation for the optical-matrix elements, can be misleading. For certain cases, the heavy- and light-hole isoenergetic surfaces form complex figures resembling the indicatrix of birefringent biaxial crystals. The symmetry of the hole wave functions causes dichroism for photon energies close to the gap energy, while the crystal becomes optically isotropic for larger photon energies. Numerical results are presented for the eight-band k⋅p model of biaxially strained bulklike \{1.3-$\mu$m–InxGa1-xAsyP1-y\} on \{InP\} being an important material in optoelectronics.},
annote = {Copyright \{(C)\} 2010 The American Physical Society; Please report any problems to prola@aps.org},
author = {Enders, P and Borwolff, A and Woerner, M and Suisky, D},
doi = {10.1103/PhysRevB.51.16695},
journal = {Physical Review B},
number = {23},
pages = {16695},
title = {k·p theory of energy bands, wave functions, and optical selection rules in strained tetrahedral semiconductors},
url = {http://link.aps.org/doi/10.1103/PhysRevB.51.16695},
volume = {51},
year = {1995}
}
@article{luttinger_quantum_1956,
abstract = {The most general form of the Hamiltonian of an electron or hole in a semiconductor such as Si or Ge, in the presence of an external homogeneous magnetic field, is given. Two methods of obtaining the corresponding energy levels are discussed. The first should yield very accurate values for the magnetic field in the (111) direction for either Si or Ge. The second is a perturbation method and is expected to give good results only for Ge.},
annote = {Copyright \{(C)\} 2010 The American Physical Society; Please report any problems to prola@aps.org},
author = {Luttinger, J M},
doi = {10.1103/PhysRev.102.1030},
journal = {Physical Review},
number = {4},
pages = {1030},
shorttitle = {Quantum Theory of Cyclotron Resonance in Semiconductors},
title = {{Quantum Theory of Cyclotron Resonance in Semiconductors: General Theory}},
url = {http://link.aps.org/doi/10.1103/PhysRev.102.1030},
volume = {102},
year = {1956}
}
@article{kira_quantum_1999,
abstract = {A fully quantum-mechanical theory for the interaction of light and electron-hole excitations in semiconductor quantum-well systems is developed. The resulting many-body hierarchy for the correlation functions is truncated using a dynamical decoupling scheme leading to coupled semiconductor luminescence and Bloch equations. For incoherent excitation conditions, the theory is used to describe nonlinear excitonic emission properties of single-quantum wells, optically coupled multiple quantum-well systems, and quantum wells in a microcavity. Resonant coherent optical excitation leads to a direct coupling between the induced coherent polarization and photoluminescence. The resulting quantum corrections to the semiclassical semiconductor Bloch equations and the coherent contributions to the semiconductor luminescence equations are discussed. The secondary emission in directions deviating from the coherent excitation direction after femtosecond-pulse excitation is studied. Coherent control and quadrature squeezing for the light emission are analyzed.},
author = {Kira, M and Jahnke, F and Hoyer, W and Koch, S W},
doi = {10.1016/S0079-6727(99)00008-7},
issn = {0079-6727},
journal = {Progress in Quantum Electronics},
number = {6},
pages = {189--279},
title = {{Quantum theory of spontaneous emission and coherent effects in semiconductor microstructures}},
url = {http://www.sciencedirect.com/science/article/B6TJD-3YF44VC-1/2/871827e2264fddc1b8f01ca83ab0b786},
volume = {23},
year = {1999}
}
@book{schafer_semiconductor_2002,
author = {Schafer, W and Wegener, M},
edition = {1},
isbn = {3540616144},
publisher = {Springer},
title = {{Semiconductor Optics and Transport Phenomena}},
year = {2002}
}
@article{martin_theory_1959,
abstract = {This is the first of a series of papers dealing with many-particle systems from a unified, nonperturbative point of view. It contains derivations and discussions of various field-theoretical techniques which will be applied in subsequent papers. In a short introduction the general method of approach is summarized, and its relationship to other field-theoretic problems indicated. In the second section the macroscopic properties of the spectra of many-particle systems are described. Asymptotic evaluations are performed which characterize these macroscopic features in terms of intensive parameters, and the relationship of these parameters to thermodynamics is discussed. The special characteristics of the ground state are shown to follow as a limiting case of the asymptotic evaluations. The third section is devoted to the time-dependent field correlation functions, or Green's functions, which describe the microscopic behavior of a multiparticle system. These functions are defined, and related to intensive macroscopic variables when the energy and number of particles are large. Spectral representations and other properties of various one-particle Green's functions are derived. In the fourth section the treatment of non-equilibrium processes is considered. As a particular example, the electromagnetic properties of a system are expressed in terms of the special two-particle Green's function which describes current correlation. The discussion yields specifically a fluctuation-dissipation theorem, a sum rule for conductivity, and certain dispersion relations. The fifth section deals with the differential equations which determine the Green's functions. The boundary conditions that characterize the Green's function equations are exhibited without reference to adiabatic decoupling. A method for solving the equations approximately, by treating the correlations among successively larger numbers of particles, is considered. The first approximation in this sequence is shown to yield a generalized Hartree-like equation. A related, but rigorous, identity for the single-particle Green's function is then derived. A second approximation, which takes certain two-particle correlations into account, is shown to produce various additional effects: The interaction between particles is altered in a manner characterized by the intensive macroscopic parameters, and the modification and spread of the energy-momentum relation come into play. In the final section compact formal expressions for the Green's functions and other physical quantities are derived. Alternative equations and systematic approximations for the Green's functions are obtained.},
annote = {Copyright \{(C)\} 2010 The American Physical Society; Please report any problems to prola@aps.org},
author = {Martin, P C and Schwinger, J},
doi = {10.1103/PhysRev.115.1342},
journal = {Physical Review},
number = {6},
pages = {1342},
title = {{Theory of \{Many-Particle\} Systems. I}},
url = {http://link.aps.org/doi/10.1103/PhysRev.115.1342},
volume = {115},
year = {1959}
}
@article{broido_effective_1985,
abstract = {The spin-split hole subband structure for a \{GaAs\} p-channel inversion layer is calculated. From it, density-of-states masses and cyclotron masses as a function of magnetic field are extracted. Results indicate significant discrepancies between calculated and measured masses. Many-body effects on the effective mass may be important.},
annote = {Copyright \{(C)\} 2010 The American Physical Society; Please report any problems to prola@aps.org},
author = {Broido, D A and Sham, L J},
doi = {10.1103/PhysRevB.31.888},
journal = {Physical Review B},
number = {2},
pages = {888},
title = {{Effective masses of holes at \{GaAs-AlGaAs\} heterojunctions}},
url = {http://link.aps.org/doi/10.1103/PhysRevB.31.888},
volume = {31},
year = {1985}
}
@article{ehrenreich_self-consistent_1959,
abstract = {The self-consistent field method in which a many-electron system is described by a time-dependent interaction of a single electron with a self-consistent electromagnetic field is shown to be equivalent for many purposes to the treatment given by Sawada and Brout. Starting with the correct many-electron Hamiltonian, it is found, when the approximations characteristic of the \{Sawada-Brout\} scheme are made, that the equation of motion for the pair creation operators is the same as that for the one-particle density matrix in the self-consistent field framework. These approximations are seen to correspond to (1) factorization of the two-particle density matrix, and (2) linearization with respect to off-diagonal components of the one-particle density matrix. The complex, frequency-dependent dielectric constant is obtained straight-forwardly from the self-consistent field approach both for a free-electron gas and a real solid. It is found to be the same as that obtained by Nozi\'{e}res and Pines in the random phase approximation. The resulting plasma dispersion relation for the solid in the limit of long wavelengths is discussed.},
author = {Ehrenreich, H and Cohen, M H},
doi = {10.1103/PhysRev.115.786},
journal = {Physical Review},
number = {4},
pages = {786},
title = {{\{Self-Consistent\} Field Approach to the \{Many-Electron\} Problem}},
url = {http://link.aps.org/doi/10.1103/PhysRev.115.786},
volume = {115},
year = {1959}
}
@misc{_nsm_????,
abstract = {These electronic archive contains frequently needed information for the most important Semiconductor Materials. We have also included basic references where one can find additional information if required. In compiling this information, we took advantage of generous help from many colleagues at the \{A.F.Ioffe\} Institute who made many excellent suggestions, and, in some cases, provided us with more accurate values of material parameters. The Auger recombination coefficients for semiconductor heterostructures are developed and computed by George Zegrya, Natalia Gunko and Anatolii Polkovnikov. The archive is under construction and in this version we present only a part of the data we have computed. We plan to place more data in the archive later.

Your questions, comments and suggestions are welcome. Please contact Vadim Siklitsky or Alexei Tolmatchev. If you find this server helpful, and use the data retrieved through the server for your research, we would appreciate acknowledging it in your papers.},
booktitle = {\{NSM\} Archive - New Semiconductor Materials. Characteristics and Properties},
howpublished = {http://www.ioffe.ru/SVA/},
title = {{\{NSM\} Archive}},
url = {http://www.ioffe.ru/SVA/}
}
